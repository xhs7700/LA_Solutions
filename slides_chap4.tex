\documentclass{beamer}
\usepackage{ctex}
\usepackage[T1]{fontenc}

% other packages
\usepackage{latexsym,xcolor,multicol,booktabs,calligra}
\usepackage{pstricks,listings,stackengine,cite}
\usepackage{anyfontsize}
\usepackage{hyperref}
\usepackage{graphicx}
\usepackage{amsmath,amsthm}

\setbeamertemplate{bibliography item}[text]
% \bibliographystyle{plain}
% 如果参考文献太多的话,可以像下面这样调整字体:
% \tiny\bibliographystyle{plain}

% \usefonttheme[onlymath]{serif}

% \newtheorem{theorem}{定理}[subsection]
% \newtheorem{lemma}{引理}[subsection]
% \newtheorem*{definition}{定义}
% \newtheorem{property}{性质}[subsection]
% \newtheorem{infer}{推论}[subsection]
\newtheorem{problem}{习题}[section]

\newcommand{\bsym}[1]{\boldsymbol{#1}}
\newcommand{\mypar}[1]{\left( #1 \right)}
\newcommand{\abs}[1]{\left|#1 \right|}
\newcommand{\Abs}[1]{\left\Vert#1\right\Vert}
\newcommand{\setof}[1]{\left\{#1 \right\}}
\newcommand{\indot}[2]{\left\langle #1,#2 \right\rangle}
\newcommand{\mat}[1]{\left[ #1 \right]}
\newcommand{\sqmat}[3]{\begin{#1}
{#2}_{11} & {#2}_{12} & \cdots & {#2}_{1{#3}} \\
{#2}_{21} & {#2}_{22} & \cdots & {#2}_{2{#3}} \\
\vdots & \vdots &        & \vdots \\
{#2}_{{#3}1} & {#2}_{{#3}2} & \cdots & {#2}_{{#3}{#3}}
\end{#1}}
\newcommand{\nullsp}[1]{\bsym{\mathrm{N}}\mypar{#1}}
\newcommand{\colsp}[1]{\bsym{\mathrm{C}}\mypar{#1}}
\newcommand{\func}[2]{\mathrm{#1}\mypar{#2}}
\newcommand{\entry}[3]{\func{entry}{#1,#2,#3}}
\newcommand{\row}[2]{\func{row}{#1,#2}}
\newcommand{\col}[2]{\func{col}{#1,#2}}
\newcommand{\trace}[1]{\func{Tr}{#1}}
\newcommand{\diag}[1]{\func{diag}{#1}}
\newcommand{\rank}[1]{\func{rank}{#1}}
\newcommand{\adj}[1]{\func{adj}{#1}}
\newcommand{\myspan}[1]{\func{span}{#1}}
\newcommand{\enums}[2]{{#1}_1,{#1}_2,\dots,{#1}_{#2}}
\newcommand{\inv}[1]{{#1}^{-1}}
\newcommand{\ortcom}[1]{{#1}^{\bot}}

\renewcommand{\det}[1]{\func{det}{#1}}

\newcommand{\mata}{\bsym{A}}
\newcommand{\matb}{\bsym{B}}
\newcommand{\matc}{\bsym{C}}
\newcommand{\matd}{\bsym{D}}
\newcommand{\mate}{\bsym{E}}
\newcommand{\matf}{\bsym{F}}
\newcommand{\matg}{\bsym{G}}
\newcommand{\matH}{\bsym{H}}
\newcommand{\mati}{\bsym{I}}
\newcommand{\matj}{\bsym{J}}
\newcommand{\matk}{\bsym{K}}
\newcommand{\matl}{\bsym{L}}
\newcommand{\matlam}{\bsym{\Lambda}}
\newcommand{\matm}{\bsym{M}}
\newcommand{\matn}{\bsym{N}}
\newcommand{\mato}{\bsym{O}}
\newcommand{\matp}{\bsym{P}}
\newcommand{\matq}{\bsym{Q}}
\newcommand{\matr}{\bsym{R}}
\newcommand{\mats}{\bsym{S}}
\newcommand{\matsig}{\bsym{\Sigma}}
\newcommand{\matt}{\bsym{T}}
\newcommand{\matu}{\bsym{U}}
\newcommand{\matv}{\bsym{V}}
\newcommand{\matw}{\bsym{W}}
\newcommand{\matx}{\bsym{X}}
\newcommand{\maty}{\bsym{Y}}
\newcommand{\matz}{\bsym{Z}}

\newcommand{\field}{\bsym{\mathrm{F}}}
\newcommand{\rea}{\bsym{\mathrm{R}}}

\newcommand{\veca}{\bsym{a}}
\newcommand{\vecal}{\bsym{\alpha}}
\newcommand{\vecb}{\bsym{b}}
\newcommand{\vecbeta}{\bsym{\beta}}
\newcommand{\vecc}{\bsym{c}}
\newcommand{\vecd}{\bsym{d}}
\newcommand{\vecdelta}{\bsym{\delta}}
\newcommand{\vece}{\bsym{e}}
\newcommand{\veceps}{\bsym{\epsilon}}
\newcommand{\vecf}{\bsym{f}}
\newcommand{\vecg}{\bsym{g}}
\newcommand{\vecgamma}{\bsym{\gamma}}
\newcommand{\vech}{\bsym{h}}
\newcommand{\veceta}{\bsym{\eta}}
\newcommand{\veci}{\bsym{i}}
\newcommand{\vecj}{\bsym{j}}
\newcommand{\veck}{\bsym{k}}
\newcommand{\vecl}{\bsym{l}}
\newcommand{\vecm}{\bsym{m}}
\newcommand{\vecn}{\bsym{n}}
\newcommand{\veco}{\bsym{o}}
\newcommand{\vecp}{\bsym{p}}
\newcommand{\vecq}{\bsym{q}}
\newcommand{\vecr}{\bsym{r}}
\newcommand{\vecs}{\bsym{s}}
\newcommand{\vect}{\bsym{t}}
\newcommand{\vecu}{\bsym{u}}
\newcommand{\vecv}{\bsym{v}}
\newcommand{\vecw}{\bsym{w}}
\newcommand{\vecx}{\bsym{x}}
\newcommand{\vecy}{\bsym{y}}
\newcommand{\vecz}{\bsym{z}}
\newcommand{\veczero}{\bsym{0}}


\author{夏海淞}
\title{线性代数习题课}
\subtitle{第四章}
% \institute{School of Computer Science, Fudan University}
\date{\today}
\usepackage{College}

% defs
\def\cmd#1{\texttt{\color{red}\footnotesize $\backslash$#1}}
\def\env#1{\texttt{\color{blue}\footnotesize #1}}
\definecolor{deepblue}{rgb}{0,0,0.5}
\definecolor{deepred}{rgb}{0.6,0,0}
\definecolor{deepgreen}{rgb}{0,0.5,0}
\definecolor{halfgray}{gray}{0.55}

\lstset{
    basicstyle=\ttfamily\small,
    keywordstyle=\bfseries\color{deepblue},
    emphstyle=\ttfamily\color{deepred},    % Custom highlighting style
    stringstyle=\color{deepgreen},
    numbers=left,
    numberstyle=\small\color{halfgray},
    rulesepcolor=\color{red!20!green!20!blue!20},
    frame=shadowbox,
}

\begin{document}
\setlength{\parskip}{0.45\baselineskip}

\begin{frame}
    \titlepage
    \begin{figure}[htpb]
        \begin{center}
            \vspace*{-0.5cm}
            \includegraphics[width=0.1\linewidth]{pic/FDU.jpeg}
        \end{center}
    \end{figure}

\end{frame}

\subsection*{4.1}
\begin{frame}
    \frametitle{题面}

    在次数不大于\(2\)的多项式线性空间\(P_2[x]\)中,试证:\(f_1=1\),\(f_2=x-1\),\(f_3=\mypar{x-2}\mypar{x-1}\)线性无关。

\end{frame}

\begin{frame}
    \frametitle{解答}



\end{frame}

\subsection*{4.2}
\begin{frame}
    \frametitle{题面}

    证明:以下三个多项式为\(P_2[x]\)的一组基:
    \begin{equation*}
        f_1=1,f_2=x-1,f_3=\mypar{x-1}^2
    \end{equation*}
    再求\(g(x)=5x^2+x+3\)在此基下的坐标。

\end{frame}

\begin{frame}
    \frametitle{解答}



\end{frame}

\subsection*{4.3}
\begin{frame}
    \frametitle{题面}

    在次数不大于\(3\)的多项式空间\(P_3[x]\)中,
    \begin{enumerate}
        \item 求由基\(1,x,x^2,x^3\)到基\(1,1+x,\mypar{1+x}^2,\mypar{1+x}^3\)的过渡矩阵;
        \item 求\(f(x)=a_0+a_1x+a_2x^2+a_3x^3\)在基\(1,1+x,\mypar{1+x}^2,\mypar{1+x}^3\)下的坐标。
    \end{enumerate}

\end{frame}

\begin{frame}
    \frametitle{解答}



\end{frame}

\subsection*{4.4}
\begin{frame}
    \frametitle{题面}

    在\(P_3[x]\)的多项式空间中,旧基为\(1,x,x^2,x^3\);新基为\(1\),\(1+x\),\(1+x+x^2\),\(1+x+x^2+x^3\)。
    \begin{enumerate}
        \item 求旧基到新基的过渡矩阵;
        \item 求多项式\(1+2x+3x^2+4x^3\)在新基下的坐标;
        \item 若多项式\(f(x)\)在新基下的坐标为\(\mat{1,2,3,4}^\top\),求它在旧基下的坐标。
    \end{enumerate}

\end{frame}

\begin{frame}
    \frametitle{解答}



\end{frame}

\subsection*{4.5}
\begin{frame}
    \frametitle{题面}

    已知\(\vecxi\)在基\(\matb_1=\setof{\vecal_1,\vecal_2,\vecal_3}\)下的坐标为\(\vecxi_{\matb_1}=\mat{1,-2,2}^\top\),求\(\vecxi\)在基\(\matb_2=\setof{\vecbeta_1,\vecbeta_2,\vecbeta_3}\)下的坐标\(\vecxi_{\matb_2}\),其中\(\vecbeta_1=\vecal_1+\vecal_2\),\(\vecbeta_2=\vecal_2+\vecal_3\),\(\vecbeta_3=\vecal_3+\vecal_1\)。

\end{frame}

\begin{frame}
    \frametitle{解答}



\end{frame}

\subsection*{4.6}
\begin{frame}
    \frametitle{题面}

    设\(\vecveps_1,\vecveps_2,\vecveps_3\)是线性空间\(V\)的一组基,且
    \begin{equation*}
        \begin{cases}
            \vecxi_1=\vecveps_1+\vecveps_3 \\
            \vecxi_2=\vecveps_2            \\
            \vecxi_3=\vecveps_1+2\vecveps_2+2\vecveps_3
        \end{cases}
    \end{equation*}
    \begin{equation*}
        \begin{cases}
            \veceta_1=\vecveps_1            \\
            \veceta_2=\vecveps_1+\vecveps_2 \\
            \veceta_3=\vecveps_1+\vecveps_2+\vecveps_3
        \end{cases}
    \end{equation*}
    \begin{enumerate}
        \item 试证\(\vecxi_1,\vecxi_2,\vecxi_3\)及\(\veceta_1,\veceta_2,\veceta_3\)都是\(V\)的一组基;
        \item 求由基\(\vecxi_1,\vecxi_2,\vecxi_3\)到基\(\veceta_1,\veceta_2,\veceta_3\)的过渡矩阵。
    \end{enumerate}

\end{frame}

\begin{frame}
    \frametitle{解答}



\end{frame}

\subsection*{4.7}
\begin{frame}[allowframebreaks]
    \frametitle{题面}

    在线性空间\(\rea^{2\times2}\)中,已知
    \begin{equation*}
        \vecal_1=\begin{bmatrix}1&0\\0&0\end{bmatrix},
        \vecal_2=\begin{bmatrix}0&1\\0&0\end{bmatrix},
        \vecal_3=\begin{bmatrix}0&0\\1&0\end{bmatrix},
        \vecal_4=\begin{bmatrix}0&0\\0&1\end{bmatrix}
    \end{equation*}
    为其一组基,若\(\rea^{2\times2}\)的另一组基为\(\vecbeta_1,\vecbeta_2,\vecbeta_3,\vecbeta_4\),由\(\vecal_1,\vecal_2,\vecal_3,\vecal_4\)到\(\vecbeta_1,\vecbeta_2,\vecbeta_3,\vecbeta_4\)的过渡矩阵为
    \begin{equation*}
        \mata=
        \begin{bmatrix}
            0 & 1 & 1 & 1 \\
            1 & 0 & 1 & 1 \\
            1 & 1 & 0 & 1 \\
            1 & 1 & 1 & 0
        \end{bmatrix}
    \end{equation*}
    求:
    \begin{enumerate}
        \item 基\(\vecbeta_1,\vecbeta_2,\vecbeta_3,\vecbeta_4\);
        \item 矩阵\begin{equation*}\begin{bmatrix}0&1\\2&-3\end{bmatrix}\end{equation*}在基\(\vecbeta_1,\vecbeta_2,\vecbeta_3,\vecbeta_4\)下的坐标。
    \end{enumerate}

\end{frame}

\begin{frame}
    \frametitle{解答}



\end{frame}

\subsection*{4.8}
\begin{frame}
    \frametitle{题面}

    已知实数域\(\rea\)上的所有\(2\)阶矩阵,对于矩阵的加法和数乘,构成\(\rea\)上的四维线性空间,记作\(V=\rea^{2\times2}\)。
    \begin{enumerate}
        \item {
              分别证明
              \begin{equation*}
                  \vecal_1=\begin{bmatrix}1&0\\0&0\end{bmatrix},
                  \vecal_2=\begin{bmatrix}1&1\\0&0\end{bmatrix},
                  \vecal_3=\begin{bmatrix}1&1\\1&0\end{bmatrix},
                  \vecal_4=\begin{bmatrix}1&1\\1&1\end{bmatrix}
              \end{equation*}
              与
              \begin{equation*}
                  \vecbeta_1=\begin{bmatrix}-1&1\\1&1\end{bmatrix},
                  \vecbeta_2=\begin{bmatrix}1&-1\\1&1\end{bmatrix},
                  \vecbeta_3=\begin{bmatrix}1&1\\-1&1\end{bmatrix},
                  \vecbeta_4=\begin{bmatrix}1&1\\1&-1\end{bmatrix}
              \end{equation*}
              均为\(\rea^{2\times2}\)的基;}
        \item 对于\(\rea^{2\times2}\)中的任意元素\(\vecal\),求\(\vecal\)在基\(\vecal_1,\vecal_2,\vecal_3,\vecal_4\)下的坐标;
        \item 求由基\(\vecal_1,\vecal_2,\vecal_3,\vecal_4\)到\(\vecbeta_1,\vecbeta_2,\vecbeta_3,\vecbeta_4\)的过渡矩阵。
    \end{enumerate}

\end{frame}

\begin{frame}
    \frametitle{解答}



\end{frame}

\subsection*{4.9}
\begin{frame}
    \frametitle{题面}

    求下列两个齐次线性方程组的解空间的基和维数:
    \begin{enumerate}
        \item \(x_1+x_2+\cdots+x_n=0\);
        \item \begin{equation*}
                  \begin{cases}
                      2x_1-4x_2+5x_3+3x_4=0 \\
                      3x_1-6x_2+4x_3+2x_4=0 \\
                      4x_1-8x_2+17x_3+11x_4=0
                  \end{cases}
              \end{equation*}
    \end{enumerate}

\end{frame}

\begin{frame}
    \frametitle{解答}



\end{frame}

\subsection*{4.10}
\begin{frame}
    \frametitle{题面}

    设\(\enums{\vecal}{n}\)是\(n\)维线性空间\(V\)的一组基,又\(V\)中向量\(\vecal_{n+1}\)在这组基下坐标\(\mypar{\enums{x}{n}}\)全都不为零。

    证明\(\enums{\vecal}{n},\vecal_{n+1}\)中任意\(n\)个向量必构成\(V\)的一组基,并求\(\vecal_1\)在基\(\vecal_2,\dots,\vecal_n,\vecal_{n+1}\)下的坐标。

\end{frame}

\begin{frame}
    \frametitle{解答}



\end{frame}

\subsection*{4.12}
\begin{frame}
    \frametitle{题面}

    设\(V_1,V_2\)是\(\rea^n\)的两个非平凡子空间,证明:在\(\rea^n\)中存在向量\(\vecal\),使\(\vecal\notin V_1\),且\(\vecal\notin V_2\),并在\(\rea^3\)中举例说明此结论。

\end{frame}

\begin{frame}
    \frametitle{解答}



\end{frame}

\subsection*{4.13}
\begin{frame}
    \frametitle{题面}

    设\(\vecal,\vecbeta,\vecgamma\in\rea^n\),\(c_1,c_2,c_3\in\rea\),且\(c_1c_3\neq0\),证明:若\(c_1\vecal+c_2\vecbeta+c_3\vecgamma=\veczero\),则\(\myspan{\vecal,\vecbeta}=\myspan{\vecbeta,\vecgamma}\)。

\end{frame}

\begin{frame}
    \frametitle{解答}



\end{frame}

\subsection*{4.14}
\begin{frame}
    \frametitle{题面}

    若
    \begin{equation*}
        \mata=\begin{bmatrix}1&1&1&0\\2&1&0&1\end{bmatrix}
    \end{equation*}
    求\(\mata\)的零空间\(\nullsp{\mata}\)的一组基。

\end{frame}

\begin{frame}
    \frametitle{解答}



\end{frame}

\subsection*{4.16}
\begin{frame}
    \frametitle{题面}

    已知向量组\(\vecal_1=\mat{2,0,1,3,-1}^\top\),\(\vecal_2=\mat{0,-2,1,5,-3}^\top\),\(\vecbeta_1=\mat{1,1,0,-1,1}^\top\),\(\vecbeta_2=\mat{1,-3,2,0,5}^\top\),且\(W_1=\myspan{\vecal_1,\vecal_2}\),\(W_2=\myspan{\vecbeta_1,\vecbeta_2}\)。试求\(W_1\cap W_2\)与\(W_1+W_2\)的维数以及各自的一组基。

\end{frame}

\begin{frame}
    \frametitle{解答}



\end{frame}

\subsection*{4.17}
\begin{frame}
    \frametitle{题面}

    设\(V\)是内积空间,试证明:
    \begin{enumerate}
        \item 对\(\forall\vecal\in V\),都有\(\indot{\veczero}{\vecdelta}=\veczero\);
        \item 若对\(\forall\vecal\in V\),恒有\(\indot{\vecal}{\vecbeta}=\veczero\),则\(\vecbeta=\veczero\);
        \item 已知\(\vecal,\vecbeta\in V\),如对\(\forall\vecgamma\in V\),都有\(\indot{\vecal}{\vecgamma}=\indot{\vecbeta}{\vecgamma}\)。试证明\(\vecal=\vecbeta\)。
    \end{enumerate}

\end{frame}

\begin{frame}
    \frametitle{解答}



\end{frame}

\subsection*{4.20}
\begin{frame}
    \frametitle{题面}

    设\(\enums{\vecal}{n}\)是内积空间的\(n\)个向量,行列式
    \begin{equation*}
        \gram{\enums{\vecal}{n}}=
        \begin{vmatrix}
            \indot{\vecal_1}{\vecal_1} & \indot{\vecal_1}{\vecal_2} & \cdots & \indot{\vecal_1}{\vecal_n} \\
            \indot{\vecal_2}{\vecal_1} & \indot{\vecal_2}{\vecal_2} & \cdots & \indot{\vecal_2}{\vecal_n} \\
            \vdots                     & \vdots                     &        & \vdots                     \\
            \indot{\vecal_n}{\vecal_1} & \indot{\vecal_n}{\vecal_2} & \cdots & \indot{\vecal_n}{\vecal_n}
        \end{vmatrix}
    \end{equation*}
    称为\(\enums{\vecal}{n}\)的Gram行列式。证明\(\gram{\enums{\vecal}{n}}=0\)的充分必要条件是\(\enums{\vecal}{n}\)线性相关。

\end{frame}

\begin{frame}
    \frametitle{解答}



\end{frame}

\subsection*{4.21}
\begin{frame}
    \frametitle{题面}

    设\(\enums{\vecal}{r}\)线性无关,\(\vecbeta_1,\vecbeta_2\)也线性无关,且后者与前者中的每个向量都正交,证明:\(\vecbeta_1,\vecbeta_2,\enums{\vecal}{r}\)也线性相关。

\end{frame}

\begin{frame}
    \frametitle{解答}



\end{frame}

\subsection*{4.22}
\begin{frame}
    \frametitle{题面}

    试把\(\mat{1,0,1,0}^\top\),\(\mat{0,1,0,2}^\top\)扩充成为\(\rea^4\)的一组标准正交基。

\end{frame}

\begin{frame}
    \frametitle{解答}



\end{frame}

\subsection*{4.23}
\begin{frame}
    \frametitle{题面}

    已知\(f_1=1\),\(f_2=x-1\),\(f_3=\mypar{x-1}^2\)为三维内积空间\(C\mat{0,1}\)的一组基,内积定义为
    \begin{equation*}
        \indot{f}{g}=\int_0^1f(x)g(x)\mathrm{d}x
    \end{equation*}
    利用Schmidt正交化方法,求与\(f_1,f_2,f_3\)等价的一组标准正交基(两个向量组可相互线性表出称为等价)。

\end{frame}

\begin{frame}
    \frametitle{解答}



\end{frame}

\subsection*{4.24}
\begin{frame}
    \frametitle{题面}

    设\(\enums{\vecal}{n}\)与\(\enums{\vecbeta}{n}\)为\(n\)维欧氏空间的两组基,其中\(\enums{\vecal}{n}\)为标准正交基,又若由\(\enums{\vecal}{n}\)到\(\enums{\vecbeta}{n}\)的过渡矩阵为\(\matm\),证明:\(\enums{\vecbeta}{n}\)为\(V\)的标准正交基的充分必要条件是\(\matm\)为正交矩阵。

\end{frame}

\begin{frame}
    \frametitle{解答}



\end{frame}

\subsection*{4.26}
\begin{frame}
    \frametitle{题面}

    设
    \begin{equation*}
        \mata=
        \begin{bmatrix}
            1 & 2  & -1 \\
            2 & 0  & 1  \\
            2 & -4 & 2  \\
            4 & 0  & 0
        \end{bmatrix},
        \vecb=\begin{bmatrix}-1\\1\\1\\-2\end{bmatrix}
    \end{equation*}
    问线性方程组\(\mata\vecx=\vecb\)是否有解?如果无解,求其最小二乘解。

\end{frame}

\begin{frame}
    \frametitle{解答}



\end{frame}

\subsection*{4.27}
\begin{frame}
    \frametitle{题面}

    求向量\(\vecb\)在矩阵\(\mata\)的列空间中的正交投影。
    \begin{equation*}
        \mata=
        \begin{bmatrix}
            1 & -2 \\
            1 & 0  \\
            1 & 1  \\
            1 & 3
        \end{bmatrix},
        \vecb=\begin{bmatrix}-4\\-3\\3\\0\end{bmatrix}
    \end{equation*}

\end{frame}

\begin{frame}
    \frametitle{解答}



\end{frame}

\subsection*{4.28}
\begin{frame}
    \frametitle{题面}

    设齐次线性方程组
    \begin{equation*}
        \begin{cases}
            2x_1+x_2-x_3+x_4-3x_5=0 \\
            x_1+x_2-x_3+x_5=0
        \end{cases}
    \end{equation*}
    \begin{enumerate}
        \item 求解空间的标准正交基;
        \item 求解空间的正交补空间的标准正交基。
    \end{enumerate}

\end{frame}

\begin{frame}
    \frametitle{解答}



\end{frame}

\subsection*{4.29}
\begin{frame}
    \frametitle{题面}

    设\(\vecal_1,\vecal_2\)和\(\vecal\)是Euclid空间\(\rea^3\)中的向量,
    \begin{equation*}
        \vecal_1=\begin{bmatrix}1\\1\\2\end{bmatrix},
        \vecal_2=\begin{bmatrix}1\\0\\1\end{bmatrix},
        \vecal=\begin{bmatrix}3\\2\\1\end{bmatrix}
    \end{equation*}
    求\(\myspan{\vecal_1,\vecal_2}\)的正交补空间以及向量\(\vecal\)分别在\(\myspan{\vecal_1,\vecal_2}\)及其正交补空间中的正交投影。

\end{frame}

\begin{frame}
    \frametitle{解答}



\end{frame}

\subsection*{4.31}
\begin{frame}
    \frametitle{题面}

    已知在\(\rea^3\)中,线性变换\(T\)在基
    \begin{equation*}
        \vecveps_1=\begin{bmatrix}8\\-6\\7\end{bmatrix},
        \vecveps_2=\begin{bmatrix}-16\\7\\-13\end{bmatrix},
        \vecveps_3=\begin{bmatrix}9\\-3\\7\end{bmatrix}
    \end{equation*}
    下的表示矩阵
    \begin{equation*}
        \mata=
        \begin{bmatrix}
            1  & -18 & 15 \\
            -1 & -22 & 20 \\
            1  & -25 & 22
        \end{bmatrix}
    \end{equation*}
    求\(T\)在基
    \begin{equation*}
        \veceta_1=\begin{bmatrix}1\\-2\\1\end{bmatrix},
        \veceta_2=\begin{bmatrix}3\\-1\\2\end{bmatrix},
        \veceta_3=\begin{bmatrix}2\\1\\2\end{bmatrix}
    \end{equation*}
    下的表示矩阵。

\end{frame}

\begin{frame}
    \frametitle{解答}



\end{frame}

\end{document}

\subsection*{4.33}
\begin{frame}
    \frametitle{题面}

    给定\(\rea^3\)的两组基:
    \begin{equation*}
        \vecveps_1=\begin{bmatrix}1\\0\\1\end{bmatrix},
        \vecveps_2=\begin{bmatrix}2\\1\\0\end{bmatrix},
        \vecveps_3=\begin{bmatrix}1\\1\\1\end{bmatrix}
    \end{equation*}
    和
    \begin{equation*}
        \veceta_1=\begin{bmatrix}1\\2\\-1\end{bmatrix},
        \veceta_2=\begin{bmatrix}2\\2\\-1\end{bmatrix},
        \veceta_3=\begin{bmatrix}2\\-1\\-1\end{bmatrix}
    \end{equation*}
    定义线性变换:\(T\mypar{\vecveps_i}=\veceta_i\),\(i\in\setof{1,2,3}\)。分别求\(T\)在基\(\vecveps_1,\vecveps_2,\vecveps_3\)与\(\veceta_1,\veceta_2,\veceta_3\)下的表示矩阵。

\end{frame}

\begin{frame}
    \frametitle{解答}



\end{frame}