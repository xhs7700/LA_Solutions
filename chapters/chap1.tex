\section{矩阵}

% 1.1
\begin{problem}\label{problem-1.1}
计算下列矩阵的乘积:

\begin{enumerate}
    \item \(\begin{bmatrix}
              4 & 3 & 1 \\1&-2&3\\5&7&0
          \end{bmatrix}\begin{bmatrix}
              7 \\2\\1
          \end{bmatrix}\);
    \item \(\begin{bmatrix}1&2&3\end{bmatrix}\begin{bmatrix}3\\2\\1\end{bmatrix}\);
    \item \(\begin{bmatrix}2\\1\\3\end{bmatrix}\begin{bmatrix}-1&2\end{bmatrix}\);
    \item \(\begin{bmatrix}
              2 & 1  & 4 & 0 \\
              1 & -1 & 3 & 4
          \end{bmatrix}\begin{bmatrix}
              1 & 3  & 1  \\
              0 & -1 & 2  \\
              1 & -3 & 1  \\
              4 & 0  & -2
          \end{bmatrix}\);
    \item \(\begin{bmatrix}
              x_1 & x_2 & x_3
          \end{bmatrix}
          \begin{bmatrix}
              a_{11} & a_{12} & a_{13} \\
              a_{12} & a_{22} & a_{23} \\
              a_{13} & a_{23} & a_{33}
          \end{bmatrix}
          \begin{bmatrix}
              x_1 \\x_2\\x_3
          \end{bmatrix}\)。
\end{enumerate}
\end{problem}
\begin{proof}
    \begin{enumerate}
        \item \(\begin{bmatrix}
                  4 & 3 & 1 \\1&-2&3\\5&7&0
              \end{bmatrix}\begin{bmatrix}
                  7 \\2\\1
              \end{bmatrix}=\begin{bmatrix}
                  35 \\6\\49
              \end{bmatrix}\);
        \item \(\begin{bmatrix}1&2&3\end{bmatrix}\begin{bmatrix}3\\2\\1\end{bmatrix}=\begin{bmatrix}10\end{bmatrix}\);
        \item \(\begin{bmatrix}2\\1\\3\end{bmatrix}\begin{bmatrix}-1&2\end{bmatrix}=\begin{bmatrix}-2&4\\-1&2\\-3&6\end{bmatrix}\);
        \item \(\begin{bmatrix}
                  2 & 1  & 4 & 0 \\
                  1 & -1 & 3 & 4
              \end{bmatrix}\begin{bmatrix}
                  1 & 3  & 1  \\
                  0 & -1 & 2  \\
                  1 & -3 & 1  \\
                  4 & 0  & -2
              \end{bmatrix}=\begin{bmatrix}
                  6  & -7 & 8  \\
                  20 & -5 & -6
              \end{bmatrix}\);
        \item
              \begin{align*}
                    & \begin{bmatrix}
                          x_1 & x_2 & x_3
                      \end{bmatrix}
                  \begin{bmatrix}
                      a_{11} & a_{12} & a_{13} \\
                      a_{12} & a_{22} & a_{23} \\
                      a_{13} & a_{23} & a_{33}
                  \end{bmatrix}
                  \begin{bmatrix}
                      x_1 \\x_2\\x_3
                  \end{bmatrix}                                                                                    \\
                  = & \begin{bmatrix}\sum_{i=1}^3a_{1i}x_i+\sum_{i=1}^3a_{2i}x_i+\sum_{i=1}^3a_{3i}x_i\end{bmatrix}
                  \begin{bmatrix}x_1 \\x_2 \\x_3\end{bmatrix}                                                       \\
                  = & \sum_{1 \leq i<j \leq 3} 2 a_{i j} x_i x_j+\sum_{k=1}^3 a_{k k} x_k^2
              \end{align*}。
    \end{enumerate}
\end{proof}

% 1.2
\begin{problem}\label{problem-1.2}
设\(\mata=\begin{bmatrix}
    1 & 1  & 1  \\
    1 & 1  & -1 \\
    1 & -1 & 1
\end{bmatrix}\),\(\matb=\begin{bmatrix}
    1  & 2  & 3 \\
    -1 & -2 & 4 \\
    0  & 5  & 1
\end{bmatrix}\),求\(3\mata\matb-2\mata\)。
\end{problem}
\begin{proof}
    \begin{align*}
         & \mata\matb           =
        \left[\begin{array}{ccc}
                      1 & 1  & 1  \\
                      1 & 1  & -1 \\
                      1 & -1 & 1
                  \end{array}\right]
        \left[\begin{array}{ccc}
                      1  & 2  & 3 \\
                      -1 & -2 & 4 \\
                      0  & 5  & 1
                  \end{array}\right]=
        \left[\begin{array}{ccc}
                      0 & 5  & 8 \\
                      0 & -5 & 6 \\
                      2 & 9  & 0
                  \end{array}\right]  \\
         & 3\mata\matb-2 \mata  =
        3\left[\begin{array}{ccc}
                       0 & 5  & 8 \\
                       0 & -5 & 6 \\
                       2 & 9  & 0
                   \end{array}\right]
        -2\left[\begin{array}{ccc}
                        1 & 1  & 1  \\
                        1 & 1  & -1 \\
                        1 & -1 & 1
                    \end{array}\right]
        =\left[\begin{array}{ccc}
                       -2 & 13  & 22 \\
                       -2 & -17 & 20 \\
                       4  & 29  & -2
                   \end{array}\right]
    \end{align*}
\end{proof}

% 1.3
\begin{problem}\label{problem-1.3}
设\(\mata=\begin{bmatrix}3&1\\1&-3\end{bmatrix}\),求\(\mata^{50}\)和\(\mata^{51}\)。
\end{problem}
\begin{proof}
    \begin{equation*}
        \begin{array}{lll}
            \mata^2=\begin{bmatrix}3&1\\1&-3\end{bmatrix}\begin{bmatrix}3&1\\1&-3\end{bmatrix}=\begin{bmatrix}10&0\\0&10\end{bmatrix}      \\
            \mata^{50}=\mypar{\mata^2}^{25}={\begin{bmatrix}10&0\\0&10\end{bmatrix}}^{25}=\begin{bmatrix}10^{25}&0\\0&10^{25}\end{bmatrix} \\
            \mata^{51}=\mata^{50}\mata=\begin{bmatrix}10^{25}&0\\0&10^{25}\end{bmatrix}\begin{bmatrix}3&1\\1&-3\end{bmatrix}=\begin{bmatrix}3\cdot10^{25}&10^{25}\\10^{25}&-3\cdot10^{25}\end{bmatrix}
        \end{array}
    \end{equation*}
\end{proof}

% 1.4
\begin{problem}\label{problem-1.4}
\begin{enumerate}
    \item 设\(\mata,\matb\)为\(n\)阶矩阵,且\(\mata\)为对称矩阵,证明\(\matb^\top\mata\matb\)也是对称矩阵;
    \item 设\(\mata,\matb\)为\(n\)阶对称矩阵,证明\(\mata\matb\)是对称矩阵的充要条件是\(\mata\matb=\matb\mata\)。
\end{enumerate}
\end{problem}
\begin{proof}
    \begin{enumerate}
        \item \begin{equation*}
                  \begin{array}{lll}
                      \mata\text{为对称矩阵}\Rightarrow \mata^\top=\mata \\
                      \mypar{\matb^\top\mata\matb}^\top=\matb^\top\mypar{\matb^\top\mata}^\top=
                      \matb^\top\mata^\top\mypar{\matb^\top}^\top=\matb^\top\mata\matb
                  \end{array}
              \end{equation*}
        \item \begin{itemize}
                  \item 充分条件:\begin{equation*}
                            \mata\matb=\matb\mata\Rightarrow\mypar{\mata\matb}^\top=\matb^\top\mata^\top=\matb\mata=\mata\matb
                        \end{equation*}
                  \item 必要条件:\begin{equation*}
                            \mata\matb=\mypar{\mata\matb}^\top\Rightarrow\mata\matb=\mypar{\mata\matb}^\top=\matb^\top\mata^\top=\matb\mata
                        \end{equation*}
              \end{itemize}
    \end{enumerate}
\end{proof}

% 1.5
\begin{problem}\label{problem-1.5}
证明矩阵\(\mata=\mato\)的充分必要条件是方阵\(\mata^\top\mata=\mato\)。
\end{problem}
\begin{proof}
    必要性显然成立。下面证明充分性。

    设矩阵\(\mata=\mat{a_{ij}}_{m\times n}\)。由\(\mata^\top\mata=\mato\)和定义1.2.5,有

    \begin{equation*}
        \mat{\mata^\top\mata}_{ii}=\sum_{k=1}^m\mat{\mata^\top}_{ik}\mat{\mata}_{ki}=\sum_{k=1}^ma_{ki}^2=0
    \end{equation*}

    对\(i=1,2,\dots,n\)均成立。因此有\(\mata=\mato\)。
\end{proof}

% 1.6
\begin{problem}\label{problem-1.6}
设

\begin{equation*}
    \mata=\begin{bmatrix}a&0&0\\0&b&0\\0&0&c\end{bmatrix},
    \matb=\begin{bmatrix}0&1&0\\0&0&1\\0&0&0\end{bmatrix},
    \matc=\begin{bmatrix}\cos \theta&\sin \theta\\-\sin \theta&\cos \theta\end{bmatrix}
\end{equation*}

求\(\mata^5,\matb^3,\matc^n\)。
\end{problem}
\begin{proof}
    \begin{equation*}
        \mata^5=\begin{bmatrix}a^5&0&0\\0&b^5&0\\0&0&c^5\end{bmatrix},\matb^3=\begin{bmatrix}0&0&0\\0&0&0\\0&0&0\end{bmatrix},\matc^n=\begin{bmatrix}\cos n\theta&\sin n\theta\\-\sin n\theta&\cos n\theta\end{bmatrix}
    \end{equation*}

    数学归纳法格式:

    猜想\(\matc^n=\begin{bmatrix}\cos n\theta&\sin n\theta\\-\sin n\theta&\cos n\theta\end{bmatrix}\)对\(n\in N^+\)成立。

    当\(n=1\)时,\(\matc^n=\begin{bmatrix}\cos \theta&\sin \theta\\-\sin \theta&\cos \theta\end{bmatrix}\),结论成立。

    设当\(n=k\)时结论成立,则当\(n=k+1\)时,
    \begin{align*}
        \matc^{k+1} & =\matc^k\matc=\begin{bmatrix}\cos k\theta&\sin k\theta\\-\sin k\theta&\cos k\theta\end{bmatrix}\begin{bmatrix}\cos \theta&\sin \theta\\-\sin \theta&\cos \theta\end{bmatrix} \\
                    & =\begin{bmatrix}\cos (k+1)\theta&\sin (k+1)\theta\\-\sin (k+1)\theta&\cos (k+1)\theta\end{bmatrix}
    \end{align*}

    由归纳公理知\(\matc^n=\begin{bmatrix}\cos n\theta&\sin n\theta\\-\sin n\theta&\cos n\theta\end{bmatrix}\)对\(n\in N^+\)成立。
\end{proof}

% 1.7
\begin{problem}\label{problem-1.7}
已知\(\begin{bmatrix}a&1&1\\3&0&1\\0&2&-1\end{bmatrix}\begin{bmatrix}3\\a\\-3\end{bmatrix}=\begin{bmatrix}b\\6\\b\end{bmatrix}\),求\(a\)和\(b\)。
\end{problem}
\begin{proof}
    计算矩阵乘法,由等式可得

    \begin{equation*}
        \begin{cases}
            3a+a-3=b       \\
            9+0\cdot a-3=6 \\
            2a+3=b
        \end{cases}
    \end{equation*}

    解线性方程组得\(a=3,b=9\)。
\end{proof}

% 1.8
\begin{problem}\label{problem-1.8}
若\(\mata\)是\(n\)阶方阵且\(\mata^n=\mato\),试证:

\begin{equation*}
    \mypar{\mati_n-\mata}\mypar{\mati_n+\sum_{i=1}^{n-1}\mata^i}
\end{equation*}
\end{problem}
\begin{proof}
    因为\(\mata^n=\mato\),我们得到

    \begin{align*}
          & \mypar{\mati_n-\mata}\mypar{\mati_n+\sum_{i=1}^{n-1}\mata^i} \\
        = & \mati_n+\sum_{i=1}^{n-1}\mata^i-\sum_{i=1}^n\mata^i          \\
        = & \mati_n-\mata^n=\mati_n
    \end{align*}
\end{proof}

% 1.9
\begin{problem}\label{problem-1.9}
计算下列矩阵的\(n\)次方幂:

\begin{enumerate}
    \item 设\(\mata=\vecal^\top\vecbeta\),其中\(\vecal=\mat{1,2,3}\),\(\vecbeta=\mat{1,\frac{1}{2},\frac{1}{3}}\),求\(\mata^n\);
    \item 设\(\matb=\begin{bmatrix}1&2&3\\2&4&6\\3&6&9\end{bmatrix}\),求\(\matb^n\);
    \item 设\(\matc=\begin{bmatrix}1&-1&-1&-1\\-1&1&-1&-1\\-1&-1&1&-1\\-1&-1&-1&1\end{bmatrix}\),求\(\matc^n\);
    \item 设\(\matd=\begin{bmatrix}1&1&0\\0&1&1\\0&0&1\end{bmatrix}\),求\(\matd^n\)。
\end{enumerate}
\end{problem}
\begin{proof}
    \begin{enumerate}
        \item 当\(n=0\)时,\(\mata^n=\begin{bmatrix}1&0&0\\0&1&0\\0&0&1\end{bmatrix}\);

              当\(n\ge1\)时,
              \begin{align*}
                  \mata^n & =\underbrace{\mypar{\vecal^\top\vecbeta}\mypar{\vecal^\top\vecbeta}\cdots\mypar{\vecal^\top\vecbeta}}_n                        \\
                          & =\vecal^\top\underbrace{\mypar{\vecbeta\vecal^\top}\mypar{\vecbeta\vecal^\top}\cdots\mypar{\vecbeta\vecal^\top}}_{n-1}\vecbeta \\
                          & =\mypar{\vecbeta\vecal^\top}^{n-1}\mypar{\vecal^\top\vecbeta}                                                                  \\
                          & =3^{n-1}\begin{bmatrix}
                                        1 & \frac{1}{2} & \frac{1}{3} \\
                                        2 & 1           & \frac{2}{3} \\
                                        3 & \frac{3}{2} & 1
                                    \end{bmatrix};
              \end{align*}
        \item 注意到\(\matb=\vecbeta\vecbeta^\top\),其中\(\vecbeta=\myvec{1,2,3}\)。

              类似地,当\(n=0\)时,\(\matb^n=\begin{bmatrix}1&0&0\\0&1&0\\0&0&1\end{bmatrix}\);

              当\(n=1\)时,
              \begin{align*}
                  \matb^n & =\underbrace{\mypar{\vecbeta\vecbeta^\top}\mypar{\vecbeta\vecbeta^\top}\cdots\mypar{\vecbeta\vecbeta^\top}}_n                          \\
                          & =\vecbeta\underbrace{\mypar{\vecbeta^\top\vecbeta}\mypar{\vecbeta^\top\vecbeta}\cdots\mypar{\vecbeta^\top\vecbeta}}_{n-1}\vecbeta^\top \\
                          & =\mypar{\vecbeta^\top\vecbeta}^{n-1}\mypar{\vecbeta\vecbeta^\top}                                                                      \\
                          & ={14}^{n-1}
                  \begin{bmatrix}
                      1 & 2 & 3 \\
                      2 & 4 & 6 \\
                      3 & 6 & 9
                  \end{bmatrix};
              \end{align*}

        \item
              {
              容易发现
              \begin{equation*}
                  \matc^2=
                  \begin{bmatrix}
                      4 & 0 & 0 & 0 \\
                      0 & 4 & 0 & 0 \\
                      0 & 0 & 4 & 0 \\
                      0 & 0 & 0 & 4
                  \end{bmatrix}=4\mati_4
              \end{equation*}
              因此根据奇偶性讨论,有
              \begin{equation*}
                  \matc^n=
                  \begin{cases}
                      2^{n-1}\matc & n=2k-1 \\
                      2^n\mati_4   & n=2k
                  \end{cases}
                  (k\in N^+)
              \end{equation*}
              }

        \item
              {
              记\(\matd'=\begin{bmatrix}0&1&0\\0&0&1\\0&0&0\end{bmatrix}\),容易发现
              \begin{equation*}
                  \matd'^2=\begin{bmatrix}0&0&1\\0&0&0\\0&0&0\end{bmatrix},\matd'^n=\mato(n\ge3)
              \end{equation*}
              因此有
              \begin{align*}
                  \matd^n & =\mypar{\mati+\matd'}^n                 \\
                          & =\mati+\sum_{i=1}^n\binom{n}{i}\matd'^i \\
                          & =\mati+n\matd'+\binom{n}{2}\matd'^2     \\
                          & =\begin{bmatrix}
                                 1 & n & n(n-1)/2 \\
                                 0 & 1 & n        \\
                                 0 & 0 & 1
                             \end{bmatrix}
              \end{align*}
              }
    \end{enumerate}
\end{proof}

% 1.10
\begin{problem}\label{problem-1.10}
试证:
\begin{enumerate}
    \item 与所有\(n\)阶对角阵乘法可交换的矩阵也必是\(n\)阶对角阵;
    \item 与所有\(n\)阶矩阵乘法可交换的矩阵是纯量阵。
\end{enumerate}
\end{problem}
\begin{proof}
    \begin{enumerate}
        \item
              {
              设矩阵\(\mata=\mat{a_{ij}}_{n\times n}\)为对角阵,矩阵\(\matb=\mat{b_{ij}}_{n\times n}\)。则有
              \begin{equation*}
                  \begin{array}{lll}
                      \mat{\mata\matb}_{ij}=\sum_{k=1}^na_{ik}b_{kj}=a_{ii}b_{ij} \\
                      \mat{\matb\mata}_{ij}=\sum_{k=1}^nb_{ik}a_{kj}=a_{jj}b_{ij}
                  \end{array}
              \end{equation*}
              由题设可知\(\mat{\mata\matb}_{ij}=\mat{\matb\mata}_{ij}\)对任意\(i,j\in\setof{1,2,\dots,n}\)成立。

              当\(i=j\)时,等式成立;当\(i\neq j\)时,由\(\mata\)的任意性知\(b_{ij}=0\),即\(\matb\)为对角阵。
              }
        \item
              {
              由(1)知,满足要求的矩阵为对角阵。

              设矩阵\(\mata=\mat{a_{ij}}_{n\times n}\)为对角阵,矩阵\(\matb=\mat{b_{ij}}_{n\times n}\)。

              同(1)理,可得\(a_{ii}b_{ij}=a_{jj}b_{ij}\)对任意\(i,j\in\setof{1,2,\dots,n}\)成立。由\(\matb\)的任意性可知\(a_{ii}=a_{jj}\)对任意\(i,j\in\setof{1,2,\dots,n}\)成立,即\(\mata\)为纯量阵。
              }
    \end{enumerate}
\end{proof}

% 1.11
\begin{problem}\label{problem-1.11}
证明:两个对角元为\(1\)的上三角阵乘积仍是对角元为\(1\)的上三角阵。
\end{problem}
\begin{proof}
    设矩阵\(\mata=\mat{a_{ij}}_{n\times n}\),矩阵\(\matb=\mat{b_{ij}}_{n\times n}\)。则有\(\mat{\mata\matb}_{ij}=\sum_{k=1}^na_{ik}b_{kj}=\sum_{k=1}^{i-1}a_{ik}b_{kj}+a_{ii}b_{ij}+\sum_{t=i+1}^na_{it}b_{tj}\)。

    当\(i>j\)时,因为\(\mata,\matb\)均为上三角矩阵,因此有\(a_{ik}=b_{tj}=0(k<i,t>=i)\),代入上式可得\(\mat{\mata\matb}_{ij}=0\);

    当\(i=j\)时,因为\(\mata,\matb\)均为对角元为\(1\)的上三角矩阵,因此有\(a_{ik}=b_{tj}=0(k<i,t>i)\),代入上式可得\(\mat{\mata\matb}_{ij}=a_{ii}b_{ij}=1\)。

    综上,\(\mata\matb\)为对角元为\(1\)的上三角矩阵。
\end{proof}

% 1.12
\begin{problem}\label{problem-1.12}
设\(n\)元向量\(\vecx=\begin{bmatrix}x_1\\x_2\\\vdots\\x_n\end{bmatrix}\),\(\vecy=\begin{bmatrix}y_1\\y_2\\\vdots\\y_n\end{bmatrix}\)。若\(\mata=\vecy\vecx^\top\),求\(\mata^k\mypar{k \in N^+}\)。
\end{problem}
\begin{proof}
    \begin{equation*}
        \begin{array}{lll}
            \mata=\vecy\vecx^\top=
            \begin{bmatrix}
                x_1y_1 & x_2y_1 & \cdots & x_ny_1 \\
                x_1y_2 & x_2y_2 & \cdots & x_ny_2 \\
                \vdots & \vdots & \ddots & \vdots \\
                x_1y_n & x_2y_n & \cdots & x_ny_n \\
            \end{bmatrix}  \\
            \vecx^\top\vecy=\sum_{i=1}^nx_iy_i \\
        \end{array}
    \end{equation*}

    因此有
    \begin{align*}
        \mata^k & =\underbrace{\mypar{\vecy\vecx^\top}\mypar{\vecy\vecx^\top}\cdots\mypar{\vecy\vecx^\top}}_k                    \\
                & =\vecy\underbrace{\mypar{\vecx^\top\vecy}\mypar{\vecx^\top\vecy}\cdots\mypar{\vecx^\top\vecy}}_{k-1}\vecx^\top \\
                & =\mypar{\vecx^\top\vecy}^{k-1}\vecy\vecx^\top=\mypar{\sum_{i=1}^nx_iy_i}^{k-1}
        \begin{bmatrix}
            x_1y_1 & x_2y_1 & \cdots & x_ny_1 \\
            x_1y_2 & x_2y_2 & \cdots & x_ny_2 \\
            \vdots & \vdots & \ddots & \vdots \\
            x_1y_n & x_2y_n & \cdots & x_ny_n \\
        \end{bmatrix}
    \end{align*}
\end{proof}

% 1.13
\begin{problem}\label{problem-1.13}
设\(n\mypar{n\ge2}\)元向量\(\vecx=\begin{bmatrix}\frac12\\0\\\vdots\\0\\\frac12\end{bmatrix}\),\(\mata=\mati_n-\vecx\vecx^\top\),\(\matb=\mati_n+2\vecx\vecx^\top\),求\(\mata\matb\)。
\end{problem}
\begin{proof}
    \begin{align*}
        \vecx^\top\vecx & =
        \begin{bmatrix}
            \frac{1}{2} & 0 & \cdots & 0 & \frac{1}{2}
        \end{bmatrix}
        \begin{bmatrix}
            \frac{1}{2} & 0 & \cdots & 0 & \frac{1}{2}
        \end{bmatrix}^\top=\frac{1}{2}                                         \\
        \mata\matb      & =\mypar{\mati_n-\vecx\vecx^\top}\mypar{\mati_n+2\vecx\vecx^\top} \\
                        & =\mati_n+\vecx\vecx^\top-2\vecx\mypar{\vecx^\top\vecx}\vecx^\top \\
                        & =\mati_n+\vecx\vecx^\top-\vecx\vecx^\top=\mati_n
    \end{align*}
\end{proof}

% 1.14
\begin{problem}\label{problem-1.14}
设\(\mata\)是\(m\times n\)矩阵。证明:若对于任何\(n\)元列向量\(\vecx\)成立\(\mata\vecx=\veczero\),则\(\mata=\mato_{m\times n}\)。
\end{problem}
\begin{proof}
    设\(\mata=\begin{bmatrix}\veca_1&\veca_2&\cdots&\veca_n\end{bmatrix}\),\(\vece_i\)表示第\(i\)个分量为\(1\),其余分量为\(0\)的\(n\)阶列向量。由题设可知

    \begin{equation*}
        \mata\vece_i=\veca_i=\veczero
    \end{equation*}

    对\(i=1,2,\dots,n\)均成立。因此有\(\veca_1=\veca_2=\cdots=\veca_n=\veczero\),即\(\mata=\mato\)。
\end{proof}

% 1.15
\begin{problem}\label{problem-1.15}
设\(n\)阶方阵\(\mata,\matb\)满足\(\mata^2=\mata\),\(\matb^2=\matb\),且\(\mypar{\mata+\matb}^2=\mata+\matb\),证明:\(\mata\matb=\mato\)。
\end{problem}
\begin{proof}
    根据\(\mata^2=\mata,\matb^2=\matb\),可将\(\mypar{\mata+\matb}^2\)展开:

    \begin{align*}
        \mypar{\mata+\matb}^2 & =\mata^2+\matb^2+\mata\matb+\matb\mata \\
                              & =\mata+\matb+\mata\matb+\matb\mata
    \end{align*}

    又因为\(\mypar{\mata+\matb}^2=\mata+\matb\),可得

    \begin{equation}\label{eq-1.15}
        \mata\matb+\matb\mata=\mato
    \end{equation}

    将\eqref{eq-1.15}式左乘\(\mata\),得到\(\mata\matb+\mata\matb\mata=\mato\);将\eqref{eq-1.15}式式左右各乘\(\mata\),得到\(2\mata\matb\mata=\mato\)。将上述两式联立解得\(\mata\matb=\mato\)。
\end{proof}

% 1.16
\begin{problem}\label{problem-1.16}
设\(\mata=\begin{bmatrix}1&0&1\\0&2&0\\1&0&1\end{bmatrix}\),求\(\mata^n-2\mata^{n-1}\mypar{n\ge2}\)。
\end{problem}
\begin{proof}
    因为\(\mata^n-2\mata^{n-1}=\mata^{n-1}(\mata-2\mati)\),容易发现

    \begin{equation*}
        \mata(\mata-2\mati)=
        \begin{bmatrix}
            1 & 0 & 1 \\0&2&0\\1&0&1
        \end{bmatrix}
        \begin{bmatrix}
            -1 & 0 & 1 \\0&0&0\\1&0&-1
        \end{bmatrix}=\mato
    \end{equation*}

    因此当\(n\ge2\)时,\(\mata^n-2\mata^{n-1}=\mata^{n-2}\mata(\mata-2\mati)=\mato\)。
\end{proof}

% 1.17
\begin{problem}\label{problem-1.17}
设\(n\)阶方阵\(\mata,\matb\)满足\(\mata^2=-\mata\),\(\matb^2=-\matb\),且\(\mypar{\mata+\matb}^2=-\mata-\matb\),证明:\(\mata\matb=\mato\)。
\end{problem}
\begin{proof}
    根据\(\mata^2=-\mata,\matb^2=-\matb\),可将\(\mypar{\mata+\matb}^2\)展开:

    \begin{align*}
        \mypar{\mata+\matb}^2 & =\mata^2+\matb^2+\mata\matb+\matb\mata \\
                              & =-\mata-\matb+\mata\matb+\matb\mata
    \end{align*}

    又因为\(\mypar{\mata+\matb}^2=-\mata-\matb\),可得

    \begin{equation}\label{eq-1.17}
        \mata\matb+\matb\mata=\mato
    \end{equation}

    将\eqref{eq-1.17}式左乘\(\mata\),得到\(\mata\matb+\mata\matb\mata=\mato\);将\eqref{eq-1.17}式左右各乘\(\mata\),得到\(2\mata\matb\mata=\mato\)。将上述两式联立解得\(\mata\matb=\mato\)。
\end{proof}

% 1.18
\begin{problem}\label{problem-1.18}
设\(\mata=\begin{bmatrix}1&1&1\\0&2&0\\1&0&1\end{bmatrix}\),\(n\)为正整数,求\(\mata^n-2\mata^{n-1}\)。
\end{problem}
\begin{proof}
    因为\(\mata^n-2\mata^{n-1}=\mata^{n-1}\mypar{\mata-2\mati}\),容易发现

    \begin{equation*}
        \mata\mypar{\mata-2\mati}=
        \begin{bmatrix}
            1 & 0 & 1 \\0&2&0\\1&0&1
        \end{bmatrix}
        \begin{bmatrix}
            -1 & 0 & 1 \\0&0&0\\1&0&-1
        \end{bmatrix}=\mato
    \end{equation*}

    因此当\(n=1\)时,\(\mata^n-2\mata^{n-1}=\mata-2\mati=\begin{bmatrix}-1 & 0 & 1 \\0&0&0\\1&0&-1\end{bmatrix}\);

    当\(n\ge2\)时,\(\mata^n-2\mata^{n-1}=\mata^{n-2}\mata\mypar{\mata-2\mati}=\mato\)。
\end{proof}

% 1.19
\begin{problem}\label{problem-1.19}
设\(\mata=\mati-\vecal\vecal^\top\),其中\(\vecal\)为非零\(n\times1\)矩阵,试证:\(\mata^2=\mata\)的充要条件是\(\vecal^\top\vecal=1\)。
\end{problem}
\begin{proof}
    \begin{itemize}
        \item 首先证明充分性:
              因为\(\vecal^\top\vecal=1\),因此有
              \begin{align*}
                  \mata^2 & =\mypar{\mati-\vecal\vecal^\top}^2                                   \\
                          & =\mati-2\vecal\vecal^\top+\vecal\mypar{\vecal^\top\vecal}\vecal^\top \\
                          & =\mati-\vecal\vecal^\top=\mata
              \end{align*}

              充分性得证。

        \item 随后证明必要性:
              因为\(\mata^2=\mata\),因此有
              \begin{align*}
                  \mata^2-\mata & =\mat{\mati-2\vecal\vecal^\top+\vecal\mypar{\vecal^\top\vecal}\vecal^\top}-\mypar{\mati-\vecal\vecal^\top} \\
                                & =\mypar{\vecal^\top\vecal-1}\vecal\vecal^\top=\mato
              \end{align*}

              因为\(\vecal\neq\veczero\),因此\(\vecal^\top\vecal-1=0\),即\(\vecal^\top\vecal=1\)。
    \end{itemize}
\end{proof}

% 1.20
\begin{problem}\label{problem-1.20}
设\(n\)维向量\(\vecal=\myvec{a,0,\cdots,0,a}\),\(a<0\),\(\mati\)为\(n\)阶单位矩阵,矩阵
\begin{equation*}
    \mata=\mati-\vecal\vecal^\top,\matb=\mati+\frac{1}{a}\vecal\vecal^\top
\end{equation*}
其中\(\mata\matb=\mati\),求\(\vecal\)。
\end{problem}
\begin{proof}
    因为\(\vecal^\top\vecal=2a^2\),所以有
    \begin{align*}
        \mata\matb & =\mypar{\mati-\vecal\vecal^\top}\mypar{\mati+\frac{1}{a}\vecal\vecal^\top} \\
                   & =\mati+\mypar{\frac{1}{a}-1-\frac{1}{a}\vecal^\top\vecal}\vecal\vecal^\top \\
                   & =\mati+\mypar{\frac{1}{a}-1-2a}\vecal\vecal^\top
    \end{align*}
    又因为\(\mata\matb=\mati\)且\(a<0\),所以有\(\frac{1}{a}-1-2a=0\)。解得\(a=-1\)。
\end{proof}

% 1.21
\begin{problem}\label{problem-1.21}
设矩阵\(\mata=\begin{bmatrix}1&1&1\\0&2&0\\1&0&1\end{bmatrix}\),矩阵\(\matx\)满足\(\mata\matx+\mati=\mata^2+\matx\),其中\(\mati\)为\(3\)阶单位矩阵,试求出矩阵\(\matx\)。
\end{problem}
\begin{proof}
    将上述等式变换后可得
    \begin{equation*}
        \mypar{\mata-\mati}\matx=\mypar{\mata^2-\mati}=\mypar{\mata-\mati}\mypar{\mata+\mati}
    \end{equation*}
    因为\(\mata-\mati=\begin{bmatrix}0&1&1\\0&1&0\\1&0&0\end{bmatrix}\)可逆,所以\(\matx=\mata+\mati=\begin{bmatrix}2&0&1\\0&3&0\\1&0&2\end{bmatrix}\)。
\end{proof}

% 1.22
\begin{problem}\label{problem-1.22}
计算下列矩阵的\(k\)次幂,其中\(k\)为正整数:

\begin{enumerate}
    \item \(\mata=\begin{bmatrix}1&a&0\\0&1&a\\0&0&1\end{bmatrix}\);
    \item \(\mata=\begin{bmatrix}1&2&4\\2&4&8\\3&6&12\end{bmatrix}\)。
\end{enumerate}
\end{problem}
\begin{proof}
    \begin{enumerate}
        \item 记\(\mata'=\begin{bmatrix}0&a&0\\0&0&a\\0&0&0\end{bmatrix}\),容易发现
              \begin{equation*}
                  \mata'^2=\begin{bmatrix}0&0&a^2\\0&0&0\\0&0&0\end{bmatrix},\mata'^n=\mato(n\ge3)
              \end{equation*}
              因此有
              \begin{align*}
                  \mata^k & =\mypar{\mati+\mata'}^k                 \\
                          & =\mati+\sum_{i=1}^k\binom{k}{i}\mata'^i \\
                          & =\mati+k\mata'+\binom{k}{2}\mata'^2     \\
                          & =\begin{bmatrix}
                                 1 & ka & \frac{k(k-1)}{2}a^2 \\
                                 0 & 1  & ka                  \\
                                 0 & 0  & 1
                             \end{bmatrix}
              \end{align*}
        \item 记\(\vecal=\begin{bmatrix}1\\2\\3\end{bmatrix}\),\(\vecbeta=\begin{bmatrix}1\\2\\4\end{bmatrix}\)。容易发现\(\mata=\vecal\vecbeta^\top\),因此有
              \begin{align*}
                  \mata^k & =\mypar{\vecal\vecbeta^\top}^k=\vecal\mypar{\vecbeta^\top\vecal}^{k-1}\vecbeta^\top \\
                          & =\mypar{17}^{k-1}\vecal\vecbeta^\top=\mypar{17}^{k-1}
                  \begin{bmatrix}
                      1 & 2 & 4 \\2&4&8\\3&6&12
                  \end{bmatrix}
              \end{align*}
    \end{enumerate}
\end{proof}

% 1.23
\begin{problem}\label{problem-1.23}
设\(\mata,\matb\)是两个\(n\)阶矩阵,若\(\trace{\mata\matb\matc}=\trace{\matc\matb\mata}\)对任意\(n\)阶矩阵\(\matc\)成立,求证\(\mata\matb=\matb\mata\)。
\end{problem}
\begin{proof}
    令\(\matc=\vece_i\vece_j^\top\),其中\(\vece_i\)表示第\(i\)个分量为\(1\),其余分量为\(0\)的\(n\)阶列向量。因此有
    \begin{align*}
        \mat{\mata\matb\matc}_{tt} & =\sum_{k=1}^n\mat{\mata\matb}_{tk}\mat{\matc}_{kt}=
        \begin{cases}
            \mat{\mata\matb}_{ji} & t=j     \\
            0                     & t\neq j
        \end{cases}                                                  \\
        \mat{\matc\matb\mata}_{tt} & =\sum_{k=1}^n\mat{\matc}_{tk}\mat{\matb\mata}_{kt}=
        \begin{cases}
            \mat{\matb\mata}_{ji} & t=i     \\
            0                     & t\neq i
        \end{cases}
    \end{align*}

    因为\(\trace{\mata\matb\matc}=\trace{\matc\matb\mata}\),因此有\(\mat{\mata\matb}_{ji}=\mat{\matb\mata}_{ji}\)对\(i,j\in\setof{1,2,\dots,n}\)成立,即\(\mata\matb=\matb\mata\)。
\end{proof}

% 1.24
\begin{problem}\label{problem-1.24}
设

\begin{equation*}
    \mata=\begin{bmatrix}1&0&0&0\\0&1&0&0\\-1&2&1&0\\1&1&0&1\end{bmatrix},
    \matb=\begin{bmatrix}1&0&1&0\\-1&2&0&1\\1&0&4&1\\-1&-1&2&0\end{bmatrix}
\end{equation*}

利用分块矩阵求\(\mata\matb\)。
\end{problem}
\begin{proof}
    定义子矩阵如下:
    \begin{equation*}
        \mata_3=
        \begin{bmatrix}
            -1 & 2 \\1&1
        \end{bmatrix},
        \matb_1=
        \begin{bmatrix}
            1 & 0 \\-1&2
        \end{bmatrix},
        \matb_3=
        \begin{bmatrix}
            1 & 0 \\-1&-1
        \end{bmatrix},
        \matb_4=
        \begin{bmatrix}
            4 & 1 \\2&0
        \end{bmatrix}
    \end{equation*}

    则有
    \begin{align*}
        \mata\matb & =
        \begin{bmatrix}
            \mati & \mato \\\mata_3&\mati
        \end{bmatrix}
        \begin{bmatrix}
            \matb_1 & \mati \\\matb_3&\matb_4
        \end{bmatrix}                        \\&=
        \begin{bmatrix}
            \matb_1 & \mati \\\mata_3\matb_1+\matb_3&\mata_3+\matb_4
        \end{bmatrix} \\&=
        \begin{bmatrix}
            1  & 0 & 1 & 0 \\
            -1 & 2 & 0 & 1 \\
            -2 & 4 & 3 & 3 \\
            -1 & 1 & 3 & 1
        \end{bmatrix}
    \end{align*}
\end{proof}

% 1.25
\begin{problem}\label{problem-1.25}
求矩阵\(\mata,\matb\)的秩,其中:
\begin{equation*}
    \mata=
    \begin{bmatrix}
        1 & 2 & 3  \\
        2 & 3 & -5 \\
        4 & 7 & 1
    \end{bmatrix},
    \matb=
    \begin{bmatrix}
        3 & 2  & 0  & 5  & 0  \\
        3 & -2 & 3  & 6  & -1 \\
        2 & 0  & 1  & 5  & -3 \\
        1 & 6  & -4 & -1 & 4
    \end{bmatrix}
\end{equation*}
\end{problem}
\begin{proof}
    \begin{equation*}
        \mata\xrightarrow[R_3-4R_1]{R_2-2R_1}
        \begin{bmatrix}
            1 & 2  & 3   \\
            0 & -1 & -11 \\
            0 & -1 & -11
        \end{bmatrix}\xrightarrow{R_3-R_2}
        \begin{bmatrix}
            1 & 2  & 3   \\
            0 & -1 & -11 \\
            0 & 0  & 0
        \end{bmatrix}
    \end{equation*}
    \begin{align*}
        \matb & \xrightarrow[R_2-3R_1\dots]{R_{14}}
        \begin{bmatrix}
            1 & 6   & -4 & -1 & 4   \\
            0 & -20 & 15 & 9  & -13 \\
            0 & -12 & 9  & 7  & -11 \\
            0 & -16 & 12 & 8  & -12
        \end{bmatrix}
              & \xrightarrow[R_3-12R_2\dots]{\frac{1}{20}R_2}
        \begin{bmatrix}
            1 & 6  & -4          & -1           & 4              \\
            0 & -1 & \frac{3}{4} & \frac{9}{20} & -\frac{13}{20} \\
            0 & 0  & 0           & \frac{8}{5}  & -\frac{16}{5}  \\
            0 & 0  & 0           & \frac{4}{5}  & -\frac{8}{5}
        \end{bmatrix}  \\
              & \xrightarrow[R_4-2R_3]{R_{34}}
        \begin{bmatrix}
            1 & 6  & -4          & -1           & 4              \\
            0 & -1 & \frac{3}{4} & \frac{9}{20} & -\frac{13}{20} \\
            0 & 0  & 0           & \frac{4}{5}  & -\frac{8}{5}   \\
            0 & 0  & 0           & 0            & 0
        \end{bmatrix}
    \end{align*}
    所以\(\rank{\mata}=2\),\(\rank{\matb}=3\)。
\end{proof}

% 1.26
\begin{problem}\label{problem-1.26}
设
\begin{equation*}
    \mata=
    \begin{bmatrix}
        1 & 2 & -1      & 1   \\
        3 & 2 & \lambda & -1  \\
        5 & 6 & 3       & \mu
    \end{bmatrix}
\end{equation*}
已知\(\rank{\mata}=2\),求\(\lambda\)和\(\mu\)的值。
\end{problem}
\begin{proof}
    \begin{equation*}
        \mata\xrightarrow[R_3-5R_1]{R_2-3R_1}
        \begin{bmatrix}
            1 & 2  & -1        & 1     \\
            0 & -4 & \lambda+3 & -4    \\
            0 & -4 & 8         & \mu-5
        \end{bmatrix}
    \end{equation*}
    因为\(\rank{\mata}=2\),所以有
    \begin{equation*}
        \begin{cases}
            \lambda+3=8 \\
            \mu-5=-4
        \end{cases}
    \end{equation*}
    解得\(\lambda=5\),\(\mu=1\)。
\end{proof}

% 1.27
\begin{problem}\label{problem-1.27}
求证:矩阵添加一行(列),其秩不变或者增加\(1\)。
\end{problem}
\begin{proof}
    若矩阵\(\mata=\mato\),结论显然成立;
    若\(\mata\neq\mato\),则其经有限次初等变换可化为标准形
    \begin{equation*}
        \begin{bmatrix}\mati_r&\mato\\\mato&\mato\end{bmatrix}
    \end{equation*}
    当\(\mata\)添加一行\(\veca\)时,设添加\(\veca\)后的矩阵为\(\mata'\),则其经有限次初等变换可化为
    \begin{equation*}
        \begin{bmatrix}\mati_r&\mato\\\veca_1&\veca_2\\\mato&\mato\end{bmatrix}\longrightarrow
        \begin{bmatrix}\mati_r&\mato\\\veczero&\veca_2\\\mato&\mato\end{bmatrix}
    \end{equation*}
    当\(\veca_2=\veczero\)时,\(\rank{\mata'}=\rank{\mata}\);当\(\veca_2\neq\veczero\)时,容易发现\(\rank{\mata'}=\rank{\mata}+1\)。

    当\(\mata\)添加一列时,同理可得相同结论。
\end{proof}

% 1.28
\begin{problem}\label{problem-1.28}
设\(\mata\)是\(n\)阶方阵,满足\(\mata^2=\mati_n\),求证:
\begin{equation*}
    \rank{\mata+\mati_n}+\rank{\mata-\mati_n}=n.
\end{equation*}
\end{problem}
\begin{proof}

\end{proof}

% 1.29
\begin{problem}\label{problem-1.29}
设\(\mata\)为\(n\)阶矩阵,则\(\rank{\mata}=1\)的充分必要条件是存在矩阵\(\matb_{n\times1}\)和\(\matc_{1\times n}\)(\(\matb\neq\mato,\matc\neq\mato\)),使得\(\mata=\matb\matc\)。
\end{problem}
\begin{proof}

\end{proof}

% 1.30
\begin{problem}\label{problem-1.30}
设\(\mata\)为\(m\times n\)矩阵,且\(\rank{\mata}=r\),从\(\mata\)中任取\(s\)行构成一个\(s\times n\)矩阵\(\matb\),证明:\(\rank{\matb}\ge r+s-m\)。
\end{problem}
\begin{proof}

\end{proof}

% 1.31
\begin{problem}\label{problem-1.31}
设\(n\mypar{n\ge3}\)阶矩阵
\begin{equation*}
    \mata=
    \begin{bmatrix}
        1      & a      & a      & \cdots & a      \\
        a      & 1      & a      & \cdots & a      \\
        a      & a      & 1      & \cdots & a      \\
        \vdots & \vdots & \vdots &        & \vdots \\
        a      & a      & a      & \cdots & 1
    \end{bmatrix}
\end{equation*}
若矩阵\(\mata\)的秩为\(n-1\),求\(a\)。
\end{problem}
\begin{proof}

\end{proof}

% 1.32
\begin{problem}\label{problem-1.32}
用初等行变换将下列矩阵化为阶梯形矩阵:
\begin{enumerate}
    \item \(\begin{bmatrix}1&2&3&4\\0&-1&0&-2\\1&1&3&2\\2&2&6&4\end{bmatrix}\);
    \item \(\begin{bmatrix}1&0&-1&5&12\\6&7&8&0&-9\\26&21&26&-10&-51\\15&14&13&-15&-54\end{bmatrix}\)
\end{enumerate}
\end{problem}
\begin{proof}
    \begin{enumerate}
        \item \begin{equation*}
                  \begin{bmatrix}
                      1 & 2 & 3 & 4 \\0&-1&0&-2\\1&1&3&2\\2&2&6&4
                  \end{bmatrix}\xrightarrow[R_4-2R_1]{R_3-R_1}
                  \begin{bmatrix}
                      1 & 2  & 3 & 4  \\
                      0 & -1 & 0 & -2 \\
                      0 & -1 & 0 & -2 \\
                      0 & -2 & 0 & -4
                  \end{bmatrix}\xrightarrow[R_4-2R_2]{R_3-R_2}
                  \begin{bmatrix}
                      1 & 2  & 3 & 4  \\
                      0 & -1 & 0 & -2 \\
                      0 & 0  & 0 & 0  \\
                      0 & 0  & 0 & 0
                  \end{bmatrix}
              \end{equation*}
        \item \begin{align*}
                  \begin{bmatrix}1&0&-1&5&12\\6&7&8&0&-9\\26&21&26&-10&-51\\15&14&13&-15&-54\end{bmatrix}
                   & \xrightarrow[\substack{R_3-26R_1 \\R_4-15R_1}]{R_2-6R_1}
                  \begin{bmatrix}
                      1 & 0  & -1 & 5    & 12   \\
                      0 & 7  & 14 & -30  & -81  \\
                      0 & 21 & 52 & -140 & -363 \\
                      0 & 14 & 28 & -90  & -234
                  \end{bmatrix}           \\
                   & \xrightarrow[R_4-2R_2]{R_3-3R_2}
                  \begin{bmatrix}
                      1 & 0 & -1 & 5   & 12   \\
                      0 & 7 & 14 & -30 & -81  \\
                      0 & 0 & 10 & -50 & -120 \\
                      0 & 0 & 0  & -30 & -72
                  \end{bmatrix}
              \end{align*}
    \end{enumerate}
\end{proof}

% 1.33
\begin{problem}\label{problem-1.33}
判断下列矩阵是否有相同的最简阶梯型:
\begin{enumerate}
    \item \(\begin{bmatrix}1&2&3\\2&4&6\\0&1&2\end{bmatrix},\begin{bmatrix}1&-1&5\\0&3&3\\0&2&2\end{bmatrix}\);
    \item \(\begin{bmatrix}-1&0&4\\3&0&-1\\0&1&-1\end{bmatrix},\begin{bmatrix}1&-1&5\\-1&4&-2\\0&3&3\end{bmatrix}\)
\end{enumerate}
\end{problem}
\begin{proof}
    \begin{enumerate}
        \item \begin{equation*}
                  \begin{bmatrix}1&2&3\\2&4&6\\0&1&2\end{bmatrix}\xrightarrow[\substack{R_3-2R_1\\R_1-2R_2}]{R_{23}}
                  \begin{bmatrix}
                      1 & 0 & -1 \\
                      0 & 1 & 2  \\
                      0 & 0 & 0
                  \end{bmatrix}
              \end{equation*}
              \begin{equation*}
                  \begin{bmatrix}1&-1&5\\0&3&3\\0&2&2\end{bmatrix}\xrightarrow[\substack{R_3-2R_2\\R_1+R_2}]{\frac{1}{3}R_2}
                  \begin{bmatrix}
                      1 & 0 & 6 \\
                      0 & 1 & 1 \\
                      0 & 0 & 0
                  \end{bmatrix}
              \end{equation*}
              两个矩阵的最简阶梯型不相同。
        \item \begin{equation*}
                  \begin{bmatrix}-1&0&4\\3&0&-1\\0&1&-1\end{bmatrix}\xrightarrow[\substack{R_3+3R_1\\-R_1}]{R_{23}}
                  \begin{bmatrix}
                      1 & 0 & -4 \\
                      0 & 1 & -1 \\
                      0 & 0 & 11
                  \end{bmatrix}\xrightarrow[\substack{R_1+4R_3\\R_2+R_3}]{\frac{1}{11}R_3}
                  \begin{bmatrix}
                      1 & 0 & 0 \\
                      0 & 1 & 0 \\
                      0 & 0 & 1
                  \end{bmatrix}
              \end{equation*}
              \begin{equation*}
                  \begin{bmatrix}1&-1&5\\-1&4&-2\\0&3&3\end{bmatrix}\xrightarrow[\substack{\frac{1}{3}R_2\\R_1+R_2}]{\substack{R_2+R_1\\R_3-R_2}}
                  \begin{bmatrix}
                      1 & 0 & 6 \\
                      0 & 1 & 1 \\
                      0 & 0 & 0
                  \end{bmatrix}
              \end{equation*}
              两个矩阵的最简阶梯型不相同。
    \end{enumerate}
\end{proof}

% 附加1.1
\begin{extraprob}\label{extra-1.1}
    证明\(\row{\mata\matb}{i}=\row{\mata}{i}\matb=\sum_{k=1}^la_{ik}\row{\matb}{k}\)。
\end{extraprob}
\begin{proof}
    设\(\mata=\normmat{bmatrix}{a}{m}{l}\),\(\matb=\normmat{bmatrix}{b}{l}{n}\)。则有
    \begin{equation*}
        \mata\matb=
        \begin{bmatrix}
            \sum_{k=1}^la_{1k}b_{k1} & \sum_{k=1}^la_{1k}b_{k2} & \cdots & \sum_{k=1}^la_{1k}b_{kn} \\
            \sum_{k=1}^la_{2k}b_{k1} & \sum_{k=1}^la_{2k}b_{k2} & \cdots & \sum_{k=1}^la_{2k}b_{kn} \\
            \vdots                   & \vdots                   & \ddots & \vdots                   \\
            \sum_{k=1}^la_{mk}b_{k1} & \sum_{k=1}^la_{mk}b_{k2} & \cdots & \sum_{k=1}^la_{mk}b_{kn}
        \end{bmatrix}
    \end{equation*}
    \begin{equation*}
        \row{\mata\matb}{i}=
        \begin{bmatrix}
            \sum_{k=1}^la_{ik}b_{k1} & \sum_{k=1}^la_{ik}b_{k2} & \cdots\sum_{k=1}^la_{ik}b_{kn}
        \end{bmatrix}
    \end{equation*}
    \begin{align*}
        \row{\mata}{i}\matb & =
        \begin{bmatrix}
            a_{i1} & a_{i2} & \cdots & a_{il}
        \end{bmatrix}\normmat{bmatrix}{b}{l}{n} \\
                            & =
        \begin{bmatrix}
            \sum_{k=1}^la_{ik}b_{k1} & \sum_{k=1}^la_{ik}b_{k2} & \cdots\sum_{k=1}^la_{ik}b_{kn}
        \end{bmatrix}
    \end{align*}
    \begin{align*}
        \sum_{k=1}^la_{ik}\row{\matb}{k} & =\sum_{k=1}^la_{ik}
        \begin{bmatrix}
            b_{k1} & b_{k2} & \cdots & b_{kn}
        \end{bmatrix}                      \\
                                         & =
        \begin{bmatrix}
            \sum_{k=1}^la_{ik}b_{k1} & \sum_{k=1}^la_{ik}b_{k2} & \cdots\sum_{k=1}^la_{ik}b_{kn}
        \end{bmatrix}
    \end{align*}

    因此有\(\row{\mata\matb}{i}=\row{\mata}{i}\matb=\sum_{k=1}^la_{ik}\row{\matb}{k}\)。
\end{proof}

% 附加1.2
\begin{extraprob}\label{extra-1.2}
    证明\(\mata\matb=\sum_{k=1}^l\col{\mata}{k}\row{\matb}{k}\)。
\end{extraprob}
\begin{proof}
    和附加习题\ref{extra-1.1}解法类似,定义矩阵\(\mata=\mat{a_ij}_{m\times l}\),\(\matb=\mat{b_{ij}}_{l\times n}\)。则有
    \begin{equation*}
        \mata\matb=
        \begin{bmatrix}
            \sum_{k=1}^la_{1k}b_{k1} & \sum_{k=1}^la_{1k}b_{k2} & \cdots & \sum_{k=1}^la_{1k}b_{kn} \\
            \sum_{k=1}^la_{2k}b_{k1} & \sum_{k=1}^la_{2k}b_{k2} & \cdots & \sum_{k=1}^la_{2k}b_{kn} \\
            \vdots                   & \vdots                   & \ddots & \vdots                   \\
            \sum_{k=1}^la_{mk}b_{k1} & \sum_{k=1}^la_{mk}b_{k2} & \cdots & \sum_{k=1}^la_{mk}b_{kn}
        \end{bmatrix}
    \end{equation*}
    \begin{align*}
          & \sum_{k=1}^l\col{\mata}{k}\row{\matb}{k}        \\ =&\sum_{k=1}^l
        \mypar{
            \begin{bmatrix}
                a_{1k} \\a_{2k}\\\vdots\\a_{mk}
            \end{bmatrix}
            \begin{bmatrix}
                b_{k1} & b_{k2} & \cdots & b_{kn}
            \end{bmatrix}
        }                                                   \\
        = & \sum_{k=1}^l
        \begin{bmatrix}
            a_{1k}b_{k1} & a_{1k}b_{k2} & \cdots & a_{1k}b_{kn} \\
            a_{2k}b_{k1} & a_{2k}b_{k2} & \cdots & a_{2k}b_{kn} \\
            \vdots       & \vdots       & \ddots & \vdots       \\
            a_{mk}b_{k1} & a_{mk}b_{k2} & \cdots & a_{mk}b_{kn}
        \end{bmatrix} \\
        = &
        \begin{bmatrix}
            \sum_{k=1}^la_{1k}b_{k1} & \sum_{k=1}^la_{1k}b_{k2} & \cdots & \sum_{k=1}^la_{1k}b_{kn} \\
            \sum_{k=1}^la_{2k}b_{k1} & \sum_{k=1}^la_{2k}b_{k2} & \cdots & \sum_{k=1}^la_{2k}b_{kn} \\
            \vdots                   & \vdots                   & \ddots & \vdots                   \\
            \sum_{k=1}^la_{mk}b_{k1} & \sum_{k=1}^la_{mk}b_{k2} & \cdots & \sum_{k=1}^la_{mk}b_{kn}
        \end{bmatrix}
    \end{align*}
    因此\(\mata\matb=\sum_{k=1}^l\col{\mata}{k}\row{\matb}{k}\)。
\end{proof}

% 附加1.3
\begin{extraprob}\label{extra-1.3}
    设\(\mata,\matb\)是数域\(P\)上的\(n\)级矩阵,则\(\abs{\mata\matb}=\abs{\mata}\abs{\matb}\)。

    提示:使用性质\(\col{\mata\matb}{j}=\mata\col{\matb}{j}=\sum_{k=1}^lb_{kj}\col{\mata}{k}\)。
\end{extraprob}
\begin{proof}
    设矩阵\(\mata=\mat{a_{ij}}_{n\times n}\),矩阵\(\matb=\mat{b_{ij}}_{n\times n}\)。构造\(2n\)阶方阵\(\matf\):
    \begin{equation*}
        \matf=
        \begin{bmatrix}
            \mata & \mato \\\matd&\matb
        \end{bmatrix},\matd=\diag{\mat{-1,-1,\dots,-1}}
    \end{equation*}

    对\(\matf\)作初等列变换,设变换后的矩阵\(\matfstar\)形如:
    \begin{equation*}
        \matfstar=\begin{bmatrix}\mata&\matc\\\matd&\mato\end{bmatrix}
    \end{equation*}

    则容易发现\(\col{\matc}{j}=\sum_{k=1}^nb_{kj}\col{\mata}{k}=\col{\mata\matb}{j}\),即\(\matc=\mata\matb\)。

    又由Laplace定理可知,\(\abs{\matf}=\abs{\mata}\abs{\matb}\),\(\abs{\matfstar}=-\abs{\matc}\abs{\matd}=\abs{\mata\matb}\),配合行列式性质即可得\(\abs{\mata\matb}=\abs{\mata}\abs{\matb}\)。

\end{proof}

% 附加1.4
\begin{extraprob}\label{extra-1.4}
    证明:
    \begin{enumerate}
        \item \(\matc\mypar{\mata+\matb}=\matc\mata+\matc\matb\)
        \item \(\mypar{\mata+\matb}\matc=\mata\matc+\matb\matc\)
        \item \(k\mypar{\mata\matb}=\mypar{k\mata}\matb=\mata\mypar{k\matb}\)
    \end{enumerate}
    其中\(k\)为数,矩阵\(\mata,\matb,\matc\)的阶数使上述各式有意义。
\end{extraprob}
\begin{proof}
    \begin{enumerate}
        \item
              {
              设\(\matc=\mat{c_{ij}}_{m\times l}\),\(\mata=\mat{a_{ij}}_{l\times n}\),\(\matb=\mat{b_{ij}}_{l\times n}\)。则有
              \begin{align*}
                  \entry{\matc\mypar{\mata+\matb}}{i}{j} & =\row{\matc}{i}\col{\mata+\matb}{j}                        \\
                                                         & =\sum_{k=1}^lc_{ik}\mypar{a_{kj}+b_{kj}}                   \\
                                                         & =\sum_{k=1}^lc_{ik}a_{kj}+\sum_{k=1}^lc_{ik}b_{kj}         \\
                                                         & =\row{\matc}{i}\col{\mata}{j}+\row{\matc}{i}\col{\matb}{j} \\
                                                         & =\entry{\matc\mata}{i}{j}+\entry{\matc\matb}{i}{j}
              \end{align*}
              因此\(\matc\mypar{\mata+\matb}=\matc\mata+\matc\matb\)成立。
              }

        \item
              {
              设\(\mata=\mat{a_{ij}}_{m\times l}\),\(\matb=\mat{b_{ij}}_{m\times l}\),\(\matc=\mat{c_{ij}}_{l\times n}\)。则有
              \begin{align*}
                  \entry{\mypar{\mata+\matb}\matc}{i}{j} & =\row{\mata+\matb}{i}\col{\matc}{j}                        \\
                                                         & =\sum_{k=1}^l\mypar{a_{ik}+b_{ik}}c_{kj}                   \\
                                                         & =\sum_{k=1}^la_{ik}c_{kj}+\sum_{k=1}^lb_{ik}c_{kj}         \\
                                                         & =\row{\mata}{i}\col{\matc}{j}+\row{\matb}{i}\col{\matc}{j} \\
                                                         & =\entry{\mata\matc}{i}{j}+\entry{\matb\matc}{i}{j}
              \end{align*}
              因此\(\mypar{\mata+\matb}\matc=\mata\matc+\matb\matc\)成立。
              }

        \item
              {
              设\(\mata=\mat{a_{ij}}_{m\times l}\),\(\matb=\mat{b_{ij}}_{l\times n}\)。则有
              \begin{align*}
                  \entry{\mypar{k\mata}\matb}{i}{j} & =\row{k\mata}{i}\col{\matb}{j}     \\
                                                    & =\sum_{t=1}^l\mypar{ka_{it}}b_{tj} \\
                                                    & =k\sum_{t=1}^la_{it}b_{tj}         \\
                                                    & =k\row{\mata}{i}\col{\matb}{j}     \\
                                                    & =k\entry{\mata\matb}{i}{j}
              \end{align*}
              因此\(\mypar{k\mata}\matb=k\mypar{\mata\matb}\)成立。同理可得\(\mata\mypar{k\matb}=k\mypar{\mata\matb}\)成立。
              }
    \end{enumerate}
\end{proof}

% 补充1.1
\begin{suplprob}\label{supl-1.1}
    设\(\mata,\matb\)为\(n\)阶方阵,且\(\mata\matb=\mata+\matb\)。证明:
    \begin{enumerate}
        \item \(\mata-\mati\)可逆;
        \item \(\mata\matb=\matb\mata\)。
    \end{enumerate}
\end{suplprob}
\begin{proof}
    \begin{enumerate}
        \item
              {
              由\(\mata\matb=\mata+\matb\)可得\(\mata\matb-\mata-\matb+\mati=\mati\),即\(\mypar{\mata-\mati}\mypar{\matb-\mati}=\mati\)。由可逆矩阵定义知\(\mata-\mati\)可逆。
              }
        \item
              {
              由可逆矩阵定义可知\(\mypar{\mata-\mati}\mypar{\matb-\mati}=\mypar{\matb-\mati}\mypar{\mata-\mati}\)。

              等式两边展开化简后即得证。
              }
    \end{enumerate}
\end{proof}

% 补充1.2
\begin{suplprob}\label{supl-1.2}
    设\(\mata,\matb\)为\(n\)阶方阵,\(\mata\)对称且可逆,且\(\mypar{\mata-\matb}^2=\mati\)。化简:
    \begin{equation*}
        \mypar{\mati+\inv{\mata}\matb^\top}^\top\inv{\mypar{\mati-\matb\inv{\mata}}}
    \end{equation*}
\end{suplprob}
\begin{proof}
    \begin{align*}
          & \mypar{\mati+\inv{\mata}\matb^\top}^\top\inv{\mypar{\mati-\matb\inv{\mata}}}          \\
        = & \mypar{\mati+\matb\inv{\mata}}\inv{\mypar{\mata\inv{\mata}-\matb\inv{\mata}}}         \\
        = & \mypar{\mata\inv{\mata}+\matb\inv{\mata}}\inv{\mypar{\mypar{\mata-\matb}\inv{\mata}}} \\
        = & \mypar{\mypar{\mata+\matb}\inv{\mata}}\mata\pinv{\mata-\matb}                         \\
        = & \mypar{\mata+\matb}\mypar{\mata-\matb}
    \end{align*}
\end{proof}