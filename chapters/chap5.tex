\section{特征值与二次型}

% 5.1(2)
\begin{problem}
求下列矩阵的特征值和特征向量:
\begin{enumerate}
    \item[(2)]{
                \begin{equation*}
                    \begin{bmatrix}
                        -1 & 2  & 2  \\
                        3  & -1 & 1  \\
                        2  & 2  & -1
                    \end{bmatrix}
                \end{equation*}
          }
\end{enumerate}
\end{problem}
\begin{proof}

\end{proof}

% 5.2
\begin{problem}
设矩阵\(\mata=\begin{bmatrix}a&0&b\\0&2&0\\b&0&-2\end{bmatrix}\)的一个特征值为\(\lambda_1=-3\),且三个特征值的积为\(-12\),则求\(a\),\(b\),以及\(\mata\)的其他特征值。
\end{problem}
\begin{proof}

\end{proof}

% 5.3
\begin{problem}
已知\(\vecal=\myvec{1,1,-1}\)是矩阵
\begin{equation*}
    \mata=
    \begin{bmatrix}
        2  & -1 & 2  \\
        5  & a  & 3  \\
        -1 & b  & -2
    \end{bmatrix}
\end{equation*}
的一个特征向量,试确定\(a\),\(b\)的值以及特征向量\(\vecal\)所对应的特征值。
\end{problem}
\begin{proof}

\end{proof}

% 5.4
\begin{problem}
设\(3\)阶矩阵\(\mata\)的三个特征值为\(\lambda_1=1\),\(\lambda_2=2\),\(\lambda_3=3\),与之对应的特征向量分别为\(\vecal_1=\myvec{2,1,-1}\),\(\vecal_2=\myvec{2,-1,2}\),\(\vecal_=\myvec{3,0,1}\),求矩阵\(\mata\)。
\end{problem}
\begin{proof}

\end{proof}

% 5.5
\begin{problem}
设\(\mata^2=\mata\),证明\(\mata\)的特征值只能是\(0\)或者\(1\)。
\end{problem}
\begin{proof}

\end{proof}

% 5.6
\begin{problem}
设\(n\)阶可逆矩阵\(\mata\)的特征值为\(n\)个非零数\(\enums{\lambda}{n}\),试证\(\mata\)的伴随矩阵的特征值为\(\enums{\det{\mata}\lambda}{n}\)。
\end{problem}
\begin{proof}

\end{proof}

% 5.7
\begin{problem}
已知\(3\)阶矩阵的特征值为\(1,2,-1\),设矩阵\(\matb=\mata^3-5\mata^2\),试计算行列式\(\det{\matb}\),\(\det{\mata-5\mati}\)。
\end{problem}
\begin{proof}

\end{proof}

% 5.8
\begin{problem}
设\(\mata=\mati-2\vecx\vecx^\top\),其中\(\vecx=\myvec{\enums{x}{n}}\),\(\mati\)为\(n\)阶单位矩阵。若\(\vecx^\top\vecx=1\),求证:\(\vecx\)为\(\mata\)的属于特征值\(-1\)的特征向量。
\end{problem}
\begin{proof}

\end{proof}

% 5.9
\begin{problem}
已知\(n\)阶矩阵\(\mata\)的特征值为\(\lambda_0\),
\begin{enumerate}
    \item 求\(k\mata\)的特征值(\(k\)为任意实数);
    \item 若\(\mata\)可逆,求\(\inv{\mata}\)的特征值;
    \item 求\(\mati+\mata\)的特征值。
\end{enumerate}
\end{problem}
\begin{proof}

\end{proof}

% 5.10
\begin{problem}
设\(\mata^2-3\mata+\mati=\mato\),证明\(\mata\)的特征值只能是\(1\)或\(2\)。
\end{problem}
\begin{proof}

\end{proof}

\setcounter{problem}{11}
% 5.12
\begin{problem}
设非零向量\(\vecal=\myvec{\enums{a}{n}}\),\(\vecbeta=\myvec{\enums{b}{n}}\),且\(\vecal^\top\vecbeta=0\),\(\mata=\vecal\vecbeta^\top\),求\(\mata\)的特征值和特征向量。
\end{problem}
\begin{proof}

\end{proof}

% 5.13
\begin{problem}
已知\(\lambda_1=2\)是\(3\)阶矩阵\(\mata\)的一个特征值,\(\vecal_1=\mat{1,2,0}\),\(\vecal_2=\mat{1,0,1}\)是\(\mata\)属于特征值\(2\)的两个特征向量,\(\vecbeta=\myvec{-1,2,-1}\),计算\(\mata\vecbeta\)。
\end{problem}
\begin{proof}

\end{proof}

% 5.14
\begin{problem}
设矩阵\(\mata=\begin{bmatrix}a&-1&c\\5&b&3\\1-c&0&-a\end{bmatrix}\),且\(\det{\mata}=-1\),伴随矩阵\(\mata^*\)有特征值\(\lambda_0\),属于\(\lambda_0\)的特征向量为\(\vecal=\myvec{-1,-1,1}\),求\(a,b,c\),以及\(\lambda_0\)。
\end{problem}
\begin{proof}

\end{proof}

% 5.15
\begin{problem}
若\(n\)阶方阵\(\mata\)的每行元素之和为常数\(c\),证明:\(c\)是\(\mata\)的一个特征值:\(\forall m\in\mathbf{N}\),\(\mata^m\)的每行元素之和为\(c^m\)。
\end{problem}
\begin{proof}

\end{proof}

\setcounter{problem}{17}
% 5.18
\begin{problem}
设\(\enums{\vecx}{n}\)是\(\mata\)的属于\(\lambda_0\)的特征向量,试证\(\enums{\vecx}{n}\)的任一非零线性组合也是\(\mata\)的属于\(\lambda_0\)的特征向量。
\end{problem}
\begin{proof}

\end{proof}

% 5.19
\begin{problem}
设\(\vecx_1,\vecx_2\)分别是属于\(\lambda_1,\lambda_2\)的特征向量,而且\(\lambda_1\neq\lambda_2\),试证\(\vecx_1+\vecx_2\)不可能是\(\mata\)的特征向量。
\end{problem}
\begin{proof}

\end{proof}

\setcounter{problem}{23}
% 5.24
\begin{problem}
设\(\mata=\vecu\vecv^\top\),其中\(\vecu,\vecv\)都是\(n\)元非零列向量,求证:\(\vecu\)是\(\mata\)的特征向量,并求出其对应的特征值。
\end{problem}
\begin{proof}

\end{proof}

% 5.25
\begin{problem}
设\(\mata=\begin{bmatrix}2&0&0\\0&0&1\\0&1&x\end{bmatrix}\),\(\matb=\begin{bmatrix}2&&\\&y&\\&&-1\end{bmatrix}\),已知\(\mata\)与\(\matb\)相似,求\(x,y\)。
\end{problem}
\begin{proof}

\end{proof}

% 5.26
\begin{problem}
设\(\mata=\begin{bmatrix}-3&2\\-2&2\end{bmatrix}\)。
\begin{enumerate}
    \item 求可逆矩阵\(\matp\)使得\(\inv{\matp}\mata\matp\)为对角矩阵;
    \item 计算\(\mata^{10}-\mata^6-\mati\)。
\end{enumerate}
\end{problem}
\begin{proof}

\end{proof}

% 5.27
\begin{problem}
已知方阵\(\mata=\begin{bmatrix}1&-1&1\\x&4&y\\-3&-3&5\end{bmatrix}\)与对角矩阵相似,且\(\lambda=2\)是\(\mata\)的二重特征值。
\begin{enumerate}
    \item 求\(x\)与\(y\)的值;
    \item 求可逆矩阵\(\matp\)使得\(\inv{\matp}\mata\matp\)为对角阵。
\end{enumerate}
\end{problem}
\begin{proof}

\end{proof}

% 5.28(2)
\begin{problem}
已知矩阵\(\mata\),求使\(\inv{\matq}\mata\matq\)为对角阵的正交矩阵\(\matq\)。
\begin{enumerate}
    \item[(2)]
        {
        \begin{equation*}
            \mata=
            \begin{bmatrix}
                3 & 2 & 4 \\
                2 & 0 & 2 \\
                4 & 2 & 3
            \end{bmatrix}
        \end{equation*}
        }
\end{enumerate}
\end{problem}
\begin{proof}

\end{proof}

\setcounter{problem}{29}
% 5.30
\begin{problem}
设矩阵\(\mata=\begin{bmatrix}0&0&1\\a&1&b\\1&0&0\end{bmatrix}\)有三个线性无关的特征向量,求\(a,b\)满足的条件,并求可逆矩阵\(\matp\),使得\(\inv{\matp}\mata\matp\)为对角阵。
\end{problem}
\begin{proof}

\end{proof}

\setcounter{problem}{31}
% 5.32
\begin{problem}
计算\({\begin{bmatrix}1&2&2\\2&1&2\\2&2&1\end{bmatrix}}^{100}\)。
\end{problem}
\begin{proof}

\end{proof}

\setcounter{problem}{33}
% 5.34(2)
\begin{problem}
求二次型\(f\)的矩阵:
\begin{enumerate}
    \item[(2)] \(f(x_1,x_2,x_3)=\mypar{x_1+x_2}^2+\mypar{x_2+x_3}^2+\mypar{x_3+x_1}^2\)
\end{enumerate}
\end{problem}
\begin{proof}

\end{proof}

% 5.35
\begin{problem}
用正交变换法将二次型
\begin{equation*}
    f\mypar{x_1,x_2,x_3}=x_1^2+4x_2^2+4x_3^2-4x_1x_2+4x_1x_3-8x_2x_3
\end{equation*}
化为标准形,写出所作的正交变换。
\end{problem}
\begin{proof}

\end{proof}

% 5.36
\begin{problem}
二次型
\begin{equation*}
    f\mypar{x_1,x_2,x_3}=\vecx^\top\mata\vecx=ax_1^2+2x_2^2-2x_3^2+2bx_1x_3\mypar{b>0}
\end{equation*}
其中二次型\(f\)的矩阵\(\mata\)的特征值之和为\(1\),特征值之积为\(12\),求
\begin{enumerate}
    \item \(a,b\)的值;
    \item 用正交变换将\(f\)化为标准形,写出所作的正交变换及对应的正交矩阵。
\end{enumerate}
\end{problem}
\begin{proof}

\end{proof}

\setcounter{problem}{37}
% 5.38
\begin{problem}
已知二次型\(f\mypar{x_1,x_2,x_3}=4x_2^2+4x_3^2+4x_1x_2+ax_1x_3-4x_2x_3\)的秩为\(2\),
\begin{enumerate}
    \item 求\(a\)的值;
    \item 求正交变换\(\vecx=\matp\vecy\)把\(f\mypar{x_1,x_2,x_3}\)化为标准形;
    \item 求\(f\mypar{x_1,x_2,x_3}=0\)的解。
\end{enumerate}
\end{problem}
\begin{proof}

\end{proof}

% 5.39
\begin{problem}
判断二次型的正定性:
\begin{enumerate}
    \item \(f\mypar{x_1,x_2,x_3}=5x_1^2+x_2^2+5x_3^2+4x_1x_2-8x_1x_3-4x_2x_3\);
    \item \(f\mypar{x_1,x_2,x_3,x_4}=x_1^2+x_2^2+14x_3^2+7x_4^2+6x_1x_3+8x_1x_4-4x_2x_3+2x_2x_4+4x_3x_4\)。
\end{enumerate}
\end{problem}
\begin{proof}

\end{proof}

% 5.40
\begin{problem}
判断二次型\(f=\sum_{i=0}^nx_i^2+\sum_{1\le i<j\le n}x_ix_j\)的正定性。
\end{problem}
\begin{proof}

\end{proof}

% 5.41
\begin{problem}
设\(\mata\)为\(3\)阶实对称矩阵,且满足\(\mata^2+2\mata=\mato\),\(\rank{\mata}=2\),
\begin{enumerate}
    \item 求\(\mata\)的特征值;
    \item 当\(k\)为何值时,矩阵\(\mata+k\mati\)为正定矩阵?
\end{enumerate}
\end{problem}
\begin{proof}

\end{proof}

% 5.42
\begin{problem}
设\(\mata\)是\(n\)阶实对称矩阵,证明\(\rank{\mata}=n\)的充要条件是存在一个\(n\)阶实矩阵\(\matb\),使得\(\mata\matb+\matb^\top\mata\)正定。
\end{problem}
\begin{proof}

\end{proof}

% 5.43
\begin{problem}
若\(\mata\)是\(n\)阶实矩阵,证明:\(\trace{\mata\mata^\top}\ge0\),且等号成立的充要条件是\(\mata=\mato\)。
\end{problem}
\begin{proof}

\end{proof}

% 5.44
\begin{problem}
设\(\mata\)是\(n\)阶正定矩阵,证明:对任意\(n\)阶矩阵\(\matb\),\(\rank{\matb^\top\mata\matb}=\rank{\matb}\)。
\end{problem}
\begin{proof}

\end{proof}