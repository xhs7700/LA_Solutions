\section{特征值与二次型}

% 5.1(2)
\begin{problem}
求下列矩阵的特征值和特征向量:
\begin{enumerate}
    \item[(2)]{
                \begin{equation*}
                    \begin{bmatrix}
                        -1 & 2  & 2  \\
                        3  & -1 & 1  \\
                        2  & 2  & -1
                    \end{bmatrix}
                \end{equation*}
          }
\end{enumerate}
\end{problem}
\begin{proof}
    记上述矩阵为\(\mata\),则矩阵\(\mata\)的特征多项式为
    \begin{equation*}
        \abs{\lambda\mati-\mata} =
        \begin{vmatrix}
            \lambda+1 & -2        & -2        \\
            -3        & \lambda+1 & -1        \\
            -2        & -2        & \lambda+1
        \end{vmatrix}=\mypar{\lambda+3}^2\mypar{\lambda-3}
    \end{equation*}
    因此\(\mata\)的特征值为\(\lambda_1=\lambda_2=-3,\lambda_3=3\)。

    将\(\lambda_1=\lambda_2=-3\)代入\(\mypar{\lambda\mati-\mata}\vecx=\veczero\),得到其基础解系为\(\myvec{1,-2,1}\)。因此属于特征值\(-3\)的特征向量为\(k_1\myvec{1,-2,1},k_1\neq0\)。

    将\(\lambda_3=3\)代入\(\mypar{\lambda\mati-\mata}\vecx=\veczero\),得到其基础解系为\(\myvec{1,1,1}\)。因此属于特征值\(3\)的特征向量为\(k_2\myvec{1,1,1},k_2\neq0\)。
\end{proof}

% 5.2
\begin{problem}
设矩阵\(\mata=\begin{bmatrix}a&0&b\\0&2&0\\b&0&-2\end{bmatrix}\)的一个特征值为\(\lambda_1=-3\),且三个特征值的积为\(-12\),则求\(a\),\(b\),以及\(\mata\)的其他特征值。
\end{problem}
\begin{proof}
    由特征向量的定义和定理5.1.2,有
    \begin{equation*}
        \begin{cases}
            \lambda_1+\lambda_2+\lambda_3=\trace{\mata}=a        \\
            \lambda_1\lambda_2\lambda_3=\det{\mata}=-4a-2b^2=-12 \\
            \det{\lambda_1\mati-\mata}=-5a+5b^2-15=0
        \end{cases}
    \end{equation*}
    解得\(a=1\),\(b=\pm2\),\(\lambda_2=\lambda_3=2\)。
\end{proof}

% 5.3
\begin{problem}
已知\(\vecal=\myvec{1,1,-1}\)是矩阵
\begin{equation*}
    \mata=
    \begin{bmatrix}
        2  & -1 & 2  \\
        5  & a  & 3  \\
        -1 & b  & -2
    \end{bmatrix}
\end{equation*}
的一个特征向量,试确定\(a\),\(b\)的值以及特征向量\(\vecal\)所对应的特征值。
\end{problem}
\begin{proof}
    设\(\vecal\)对应的特征值为\(\lambda\),则由特征值和特征向量的定义可知\(\mata\vecal=\lambda\vecal\)。将\(\mata,\vecal\)代入即可解得\(\lambda=-1\),\(a=-3\),\(b=0\)。
\end{proof}

% 5.4
\begin{problem}
设\(3\)阶矩阵\(\mata\)的三个特征值为\(\lambda_1=1\),\(\lambda_2=2\),\(\lambda_3=3\),与之对应的特征向量分别为\(\vecal_1=\myvec{2,1,-1}\),\(\vecal_2=\myvec{2,-1,2}\),\(\vecal_=\myvec{3,0,1}\),求矩阵\(\mata\)。
\end{problem}
\begin{proof}
    由特征值和特征向量的定义可知
    \begin{equation*}
        \mata\begin{bmatrix}\vecal_1&\vecal_2&\vecal_3\end{bmatrix}=\begin{bmatrix}\lambda_1\vecal_1&\lambda_2\vecal_2&\lambda_3\vecal_3\end{bmatrix}
    \end{equation*}
    由定理5.1.5知\(\mat{\vecal_1,\vecal_2,\vecal_3}\)可逆,因此有
    \begin{equation*}
        \mata=\begin{bmatrix}\lambda_1\vecal_1&\lambda_2\vecal_2&\lambda_3\vecal_3\end{bmatrix}\inv{\begin{bmatrix}\vecal_1&\vecal_2&\vecal_3\end{bmatrix}}=
        \begin{bmatrix}
            -3 & 26 & 18 \\
            -1 & 6  & 3  \\
            0  & 2  & 3
        \end{bmatrix}
    \end{equation*}
\end{proof}

% 5.5
\begin{problem}
设\(\mata^2=\mata\),证明\(\mata\)的特征值只能是\(0\)或者\(1\)。
\end{problem}
\begin{proof}
    设\(\mata\)的特征向量为\(\vecal\neq\veczero\)。由题设及特征向量的定义,有
    \begin{equation*}
        \lambda\vecal=\mata\vecal=\mata^2\vecal=\lambda^2\vecal
    \end{equation*}
    因此可得\(\mypar{\lambda^2-\lambda}\vecal=\veczero\)。由\(\vecal\neq\veczero\)解得\(\lambda=0\)或\(\lambda=1\)。结论成立。
\end{proof}

% 5.6
\begin{problem}
记\(\mata\)的伴随矩阵为\(\mata^*\)。设\(n\)阶可逆矩阵\(\mata\)的特征值为\(n\)个非零数\(\enums{\lambda}{n}\),试证\(\mata\)的伴随矩阵的特征值为\(\det{\mata}\inv{\lambda_1}\),\(\det{\mata}\inv{\lambda_2}\),\(\dots\),\(\det{\mata}\inv{\lambda_n}\)。
\end{problem}
\begin{proof}
    对\(\forall i\in\setof{1,2,\dots,n}\),设\(\vecal_i\)为\(\lambda_i\)对应的一个特征向量。则由定理3.3.1和特征向量的定义,有
    \begin{equation*}
        \det{\mata}\vecal=\mata^*\mata\vecal=\lambda_i\mata^*\vecal
    \end{equation*}
    所以\(\det{\mata}\inv{\lambda_i}\mypar{i=1,2,\dots,n}\)是\(\mata^*\)的特征值。
\end{proof}

% 5.7
\begin{problem}
已知\(3\)阶方阵\(\mata\)的特征值为\(1,2,-1\),设矩阵\(\matb=\mata^3-5\mata^2\)。试计算行列式\(\det{\matb}\),\(\det{\mata-5\mati}\)。
\end{problem}
\begin{proof}
    由推论5.2.1可知存在可逆矩阵\(\matp\),满足\(\mata=\matp\diag{\lambda_1,\lambda_2,\lambda_3}\inv{\matp}\)。因此有
    \begin{align*}
        \mata-5\mati & =\matp\diag{\lambda_1,\lambda_2,\lambda_3}\inv{\matp}-5\matp\mati\inv{\matp} \\
                     & =\matp\diag{\lambda_1-5,\lambda_2-5,\lambda_3-5}\inv{\matp}
    \end{align*}
    所以\(\det{\mata-5\mati}=\prod_{i=1}^3\mypar{\lambda_i-5}=-72\)。

    同理可得\(\det{\matb}=\prod_{i=1}^3\mypar{\lambda_i^3-5\lambda_i^2}=-288\)。
\end{proof}

% 5.8
\begin{problem}
设\(\mata=\mati-2\vecx\vecx^\top\),其中\(\vecx=\myvec{\enums{x}{n}}\),\(\mati\)为\(n\)阶单位矩阵。若\(\vecx^\top\vecx=1\),求证:\(\vecx\)为\(\mata\)的属于特征值\(-1\)的特征向量。
\end{problem}
\begin{proof}
    由\(\vecx^\top\vecx=1\)可得
    \begin{align*}
        \mata\vecx & =\mypar{\mati-2\vecx\vecx^\top}\vecx \\
                   & =\mati\vecx-2\vecx\vecx^\top\vecx    \\
                   & =\vecx-2\vecx=-\vecx
    \end{align*}
    由特征值和特征向量的定义可知\(\vecx\)为\(\mata\)的属于特征值\(-1\)的特征向量。
\end{proof}

% 5.9
\begin{problem}
已知\(n\)阶矩阵\(\mata\)的特征值为\(\lambda_0\),
\begin{enumerate}
    \item 求\(k\mata\)的特征值(\(k\)为任意实数);
    \item 若\(\mata\)可逆,求\(\inv{\mata}\)的特征值;
    \item 求\(\mati+\mata\)的特征值。
\end{enumerate}
\end{problem}
\begin{proof}
    设\(\mata\)的特征值为\(\lambda_0\)对应的特征向量为\(\vecal_0\)。
    \begin{enumerate}
        \item 当\(k\neq0\)时,有\(\mata\vecal_0=\lambda_0\vecal_0\iff k\mata\vecal_0=k\lambda_0\vecal_0\)。此时\(k\mata\)的特征值为\(k\lambda_0\)。当\(k=0\)时结论依然成立。
        \item 因为\(\mata\)可逆,所以\(\lambda_0\neq0\)。因此有\(\mata\vecal_0=\lambda_0\vecal_0\iff\inv{\lambda_0}\vecal_0=\inv{\mata}\vecal_0\)。所以\(\inv{\mata}\)的特征值为\(\inv{\lambda_0}\)。
        \item 同理,可得\(\mata\vecal_0=\lambda_0\vecal_0\iff\mypar{\mata+\mati}\vecal_0=\mypar{\lambda_0+1}\vecal_0\)。所以\(\mati+\mata\)的特征值为\(\lambda_0+1\)。
    \end{enumerate}
\end{proof}

% 5.10
\begin{problem}
设\(\mata^2-3\mata+2\mati=\mato\),证明\(\mata\)的特征值只能是\(1\)或\(2\)。
\end{problem}
\begin{proof}
    设\(\mata\)的特征向量为\(\vecal\neq\veczero\)。由题设及特征向量的定义,有
    \begin{equation*}
        \mypar{\mata^2-3\mata+2\mati}\vecal=\mypar{\lambda^2-3\lambda+2}\vecal=\veczero
    \end{equation*}
    由\(\vecal\neq\veczero\)解得\(\lambda=1\)或\(\lambda=2\)。结论成立。
\end{proof}

\setcounter{problem}{11}
% 5.12
\begin{problem}
设非零向量\(\vecal=\myvec{\enums{a}{n}}\),\(\vecbeta=\myvec{\enums{b}{n}}\),且\(\vecal^\top\vecbeta=0\),\(\mata=\vecal\vecbeta^\top\),求\(\mata\)的特征值和特征向量。
\end{problem}
\begin{proof}
    设\(\mata\)的特征值为\(\lambda\),对应的特征向量为\(\vecx\neq\veczero\)。由\(\vecal^\top\vecbeta=0\)和特征值的定义可得
    \begin{equation*}
        \lambda^2\vecx=\mata^2\vecx=\vecal\mypar{\vecbeta^\top\vecal}\vecbeta^\top\vecx=\veczero
    \end{equation*}
    解得\(\lambda=0\),即\(\mata\)的特征值均为\(0\)。

    又因为\(\vecal,\vecbeta\)为非零向量,所以\(\rank{\mata}=\rank{\vecal\vecbeta^\top}=\rank{\vecbeta^\top}\)。因此方程\(\vecal\vecbeta^\top\vecx\)和方程\(\vecbeta^\top\vecx\)同解。所以\(\mata\)的特征向量为方程
    \begin{equation*}
        \vecbeta^\top\vecx=b_1x_1+b_2x_2+\dots+b_nx_n=0
    \end{equation*}
    的解。不妨设\(b_i\neq0\),则\(\mata\)线性无关的特征向量分别为
    \begin{gather*}
        \begin{bmatrix}b_i&0&\cdots&0&-b_1&0&\cdots&0&0\end{bmatrix}^\top\\
        \begin{bmatrix}0&b_i&\cdots&0&-b_2&0&\cdots&0&0\end{bmatrix}^\top\\
        \vdots\\
        \begin{bmatrix}0&0&\cdots&0&-b_n&0&\cdots&0&b_i\end{bmatrix}^\top
    \end{gather*}
\end{proof}

% 5.13
\begin{problem}
已知\(\lambda_1=2\)是\(3\)阶矩阵\(\mata\)的一个特征值,\(\vecal_1=\mat{1,2,0}\),\(\vecal_2=\mat{1,0,1}\)是\(\mata\)属于特征值\(2\)的两个特征向量,\(\vecbeta=\myvec{-1,2,-2}\),计算\(\mata\vecbeta\)。
\end{problem}
\begin{proof}
    容易发现\(\vecal_1,\vecal_2,\vecbeta\)线性相关:\(\vecbeta=\vecal_1-2\vecal_2\)。因此由特征值和特征向量的定义,有
    \begin{equation*}
        \mata\vecbeta=\mata\mypar{\vecal_1-2\vecal_2}=2\mypar{\vecal_1-2\vecal_2}=\begin{bmatrix}2&4&-4\end{bmatrix}^\top
    \end{equation*}
\end{proof}

% 5.14
\begin{problem}
设矩阵\(\mata=\begin{bmatrix}a&-1&c\\5&b&3\\1-c&0&-a\end{bmatrix}\),且\(\det{\mata}=-1\),伴随矩阵\(\mata^*\)有特征值\(\lambda_0\),属于\(\lambda_0\)的特征向量为\(\vecal=\myvec{-1,-1,1}\),求\(a,b,c\),以及\(\lambda_0\)。
\end{problem}
\begin{proof}
    由题意可得\(\mata^*\vecal=\lambda_0\vecal\),通过定理3.3.1可转化为
    \begin{equation*}
        \mata\vecal=\inv{\lambda_0}\det{\mata}\vecx=-\inv{\lambda_0}\vecx
    \end{equation*}
    将其与\(\det{\mata}=-1\)联立可得
    \begin{equation*}
        \begin{cases}
            -a^2b-\mypar{3+bc}\mypar{1-c}-5a=-1 \\
            -a+1+c=-\inv{\lambda_0}             \\
            -b-2=-\inv{\lambda_0}               \\
            c-1-a=-\inv{\lambda_0}
        \end{cases}
    \end{equation*}
    解得\(a=c=2,b=-3,\lambda_0=1\)。
\end{proof}

% 5.15
\begin{problem}
若\(n\)阶方阵\(\mata\)的每行元素之和为常数\(c\),证明:\(c\)是\(\mata\)的一个特征值:\(\forall m\in\mathbf{N}\),\(\mata^m\)的每行元素之和为\(c^m\)。
\end{problem}
\begin{proof}
    记\(\rea^n\)中各分量均为\(1\)的列向量为\(\vecone\)。则由题设可知\(\mata\vecone=c\vecone\)。

    所以\(c\)是\(\mata\)的一个特征值,\(\vecone\)是其对应的特征向量。因此有\(\mata^m\vecone=c^m\vecone\),结论成立。
\end{proof}

\setcounter{problem}{17}
% 5.18
\begin{problem}
设\(\enums{\vecx}{n}\)是\(\mata\)的属于\(\lambda_0\)的特征向量,试证\(\enums{\vecx}{n}\)的任一非零线性组合也是\(\mata\)的属于\(\lambda_0\)的特征向量。
\end{problem}
\begin{proof}
    设\(\rea^n\)中的任意非零向量为\(\vecy\),则\(\enums{\vecx}{n}\)的任一非零线性组合可表示为\(\mat{\enums{\vecx}{n}}\vecy\)。因此有
    \begin{align*}
          & \mata\begin{bmatrix}\vecx_1&\vecx_2&\cdots&\vecx_n\end{bmatrix}\vecy      \\
        = & \begin{bmatrix}\mata\vecx_1&\mata\vecx_2&\cdots&\mata\vecx_n\end{bmatrix} \\
        = & \lambda_0\begin{bmatrix}\vecx_1&\vecx_2&\cdots&\vecx_n\end{bmatrix}\vecy
    \end{align*}
    结论成立。
\end{proof}

% 5.19
\begin{problem}
设\(\vecx_1,\vecx_2\)分别是属于\(\lambda_1,\lambda_2\)的特征向量,而且\(\lambda_1\neq\lambda_2\),试证\(\vecx_1+\vecx_2\)不可能是\(\mata\)的特征向量。
\end{problem}
\begin{proof}
    反设结论不成立,即\(\vecx_1+\vecx_2\)是\(\mata\)的特征值为\(\lambda\)的特征向量。则有
    \begin{equation*}
        \lambda_1\vecx_1+\lambda_2\vecx_2=\mata\vecx_1+\mata\vecx_2=\mata\mypar{\vecx_1+\vecx_2}=\lambda\mypar{\vecx_1+\vecx_2}
    \end{equation*}
    化简得到\(\mypar{\lambda_1-\lambda}\vecx_1+\mypar{\lambda_2-\lambda}\vecx_2=\veczero\)。由\(\lambda_1\neq\lambda_2\)可知\(\vecx_1,\vecx_2\)线性相关,和定理5.1.5矛盾。

    所以\(\vecx_1+\vecx_2\)不可能是\(\mata\)的特征向量。
\end{proof}

\setcounter{problem}{23}
% 5.24
\begin{problem}
设\(\mata=\vecu\vecv^\top\),其中\(\vecu,\vecv\)都是\(n\)元非零列向量,求证:\(\vecu\)是\(\mata\)的特征向量,并求出其对应的特征值。
\end{problem}
\begin{proof}
    因为\(\mata\vecu=\vecu\vecv^\top\vecu=\mypar{\vecu^\top\vecv}\vecu\),由特征值和特征向量的定义可知\(\vecu\)是\(\mata\)的特征向量,对应的特征值是\(\vecu^\top\vecv\)。
\end{proof}

% 5.25
\begin{problem}
设\(\mata=\begin{bmatrix}2&0&0\\0&0&1\\0&1&x\end{bmatrix}\),\(\matb=\begin{bmatrix}2&&\\&y&\\&&-1\end{bmatrix}\),已知\(\mata\)与\(\matb\)相似,求\(x,y\)。
\end{problem}
\begin{proof}
    由定理5.1.2和定理5.1.3可知,相似矩阵的迹和行列式相等。因此有
    \begin{equation*}
        \begin{cases}
            -2=-2y \\
            2+x=1+y
        \end{cases}
    \end{equation*}
    解得\(x=0,y=1\)。
\end{proof}

% 5.26
\begin{problem}
设\(\mata=\begin{bmatrix}-3&2\\-2&2\end{bmatrix}\)。
\begin{enumerate}
    \item 求可逆矩阵\(\matp\)使得\(\inv{\matp}\mata\matp\)为对角矩阵;
    \item 计算\(\mata^{10}-\mata^6-\mati\)。
\end{enumerate}
\end{problem}
\begin{proof}
    \begin{enumerate}
        \item {
              \(\mata\)的特征多项式为
              \begin{equation*}
                  \abs{\lambda\mati-\mata}=\begin{vmatrix}\lambda+3&-2\\2&\lambda-2\end{vmatrix}=\mypar{\lambda-1}\mypar{\lambda+2}
              \end{equation*}
              因此\(\mata\)的特征值为\(\lambda_1=1,\lambda_2=-2\)。

              将\(\lambda_1=1\)代入特征方程组,解得其线性无关的特征向量为\(\myvec{1,2}\);

              将\(\lambda_2=-2\)代入特征方程组,解得其线性无关的特征向量为\(\myvec{2,1}\)。

              因此得到满足条件的\(\matp=\begin{bmatrix}1&2\\2&1\end{bmatrix}\)。
              }
        \item {
              由(1)可得\(\mata=\matp\matlam\inv{\matp}\),其中\(\matlam=\begin{bmatrix}1&0\\0&-2\end{bmatrix}\)。因此有
              \begin{align*}
                  \mata^{10}-\mata^6-\mati & =\matp\mypar{\matlam^{10}-\matlam^6-\mati}\inv{\matp}                                                                      \\
                                           & =\begin{bmatrix}1&2\\2&1\end{bmatrix}\begin{bmatrix}-1&0\\0&959\end{bmatrix}\begin{bmatrix}-1/3&2/3\\2/3&-1/3\end{bmatrix} \\
                                           & =\begin{bmatrix}1279&-640\\640&-321\end{bmatrix}
              \end{align*}
              }
    \end{enumerate}
\end{proof}

% 5.27
\begin{problem}
已知方阵\(\mata=\begin{bmatrix}1&-1&1\\x&4&y\\-3&-3&5\end{bmatrix}\)与对角矩阵相似,且\(\lambda=2\)是\(\mata\)的二重特征值。
\begin{enumerate}
    \item 求\(x\)与\(y\)的值;
    \item 求可逆矩阵\(\matp\)使得\(\inv{\matp}\mata\matp\)为对角阵。
\end{enumerate}
\end{problem}
\begin{proof}
    \begin{enumerate}
        \item {
              因为\(\mata\)可对角化,由定理5.2.2可知\(\mata\)的代数重数与几何重数相等。考虑
              \begin{equation*}
                  \lambda\mati-\mata=2\mati-\mata=
                  \begin{bmatrix}
                      1  & 1  & -1 \\
                      -x & -2 & -y \\
                      3  & 3  & -3
                  \end{bmatrix}\to
                  \begin{bmatrix}
                      1   & 1 & -1  \\
                      2-x & 0 & 2-y \\
                      0   & 0 & 0
                  \end{bmatrix}
              \end{equation*}
              因为\(\lambda=2\)是\(\mata\)的二重特征值,所以\(\rank{\lambda\mati-\mata}=1\),由此得到\(x=2,y=-2\)。
              }
        \item {
              由定理5.1.2可知\(\mata\)的另一个特征值为\(\trace{\mata}-2\lambda=6\)。
              将\(\lambda_1=\lambda_2=2\),\(\lambda_3=6\)代入特征方程组\(\mypar{\lambda\mati-\mata}\vecal=0\),得到\(\mata\)对应的线性无关的特征向量分别为\(\myvec{-1,1,0}\),\(\myvec{1,0,1}\)和\(\myvec{1,-2,3}\)。所以对应的可逆矩阵\(\matp\)为\(\begin{bmatrix}-1&1&1\\1&0&-2\\0&1&3\end{bmatrix}\)。
              }
    \end{enumerate}
\end{proof}

% 5.28(2)
\begin{problem}
已知矩阵\(\mata\),求使\(\inv{\matq}\mata\matq\)为对角阵的正交矩阵\(\matq\)。
\begin{enumerate}
    \item[(2)]
        {
        \begin{equation*}
            \mata=
            \begin{bmatrix}
                3 & 2 & 4 \\
                2 & 0 & 2 \\
                4 & 2 & 3
            \end{bmatrix}
        \end{equation*}
        }
\end{enumerate}
\end{problem}
\begin{proof}
    \(\mata\)的特征多项式为
    \begin{equation*}
        \abs{\lambda\mati-\mata}=\begin{vmatrix}\lambda-1&-1&-1\\-1&\lambda-1&-1\\-1&\lambda-1&-1\end{vmatrix}=\lambda^2\mypar{\lambda-3}
    \end{equation*}
    所以\(\mata\)的特征值为\(\lambda_1=\lambda_2=0\),\(\lambda_3=3\)。

    将上述特征值代入特征方程组\(\mypar{\lambda\mati-\mata}\vecal=0\),得到\(\lambda_1,\lambda_2\)对应的线性无关的特征向量为\(\vecal_1=\myvec{-1,1,0}\)和\(\vecal_2=\myvec{-1,0,1}\),\(\lambda_3\)对应的线性无关的特征向量为\(\vecal_3=\myvec{1,1,1}\)。

    对向量组\(\vecal_1,\vecal_2\)和向量组\(\vecal_3\)分别做Schmidt正交化,得到正交矩阵
    \begin{equation*}
        \matq=\begin{bmatrix}\vecq_1&\vecq_2&\vecq_3\end{bmatrix}=
        \begin{bmatrix}
            -\frac{\sqrt{2}}{2} & -\frac{\sqrt{6}}{6} & \frac{\sqrt{3}}{3} \\
            \frac{\sqrt{2}}{2}  & -\frac{\sqrt{6}}{6} & \frac{\sqrt{3}}{3} \\
            0                   & \frac{2\sqrt{6}}{6} & \frac{\sqrt{3}}{3}
        \end{bmatrix}
    \end{equation*}
\end{proof}

\setcounter{problem}{29}
% 5.30
\begin{problem}
设矩阵\(\mata=\begin{bmatrix}0&0&1\\a&1&b\\1&0&0\end{bmatrix}\)有三个线性无关的特征向量,求\(a,b\)满足的条件,并求可逆矩阵\(\matp\),使得\(\inv{\matp}\mata\matp\)为对角阵。
\end{problem}
\begin{proof}
    \(\mata\)的特征多项式为
    \begin{equation*}
        \abs{\lambda\mati-\mata}=\begin{vmatrix}\lambda&0&-1\\-a&\lambda-1&-b\\-1&0&\lambda\end{vmatrix}=\mypar{\lambda-1}^2\mypar{\lambda+1}
    \end{equation*}
    所以\(\mata\)的特征值为\(\lambda_1=\lambda_2=1\),\(\lambda_3=-1\)。

    由定理5.2.1和定理5.2.2可知\(\mata\)的代数重数与几何重数相等。考虑
    \begin{equation*}
        \lambda_1\mati-\mata=\mati-\mata=
        \begin{bmatrix}
            1  & 0 & -1 \\
            -a & 0 & -b \\
            -1 & 0 & 1
        \end{bmatrix}\to
        \begin{bmatrix}
            1  & 0 & -1 \\
            -a & 0 & -b \\
            0  & 0 & 0
        \end{bmatrix}
    \end{equation*}
    因为\(\lambda=1\)是\(\mata\)的二重特征值,所以\(\rank{\lambda\mati-\mata}=1\),由此得到\(a=b=0\)。

    将\(\lambda_1=\lambda_2=1\),\(\lambda_3=-1\)代入特征方程组\(\mypar{\lambda\mati-\mata}\vecal=0\),得到\(\mata\)对应的线性无关的特征向量分别为\(\myvec{0,1,0}\),\(\myvec{1,0,1}\)和\(\myvec{-1,0,1}\)。所以对应的可逆矩阵\(\matp\)为\(\begin{bmatrix}0&1&-1\\1&0&0\\0&1&1\end{bmatrix}\)。
\end{proof}

\setcounter{problem}{31}
% 5.32
\begin{problem}
计算\({\begin{bmatrix}1&2&2\\2&1&2\\2&2&1\end{bmatrix}}^{100}\)。
\end{problem}
\begin{proof}
    记矩阵\(\begin{bmatrix}1&2&2\\2&1&2\\2&2&1\end{bmatrix}\)为\(\mata\)。则\(\mata\)的特征多项式为
    \begin{equation*}
        \abs{\lambda\mati-\mata}=\begin{vmatrix}\lambda-1&-2&-2\\-2&\lambda-1&-2\\-2&-2&\lambda-1\end{vmatrix}=\mypar{\lambda+1}^2\mypar{\lambda-5}
    \end{equation*}
    所以\(\mata\)的特征值为\(\lambda_1=\lambda_2=-1\),\(\lambda_3=5\)。

    将\(\lambda_1=\lambda_2=-1\),\(\lambda_3=5\)代入特征方程组\(\mypar{\lambda\mati-\mata}\vecal=0\),得到\(\mata\)对应的线性无关的特征向量分别为\(\myvec{-1,1,0}\),\(\myvec{-1,0,1}\)和\(\myvec{1,1,1}\)。所以有
    \begin{equation*}
        \matp=\begin{bmatrix}-1&-1&1\\1&0&1\\0&1&1\end{bmatrix},\matlam=\begin{bmatrix}-1&0&0\\0&-1&0\\0&0&5\end{bmatrix}
    \end{equation*}
    \begin{align*}
        \mata^{100} & =\mypar{\matp\matlam\inv{\matp}}^{100}=\matp\matlam^{100}\inv{\matp}                                                                                                  \\
                    & =\frac{1}{3}\begin{bmatrix}-1&-1&1\\1&0&1\\0&1&1\end{bmatrix}\begin{bmatrix}1&0&0\\0&1&0\\0&0&5^{100}\end{bmatrix}\begin{bmatrix}-1&2&-1\\-1&-1&2\\1&1&1\end{bmatrix} \\
                    & =\frac{1}{3}\begin{bmatrix}5^{100}+2&5^{100}-1&5^{100}-1\\5^{100}-1&5^{100}+2&5^{100}-1\\5^{100}-1&5^{100}-1&5^{100}+2\end{bmatrix}
    \end{align*}
\end{proof}

\setcounter{problem}{33}
% 5.34(2)
\begin{problem}
求二次型\(f\)的矩阵:
\begin{enumerate}
    \item[(2)] \(f(x_1,x_2,x_3)=\mypar{x_1+x_2}^2+\mypar{x_2-x_3}^2+\mypar{x_3+x_1}^2\)
\end{enumerate}
\end{problem}
\begin{proof}
    \begin{align*}
        f(x_1,x_2,x_3) & =\mypar{x_1+x_2}^2+\mypar{x_2-x_3}^2+\mypar{x_3+x_1}^2 \\
                       & =2x_1^2+2x_2^2+2x_3^2+2x_1x_2-2x_2x_3+2x_3x_1
    \end{align*}

    所以二次型\(f\)的矩阵为\(\mata=\begin{bmatrix}2&1&1\\1&2&-1\\1&-1&2\end{bmatrix}\)。
\end{proof}

% 5.35
\begin{problem}
用正交变换法将二次型
\begin{equation*}
    f\mypar{x_1,x_2,x_3}=x_1^2+4x_2^2+4x_3^2-4x_1x_2+4x_1x_3-8x_2x_3
\end{equation*}
化为标准形,写出所作的正交变换。
\end{problem}
\begin{proof}
    记二次型\(f\)的矩阵为\(\mata=\begin{bmatrix}1&-2&2\\-2&4&-4\\2&-4&4\end{bmatrix}\)。则\(\mata\)的特征多项式为
    \begin{equation*}
        \abs{\lambda\mati-\mata}=\begin{vmatrix}\lambda-1&2&-2\\2&\lambda-4&4\\-2&4&\lambda-4\end{vmatrix}=\lambda^2\mypar{\lambda-9}
    \end{equation*}
    所以\(\mata\)的特征值为\(\lambda_1=\lambda_2=0\),\(\lambda_3=9\)。

    将上述特征值代入特征方程组\(\mypar{\lambda\mati-\mata}\vecal=0\),得到\(\lambda_1,\lambda_2\)对应的线性无关的特征向量为\(\vecal_1=\myvec{2,0,-1}\)和\(\vecal_2=\myvec{0,1,1}\),\(\lambda_3\)对应的线性无关的特征向量为\(\vecal_3=\myvec{1,-2,2}\)。

    对向量组\(\vecal_1,\vecal_2\)和向量组\(\vecal_3\)分别做Schmidt正交化,得到正交变换的表示矩阵
    \begin{equation*}
        \matp=
        \begin{bmatrix}
            \frac{2\sqrt{5}}{5} & \frac{2\sqrt{39}}{39} & \frac{1}{3}  \\
            0                   & \frac{5\sqrt{39}}{39} & -\frac{2}{3} \\
            -\frac{\sqrt{5}}{5} & \frac{4\sqrt{39}}{39} & \frac{2}{3}
        \end{bmatrix}
    \end{equation*}
    即二次型\(f\)在正交变换\(\vecy=\matp^\top\vecx\)下化为标准形\(f\mypar{x_1,x_2,x_3}=9y_3^2\)。
\end{proof}

% 5.36
\begin{problem}
二次型
\begin{equation*}
    f\mypar{x_1,x_2,x_3}=\vecx^\top\mata\vecx=ax_1^2+2x_2^2-2x_3^2+2bx_1x_3\mypar{b>0}
\end{equation*}
其中二次型\(f\)的矩阵\(\mata\)的特征值之和为\(1\),特征值之积为\(-12\),求
\begin{enumerate}
    \item \(a,b\)的值;
    \item 用正交变换将\(f\)化为标准形,写出所作的正交变换及对应的正交矩阵。
\end{enumerate}
\end{problem}
\begin{proof}
    \begin{enumerate}
        \item {
              由题意,\(\mata=\begin{bmatrix}a&0&b\\0&2&0\\b&0&-2\end{bmatrix}\)。由定理5.1.2,有
              \begin{equation*}
                  \begin{cases}
                      \trace{\mata}=a=1 \\
                      \det{\mata}=-4a-2b^2=-12
                  \end{cases}
              \end{equation*}
              解得\(a=1,b=2\)。
              }
        \item {
              \(\mata\)的特征多项式为
              \begin{equation*}
                  \abs{\lambda\mati-\mata}=\begin{vmatrix}\lambda-1&0&-2\\0&\lambda-2&0\\-2&0&\lambda+2\end{vmatrix}=\mypar{\lambda-2}^2\mypar{\lambda+3}
              \end{equation*}
              所以\(\mata\)的特征值为\(\lambda_1=\lambda_2=2\),\(\lambda_3=-3\)。

              将上述特征值代入特征方程组\(\mypar{\lambda\mati-\mata}\vecal=0\),得到\(\lambda_1,\lambda_2\)对应的线性无关的特征向量为\(\vecal_1=\myvec{2,0,1}\)和\(\vecal_2=\myvec{0,1,0}\),\(\lambda_3\)对应的线性无关的特征向量为\(\vecal_3=\myvec{1,0,-2}\)。

              对向量组\(\vecal_1,\vecal_2\)和向量组\(\vecal_3\)分别做Schmidt正交化,得到正交变换的表示矩阵
              \begin{equation*}
                  \matp=
                  \begin{bmatrix}
                      \frac{2\sqrt{5}}{5} & 0 & \frac{\sqrt{5}}{5}   \\
                      0                   & 1 & 0                    \\
                      \frac{\sqrt{5}}{5}  & 0 & -\frac{2\sqrt{5}}{5}
                  \end{bmatrix}
              \end{equation*}
              即二次型\(f\)在正交变换\(\vecy=\matp^\top\vecx\)下化为标准形\(f\mypar{x_1,x_2,x_3}=2y_1^2+2y_2^2-3y_3^2\)。
              }
    \end{enumerate}
\end{proof}

\setcounter{problem}{37}
% 5.38
\begin{problem}
已知二次型\(f\mypar{x_1,x_2,x_3}=4x_1^2+4x_2^2+4x_3^2+4x_1x_2+ax_1x_3-4x_2x_3\)的秩为\(2\),
\begin{enumerate}
    \item 求\(a\)的值;
    \item 求正交变换\(\vecx=\matp\vecy\)把\(f\mypar{x_1,x_2,x_3}\)化为标准形;
    \item 求\(f\mypar{x_1,x_2,x_3}=0\)的解。
\end{enumerate}
\end{problem}
\begin{proof}
    记二次型\(f\)的矩阵为\(\mata=\begin{bmatrix}4&2&a/2\\2&4&2\\a/2&2&4\end{bmatrix}\)。
    \begin{enumerate}
        \item {
              因为\(\rank{\mata}=2\neq3\),因此有
              \begin{equation*}
                  \det{\mata}=32+4a-a^2=0
              \end{equation*}
              解得\(a=8\)或\(a=-4\),将\(a\)代入验证\(\rank{\mata}=2\),均满足条件。
              }
        \item {
              当\(a=8\)时,\(\mata\)的特征多项式为
              \begin{equation*}
                  \abs{\lambda\mati-\mata}=\begin{vmatrix}\lambda-4&-2&-4\\-2&\lambda-4&-2\\-4&-2&\lambda-4\end{vmatrix}=\lambda\mypar{\lambda^2-12\lambda+24}
              \end{equation*}
              所以\(\mata\)的特征值为\(\lambda_1=0\),\(\lambda_2=6+2\sqrt{3}\),\(\lambda_3=6-2\sqrt{3}\)。

              将上述特征值代入特征方程组\(\mypar{\lambda\mati-\mata}\vecal=0\),得到\(\mata\)的线性无关的特征向量为\(\vecal_1=\myvec{-1,0,1}\),\(\vecal_2=\myvec{1,-1+\sqrt{3},1}\),\(\vecal_3=\myvec{1,-1-\sqrt{3},1}\)。

              对\(\vecal_1,\vecal_2,\vecal_3\)标准化,得到正交变换的表示矩阵
              \begin{equation*}
                  \matp_1=
                  \begin{bmatrix}
                      -0.707 & 0.325  & -0.628 \\
                      0      & -0.888 & -0.460 \\
                      0.707  & 0.325  & -0.628
                  \end{bmatrix}
              \end{equation*}
              该正交变换将二次型\(f\)化为标准形\(f\mypar{x_1,x_2,x_3}=\mypar{6+2\sqrt{3}}y_2^2+\mypar{6-2\sqrt{3}}y_3^2\)。

              当\(a=-4\)时,\(\mata\)的特征多项式为
              \begin{equation*}
                  \abs{\lambda\mati-\mata}=\begin{vmatrix}\lambda-4&-2&2\\-2&\lambda-4&-2\\2&-2&\lambda-4\end{vmatrix}=\lambda\mypar{\lambda-6}^2
              \end{equation*}
              所以\(\mata\)的特征值为\(\lambda_1=\lambda_2=6\),\(\lambda_3=0\)。

              将上述特征值代入特征方程组\(\mypar{\lambda\mati-\mata}\vecal=0\),得到\(\lambda_1,\lambda_2\)对应的线性无关的特征向量为\(\vecal_1=\myvec{1,1,0}\)和\(\vecal_2=\myvec{-1,0,1}\),\(\lambda_3\)对应的线性无关的特征向量为\(\vecal_3=\myvec{1,-1,1}\)。

              对向量组\(\vecal_1,\vecal_2\)和向量组\(\vecal_3\)分别做Schmidt正交化,得到正交变换的表示矩阵
              \begin{equation*}
                  \matp_2=
                  \begin{bmatrix}
                      \frac{\sqrt{2}}{2} & -\frac{\sqrt{6}}{6} & \frac{\sqrt{3}}{3}  \\
                      \frac{\sqrt{2}}{2} & \frac{\sqrt{6}}{6}  & \frac{-\sqrt{3}}{3} \\
                      0                  & \frac{2\sqrt{6}}{6} & \frac{\sqrt{3}}{3}
                  \end{bmatrix}
              \end{equation*}
              该正交变换将二次型\(f\)化为标准形\(f\mypar{x_1,x_2,x_3}=6y_1^2+6y_2^2\)。
              }
        \item {
              当\(a=8\)时,因为正交变换\(\matp_1\)将二次型\(f\)化为标准形\(f\mypar{x_1,x_2,x_3}=\mypar{6+2\sqrt{3}}y_2^2+\mypar{6-2\sqrt{3}}y_3^2\),因此\(f\mypar{x_1,x_2,x_3}=0\)的解空间的基为
              \begin{equation*}
                  \matp_1\begin{bmatrix}1&0&0\end{bmatrix}^\top=\begin{bmatrix}-\frac{\sqrt{2}}{2}&0&\frac{\sqrt{2}}{2}\end{bmatrix}^\top
              \end{equation*}

              当\(a=-4\)时,因为正交变换\(\matp_2\)将二次型\(f\)化为标准形\(f\mypar{x_1,x_2,x_3}=6y_1^2+6y_2^2\),因此\(f\mypar{x_1,x_2,x_3}=0\)的解空间的基为
              \begin{equation*}
                  \matp_2\begin{bmatrix}0&0&1\end{bmatrix}^\top=\begin{bmatrix}\frac{\sqrt{3}}{3}&-\frac{\sqrt{3}}{3}&\frac{\sqrt{3}}{3}\end{bmatrix}^\top
              \end{equation*}
              }
    \end{enumerate}
\end{proof}

% 5.39
\begin{problem}
判断二次型的正定性:
\begin{enumerate}
    \item \(f\mypar{x_1,x_2,x_3}=5x_1^2+x_2^2+5x_3^2+4x_1x_2-8x_1x_3-4x_2x_3\);
    \item \(f\mypar{x_1,x_2,x_3,x_4}=x_1^2+x_2^2+14x_3^2+7x_4^2+6x_1x_3+8x_1x_4-4x_2x_3+2x_2x_4+4x_3x_4\)。
\end{enumerate}
\end{problem}
\begin{proof}
    \begin{enumerate}
        \item {
              记二次型\(f\)的矩阵为\(\mata=\begin{bmatrix}5&2&-4\\2&1&-2\\-4&-2&5\end{bmatrix}\)。则其顺序主子式为
              \begin{equation*}
                  \Delta_1=\begin{vmatrix}5\end{vmatrix}=5>0,\Delta_2=\begin{vmatrix}5&2\\2&1\end{vmatrix}=1>0,\Delta_3=\begin{vmatrix}5&2&-4\\2&1&-2\\-4&-2&5\end{vmatrix}=1>0
              \end{equation*}
              所以该二次型是正定的。
              }
        \item {
              记二次型\(f\)的矩阵为\(\mata=\begin{bmatrix}1&0&3&4\\0&1&-2&1\\3&-2&14&2\\4&1&2&7\end{bmatrix}\)。因为\(\mata\)的\(4\)阶顺序主子式为
              \begin{equation*}
                  \Delta_4=\begin{vmatrix}1&0&3&4\\0&1&-2&1\\3&-2&14&2\\4&1&2&7\end{vmatrix}=-74<0
              \end{equation*}
              所以该二次型不是正定的。
              }
    \end{enumerate}
\end{proof}

% 5.40
\begin{problem}
判断二次型\(f=\sum_{i=1}^nx_i^2+\sum_{1\le i<j\le n}x_ix_j\)的正定性。
\end{problem}
\begin{proof}
    记二次型\(f_n\)的矩阵为\(\mata_n\in\rea^{n\times n}\),则有
    \begin{gather*}
        A_n =
        \begin{bmatrix}
            1           & \frac{1}{2} & \cdots & \frac{1}{2} \\
            \frac{1}{2} & 1           & \cdots & \frac{1}{2} \\
            \vdots      & \vdots      &        & \vdots      \\
            \frac{1}{2} & \frac{1}{2} & \cdots & 1
        \end{bmatrix}\xrightarrow[i=2,3,\dots,n]{C_1+C_i}
        \begin{bmatrix}
            \frac{n+1}{2} & \frac{1}{2} & \cdots & \frac{1}{2} \\
            \frac{n+1}{2} & 1           & \cdots & \frac{1}{2} \\
            \vdots        & \vdots      &        & \vdots      \\
            \frac{n+1}{2} & \frac{1}{2} & \cdots & 1
        \end{bmatrix} \\
        \xrightarrow[i=2,3,\dots,n]{R_i-R_1}
        \begin{bmatrix}
            \frac{n+1}{2} & \frac{1}{2} & \cdots & \frac{1}{2} \\
            0             & \frac{1}{2} & \cdots & \frac{1}{2} \\
            \vdots        & \vdots      &        & \vdots      \\
            0             & 0           & \cdots & \frac{1}{2}
        \end{bmatrix}
    \end{gather*}
    所以\(\det{\mata_n}=\mypar{n+1}2^{-n}>0\)。又因为\(\mata_n\)的顺序主子式为\(\Delta_i=\det{\mata_i}\),由定理5.4.2知\(\mata_n\)正定,即二次型\(f_n\)正定。
\end{proof}

% 5.41
\begin{problem}
设\(\mata\)为\(3\)阶实对称矩阵,且满足\(\mata^2+2\mata=\mato\),\(\rank{\mata}=2\),
\begin{enumerate}
    \item 求\(\mata\)的特征值;
    \item 当\(k\)为何值时,矩阵\(\mata+k\mati\)为正定矩阵?
\end{enumerate}
\end{problem}
\begin{proof}
    \begin{enumerate}
        \item {
              设\(\mata\)特征值为\(\lambda\)对应的特征向量为\(\vecal\neq\veczero\),则有
              \begin{align*}
                  \mypar{\lambda^2+2\lambda}\vecal & =\mypar{\mata^2+2\mata}\vecal \\
                                                   & =\mato\vecal=\veczero
              \end{align*}
              所以\(\lambda^2+2\lambda=0\),解得\(\lambda=0\)或\(\lambda=-2\)。
              }
        \item {
              因为\(\rank{\mata}=2\)且\(\mata\)为实对称矩阵,因此存在正交矩阵\(\matp\),满足\(\mata+k\mati=\matp\matlam\matp^\top\),其中\(\matlam=\diag{\mat{k-2,k-2,k}}\)。由正定矩阵定义可知当\(k>2\)时,\(\mata+k\mati\)正定。
              }
    \end{enumerate}
\end{proof}

% 5.42
\begin{problem}
设\(\mata\)是\(n\)阶实对称矩阵,证明\(\rank{\mata}=n\)的充要条件是存在一个\(n\)阶实矩阵\(\matb\),使得\(\mata\matb+\matb^\top\mata\)正定。
\end{problem}
\begin{proof}
    首先证明必要性。

    因为\(\rank{\mata}=n\)且\(\mata\)为实对称矩阵,可知\(\mata\)的特征值为非零实数\(\enums{\lambda}{n}\)。令\(\matb=\mata\),则\(\mata\matb+\matb^\top\mata=2\mata^2\)特征值为\(2\lambda_1^2,2\lambda_2^2\dots,2\lambda_n^2\)。因此\(2\mata^2\)正定,结论成立。

    随后用反证法证明充分性。

    因为\(\rank{\mata}<n\),所以存在\(\vecx\in\rea^n\neq\veczero\)满足\(\mata\vecx=\veczero\)。因此对\(\forall\matb\in\rea^{n\times n}\),有
    \begin{align*}
        \vecx^\top\mypar{\mata\matb+\matb^\top\mata}\vecx & =\vecx^\top\mypar{\mata\matb}\vecx+\vecx^\top\mypar{\matb^\top\mata}\vecx \\
                                                          & =\mypar{\mata\vecx}^\top\matb\vecx+\mypar{\matb\vecx}^\top\mata\vecx=0
    \end{align*}
    这和题设矛盾。
\end{proof}

% 5.43
\begin{problem}
若\(\mata\)是\(n\)阶实矩阵,证明:\(\trace{\mata\mata^\top}\ge0\),且等号成立的充要条件是\(\mata=\mato\)。
\end{problem}
\begin{proof}
    因为\(\mata\mata^\top\)为实对称矩阵,因此它可被对角化为\(\matp\matlam\matp^\top\),其中\(\matp\)为正交矩阵。

    设\(\vece_i\in\rea^n\)表示第\(i\)个分量为\(1\),其余分量为\(0\)的单位向量。因此得到
    \begin{align*}
        \vece_i^\top\matlam\vece_i & =\vece_i^\top\matp^\top\mata\mata^\top\matp\vece_i \\
                                   & =\Abs{\mata^\top\matp\vece_i}^2\ge0
    \end{align*}
    因此有\(\trace{\mata\mata^\top}=\sum_{i=1}^n\vece_i^\top\matlam\vece_i\ge0\)。

    注意到当\(\vece_i^\top\matlam\vece_i=0\)对\(i=1,2,\dots,n\)均成立时等号成立,即\(\matlam=\mato\)。此时对\(\forall\vecx\in\rea^n\),有
    \begin{equation*}
        \Abs{\mata^\top\vecx}^2=\vecx^\top\mata\mata^\top\vecx=\vecx^\top\matp\matlam\matp^\top\vecx=0
    \end{equation*}
    因此\(\mata^\top\vecx=\veczero\)对\(\forall\vecx\in\rea^n\)均成立,即\(\mata=\mata^\top=\mato\)。
\end{proof}

% 5.44
\begin{problem}
设\(\mata\)是\(n\)阶正定矩阵,证明:对任意\(n\)阶矩阵\(\matb\),\(\rank{\matb^\top\mata\matb}=\rank{\matb}\)。
\end{problem}
\begin{proof}
    设\(\mata\)的特征值为\(\lambda_1>\lambda_2>\cdots>\lambda_n\)。因为\(\mata\)是正定矩阵,因此\(\mata\)可被对角化为\(\matp\matlam\matlam^\top\matp^\top\),其中\(\matp\)为正交矩阵,\(\matlam=\diag{\mat{\sqrt{\lambda_1},\sqrt{\lambda_2},\dots,\sqrt{\lambda_n}}}\)。因此有
    \begin{equation*}
        \matb^\top\mata\matb=\matb^\top\matp\matlam\matlam^\top\matp^\top\matb=\mypar{\matlam^\top\matp^\top\matb}^\top\mypar{\matlam^\top\matp^\top\matb}
    \end{equation*}
    所以有
    \begin{align*}
        \rank{\matb^\top\mata\matb} & =\rank{\mypar{\matlam^\top\matp^\top\matb}^\top\mypar{\matlam^\top\matp^\top\matb}} \\
                                    & =\rank{\matlam^\top\matp^\top\matb}=\rank{\matb}
    \end{align*}
\end{proof}