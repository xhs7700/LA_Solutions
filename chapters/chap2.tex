\section{线性方程组}

% 2.1
\begin{problem}
设\(\mata\)为\(m\times n\)矩阵,判断下列命题是否正确。
\begin{enumerate}
    \item 非齐次线性方程组\(\mata\vecx=\vecb\),当\(m<n\)时,有无穷多解;当\(m=n\)时,有唯一解;当\(m>n\)时,无解;
    \item 齐次线性方程组\(\mata\vecx=\veczero\),当\(m<n\)时,必有非零解;
    \item 非齐次线性方程组\(\mata\vecx=\vecb\),当\(\rank{\mata}=m\),则一定有解。
\end{enumerate}
\end{problem}
\begin{proof}

\end{proof}

% 2.2
\begin{problem}
解下列线性方程组:
\begin{enumerate}
    \item \begin{equation*}
              \begin{cases}
                  x_1-2x_2+3x_3-4x_4=0 \\
                  x_2-x_3+x_4=0        \\
                  x_1+3x_2-3x_4=0      \\
                  x_1-4x_2+3x_3-2x_4=0
              \end{cases}
          \end{equation*}
    \item \begin{equation*}
              \begin{cases}
                  x_1+x_2+x_3+x_4+x_5=0    \\
                  3x_1+2x_2+x_3+x_4-3x_5=0 \\
                  x_2+2x_3+2x_4+6x_5=0     \\
                  5x_1+4x_2+3x_3+3x_4-x_5=0
              \end{cases}
          \end{equation*}
    \item \begin{equation*}
              \begin{cases}
                  x_1+2x_2+x_3-x_4=6         \\
                  2x_1-x_2+x_3+3x_4+4x_5=-7  \\
                  2x_1-x_2+2x_3+x_4-2x_5=-4  \\
                  2x_1-3x_2+x_3+2x_4-2x_5=-9 \\
                  x_1+x_3-2x_4-6x_5=4
              \end{cases}
          \end{equation*}
    \item \begin{equation*}
              \begin{cases}
                  2x_1-x_2-2x_3+x_4=0 \\
                  x_1+2x_2+2x_3+x_4=6 \\
                  3x_1+x_2-x_3-2x_4=1 \\
                  x_1+2x_2+x_3-3x_4=2 \\
                  2x_1+4x_2+3x_3-2x_4=7
              \end{cases}
          \end{equation*}
\end{enumerate}
\end{problem}
\begin{proof}

\end{proof}

% 2.3
\begin{problem}
求线性方程组
\begin{equation*}
    \begin{cases}
        x_1-x_2+2x_3+x_4=1  \\
        2x_1-x_2+x_3+2x_4=3 \\
        x_1-x_3+x_4=2       \\
        3x_1-x_2+3x_4=5
    \end{cases}
\end{equation*}
的通解,并求出满足\(x_1^2=x_2^2\)的全部解。
\end{problem}
\begin{proof}

\end{proof}

% 2.4
\begin{problem}
讨论下列线性方程组,当\(\lambda\)取什么值时方程组有唯一解?取什么值时有无穷多解?取什么值时无解?
\begin{enumerate}
    \item \begin{equation*}
              \begin{cases}
                  \mypar{\lambda+3}x_1+x_2+2x_3=\lambda        \\
                  \lambda x_1+\mypar{\lambda-1}x_2+x_3=\lambda \\
                  3\mypar{\lambda+1}x_1+\lambda x_1+\mypar{\lambda+3}x_3=3
              \end{cases}
          \end{equation*}
    \item
          {
          \begin{equation*}
              \begin{cases}
                  x_1+x_2+\lambda x_3=2 \\
                  2x_1+\lambda x_2+8x_3=\lambda
              \end{cases}
          \end{equation*}
          }
    \item \begin{equation*}
              \begin{cases}
                  x_1+x_2+\lambda x_3=2  \\
                  3x_1+4x_2+2x_3=\lambda \\
                  2x_1+3x_2-x_3=3        \\
              \end{cases}
          \end{equation*}
\end{enumerate}
\end{problem}
\begin{proof}

\end{proof}

% 2.5
\begin{problem}
讨论\(a,b\)为何值时,线性方程组
\begin{equation*}
    \begin{cases}
        x_1-x_2+2x_3=1             \\
        2x_1-x_2+3x_3-x_4=4        \\
        x_2+ax_3+bx_4=b            \\
        x_1-3x_2+\mypar{3-a}x_3=-4 \\
    \end{cases}
\end{equation*}
\end{problem}
\begin{proof}

\end{proof}

% 2.6
\begin{problem}
设\(x_1-x_2=a_1\),\(x_2-x_3=a_2\),\(x_3-x_4=a_3\),\(x_4-x_5=a_4\),\(x_5-x_1=a_5\)。证明:这个方程组有解的充分必要条件为
\begin{equation*}
    \sum_{i=1}^5a_i=0
\end{equation*}
并在有解的情况下求出它的一般解。
\end{problem}
\begin{proof}

\end{proof}

% 2.7
\begin{problem}
已知平面上三条不同直线的方程分别为:
\begin{align*}
    l_1: & ax+2by+3c=0 \\
    l_2: & bx+2cy+3a=0 \\
    l_3: & cx+2ay+3b=0
\end{align*}
试证这三条直线交于一点的充要条件是\(a+b+c=0\)。
\end{problem}
\begin{proof}

\end{proof}

% 2.8
\begin{problem}
设有\(n\)元齐次线性方程组(\(n\ge2\))
\begin{equation*}
    \begin{cases}
        \mypar{1+a}x_1+x_2+\cdots+x_n=0   \\
        2x_1+\mypar{2+a}x_2+\cdots+2x_n=0 \\
        \vdots                            \\
        nx_1+nx_2+\cdots+\mypar{n+a}x_n=0
    \end{cases}
\end{equation*}
试问\(a\)取何值时,该方程组有非零解,并求出其通解。
\end{problem}
\begin{proof}

\end{proof}

% 2.9
\begin{problem}
已知非齐次线性方程组
\begin{equation*}
    x_1+x_2+x_3+x_4=-1\\
    4x_1+3x_2+5x_3-x_4=-1\\
    ax_1+x_2+3x_3+bx_4=1
\end{equation*}
有\(3\)个线性无关的解,
\begin{enumerate}
    \item 证明方程组系数矩阵\(\mata\)的秩\(\rank{\mata}=2\);
    \item 求\(a,b\)的值及方程组的通解。
\end{enumerate}
\end{problem}
\begin{proof}

\end{proof}

% 2.10
\begin{problem}
设线性方程组
\begin{equation*}
    \begin{cases}
        x_1+x_2+x_3=0   \\
        x_1+2x_2+ax_3=0 \\
        x_1+4x_2+a^2x_3=0
    \end{cases}
\end{equation*}
与方程\(x_1+2x_2+x_3=a-1\)有公共解,求\(a\)的值及所有公共解。
\end{problem}
\begin{proof}

\end{proof}

% 2.11
\begin{problem}
设线性方程组\(\mata\vecx=\vecb\)存在两个不同的解,其中
\begin{equation*}
    \mata=
    \begin{bmatrix}
        \lambda & 1         & 1       \\
        0       & \lambda-1 & 0       \\
        1       & 1         & \lambda
    \end{bmatrix},
    \vecb=
    \begin{bmatrix}
        a \\1\\1
    \end{bmatrix}
\end{equation*}
\begin{enumerate}
    \item 求\(\lambda,a\)的值;
    \item 求方程组\(\mata\vecx=\vecb\)的通解。
\end{enumerate}
\end{problem}
\begin{proof}

\end{proof}

% 2.12
\begin{problem}
设
\begin{equation*}
    \mata=
    \begin{bmatrix}
        1 & a & 0 & 0 \\
        0 & 1 & a & 0 \\
        0 & 0 & 1 & a \\
        a & 0 & 0 & 1
    \end{bmatrix},
    \vecb=
    \begin{bmatrix}
        1 \\-1\\0\\0
    \end{bmatrix}
\end{equation*}
当实数\(a\)为何值时,方程组\(\mata\vecx=\vecb\)有无穷多解,并求其通解。
\end{problem}
\begin{proof}

\end{proof}

% 2.13
\begin{problem}
已知\(4\)阶矩阵\(\mata=\mat{\vecal_1,\vecal_2,\vecal_3,\vecal_4}\),\(\vecal_1,\vecal_2,\vecal_3,\vecal_4\)均为\(4\)元列向量,其中\(\vecal_2,\vecal_3,\vecal_4\)线性无关,\(\vecal_1=2\vecal_2-\vecal_3\),若\(\vecbeta=\vecal_1+\vecal_2+\vecal_3+\vecal_4\),求线性方程组\(\mata\vecx=\vecbeta\)的通解。
\end{problem}
\begin{proof}

\end{proof}

% 2.14
\begin{problem}
已知\(3\)阶矩阵\(\mata\)的第一行是\(\mat{a,b,c}\),\(a,b,c\)不全为零,
\begin{equation*}
    \matb=
    \begin{bmatrix}
        1 & 2 & 3 \\
        2 & 4 & 6 \\
        3 & 6 & k
    \end{bmatrix}
\end{equation*}
\(k\)为常数,且\(\mata\matb=\mato\),求线性方程组\(\mata\vecx=\veczero\)的通解。
\end{problem}
\begin{proof}

\end{proof}

% 2.15
\begin{problem}
设\(\vecal,\vecbeta\)为三维列向量,矩阵\(\mata=\vecal\vecal^\top+\vecbeta\vecbeta^\top\),证明:
\begin{enumerate}
    \item 秩\(\rank{\mata}\le2\);
    \item 若\(\vecal,\vecbeta\)线性相关,则秩\(\rank{\mata}<2\)。
\end{enumerate}
\end{problem}
\begin{proof}

\end{proof}

% 2.16
\begin{problem}
设
\begin{equation*}
    \mata=
    \begin{bmatrix}
        1  & -1 & -1 \\
        -1 & 1  & 1  \\
        0  & -4 & -2
    \end{bmatrix},
    \vecxi_1=\begin{bmatrix}-1\\1\\-2\end{bmatrix}
\end{equation*}
\begin{enumerate}
    \item 求满足\(\mata\vecxi_2=\vecxi_1\),\(\mata\vecxi_3^2=\vecxi_1\)的所有向量\(\vecxi_2,\vecxi_3\);
    \item 对(1)中的任意向量\(\vecxi_2,\vecxi_3\),证明\(\vecxi_1,\vecxi_2,\vecxi_3\)线性无关。
\end{enumerate}
\end{problem}
\begin{proof}

\end{proof}

% 2.17
\begin{problem}
设向量组\(\vecal_1=\myvec{1,0,1}\),\(\vecal_2=\myvec{0,1,1}\),\(\vecal_3=\myvec{1,3,5}\)不能由向量组\(\vecbeta_1=\myvec{1,1,1}\),\(\vecbeta_2=\myvec{1,2,3}\),\(\vecbeta_3=\myvec{3,4,a}\)线性表出,
\begin{enumerate}
    \item 求\(a\)的值;
    \item 将\(\vecbeta_1,\vecbeta_2,\vecbeta_3\)由\(\vecal_1,\vecal_2,\vecal_3\)线性表出。
\end{enumerate}
\end{problem}
\begin{proof}

\end{proof}

% 2.18
\begin{problem}
设矩阵
\begin{equation*}
    \mata=
    \begin{bmatrix}
        1 & -2 & 3  & -4 \\
        0 & 1  & -1 & 1  \\
        1 & 2  & 0  & -3
    \end{bmatrix}
\end{equation*}
\begin{enumerate}
    \item 求方程组\(\mata\vecx=\veczero\)的一个基础解系;
    \item 求满足\(\mata\matb=\mati\)的所有矩阵\(\matb\),其中\(\mati\)是\(3\)阶单位矩阵。
\end{enumerate}
\end{problem}
\begin{proof}

\end{proof}

% 2.19
\begin{problem}
\begin{enumerate}
    \item 设\(\mata\)是\(2\)阶方阵,求满足\(\mata^2=\mato\)的所有矩阵\(\mata\);
    \item 证明:若\(\mata^2=\mato\),且\(\mata=\mata^\top\),则\(\mata=\mato\)。
\end{enumerate}
\end{problem}
\begin{proof}

\end{proof}

% 2.20
\begin{problem}
设
\begin{equation*}
    \mata=
    \begin{bmatrix}
        \lambda-1 & 1         & \lambda   \\
        1         & \lambda-1 & 0         \\
        \lambda   & 1         & \lambda-1
    \end{bmatrix}
\end{equation*}
若存在非零\(3\)阶矩阵\(\matb\),使得\(\mata\matb=\mato\),求\(\lambda\)的值和齐次线性方程组\(\mata\vecx=\veczero\)的通解。
\end{problem}
\begin{proof}

\end{proof}

% 2.21
\begin{problem}
设\(\mata\)是\(s\times n\)矩阵,\(\matb\)是\(n\times m\)矩阵,\(n<m\),证明:齐次线性方程组\(\mypar{\mata\matb}\vecx=\veczero\)有非零解。
\end{problem}
\begin{proof}

\end{proof}

% 2.22
\begin{problem}
设\(\mata\)是\(m\times s\)矩阵,\(\matb\)是\(s\times n\)矩阵,\(\vecx\)是\(n\)元列向量。证明:若\(\mypar{\mata\matb}\vecx=\veczero\)与\(\matb\vecx=\veczero\)是同解方程组,则\(\rank{\mata\matb}=\rank{\matb}\)。
\end{problem}
\begin{proof}

\end{proof}

% 2.23
\begin{problem}
试将向量\(\vecbeta\)表示成向量组\(\vecal_1,\vecal_2,\vecal_3\)的线性组合:
\begin{enumerate}
    \item \(\vecbeta=\myvec{3,4,1}\),\(\vecal_1=\myvec{1,-1,1}\),\(\vecal_2=\myvec{2,2,0}\),\(\vecal_3=\myvec{1,1,1}\);
    \item \(\vecbeta=\myvec{3,5,9}\),\(\vecal_1=\myvec{0,2,-1}\),\(\vecal_2=\myvec{1,0,4}\),\(\vecal_3=\myvec{1,3,1}\);
    \item \(\vecbeta=\myvec{4,1,2,-1}\),\(\vecal_1=\myvec{4,4,8,1}\),\(\vecal_2=\myvec{3,4,7,2}\),\(\vecal_3=\myvec{1,3,-1,7}\);
\end{enumerate}
\end{problem}
\begin{proof}

\end{proof}

% 2.24
\begin{problem}
问\(y\)为何值时,向量\(\vecbeta=\myvec{4,6,y,-2}\)可以由向量组\(\vecal_1=\myvec{2,3,1,-2}\),\(\vecal_2=\myvec{3,-2,-5,3}\),\(\vecal_3=\myvec{-3,2,2,-1}\)线性表出。
\end{problem}
\begin{proof}

\end{proof}

% 2.25
\begin{problem}
判断下列向量组的线性相关性:
\begin{enumerate}
    \item \(\vecal_1=\myvec{2,2,7,-1}\),\(\vecal_2=\myvec{3,-1,2,4}\),\(\vecal_3=\myvec{1,1,3,1}\);
    \item \(\vecal_1=\myvec{1,-2,4,-1}\),\(\vecal_2=\myvec{1,3,2,1}\),\(\vecal_3=\myvec{1,4,1,0}\),\(\vecal_4=\myvec{3,5,6,1}\);
    \item \(\vecal_1=\myvec{1,0,2,1}\),\(\vecal_2=\myvec{1,2,0,1}\),\(\vecal_3=\myvec{1,5,-3,3}\),\(\vecal_4=\myvec{2,1,3,0}\)。
\end{enumerate}
\end{problem}
\begin{proof}

\end{proof}

% 2.26
\begin{problem}
求下列向量组的秩和极大线性无关组:
\begin{enumerate}
    \item \(\vecal_1=\myvec{2,1,0}\),\(\vecal_2=\myvec{3,1,1}\),\(\vecal_3=\myvec{2,0,2}\),\(\vecal_4=\myvec{4,2,0}\);
    \item \(\vecal_1=\myvec{1,1,1,1}\),\(\vecal_2=\myvec{1,1,-1,-1}\),\(\vecal_3=\myvec{1,-1,-1,1}\),\(\vecal_4=\myvec{-1,-1,-1,1}\);
    \item \(\vecal_1=\myvec{6,4,1,-1,2}\),\(\vecal_2=\myvec{1,0,2,3,-4}\),\(\vecal_3=\myvec{1,4,-9,-16,22}\),\(\vecal_4=\myvec{7,1,0,-1,3}\)。
\end{enumerate}
\end{problem}
\begin{proof}

\end{proof}

% 2.27
\begin{problem}
已知向量组\(\vecal_1=\myvec{1,-1,2,1,0}\),\(\vecal_2=\myvec{2,-2,4,-2,0}\),\(\vecal_3=\myvec{3,0,6,-1,1}\),\(\vecal_4=\myvec{0,x,0,0,1}\)有\(4\)个不同极大无关组,求\(x\)的值。
\end{problem}
\begin{proof}

\end{proof}

% 2.28(1)
\begin{problem}
求下列齐次线性方程组的一个基础解系:
\begin{enumerate}
    \item
          {
          \begin{equation*}
              \begin{cases}
                  x_1-2x_2+4x_3-7x_4=0 \\
                  2x_1+x_2-2x_3+x_4=0  \\
                  3x_1-x_2+2x_3-4x_4=0 \\
              \end{cases}
          \end{equation*}
          }
\end{enumerate}
\end{problem}
\begin{proof}

\end{proof}

% 2.29
\begin{problem}
求下列线性方程组的通解,并表示成列向量线性组合的形式:
\begin{enumerate}
    \item
          {
          \begin{equation*}
              \begin{cases}
                  x_1-2x_2-x_3-x_4+x_5=0    \\
                  2x_1+x_2-x_3+2x_4-3x_5=0  \\
                  3x_1-2x_2-x_3+x_4-2x_5=0  \\
                  2x_1-5x_2+x_3-2x_4+2x_5=0 \\
              \end{cases}
          \end{equation*}
          }
    \item \begin{equation*}
              \begin{cases}
                  x_1+x_2+x_3+x_4+x_5=1    \\
                  3x_1+2x_2+x_3+x_4-3x_5=0 \\
                  x_2+2x_3+2x_4+6x_5=3     \\
                  5x_1+4x_2+3x_3+3x_4-x_5=2
              \end{cases}
          \end{equation*}
\end{enumerate}
\end{problem}
\begin{proof}

\end{proof}

% 2.30
\begin{problem}
已知:向量组I可用向量组II线性表出,向量组II可用向量组III线性表出。求证:向量组I可用向量组III线性表出。
\end{problem}
\begin{proof}

\end{proof}

% 2.31
\begin{problem}
设向量组\(\enums{\vecal}{n}\)线性无关。证明:当且仅当\(n\)为奇数时,向量组\(\vecal_1+\vecal_2\),\(\vecal_2+\vecal_3\),\(\cdots\),\(\vecal_{n-1}+\vecal_n\),\(\vecal_n+\vecal_1\)也线性无关。
\end{problem}
\begin{proof}

\end{proof}

% 2.32
\begin{problem}
设向量组\(\enums{\vecal}{n}\)线性无关,而向量组\(\enums{\vecal}{n},\vecbeta,\vecgamma\)线性相关。证明:或者\(\vecbeta\)与\(\vecgamma\)中至少有一个可由\(\enums{\vecal}{n}\)线性表出,或者向量组\(\enums{\vecal}{n},\vecbeta\)与向量组\(\enums{\vecal}{n},\vecgamma\)可相互线性表出。
\end{problem}
\begin{proof}

\end{proof}

% 2.33
\begin{problem}
设\(\vecbeta_1=\vecal_2+\vecal_3+\cdots+\vecal_{n}\),\(\vecbeta_2=\vecal_1+\vecal_3+\cdots+\vecal_{n}\),\(\cdots\),\(\vecbeta_n=\vecal_1+\vecal_2+\cdots+\vecal_{n-1}\),其中\(m>1\)。证明:向量组\(\enums{\vecal}{n}\)与向量组\(\enums{\vecbeta}{n}\)可相互线性表出。
\end{problem}
\begin{proof}

\end{proof}

% 2.34
\begin{problem}
设向量组\(\enums{\vecal}{n}\)与\(\enums{\vecbeta}{n}\)满足:
\begin{equation*}
    \vecal_i=\begin{bmatrix}x_{1i}\\x_{2i}\\\vdots\\x_{ri}\end{bmatrix},
    \vecbeta_i=\begin{bmatrix}x_{1i}\\x_{2i}\\\vdots\\x_{ri}\\x_{\mypar{r+1}i}\\\vdots\\x_{mi}\end{bmatrix},
    i=1,2,\cdots,n
\end{equation*}
证明:若\(\enums{\vecbeta}{n}\)线性相关,则\(\enums{\vecal}{n}\)线性相关;若\(\enums{\vecal}{n}\)线性无关,则\(\enums{\vecbeta}{n}\)线性无关。
\end{problem}
\begin{proof}

\end{proof}

% 2.35
\begin{problem}
给定\(n\)个非零的数\(\enums{a}{n}\),求下面向量组的秩:
\begin{align*}
    \veceta_1 & =\mat{1+a_1,1,\cdots,1} \\
    \veceta_2 & =\mat{1,1+a_2,\cdots,1} \\
              & \vdots                  \\
    \veceta_n & =\mat{1,1,\cdots,1+a_n}
\end{align*}
\end{problem}
\begin{proof}

\end{proof}

% 2.36
\begin{problem}
设\(\mata,\matb\)分别是\(m\times n\)和\(n\times s\)的矩阵,且\(\mata\matb=\mato\)。证明:\(\rank{\mata}+\rank{\matb}\le n\)。
\end{problem}
\begin{proof}

\end{proof}

% 2.37
\begin{problem}
设\(\enums{\veceta}{t}\)是某一非齐次线性方程组的解。证明:\(\mu_1\veceta_1+\mu_2\veceta_2+\cdots+\mu_t\veceta_t\)也是该非齐次线性方程组的解的充要条件是\(\mu_1+\mu_2+\cdots+\mu_t=1\)。
\end{problem}
\begin{proof}

\end{proof}

% 2.38
\begin{problem}
证明:方程组
\begin{equation*}
    \begin{cases}
        a_{11}x_1+a_{12}x_2+\cdots+a_{1n}x_n=0 \\
        a_{21}x_1+a_{22}x_2+\cdots+a_{2n}x_n=0 \\
        \vdots                                 \\
        a_{m1}x_1+a_{m2}x_2+\cdots+a_{mn}x_n=0
    \end{cases}
\end{equation*}
的解全是方程\(b_1x_1+b_2x_2+\cdots+b_nx_n=0\)的解的充要条件是:向量\(\vecbeta=\mat{\enums{b}{n}}\)可由向量组\(\enums{\vecal}{m}\)线性表出,其中
\begin{equation*}
    \vecal_i=\mat{a_{i1},a_{i2},\cdots,a_{im}},i=1,2,\cdots,n
\end{equation*}
\end{problem}
\begin{proof}

\end{proof}

% 2.39
\begin{problem}
求下列矩阵的逆矩阵:
\begin{enumerate}
    \item \(\begin{bmatrix}
              1 & 1  & 1  & 1  \\
              1 & 1  & -1 & -1 \\
              1 & -1 & 1  & -1 \\
              1 & -1 & -1 & 1
          \end{bmatrix}\);
    \item \(\begin{bmatrix}
              5 & 2 & 0 & 0  \\
              2 & 1 & 0 & 0  \\
              0 & 0 & 1 & -2 \\
              0 & 0 & 1 & 1
          \end{bmatrix}\);
    \item \(\begin{bmatrix}
              1 & a & a^2 & a^3 \\
              0 & 1 & a   & a^2 \\
              0 & 0 & 1   & a   \\
              0 & 0 & 0   & 1
          \end{bmatrix}\);
    \item \(\begin{bmatrix}
              0      & a_1    & 0      & \cdots & 0       \\
              0      & 0      & a_2    & \cdots & 0       \\
              \vdots & \vdots & \vdots &        & \vdots  \\
              0      & 0      & 0      & \cdots & a_{n-1} \\
              a_n    & 0      & 0      & \cdots & 0
          \end{bmatrix},a_i\neq0,i=1,2,\cdots,n\)。
\end{enumerate}
\end{problem}
\begin{proof}

\end{proof}

% 2.40
\begin{problem}
解下列矩阵方程:
\begin{enumerate}
    \item \(\begin{bmatrix}1&1&1\\0&1&1\\0&0&1\end{bmatrix}\matx=\begin{bmatrix}5&6\\3&4\\1&2\end{bmatrix}\);
    \item \(\matx\begin{bmatrix}
              1 & 2  & -3 \\
              3 & 2  & -4 \\
              2 & -1 & 0
          \end{bmatrix}=\begin{bmatrix}
              1  & -3 & 0 \\
              10 & 2  & 7 \\
              10 & 7  & 8
          \end{bmatrix}\)。
\end{enumerate}
\end{problem}
\begin{proof}

\end{proof}

% 2.41
\begin{problem}
设\(\mata,\matb,\matc\)均为\(n\)阶方阵,若\(\mata\matb\matc=\mati\),则下列乘积:\(\mata\matc\matb\),\(\matb\mata\matc\),\(\matb\matc\mata\),\(\matc\mata\matb\),\(\matc\matb\mata\)中哪些必等于单位阵\(\mati\)。
\end{problem}
\begin{proof}

\end{proof}

% 2.42
\begin{problem}
\begin{enumerate}
    \item 设\(\mata\)为方阵,正整数\(k>1\),\(\mata^k=\mato\),证明:
          \begin{equation*}
              \pinv{\mati-\mata}=\mati+\mata+\mata^2+\cdots+\mata^{k-1}
          \end{equation*}
    \item 设\(\matj_n\)为所有元素全为\(1\)的\(n\)阶矩阵,证明:
          \begin{equation*}
              \pinv{\mati-\matj_n}=\mati-\frac{1}{n-1}\matj_n
          \end{equation*}
\end{enumerate}
\end{problem}
\begin{proof}

\end{proof}

% 2.43
\begin{problem}
设\(\mata=\begin{bmatrix}\mato&\matb\\\matc&\mato\end{bmatrix}\),其中\(\matb\)是\(n\)阶可逆矩阵,\(\matc\)是\(m\)阶可逆矩阵。证明:\(\mata\)可逆,并求出\(\inv{\mata}\)。
\end{problem}
\begin{proof}

\end{proof}

% 2.44
\begin{problem}
设\(n\)阶可逆矩阵\(\mata\)每行元素之和都等于常数\(c\)。证明:\(c\neq0\),且\(\inv{\mata}\)中每行元素之和都等于\(\inv{c}\)。
\end{problem}
\begin{proof}

\end{proof}

% 2.45
\begin{problem}
设\(\mata\)为方阵,又\(\mata+\mati\)可逆,\(f\mypar{\mata}=\mypar{\mati-\mata}\pinv{\mati+\mata}\)。证明:
\begin{enumerate}
    \item \(\mypar{\mati+f\mypar{\mata}}\mypar{\mati+\mata}=2\mati\);
    \item \(f\mypar{f\mypar{\mata}}=\mata\)。
\end{enumerate}
\end{problem}
\begin{proof}

\end{proof}

% 2.46
\begin{problem}
设\(\mata,\matb\)均为\(n\)阶可逆矩阵。证明:如果\(\mata+\matb\)可逆,则\(\inv{\mata}+\inv{\matb}\)也可逆,并求其逆矩阵。
\end{problem}
\begin{proof}

\end{proof}

% 2.47
\begin{problem}
\begin{enumerate}
    \item 设\(\mata\)为\(n\)阶方阵,满足\(\mata^3+2\mata^2-2\mata-\mati_n=\mato\),证明:\(\mata+\mati_n\)是可逆矩阵,并求其逆矩阵;
    \item
          {
          设\(\mata,\matb\)为\(n\)阶方阵,且\(\mata-\mati_n\)和\(\matb\)可逆。证明:若\(\pinv{\mata-\mati_n}=\mypar{\matb-\mati_n}^\top\),则\(\mata\)可逆;
          }
    \item 设\(n\)阶方阵\(\mata\)满足\(\mata^2+\mata-6\mati_n=\mato\),证明:\(\mata\mata+\mati_n\)和\(\mata+4\mati_n\)都可逆,并求它们的逆矩阵。
\end{enumerate}
\end{problem}
\begin{proof}

\end{proof}

% 2.48
\begin{problem}
已知\(\mata,\matb\)为\(3\)阶方阵,且满足\(2\inv{\mata}\matb=\matb-4\mati\)。
\begin{enumerate}
    \item 矩阵\(\mata-2\mati\)可逆;
    \item 若
          \begin{equation*}
              \matb=
              \begin{bmatrix}
                  1 & -2 & 0 \\
                  1 & 2  & 0 \\
                  0 & 0  & 2
              \end{bmatrix}
          \end{equation*}
          求\(\mata\)。
\end{enumerate}
\end{problem}
\begin{proof}

\end{proof}

% 2.49
\begin{problem}
设\(\vecal\)是\(n\)维非零列向量,记\(\mata=\mati_n-\vecal\vecal^\top\)。证明:
\begin{enumerate}
    \item \(\mata^2=\mata\)的充分必要条件是\(\vecal\vecal^\top=1\);
    \item 当\(\vecal^\top\vecal=1\)时,\(\mata\)是不可逆矩阵。
\end{enumerate}
\end{problem}
\begin{proof}

\end{proof}

% 2.50
\begin{problem}
设\(\matp\)是\(m\times n\)矩阵,且\(\matp\matp^\top\)可逆。记\(\mata=\mati_n-\matp^\top\pinv{\matp\matp^\top}\matp\),证明:\(\mata\)是对称矩阵,且\(\mata^2=\mata\)。
\end{problem}
\begin{proof}

\end{proof}

% 2.51
\begin{problem}
证明下列矩阵是正交阵:
\begin{enumerate}
    \item \(\begin{bmatrix}
              1/\sqrt{3}  & 0           & 2/\sqrt{6} \\
              -1/\sqrt{3} & 1/\sqrt{2}  & 1/\sqrt{6} \\
              -1/\sqrt{3} & -1/\sqrt{2} & 1/\sqrt{6}
          \end{bmatrix}\);
    \item \(\begin{bmatrix}
              1/2 & -1/2 & 1/2  & 1/2  \\
              1/2 & 1/2  & -1/2 & 1/2  \\
              1/2 & -1/2 & -1/2 & -1/2 \\
              1/2 & 1/2  & 1/2  & -1/2
          \end{bmatrix}\)
\end{enumerate}
\end{problem}
\begin{proof}

\end{proof}

% 2.52
\begin{problem}
设\(\mata\)为\(n\)阶对角阵,且为正交矩阵,求\(\mata\)。
\end{problem}
\begin{proof}

\end{proof}