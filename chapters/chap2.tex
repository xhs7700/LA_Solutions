\section{线性方程组}

% 2.1
\begin{problem}
设\(\mata\)为\(m\times n\)矩阵,判断下列命题是否正确。
\begin{enumerate}
    \item 非齐次线性方程组\(\mata\vecx=\vecb\),当\(m<n\)时,有无穷多解;当\(m=n\)时,有唯一解;当\(m>n\)时,无解;
    \item 齐次线性方程组\(\mata\vecx=\veczero\),当\(m<n\)时,必有非零解;
    \item 非齐次线性方程组\(\mata\vecx=\vecb\),当\(\rank{\mata}=m\),则一定有解。
\end{enumerate}
\end{problem}
\begin{proof}

\end{proof}

% 2.2
\begin{problem}
解下列线性方程组:
\begin{enumerate}
    \item \begin{equation*}
              \begin{cases}
                  x_1-2x_2+3x_3-4x_4=0 \\
                  x_2-x_3+x_4=0        \\
                  x_1+3x_2-3x_4=0      \\
                  x_1-4x_2+3x_3-2x_4=0
              \end{cases}
          \end{equation*}
    \item \begin{equation*}
              \begin{cases}
                  x_1+x_2+x_3+x_4+x_5=0    \\
                  3x_1+2x_2+x_3+x_4-3x_5=0 \\
                  x_2+2x_3+2x_4+6x_5=0     \\
                  5x_1+4x_2+3x_3+3x_4-x_5=0
              \end{cases}
          \end{equation*}
    \item \begin{equation*}
              \begin{cases}
                  x_1+2x_2+x_3-x_4=6         \\
                  2x_1-x_2+x_3+3x_4+4x_5=-7  \\
                  2x_1-x_2+2x_3+x_4-2x_5=-4  \\
                  2x_1-3x_2+x_3+2x_4-2x_5=-9 \\
                  x_1+x_3-2x_4-6x_5=4
              \end{cases}
          \end{equation*}
    \item \begin{equation*}
              \begin{cases}
                  2x_1-x_2-2x_3+x_4=0 \\
                  x_1+2x_2+2x_3+x_4=6 \\
                  3x_1+x_2-x_3-2x_4=1 \\
                  x_1+2x_2+x_3-3x_4=2 \\
                  2x_1+4x_2+3x_3-2x_4=7
              \end{cases}
          \end{equation*}
\end{enumerate}
\end{problem}
\begin{proof}

\end{proof}

% 2.3
\begin{problem}
求线性方程组
\begin{equation*}
    \begin{cases}
        x_1-x_2+2x_3+x_4=1  \\
        2x_1-x_2+x_3+2x_4=3 \\
        x_1-x_3+x_4=2       \\
        3x_1-x_2+3x_4=5
    \end{cases}
\end{equation*}
的通解,并求出满足\(x_1^2=x_2^2\)的全部解。
\end{problem}
\begin{proof}

\end{proof}

% 2.4(2)
\begin{problem}
讨论下列线性方程组,当\(\lambda\)取什么值时方程组有唯一解?取什么值时有无穷多解?取什么值时无解?
\begin{enumerate}
    \item \begin{equation*}
              \begin{cases}
                  \mypar{\lambda+3}x_1+x_2+2x_3=\lambda        \\
                  \lambda x_1+\mypar{\lambda-1}x_2+x_3=\lambda \\
                  3\mypar{\lambda+1}x_1+\lambda x_1+\mypar{\lambda+3}x_3=3
              \end{cases}
          \end{equation*}
    \item
          {
          \begin{equation*}
              \begin{cases}
                  x_1+x_2+\lambda x_3=2 \\
                  2x_1+\lambda x_2+8x_3=\lambda
              \end{cases}
          \end{equation*}
          }
    \item \begin{equation*}
              \begin{cases}
                  x_1+x_2+\lambda x_3=2  \\
                  3x_1+4x_2+2x_3=\lambda \\
                  2x_1+3x_2-x_3=3        \\
              \end{cases}
          \end{equation*}
\end{enumerate}
\end{problem}
\begin{proof}

\end{proof}

% 2.5
\begin{problem}
讨论\(a,b\)为何值时,线性方程组
\begin{equation*}
    \begin{cases}
        x_1-x_2+2x_3=1             \\
        2x_1-x_2+3x_3-x_4=4        \\
        x_2+ax_3+bx_4=b            \\
        x_1-3x_2+\mypar{3-a}x_3=-4 \\
    \end{cases}
\end{equation*}
\end{problem}
\begin{proof}

\end{proof}

% 2.6
\begin{problem}
设\(x_1-x_2=a_1\),\(x_2-x_3=a_2\),\(x_3-x_4=a_3\),\(x_4-x_5=a_4\),\(x_5-x_1=a_5\)。证明:这个方程组有解的充分必要条件为
\begin{equation*}
    \sum_{i=1}^5a_i=0
\end{equation*}
并在有解的情况下求出它的一般解。
\end{problem}
\begin{proof}

\end{proof}

% 2.7
\begin{problem}
已知平面上三条不同直线的方程分别为:
\begin{align*}
    l_1: & ax+2by+3c=0 \\
    l_2: & bx+2cy+3a=0 \\
    l_3: & cx+2ay+3b=0
\end{align*}
试证这三条直线交于一点的充要条件是\(a+b+c=0\)。
\end{problem}
\begin{proof}

\end{proof}

% 2.8
\begin{problem}
设有\(n\)元齐次线性方程组(\(n\ge2\))
\begin{equation*}
    \begin{cases}
        \mypar{1+a}x_1+x_2+\cdots+x_n=0   \\
        2x_1+\mypar{2+a}x_2+\cdots+2x_n=0 \\
        \vdots                            \\
        nx_1+nx_2+\cdots+\mypar{n+a}x_n=0
    \end{cases}
\end{equation*}
试问\(a\)取何值时,该方程组有非零解,并求出其通解。
\end{problem}
\begin{proof}

\end{proof}

% 2.9
\begin{problem}
已知非齐次线性方程组
\begin{equation*}
    x_1+x_2+x_3+x_4=-1\\
    4x_1+3x_2+5x_3-x_4=-1\\
    ax_1+x_2+3x_3+bx_4=1
\end{equation*}
有\(3\)个线性无关的解,
\begin{enumerate}
    \item 证明方程组系数矩阵\(\mata\)的秩\(\rank{\mata}=2\);
    \item 求\(a,b\)的值及方程组的通解。
\end{enumerate}
\end{problem}
\begin{proof}

\end{proof}

% 2.10
\begin{problem}
设线性方程组
\begin{equation*}
    \begin{cases}
        x_1+x_2+x_3=0   \\
        x_1+2x_2+ax_3=0 \\
        x_1+4x_2+a^2x_3=0
    \end{cases}
\end{equation*}
与方程\(x_1+2x_2+x_3=a-1\)有公共解,求\(a\)的值及所有公共解。
\end{problem}
\begin{proof}

\end{proof}

% 2.11
\begin{problem}
设线性方程组\(\mata\vecx=\vecb\)存在两个不同的解,其中
\begin{equation*}
    \mata=
    \begin{bmatrix}
        \lambda & 1         & 1       \\
        0       & \lambda-1 & 0       \\
        1       & 1         & \lambda
    \end{bmatrix},
    \vecb=
    \begin{bmatrix}
        a \\1\\1
    \end{bmatrix}
\end{equation*}
\begin{enumerate}
    \item 求\(\lambda,a\)的值;
    \item 求方程组\(\mata\vecx=\vecb\)的通解。
\end{enumerate}
\end{problem}
\begin{proof}

\end{proof}

% 2.12
\begin{problem}
设
\begin{equation*}
    \mata=
    \begin{bmatrix}
        1 & a & 0 & 0 \\
        0 & 1 & a & 0 \\
        0 & 0 & 1 & a \\
        a & 0 & 0 & 1
    \end{bmatrix},
    \vecb=
    \begin{bmatrix}
        1 \\-1\\0\\0
    \end{bmatrix}
\end{equation*}
当实数\(a\)为何值时,方程组\(\mata\vecx=\vecb\)有无穷多解,并求其通解。
\end{problem}
\begin{proof}

\end{proof}

% 2.13
\begin{problem}
已知\(4\)阶矩阵\(\mata=\mat{\vecal_1,\vecal_2,\vecal_3,\vecal_4}\),\(\vecal_1,\vecal_2,\vecal_3,\vecal_4\)均为\(4\)元列向量,其中\(\vecal_2,\vecal_3,\vecal_4\)线性无关,\(\vecal_1=2\vecal_2-\vecal_3\),若\(\vecbeta=\vecal_1+\vecal_2+\vecal_3+\vecal_4\),求线性方程组\(\mata\vecx=\vecbeta\)的通解。
\end{problem}
\begin{proof}

\end{proof}

% 2.14
\begin{problem}
已知\(3\)阶矩阵\(\mata\)的第一行是\(\mat{a,b,c}\),\(a,b,c\)不全为零,
\begin{equation*}
    \matb=
    \begin{bmatrix}
        1 & 2 & 3 \\
        2 & 4 & 6 \\
        3 & 6 & k
    \end{bmatrix}
\end{equation*}
\(k\)为常数,且\(\mata\matb=\mato\),求线性方程组\(\mata\vecx=\veczero\)的通解。
\end{problem}
\begin{proof}

\end{proof}

% 2.15
\begin{problem}
设\(\vecal,\vecbeta\)为三维列向量,矩阵\(\mata=\vecal\vecal^\top+\vecbeta\vecbeta^\top\),证明:
\begin{enumerate}
    \item 秩\(\rank{\mata}\le2\);
    \item 若\(\vecal,\vecbeta\)线性相关,则秩\(\rank{\mata}<2\)。
\end{enumerate}
\end{problem}
\begin{proof}

\end{proof}

% 2.16
\begin{problem}

\end{problem}
\begin{proof}

\end{proof}

% 2.17
\begin{problem}

\end{problem}
\begin{proof}

\end{proof}

% 2.18
\begin{problem}

\end{problem}
\begin{proof}

\end{proof}

% 2.19
\begin{problem}

\end{problem}
\begin{proof}

\end{proof}

% 2.20
\begin{problem}

\end{problem}
\begin{proof}

\end{proof}

% 2.21
\begin{problem}
设\(\mata\)是\(s\times n\)矩阵,\(\matb\)是\(n\times m\)矩阵,\(n<m\),证明:齐次线性方程组\(\mypar{\mata\matb}\vecx=\veczero\)有非零解。
\end{problem}
\begin{proof}

\end{proof}

% 2.22
\begin{problem}
设\(\mata\)是\(m\times s\)矩阵,\(\matb\)是\(s\times n\)矩阵,\(\vecx\)是\(n\)元列向量。证明:若\(\mypar{\mata\matb}\vecx=\veczero\)与\(\matb\vecx=\veczero\)是同解方程组,则\(\rank{\mata\matb}=\rank{\matb}\)。
\end{problem}
\begin{proof}

\end{proof}

% 2.23
\begin{problem}

\end{problem}
\begin{proof}

\end{proof}

% 2.24
\begin{problem}

\end{problem}
\begin{proof}

\end{proof}

% 2.25
\begin{problem}

\end{problem}
\begin{proof}

\end{proof}

% 2.26(3)
\begin{problem}
求下列向量组的秩和极大线性无关组:
\begin{enumerate}
    \item[(3)] \(\vecal_1=\myvec{6,4,1,-1,2}\),\(\vecal_2=\myvec{1,0,2,3,-4}\),\(\vecal_3=\myvec{1,4,-9,-16,22}\),\(\vecal_4=\myvec{7,1,0,-1,3}\)
\end{enumerate}
\end{problem}
\begin{proof}

\end{proof}

% 2.27
\begin{problem}

\end{problem}
\begin{proof}

\end{proof}

% 2.28(1)
\begin{problem}
求下列齐次线性方程组的一个基础解系:
\begin{enumerate}
    \item
          {
          \begin{equation*}
              \begin{cases}
                  x_1-2x_2+4x_3-7x_4=0 \\
                  2x_1+x_2-2x_3+x_4=0  \\
                  3x_1-x_2+2x_3-4x_4=0 \\
              \end{cases}
          \end{equation*}
          }
\end{enumerate}
\end{problem}
\begin{proof}

\end{proof}

% 2.29(1)
\begin{problem}
求下列线性方程组的通解,并表示成列向量线性组合的形式:
\begin{enumerate}
    \item
          {
          \begin{equation*}
              \begin{cases}
                  x_1-2x_2-x_3-x_4+x_5=0    \\
                  2x_1+x_2-x_3+2x_4-3x_5=0  \\
                  3x_1-2x_2-x_3+x_4-2x_5=0  \\
                  2x_1-5x_2+x_3-2x_4+2x_5=0 \\
              \end{cases}
          \end{equation*}
          }
\end{enumerate}
\end{problem}
\begin{proof}

\end{proof}

% 2.30
\begin{problem}
已知:向量组I可用向量组II线性表出,向量组II可用向量组III线性表出。求证:向量组I可用向量组III线性表出。
\end{problem}
\begin{proof}

\end{proof}

% 2.31
\begin{problem}
设向量组\(\enums{\vecal}{n}\)线性无关。证明:当且仅当\(n\)为奇数时,向量组\(\vecal_1+\vecal_2\),\(\vecal_2+\vecal_3\),\(\cdots\),\(\vecal_{n-1}+\vecal_n\),\(\vecal_n+\vecal_1\)也线性无关。
\end{problem}
\begin{proof}

\end{proof}

% 2.32
\begin{problem}
设向量组\(\enums{\vecal}{n}\)线性无关,而向量组\(\enums{\vecal}{n},\vecbeta,\vecgamma\)线性相关。证明:或者\(\vecbeta\)与\(\vecgamma\)中至少有一个可由\(\enums{\vecal}{n}\)线性表出,或者向量组\(\enums{\vecal}{n},\vecbeta\)与向量组\(\enums{\vecal}{n},\vecgamma\)可相互线性表出。
\end{problem}
\begin{proof}

\end{proof}

% 2.33
\begin{problem}
设\(\vecbeta_1=\vecal_2+\vecal_3+\cdots+\vecal_{n}\),\(\vecbeta_2=\vecal_1+\vecal_3+\cdots+\vecal_{n}\),\(\cdots\),\(\vecbeta_n=\vecal_1+\vecal_2+\cdots+\vecal_{n-1}\),其中\(m>1\)。证明:向量组\(\enums{\vecal}{n}\)与向量组\(\enums{\vecbeta}{n}\)可相互线性表出。
\end{problem}
\begin{proof}

\end{proof}

% 2.34
\begin{problem}

\end{problem}
\begin{proof}

\end{proof}

% 2.35
\begin{problem}
给定\(n\)个非零的数\(\enums{a}{n}\),求下面向量组的秩:
\begin{align*}
    \veceta_1 & =\mat{1+a_1,1,\cdots,1} \\
    \veceta_2 & =\mat{1,1+a_2,\cdots,1} \\
              & \vdots                  \\
    \veceta_n & =\mat{1,1,\cdots,1+a_n}
\end{align*}
\end{problem}
\begin{proof}

\end{proof}

% 2.36
\begin{problem}

\end{problem}
\begin{proof}

\end{proof}

% 2.37
\begin{problem}
设\(\enums{\veceta}{t}\)是某一非齐次线性方程组的解。证明:\(\mu_1\veceta_1+\mu_2\veceta_2+\cdots+\mu_t\veceta_t\)也是该非齐次线性方程组的解的充要条件是\(\mu_1+\mu_2+\cdots+\mu_t=1\)。
\end{problem}
\begin{proof}

\end{proof}

% 2.38
\begin{problem}

\end{problem}
\begin{proof}

\end{proof}

% 2.39
\begin{problem}

\end{problem}
\begin{proof}

\end{proof}

% 2.40(1)
\begin{problem}
解下列矩阵方程:
\begin{enumerate}
    \item \(\begin{bmatrix}1&1&1\\0&1&1\\0&0&1\end{bmatrix}\matx=\begin{bmatrix}5&6\\3&4\\1&2\end{bmatrix}\)
\end{enumerate}
\end{problem}
\begin{proof}

\end{proof}

% 2.41
\begin{problem}
设\(\mata,\matb,\matc\)均为\(n\)阶方阵,若\(\mata\matb\matc=\mati\),则下列乘积:\(\mata\matc\matb\),\(\matb\mata\matc\),\(\matb\matc\mata\),\(\matc\mata\matb\),\(\matc\matb\mata\)中哪些必等于单位阵\(\mati\)。
\end{problem}
\begin{proof}

\end{proof}

% 2.42
\begin{problem}

\end{problem}
\begin{proof}

\end{proof}

% 2.43
\begin{problem}

\end{problem}
\begin{proof}

\end{proof}

% 2.44
\begin{problem}
设\(n\)阶可逆矩阵\(\mata\)每行元素之和都等于常数\(c\)。证明:\(c\neq0\),且\(\inv{\mata}\)中每行元素之和都等于\(\inv{c}\)。
\end{problem}
\begin{proof}

\end{proof}

% 2.45
\begin{problem}

\end{problem}
\begin{proof}

\end{proof}

% 2.46
\begin{problem}
设\(\mata,\matb\)均为\(n\)阶可逆矩阵。证明:如果\(\mata+\matb\)可逆,则\(\inv{\mata}+\inv{\matb}\)也可逆,并求其逆矩阵。
\end{problem}
\begin{proof}

\end{proof}

% 2.47(2)
\begin{problem}
\begin{enumerate}
    \item[(2)]
        {
        设\(\mata,\matb\)为\(n\)阶方阵,且\(\mata-\mati_n\)和\(\matb\)可逆。证明:若\(\inv{\mypar{\mata-\mati_n}}=\mypar{\matb-\mati_n}^\top\),则\(\mata\)可逆。
        }
\end{enumerate}
\end{problem}
\begin{proof}

\end{proof}

% 2.48
\begin{problem}

\end{problem}
\begin{proof}

\end{proof}

% 2.49
\begin{problem}

\end{problem}
\begin{proof}

\end{proof}

% 2.50
\begin{problem}

\end{problem}
\begin{proof}

\end{proof}

% 2.51
\begin{problem}

\end{problem}
\begin{proof}

\end{proof}

% 2.52
\begin{problem}

\end{problem}
\begin{proof}

\end{proof}