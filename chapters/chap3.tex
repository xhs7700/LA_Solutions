\section{行列式}

% 3.1
\begin{problem}
根据行列式定义,计算
\begin{equation*}
    f\mypar{x}=
    \begin{vmatrix}
        2x & x & 1 & 2  \\
        1  & x & 1 & -1 \\
        3  & 2 & x & 1  \\
        1  & 1 & 1 & x
    \end{vmatrix}
\end{equation*}
中\(x^4\)与\(x^3\)的系数。
\end{problem}
\begin{proof}

\end{proof}

% 3.2
\begin{problem}
使用行列式的定义证明:
\begin{equation*}
    D=
    \begin{vmatrix}
        a_1 & a_2 & a_3 & a_4 & a_5 \\
        b_1 & b_2 & b_3 & b_4 & b_5 \\
        c_1 & c_2 & 0   & 0   & 0   \\
        d_1 & d_2 & 0   & 0   & 0   \\
        e_1 & e_2 & 0   & 0   & 0   \\
    \end{vmatrix}=0
\end{equation*}
\end{problem}
\begin{proof}

\end{proof}

% 3.3
\begin{problem}
证明:一个\(n\)阶行列式中等于零的元素个数如果比\(n^2-n\)多,则此行列式必等于零。
\end{problem}
\begin{proof}

\end{proof}

% 3.4
\begin{problem}
通过计算以下行列式证明:奇偶排列各半。
\begin{equation*}
    D=
    \begin{vmatrix}
        1      & 1      & \cdots & 1      \\
        1      & 1      & \cdots & 1      \\
        \vdots & \vdots &        & \vdots \\
        1      & 1      & \cdots & 1
    \end{vmatrix}
\end{equation*}
\end{problem}
\begin{proof}

\end{proof}

% 3.5
\begin{problem}
计算下列行列式的值:
\begin{enumerate}
    \item \(\begin{vmatrix}2&0&0\\4&1&0\\7&3&-2\end{vmatrix}\);
    \item \(\begin{vmatrix}3&0&0\\2&1&1\\1&2&2\end{vmatrix}\);
    \item \(\begin{vmatrix}4&0&2&1\\5&0&4&2\\2&0&3&4\\1&0&2&3\end{vmatrix}\);
    \item \(\begin{vmatrix}1&1&1&3\\0&3&1&1\\0&0&2&2\\-1&-1&-1&2\end{vmatrix}\);
\end{enumerate}
\end{problem}
\begin{proof}

\end{proof}

\setcounter{problem}{6}
% 3.7
\begin{problem}
计算\(n\mypar{n>1}\)阶行列式:
\begin{equation*}
    D_n=
    \begin{vmatrix}
        x      & y      & 0      & \cdots & 0      & 0      \\
        0      & x      & y      & \cdots & 0      & 0      \\
        \vdots & \vdots & \vdots &        & \vdots & \vdots \\
        0      & 0      & 0      & \cdots & x      & y      \\
        y      & 0      & 0      & \cdots & 0      & x
    \end{vmatrix}
\end{equation*}
\end{problem}
\begin{proof}

\end{proof}

\setcounter{problem}{8}
% 3.9
\begin{problem}
计算\(n\)阶行列式:
\begin{equation*}
    D_n=
    \begin{vmatrix}
        x+y    & xy     & 0      & \cdots & 0      & 0      & 0      \\
        1      & x+y    & xy     & \cdots & 0      & 0      & 0      \\
        0      & 1      & x+y    & \cdots & 0      & 0      & 0      \\
        \vdots & \vdots & \vdots &        & \vdots & \vdots & \vdots \\
        0      & 0      & 0      & \cdots & 0      & 1      & x+y
    \end{vmatrix}
\end{equation*}
\end{problem}
\begin{proof}

\end{proof}

\setcounter{problem}{11}
% 3.12
\begin{problem}
计算行列式:
\begin{equation*}
    \begin{vmatrix}
        x      & -1      & 0       & \cdots & 0      & 0      \\
        0      & x       & -1      & \cdots & 0      & 0      \\
        \vdots & \vdots  & \vdots  &        & \vdots & \vdots \\
        0      & 0       & 0       & \cdots & x      & -1     \\
        a_n    & a_{n-1} & a_{n-2} & \cdots & a_2    & a_1+x
    \end{vmatrix}
\end{equation*}
\end{problem}
\begin{proof}

\end{proof}

% 3.13
\begin{problem}
计算\(n\)阶行列式:
\begin{equation*}
    \begin{vmatrix}
        1      & 2      & 3      & \cdots & n      \\
        2      & 3      & 4      & \cdots & 1      \\
        3      & 4      & 5      & \cdots & 2      \\
        \vdots & \vdots & \vdots &        & \vdots \\
        n      & 1      & 2      & \cdots & n-1
    \end{vmatrix}
\end{equation*}
\end{problem}
\begin{proof}

\end{proof}

% 3.14
\begin{problem}
计算\(n\)阶行列式:
\begin{equation*}
    \begin{vmatrix}
        x_{1}+a & a       & \cdots & a         & a      \\
        a       & x_{2}+a & \cdots & a         & a      \\
        \vdots  & \vdots  &        & \vdots    & \vdots \\
        a       & a       & \cdots & x_{n-1}+a & a      \\
        a       & a       & \cdots & a         & a
    \end{vmatrix}
\end{equation*}
\end{problem}
\begin{proof}

\end{proof}

\setcounter{problem}{15}
% 3.16
\begin{problem}
计算行列式:
\begin{equation*}
    \begin{vmatrix}
        1 & 1  & 1  & 1  \\
        1 & 1  & -1 & -1 \\
        1 & -1 & 1  & -1 \\
        1 & -1 & -1 & 1
    \end{vmatrix}
\end{equation*}
\end{problem}
\begin{proof}

\end{proof}

% 3.17
\begin{problem}
计算行列式:
\begin{equation*}
    D_n=
    \begin{vmatrix}
        a_{1}+b_{1} & a_{1}+b_{2} & \cdots & a_{1}+b_{n} \\
        a_{2}+b_{1} & a_{2}+b_{2} & \cdots & a_{2}+b_{n} \\
        \vdots      & \vdots      &        & \vdots      \\
        a_{n}+b_{1} & a_{n}+b_{2} & \cdots & a_{n}+b_{n}
    \end{vmatrix}
\end{equation*}
\end{problem}
\begin{proof}

\end{proof}

\setcounter{problem}{20}
% 3.21
\begin{problem}
设分块\(n\)阶方阵\(\matm=\begin{bmatrix}\mata&\matc\\\mato&\matb\end{bmatrix}\),其中\(\mata\)为\(k\)阶方阵,证明:\(\det{\matm}=\det{\mata}\det{\matb}\)。
\end{problem}
\begin{proof}

\end{proof}

\setcounter{problem}{22}
% 3.23
\begin{problem}
计算行列式(设\(n>2\)):
\begin{equation*}
    \begin{vmatrix}
        \sin 2\alpha_1                    & \sin\mypar{\alpha_{1}+\alpha_{2}} & \cdots & \sin\mypar{\alpha_{1}+\alpha_{n}} \\
        \sin\mypar{\alpha_{2}+\alpha_{1}} & \sin 2\alpha_2                    & \cdots & \sin\mypar{\alpha_{2}+\alpha_{n}} \\
        \vdots                            & \vdots                            &        & \vdots                            \\
        \sin\mypar{\alpha_{n}+\alpha_{1}} & \sin\mypar{\alpha_{n}+\alpha_{2}} & \cdots & \sin 2\alpha_n
    \end{vmatrix}
\end{equation*}
\end{problem}
\begin{proof}

\end{proof}

\setcounter{problem}{26}
% 3.27
\begin{problem}
计算\(n\)阶行列式:
\begin{equation*}
    D_n=
    \begin{vmatrix}
        1+x_{1}y_{1} & 1+x_{1}y_{2} & \cdots & 1+x_{1}y_{n} \\
        1+x_{2}y_{1} & 1+x_{2}y_{2} & \cdots & 1+x_{2}y_{n} \\
        \vdots       & \vdots       &        & \vdots       \\
        1+x_{n}y_{1} & 1+x_{n}y_{2} & \cdots & 1+x_{n}y_{n}
    \end{vmatrix}
\end{equation*}
\end{problem}
\begin{proof}

\end{proof}