\documentclass{beamer}
\usepackage{ctex, hyperref}
\usepackage[T1]{fontenc}

% other packages
\usepackage{latexsym,amsmath,xcolor,multicol,booktabs,calligra}
\usepackage{graphicx,pstricks,listings,stackengine,cite}

\setbeamertemplate{bibliography item}[text]
% \bibliographystyle{plain}
% 如果参考文献太多的话,可以像下面这样调整字体:
% \tiny\bibliographystyle{plain}

\author{夏海淞}
\title{线性代数习题课}
\subtitle{第三章}
% \institute{School of Computer Science, Fudan University}
\date{\today}
\usepackage{College}

% defs
\def\cmd#1{\texttt{\color{red}\footnotesize $\backslash$#1}}
\def\env#1{\texttt{\color{blue}\footnotesize #1}}
\definecolor{deepblue}{rgb}{0,0,0.5}
\definecolor{deepred}{rgb}{0.6,0,0}
\definecolor{deepgreen}{rgb}{0,0.5,0}
\definecolor{halfgray}{gray}{0.55}

\renewcommand{\proofname}{解}

\lstset{
    basicstyle=\ttfamily\small,
    keywordstyle=\bfseries\color{deepblue},
    emphstyle=\ttfamily\color{deepred},    % Custom highlighting style
    stringstyle=\color{deepgreen},
    numbers=left,
    numberstyle=\small\color{halfgray},
    rulesepcolor=\color{red!20!green!20!blue!20},
    frame=shadowbox,
}

\begin{document}

\fangsong
\begin{frame}
    \titlepage
    \begin{figure}[htpb]
        \begin{center}
            \vspace*{-0.5cm}
            \includegraphics[width=0.1\linewidth]{pic/FDU.jpeg}
        \end{center}
    \end{figure}

\end{frame}

% \begin{frame}
%     \tableofcontents[sectionstyle=show,subsectionstyle=show/shaded/hide,subsubsectionstyle=show/shaded/hide]
% \end{frame}

\begin{frame}
    \frametitle{3.1}
    \begin{proof}
        \(x^4\)由\((-1)^{\tau(1234)}a_{11}a_{22}a_{33}a_{44}\)得到,所以其系数为\(1\times2\times1\times1\times1=2\)。\\
        \(x^3\)由\((-1)^{\tau(2134)}a_{12}a_{21}a_{33}a_{44}\)得到,所以其系数为\(-1\times1\times1\times1\times1=-1\)。\\
    \end{proof}

\end{frame}

% \begin{frame}[allowframebreaks]
%     \nocite{*}
%     \bibliography{ref}
% \end{frame}

\end{document}