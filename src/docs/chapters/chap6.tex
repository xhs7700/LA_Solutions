\section{矩阵分解}

% 6.1
\begin{problem}
对下列矩阵进行LU分解:
\begin{enumerate}
    \item \(\begin{bmatrix}3&1\\9&5\end{bmatrix}\);
    \item \(\begin{bmatrix}2&4\\-2&1\end{bmatrix}\);
    \item \(\begin{bmatrix}2&4&2\\1&5&2\\4&-1&9\end{bmatrix}\);
    \item \(\begin{bmatrix}2&3&1\\4&1&4\\3&4&6\end{bmatrix}\);
    \item \(\begin{bmatrix}1&1&1\\3&5&6\\-2&2&7\end{bmatrix}\);
    \item \(\begin{bmatrix}-2&1&2\\4&1&-2\\-6&-3&4\end{bmatrix}\)。
\end{enumerate}
\end{problem}
\begin{proof}
\end{proof}

% 6.2
\begin{problem}
对下列矩阵进行QR分解:
\begin{enumerate}
    \item \(\begin{bmatrix}1&2&3\\-1&0&3\\0&-2&3\end{bmatrix}\);
    \item \(\begin{bmatrix}1&2&4\\0&0&5\\0&3&6\end{bmatrix}\);
    \item \(\begin{bmatrix}1&2&-1\\2&0&1\\2&-4&2\\4&0&0\end{bmatrix}\);
    \item \(\begin{bmatrix}1&-1&4\\1&4&-2\\1&4&2\\1&-1&0\end{bmatrix}\)。
\end{enumerate}
\end{problem}
\begin{proof}
\end{proof}

% 6.3
\begin{problem}
对下列矩阵进行Cholesky分解:
\begin{enumerate}
    \item \(\begin{bmatrix}4&2\\2&10\end{bmatrix}\);
    \item \(\begin{bmatrix}9&-3\\-3&2\end{bmatrix}\);
    \item \(\begin{bmatrix}16&8&4\\8&6&0\\4&0&7\end{bmatrix}\);
    \item \(\begin{bmatrix}9&3&-6\\3&4&1\\-6&1&9\end{bmatrix}\);
    \item \(\begin{bmatrix}2&-1&0&0\\-1&2&-1&0\\0&-1&2&-1\\0&0&-1&2\end{bmatrix}\)。
\end{enumerate}
\end{problem}
\begin{proof}
\end{proof}

% 6.4
\begin{problem}
对下列矩阵进行谱分解:
\begin{enumerate}
    \item \(\begin{bmatrix}5&4\\4&5\end{bmatrix}\);
    \item \(\begin{bmatrix}0&2&-1\\2&3&-2\\-1&-2&0\end{bmatrix}\)。
\end{enumerate}
\end{problem}
\begin{proof}
\end{proof}

% 6.5
\begin{problem}
设\(\mata\)是\(m\times n\)矩阵,证明\(\mata^\top\mata\)和\(\mata\mata^\top\)具有相同的非零特征值,且它们的每个非零特征值的代数重数与几何重数均相等。
\end{problem}
\begin{proof}
\end{proof}

% 6.6
\begin{problem}
对下列矩阵进行奇异值分解:
\begin{enumerate}
    \item \(\begin{bmatrix}2&2\\1&1\end{bmatrix}\);
    \item \(\begin{bmatrix}2&2\\-1&1\end{bmatrix}\);
    \item \(\begin{bmatrix}1&1&0\\0&1&1\end{bmatrix}\);
    \item \(\begin{bmatrix}1&1\\1&1\\0&0\end{bmatrix}\)。
\end{enumerate}
\end{problem}
\begin{proof}
\end{proof}

% 6.7
\begin{problem}
设\(m\times n\)矩阵\(\mata\)的秩\(\rank{\mata}=r\),试证明矩阵\(\mata\)可以表示为\(r\)个秩为\(1\)的\(m\times n\)矩阵之和。
\end{problem}
\begin{proof}
\end{proof}

% 6.8
\begin{problem}
设实对称矩阵\(\mata\)正定,证明\(\mata\)的奇异值分解即为\(\mata\)的谱分解。
\end{problem}
\begin{proof}
\end{proof}

% 6.9
\begin{problem}
求下列矩阵的伪逆:
\begin{enumerate}
    \item \(\begin{bmatrix}7&1\\0&0\\5&5\end{bmatrix}\);
    \item \(\begin{bmatrix}3&2&2\\2&3&-2\end{bmatrix}\)。
\end{enumerate}
\end{problem}
\begin{proof}
\end{proof}

% 6.10
\begin{problem}
设线性方程组\(\mata\vecx=\vecb\)无解,系数矩阵\(\mata\)的列向量线性相关,证明\(\hat{\vecx}=\mata^\dagger\vecb\)是\(\mata\vecx=\vecb\)的具有最小模长的最小二乘解,其中\(\mata^\dagger\)是矩阵\(\mata\)的伪逆。
\end{problem}
\begin{proof}
\end{proof}