\section{行列式}

% 3.1
\begin{problem}
根据行列式定义,计算
\begin{equation*}
    f\mypar{x}=
    \begin{vmatrix}
        2x & x & 1 & 2  \\
        1  & x & 1 & -1 \\
        3  & 2 & x & 1  \\
        1  & 1 & 1 & x
    \end{vmatrix}
\end{equation*}
中\(x^4\)与\(x^3\)的系数。
\end{problem}
\begin{proof}
\end{proof}

% 3.2
\begin{problem}
使用行列式的定义证明:
\begin{equation*}
    D=
    \begin{vmatrix}
        a_1 & a_2 & a_3 & a_4 & a_5 \\
        b_1 & b_2 & b_3 & b_4 & b_5 \\
        c_1 & c_2 & 0   & 0   & 0   \\
        d_1 & d_2 & 0   & 0   & 0   \\
        e_1 & e_2 & 0   & 0   & 0   \\
    \end{vmatrix}=0
\end{equation*}
\end{problem}
\begin{proof}
\end{proof}

% 3.3
\begin{problem}
证明:一个\(n\)阶行列式中等于零的元素个数如果比\(n^2-n\)多,则此行列式必等于零。
\end{problem}
\begin{proof}
\end{proof}

% 3.4
\begin{problem}
通过计算以下行列式证明:奇偶排列各半。
\begin{equation*}
    D=
    \begin{vmatrix}
        1      & 1      & \cdots & 1      \\
        1      & 1      & \cdots & 1      \\
        \vdots & \vdots &        & \vdots \\
        1      & 1      & \cdots & 1
    \end{vmatrix}
\end{equation*}
\end{problem}
\begin{proof}
\end{proof}

% 3.5
\begin{problem}
计算下列行列式的值:
\begin{enumerate}
    \item \(\begin{vmatrix}2&0&0\\4&1&0\\7&3&-2\end{vmatrix}\);
    \item \(\begin{vmatrix}3&0&0\\2&1&1\\1&2&2\end{vmatrix}\);
    \item \(\begin{vmatrix}4&0&2&1\\5&0&4&2\\2&0&3&4\\1&0&2&3\end{vmatrix}\);
    \item \(\begin{vmatrix}1&1&1&3\\0&3&1&1\\0&0&2&2\\-1&-1&-1&2\end{vmatrix}\);
\end{enumerate}
\end{problem}
\begin{proof}
\end{proof}

% 3.6
\begin{problem}
计算\(2n\)阶行列式:
\begin{equation*}
    D_{2n}=
    \begin{vmatrix}
        x      & 0      & \cdots & 0      & y      \\
        0      & x      & \cdots & y      & 0      \\
        \vdots & \vdots &        & \vdots & \vdots \\
        0      & y      & \cdots & x      & 0      \\
        y      & 0      & \cdots & 0      & x
    \end{vmatrix}
\end{equation*}
\end{problem}
\begin{proof}
\end{proof}

% 3.7
\begin{problem}
计算\(n\mypar{n>1}\)阶行列式:
\begin{equation*}
    D_n=
    \begin{vmatrix}
        x      & y      & 0      & \cdots & 0      & 0      \\
        0      & x      & y      & \cdots & 0      & 0      \\
        \vdots & \vdots & \vdots &        & \vdots & \vdots \\
        0      & 0      & 0      & \cdots & x      & y      \\
        y      & 0      & 0      & \cdots & 0      & x
    \end{vmatrix}
\end{equation*}
\end{problem}
\begin{proof}
\end{proof}

% 3.8
\begin{problem}
计算\(n\)阶行列式:
\begin{equation*}
    D_n=
    \begin{vmatrix}
        2      & 1      & 0      & \cdots & 0      & 0      & 0      \\
        1      & 2      & 1      & \cdots & 0      & 0      & 0      \\
        \vdots & \vdots & \vdots &        & \vdots & \vdots & \vdots \\
        0      & 0      & 0      & \cdots & 1      & 2      & 1      \\
        0      & 0      & 0      & \cdots & 0      & 1      & 2
    \end{vmatrix}
\end{equation*}
\end{problem}
\begin{proof}
\end{proof}

% 3.9
\begin{problem}
计算\(n\)阶行列式:
\begin{equation*}
    D_n=
    \begin{vmatrix}
        x+y    & xy     & 0      & \cdots & 0      & 0      & 0      \\
        1      & x+y    & xy     & \cdots & 0      & 0      & 0      \\
        0      & 1      & x+y    & \cdots & 0      & 0      & 0      \\
        \vdots & \vdots & \vdots &        & \vdots & \vdots & \vdots \\
        0      & 0      & 0      & \cdots & 0      & 1      & x+y
    \end{vmatrix}
\end{equation*}
\end{problem}
\begin{proof}
\end{proof}

% 3.10
\begin{problem}
计算\(n\)阶行列式:
\begin{equation*}
    D_n=
    \begin{vmatrix}
        x+y    & xy     & 0      & \cdots & 0      & 0      & 0      \\
        1      & x+y    & xy     & \cdots & 0      & 0      & 0      \\
        0      & 1      & x+y    & \cdots & 0      & 0      & 0      \\
        \vdots & \vdots & \vdots &        & \vdots & \vdots & \vdots \\
        0      & 0      & 0      & \cdots & 0      & 1      & x+y
    \end{vmatrix}
\end{equation*}
\end{problem}
\begin{proof}
\end{proof}

% 3.11
\begin{problem}
计算以下\(n\)阶“带”形行列式:
\begin{equation*}
    D_n=
    \begin{vmatrix}
        a      & b      & 0      & \cdots & 0      & 0      & 0      \\
        c      & a      & b      & \cdots & 0      & 0      & 0      \\
        \vdots & \vdots & \vdots &        & \vdots & \vdots & \vdots \\
        0      & 0      & 0      & \cdots & c      & a      & b      \\
        0      & 0      & 0      & \cdots & 0      & c      & a
    \end{vmatrix}
\end{equation*}
\end{problem}
\begin{proof}
\end{proof}

% 3.12
\begin{problem}
计算行列式:
\begin{equation*}
    \begin{vmatrix}
        x      & -1      & 0       & \cdots & 0      & 0      \\
        0      & x       & -1      & \cdots & 0      & 0      \\
        \vdots & \vdots  & \vdots  &        & \vdots & \vdots \\
        0      & 0       & 0       & \cdots & x      & -1     \\
        a_n    & a_{n-1} & a_{n-2} & \cdots & a_2    & a_1+x
    \end{vmatrix}
\end{equation*}
\end{problem}
\begin{proof}
\end{proof}

% 3.13
\begin{problem}
计算\(n\)阶行列式:
\begin{equation*}
    \begin{vmatrix}
        1      & 2      & 3      & \cdots & n      \\
        2      & 3      & 4      & \cdots & 1      \\
        3      & 4      & 5      & \cdots & 2      \\
        \vdots & \vdots & \vdots &        & \vdots \\
        n      & 1      & 2      & \cdots & n-1
    \end{vmatrix}
\end{equation*}
\end{problem}
\begin{proof}
\end{proof}

% 3.14
\begin{problem}
计算\(n\)阶行列式:
\begin{equation*}
    \begin{vmatrix}
        x_{1}+a & a       & \cdots & a         & a      \\
        a       & x_{2}+a & \cdots & a         & a      \\
        \vdots  & \vdots  &        & \vdots    & \vdots \\
        a       & a       & \cdots & x_{n-1}+a & a      \\
        a       & a       & \cdots & a         & a
    \end{vmatrix}
\end{equation*}
\end{problem}
\begin{proof}
\end{proof}

% 3.15
\begin{problem}
计算\(n\)阶行列式:
\begin{equation*}
    \begin{vmatrix}
        x_1+a  & a      & \cdots & a         & a      \\
        a      & x_2+a  & \cdots & a         & a      \\
        \vdots & \vdots &        & \vdots    & \vdots \\
        a      & a      & \cdots & x_{n-1}+a & a      \\
        a      & a      & \cdots & a         & a
    \end{vmatrix}
\end{equation*}
\end{problem}
\begin{proof}
\end{proof}

% 3.16
\begin{problem}
计算行列式:
\begin{equation*}
    \begin{vmatrix}
        1 & 1  & 1  & 1  \\
        1 & 1  & -1 & -1 \\
        1 & -1 & 1  & -1 \\
        1 & -1 & -1 & 1
    \end{vmatrix}
\end{equation*}
\end{problem}
\begin{proof}
\end{proof}

% 3.17
\begin{problem}
计算行列式:
\begin{equation*}
    D_n=
    \begin{vmatrix}
        a_{1}+b_{1} & a_{1}+b_{2} & \cdots & a_{1}+b_{n} \\
        a_{2}+b_{1} & a_{2}+b_{2} & \cdots & a_{2}+b_{n} \\
        \vdots      & \vdots      &        & \vdots      \\
        a_{n}+b_{1} & a_{n}+b_{2} & \cdots & a_{n}+b_{n}
    \end{vmatrix}
\end{equation*}
\end{problem}
\begin{proof}
\end{proof}

% 3.18
\begin{problem}
计算行列式:
\begin{equation*}
    D_n=
    \begin{vmatrix}
        a_1+1  & 1      & \cdots & 1      \\
        1      & a_2+1  & \cdots & 1      \\
        \vdots & \vdots &        & \vdots \\
        1      & 1      & \cdots & a_n+1
    \end{vmatrix}
\end{equation*}
\end{problem}
\begin{proof}
\end{proof}

% 3.19
\begin{problem}
计算行列式:
\begin{equation*}
    D_n=
    \begin{vmatrix}
        a_1    & x      & x      & x      & \cdots & x      \\
        x      & a_2    & x      & x      & \cdots & x      \\
        x      & x      & a_3    & x      & \cdots & x      \\
        \vdots & \vdots & \vdots & \vdots &        & \vdots \\
        x      & x      & x      & x      & \cdots & x      \\
        x      & x      & x      & x      & \cdots & a_n
    \end{vmatrix}
\end{equation*}
\end{problem}
\begin{proof}
\end{proof}

% 3.20
\begin{problem}
计算\(n\)阶行列式:
\begin{equation*}
    \begin{vmatrix}
        x+y    & xy     & 0      & 0      & 0      & \cdots & 0      & 0      & 0      \\
        1      & x+y    & xy     & 0      & 0      & \cdots & 0      & 0      & 0      \\
        0      & 1      & x+y    & xy     & 0      & \cdots & 0      & 0      & 0      \\
        \vdots & \vdots & \vdots & \vdots & \vdots &        & \vdots & \vdots & \vdots \\
        0      & 0      & 0      & 0      & 0      & \cdots & 1      & x+y    & xy
    \end{vmatrix}
\end{equation*}
\end{problem}
\begin{proof}
\end{proof}

% 3.21
\begin{problem}
设分块\(n\)阶方阵\(\matm=\begin{bmatrix}\mata&\matc\\\mato&\matb\end{bmatrix}\),其中\(\mata\)为\(k\)阶方阵,证明:\(\det{\matm}=\det{\mata}\det{\matb}\)。
\end{problem}
\begin{proof}
\end{proof}

% 3.22
\begin{problem}
计算\(2n\)阶行列式:
\begin{equation*}
    D_{2n}=
    \begin{vmatrix}
        a_1 &     &                                          &     &     &                                          &     & c_1 \\
            & a_2 &                                          &     &     &                                          & c_2 &     \\
            &     & \ddots                                   &     &     & \begin{sideways}\(\ddots\)\end{sideways} &     &     \\
            &     &                                          & a_n & c_n &                                          &     &     \\
            &     &                                          & d_n & b_n &                                          &     &     \\
            &     & \begin{sideways}\(\ddots\)\end{sideways} &     &     & \ddots                                   &     &     \\
            & d_2 &                                          &     &     &                                          & b_2 &     \\
        d_1 &     &                                          &     &     &                                          &     & b_1
    \end{vmatrix}
\end{equation*}
\end{problem}
\begin{proof}
\end{proof}

% 3.23
\begin{problem}
计算行列式(设\(n>2\)):
\begin{equation*}
    \begin{vmatrix}
        \sin 2\alpha_1                    & \sin\mypar{\alpha_{1}+\alpha_{2}} & \cdots & \sin\mypar{\alpha_{1}+\alpha_{n}} \\
        \sin\mypar{\alpha_{2}+\alpha_{1}} & \sin 2\alpha_2                    & \cdots & \sin\mypar{\alpha_{2}+\alpha_{n}} \\
        \vdots                            & \vdots                            &        & \vdots                            \\
        \sin\mypar{\alpha_{n}+\alpha_{1}} & \sin\mypar{\alpha_{n}+\alpha_{2}} & \cdots & \sin 2\alpha_n
    \end{vmatrix}
\end{equation*}
\end{problem}
\begin{proof}
\end{proof}

% 3.24
\begin{problem}
证明\(n\)阶行列式:
\begin{equation*}
    \begin{vmatrix}
        \cos\alpha & 1           & 0           & \cdots & 0           & 0           \\
        1          & 2\cos\alpha & 1           & \cdots & 0           & 0           \\
        0          & 1           & 2\cos\alpha & \cdots & 0           & 0           \\
        \vdots     & \vdots      & \vdots      &        & \vdots      & \vdots      \\
        0          & 0           & 0           & \cdots & 2\cos\alpha & 1           \\
        0          & 0           & 0           & \cdots & 1           & 2\cos\alpha
    \end{vmatrix}=\cos n\alpha
\end{equation*}
\end{problem}
\begin{proof}
\end{proof}

% 3.25
\begin{problem}
计算以下\(n\)阶行列式:
\begin{equation*}
    D_n=
    \begin{vmatrix}
        a^{}+x_1^{}   & a^{}+x_2^{}   & \cdots & a^{}+x_n^{}   \\
        a^{2}+x_1^{2} & a^{2}+x_2^{2} & \cdots & a^{2}+x_n^{2} \\
        \vdots        & \vdots        &        & \vdots        \\
        a^{n}+x_1^{n} & a^{n}+x_2^{n} & \cdots & a^{n}+x_n^{n}
    \end{vmatrix}
\end{equation*}
\end{problem}
\begin{proof}
\end{proof}

% 3.26
\begin{problem}
计算以下\(n\)阶行列式:
\begin{equation*}
    D_n=
    \begin{vmatrix}
        a      & b      & b      & \cdots & b      \\
        c      & a      & b      & \cdots & b      \\
        c      & c      & a      & \cdots & b      \\
        \vdots & \vdots & \vdots &        & \vdots \\
        c      & c      & c      & \cdots & a
    \end{vmatrix}
\end{equation*}
\end{problem}
\begin{proof}
\end{proof}

% 3.27
\begin{problem}
计算\(n\)阶行列式:
\begin{equation*}
    D_n=
    \begin{vmatrix}
        1+x_{1}y_{1} & 1+x_{1}y_{2} & \cdots & 1+x_{1}y_{n} \\
        1+x_{2}y_{1} & 1+x_{2}y_{2} & \cdots & 1+x_{2}y_{n} \\
        \vdots       & \vdots       &        & \vdots       \\
        1+x_{n}y_{1} & 1+x_{n}y_{2} & \cdots & 1+x_{n}y_{n}
    \end{vmatrix}
\end{equation*}
\end{problem}
\begin{proof}
\end{proof}

% 3.28
\begin{problem}
利用伴随矩阵求以下矩阵的逆矩阵:
\begin{enumerate}
    \item \(\mata=\begin{bmatrix}
              1 & 1  & -1 \\
              2 & 1  & 0  \\
              1 & -1 & 0
          \end{bmatrix}\);
    \item \(\matb=\begin{bmatrix}
              2  & 2  & 3 \\
              1  & -1 & 0 \\
              -1 & 2  & 1
          \end{bmatrix}\);
    \item \(\matc=\begin{bmatrix}
              1 & 0 & -1 & 0  \\
              0 & 1 & 0  & 0  \\
              0 & 0 & -1 & 1  \\
              0 & 0 & 0  & -1
          \end{bmatrix}\)
\end{enumerate}
\end{problem}
\begin{proof}
\end{proof}

% 3.29
\begin{problem}
设\(\mata\)为\(n\)阶方阵,\(\mata^*\)为\(\mata\)的伴随矩阵。已知\(\abs{\mata}=3\),求\(\abs{\inv{\mypar{\frac{1}{4}\mata}}-2\mata^*}\)。
\end{problem}
\begin{proof}
\end{proof}

% 3.30
\begin{problem}
设矩阵\(\mata\)的伴随矩阵
\begin{equation*}
    \mata^*=
    \begin{bmatrix}
        1 & 0  & 0 & 0 \\
        0 & 1  & 0 & 0 \\
        1 & 0  & 1 & 0 \\
        0 & -3 & 0 & 8
    \end{bmatrix}
\end{equation*}
且\(\mata\matb\inv{\mata}=\matb\inv{\mata}+3\mati\),其中\(\mati\)为\(4\)阶单位矩阵,求矩阵\(\matb\)。
\end{problem}
\begin{proof}
\end{proof}

% 3.31
\begin{problem}
设\(\mata\)是\(n\)阶可逆矩阵,证明:
\begin{enumerate}
    \item \(\det{\mata^*}=\mypar{\det{\mata}}^{n-1}\);
    \item \(\pinv{\mata^*}=\mypar{\inv{\mata}}^*\);
    \item \(\mypar{\mata^*}^\top=\mypar{\mata^\top}^*\)。
\end{enumerate}
\end{problem}
\begin{proof}
\end{proof}

% 3.32
\begin{problem}
设\(\enums{\vecal}{s}\)是线性无关向量组,而
\begin{equation*}
    \vecbeta_i=\sum_{j=1}^sa_{ij}\vecal_j,i=1,2,\cdots,s
\end{equation*}
证明\(\enums{\vecbeta}{s}\)线性无关的充分必要条件是下面\(s\)阶行列式
\begin{equation*}
    \begin{vmatrix}
        a_{11} & a_{12} & \cdots & a_{1s} \\
        a_{21} & a_{22} & \cdots & a_{2s} \\
        \vdots & \vdots &        & \vdots \\
        a_{s1} & a_{s2} & \cdots & a_{ss}
    \end{vmatrix}\neq0
\end{equation*}
\end{problem}
\begin{proof}
\end{proof}

% 3.33
\begin{problem}
设\(a_{ij}\mypar{t}\)为自变量\(t\)的实连续函数,\(i,j=1,2,\cdots,n\)。证明:
\begin{equation*}
    \fracdif{t}
    \begin{vmatrix}
        a_{11}\mypar{t} & \cdots & a_{1n}\mypar{t} \\
        \vdots          &        & \vdots          \\
        a_{n1}\mypar{t} & \cdots & a_{nn}\mypar{t}
    \end{vmatrix}=\sum_{j=1}^n
    \begin{vmatrix}
        a_{11}\mypar{t} & \cdots & \fracdif{t}a_{1j}\mypar{t} & \cdots & a_{1n}\mypar{t} \\
        \vdots          &        & \vdots                     &        & \vdots          \\
        a_{n1}\mypar{t} & \cdots & \fracdif{t}a_{nj}\mypar{t} & \cdots & a_{nn}\mypar{t}
    \end{vmatrix}
\end{equation*}
\end{problem}
\begin{proof}
\end{proof}

% 3.34
\begin{problem}
利用子矩阵的行列式计算下列矩阵的秩:
\begin{enumerate}
    \item \(\begin{bmatrix}
              17 & 18 & 40 & 10 \\
              3  & 7  & 17 & 1  \\
              1  & 4  & 10 & 0  \\
              7  & 8  & 18 & 4
          \end{bmatrix}\);
    \item \(\begin{bmatrix}
              3 & -5 & 4 & 2 & 7 \\
              2 & 0  & 3 & 1 & 4 \\
              1 & 5  & 2 & 0 & 1
          \end{bmatrix}\)。
\end{enumerate}
\end{problem}
\begin{proof}
\end{proof}