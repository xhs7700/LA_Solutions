\section{行列式}

% 1 2 3 4 5 7 9 12 13 14 16 17 21 23 27

% 3.1
\begin{problem}
根据行列式定义,计算
\begin{equation*}
    f(x)=
    \begin{vmatrix}
        2x & x & 1 & 2  \\
        1  & x & 1 & -1 \\
        3  & 2 & x & 1  \\
        1  & 1 & 1 & x
    \end{vmatrix}
\end{equation*}
中\(x^4\)与\(x^3\)的系数。
\end{problem}
\begin{proof}
    \(x^4\)的系数为\((-1)^{\tau(1234)}\times2\times1\times1\times1=2\);

    \(x^3\)的系数为\((-1)^{\tau(2134)}\times1\times1\times1\times1=-1\)。
\end{proof}

% 3.2
\begin{problem}
使用行列式的定义证明:
\begin{equation*}
    D=
    \begin{vmatrix}
        a_1 & a_2 & a_3 & a_4 & a_5 \\
        b_1 & b_2 & b_3 & b_4 & b_5 \\
        c_1 & c_2 & 0   & 0   & 0   \\
        d_1 & d_2 & 0   & 0   & 0   \\
        e_1 & e_2 & 0   & 0   & 0   \\
    \end{vmatrix}=0
\end{equation*}
\end{problem}
\begin{proof}
    记\(c_i=d_i=e_i=0\),\(i\in\setof{3,4,5}\)。则由行列式定义可知
    \begin{equation*}
        D=\sum_{i_1i_2i_3i_4i_5}(-1)^{\tau(i_1i_2i_3i_4i_5)}a_{i_1}b_{i_2}c_{i_3}d_{i_4}e_{i_5}
    \end{equation*}
    因为\(i_1i_2i_3i_4i_5\)是长度为\(5\)的排列,\(\setof{i_3i_4i_5}\cap\setof{3,4,5}\neq\varnothing\)。

    所以\(a_{i_1}b_{i_2}c_{i_3}d_{i_4}e_{i_5}=0\)恒成立,即\(D=0\)。
\end{proof}

% 3.3
\begin{problem}
证明:一个\(n\)阶行列式中等于零的元素个数如果比\(n^2-n\)多,则此行列式必等于零。
\end{problem}
\begin{proof}
    由题设可知该\(n\)阶行列式中非零元素个数小于\(n^2-\spar{n^2-n}=n\)。记该行列式\(i\)行\(j\)列元素为\(a_{ij}\),则对任意\(n\)阶排列\(i_1i_2\cdots i_n\),均满足\(\prod_{j=1}^{n}a_{i_jj}=0\)。由行列式定义可知该行列式为\(0\)。
\end{proof}

% 3.4
\begin{problem}
通过计算以下行列式证明:奇偶排列各半。
\begin{equation*}
    D=
    \begin{vmatrix}
        1      & 1      & \cdots & 1      \\
        1      & 1      & \cdots & 1      \\
        \vdots & \vdots &        & \vdots \\
        1      & 1      & \cdots & 1
    \end{vmatrix}
\end{equation*}
\end{problem}
\begin{proof}
    注意到\(D\)中存在两行(列)元素相同,由推论3.2.1可知\(D=0\)。将\(D\)按定义展开:
    \begin{equation*}
        D=\sum_{i_1i_2\cdots i_n}(-1)^{\tau(i_1i_2\cdots i_n)}=0
    \end{equation*}
    当\(i_1i_2\cdots i_n\)为奇排列时,\((-1)^{\tau(i_1i_2\cdots i_n)}=-1\);当\(i_1i_2\cdots i_n\)为偶排列时,\((-1)^{\tau(i_1i_2\cdots i_n)}=1\)。因此可知奇偶排列各半。
\end{proof}

% 3.5
\begin{problem}
计算下列行列式的值:
\begin{enumerate}
    \item \(\begin{vmatrix}2&0&0\\4&1&0\\7&3&-2\end{vmatrix}\);
    \item \(\begin{vmatrix}3&0&0\\2&1&1\\1&2&2\end{vmatrix}\);
    \item \(\begin{vmatrix}4&0&2&1\\5&0&4&2\\2&0&3&4\\1&0&2&3\end{vmatrix}\);
    \item \(\begin{vmatrix}1&1&1&3\\0&3&1&1\\0&0&2&2\\-1&-1&-1&2\end{vmatrix}\);
\end{enumerate}
\end{problem}
\begin{proof}
    \begin{equation*}
        \begin{array}[2]{ll}
            \begin{vmatrix}2&0&0\\4&1&0\\7&3&-2\end{vmatrix}=-4,              & \begin{vmatrix}3&0&0\\2&1&1\\1&2&2\end{vmatrix}=0,                    \\
            \begin{vmatrix}4&0&2&1\\5&0&4&2\\2&0&3&4\\1&0&2&3\end{vmatrix}=0, & \begin{vmatrix}1&1&1&3\\0&3&1&1\\0&0&2&2\\-1&-1&-1&2\end{vmatrix}=30.
        \end{array}
    \end{equation*}
\end{proof}

% 3.6
\begin{problem}
计算\(2n\)阶行列式:
\begin{equation*}
    D_{2n}=
    \begin{vmatrix}
        x      & 0      & \cdots & 0      & y      \\
        0      & x      & \cdots & y      & 0      \\
        \vdots & \vdots &        & \vdots & \vdots \\
        0      & y      & \cdots & x      & 0      \\
        y      & 0      & \cdots & 0      & x
    \end{vmatrix}
\end{equation*}
\end{problem}
\begin{proof}
    将行列式按第一行展开,可得
    \begin{align*}
        D_{2n} & =x
        \begin{vmatrix}
            x      & \cdots & y      & 0      \\
            \vdots &        & \vdots & \vdots \\
            y      & \cdots & x      & 0      \\
            0      & \cdots & 0      & x
        \end{vmatrix}-y
        \begin{vmatrix}
            0      & x      & \cdots & y      \\
            \vdots & \vdots &        & \vdots \\
            0      & y      & \cdots & x      \\
            y      & 0      & \cdots & 0
        \end{vmatrix}     \\
               & =x^2D_{2(n-1)}-y^2D_{2(n-1)} \\
               & =\spar{x^2-y^2}D_{2(n-1)}.
    \end{align*}
    又因为\(D_2=\begin{vmatrix}x&y\\y&x\end{vmatrix}=x^2-y^2\),所以有
    \begin{equation*}
        D_{2n}=\spar{x^2-y^2}^{n}.
    \end{equation*}
\end{proof}

% 3.7
\begin{problem}
计算\(n\spar{n>1}\)阶行列式:
\begin{equation*}
    D_n=
    \begin{vmatrix}
        x      & y      & 0      & \cdots & 0      & 0      \\
        0      & x      & y      & \cdots & 0      & 0      \\
        \vdots & \vdots & \vdots &        & \vdots & \vdots \\
        0      & 0      & 0      & \cdots & x      & y      \\
        y      & 0      & 0      & \cdots & 0      & x
    \end{vmatrix}
\end{equation*}
\end{problem}
\begin{proof}
    将行列式按第一列展开,可得
    \begin{align*}
        D_n & =x
        \begin{vmatrix}
            x      & y      & \cdots & 0      & 0      \\
            \vdots & \vdots &        & \vdots & \vdots \\
            0      & 0      & \cdots & x      & y      \\
            0      & 0      & \cdots & 0      & x
        \end{vmatrix}+(-1)^{n+1}y
        \begin{vmatrix}
            y      & 0      & \cdots & 0      & 0      \\
            x      & y      & \cdots & 0      & 0      \\
            \vdots & \vdots &        & \vdots & \vdots \\
            0      & 0      & \cdots & x      & y
        \end{vmatrix} \\
            & =x^n+(-1)^{n+1}y^n.
    \end{align*}
\end{proof}

% 3.8
\begin{problem}
计算\(n\)阶行列式:
\begin{equation*}
    D_n=
    \begin{vmatrix}
        2      & 1      & 0      & \cdots & 0      & 0      & 0      \\
        1      & 2      & 1      & \cdots & 0      & 0      & 0      \\
        \vdots & \vdots & \vdots &        & \vdots & \vdots & \vdots \\
        0      & 0      & 0      & \cdots & 1      & 2      & 1      \\
        0      & 0      & 0      & \cdots & 0      & 1      & 2
    \end{vmatrix}
\end{equation*}
\end{problem}
\begin{proof}
\end{proof}

% 3.9
\begin{problem}
计算\(n\)阶行列式:
\begin{equation*}
    D_n=
    \begin{vmatrix}
        x+y    & xy     & 0      & \cdots & 0      & 0      & 0      \\
        1      & x+y    & xy     & \cdots & 0      & 0      & 0      \\
        0      & 1      & x+y    & \cdots & 0      & 0      & 0      \\
        \vdots & \vdots & \vdots &        & \vdots & \vdots & \vdots \\
        0      & 0      & 0      & \cdots & 0      & 1      & x+y
    \end{vmatrix}
\end{equation*}
\end{problem}
\begin{proof}
    将\(D_n\)按第一行展开,后项按第一列展开后,得
    \begin{equation*}
        D_n=\spar{x+y}D_{n-1}-xyD_{n-2}.
    \end{equation*}
    该递推式的特征方程为\(\lambda^2-\spar{x+y}\lambda+xy=0\),根为\(\lambda_1=x\),\(\lambda_2=y\)。

    当\(x=y\)时,递推式的通解为\(D_n=c_1x^n+c_2nx^n\)。代入初值解得\(D_n=\spar{1+n}x^n\);

    当\(x\neq y\)时,递推式的通解为\(D_n=c_1x^n+c_2y^n\)。代入初值解得
    \begin{equation*}
        D_n=\frac{x}{x-y}x^n-\frac{y}{x-y}y^n=\sum_{i=0}^{n}x^{n-i}y_i.
    \end{equation*}
    综上,\(D_n=\sum_{i=0}^{n}x^{n-i}y^i\)。
\end{proof}

% 3.10
\begin{problem}
计算\(n\)阶行列式:
\begin{equation*}
    D_n=
    \begin{vmatrix}
        x+y    & xy     & 0      & \cdots & 0      & 0      & 0      \\
        1      & x+y    & xy     & \cdots & 0      & 0      & 0      \\
        0      & 1      & x+y    & \cdots & 0      & 0      & 0      \\
        \vdots & \vdots & \vdots &        & \vdots & \vdots & \vdots \\
        0      & 0      & 0      & \cdots & 0      & 1      & x+y
    \end{vmatrix}
\end{equation*}
\end{problem}
\begin{proof}
\end{proof}

% 3.11
\begin{problem}
计算以下\(n\)阶“带”形行列式:
\begin{equation*}
    D_n=
    \begin{vmatrix}
        a      & b      & 0      & \cdots & 0      & 0      & 0      \\
        c      & a      & b      & \cdots & 0      & 0      & 0      \\
        \vdots & \vdots & \vdots &        & \vdots & \vdots & \vdots \\
        0      & 0      & 0      & \cdots & c      & a      & b      \\
        0      & 0      & 0      & \cdots & 0      & c      & a
    \end{vmatrix}
\end{equation*}
\end{problem}
\begin{proof}
\end{proof}

% 3.12
\begin{problem}
计算行列式:
\begin{equation*}
    \begin{vmatrix}
        x      & -1      & 0       & \cdots & 0      & 0      \\
        0      & x       & -1      & \cdots & 0      & 0      \\
        \vdots & \vdots  & \vdots  &        & \vdots & \vdots \\
        0      & 0       & 0       & \cdots & x      & -1     \\
        a_n    & a_{n-1} & a_{n-2} & \cdots & a_2    & a_1+x
    \end{vmatrix}
\end{equation*}
\end{problem}
\begin{proof}
    当\(x\neq0\)时,可利用初等列变换将行列式转化为:
    \begin{align*}
          &
        \begin{vmatrix}
            x      & 0                     & 0       & \cdots & 0      & 0      \\
            0      & x                     & -1      & \cdots & 0      & 0      \\
            \vdots & \vdots                & \vdots  &        & \vdots & \vdots \\
            0      & 0                     & 0       & \cdots & x      & -1     \\
            a_n    & a_{n-1}+\frac{a_n}{x} & a_{n-2} & \cdots & a_2    & a_1+x
        \end{vmatrix}                                                                                                       \\
        = &
        \begin{vmatrix}
            x      & 0                     & 0                                         & \cdots & 0      & 0      \\
            0      & x                     & 0                                         & \cdots & 0      & 0      \\
            \vdots & \vdots                & \vdots                                    &        & \vdots & \vdots \\
            0      & 0                     & 0                                         & \cdots & x      & -1     \\
            a_n    & a_{n-1}+\frac{a_n}{x} & a_{n-2}+\frac{a_{n-1}}{x}+\frac{a_n}{x^2} & \cdots & a_2    & a_1+x
        \end{vmatrix}                                               \\
        = &
        \begin{vmatrix}
            x      & 0      & 0      & \cdots & 0      & 0                                                        \\
            0      & x      & 0      & \cdots & 0      & 0                                                        \\
            \vdots & \vdots & \vdots &        & \vdots & \vdots                                                   \\
            0      & 0      & 0      & \cdots & x      & 0                                                        \\
            a_n    & \cdots & \cdots & \cdots & \cdots & \frac{a_n}{x^{n-1}}+\frac{a_{n-1}}{x^{n-2}}+\cdots+a_1+x
        \end{vmatrix} \\
        = & x^n+\sum_{i=1}^{n}a_ix^{n-i}
    \end{align*}

    当\(x=0\)时,可将行列式按第一列展开:
    \begin{align*}
        \begin{vmatrix}
            0      & -1      & 0       & \cdots & 0      & 0      \\
            0      & 0       & -1      & \cdots & 0      & 0      \\
            \vdots & \vdots  & \vdots  &        & \vdots & \vdots \\
            0      & 0       & 0       & \cdots & 0      & -1     \\
            a_n    & a_{n-1} & a_{n-2} & \cdots & a_2    & a_1
        \end{vmatrix}=(-1)^{n+1}a_n(-1)^{n-1}=a_n
    \end{align*}
    综上,行列式的值为\(x^n+\sum_{i=1}^{n}a_ix^{n-i}\)。
\end{proof}

% 3.13
\begin{problem}
计算\(n\)阶行列式:
\begin{equation*}
    \begin{vmatrix}
        1      & 2      & 3      & \cdots & n      \\
        2      & 3      & 4      & \cdots & 1      \\
        3      & 4      & 5      & \cdots & 2      \\
        \vdots & \vdots & \vdots &        & \vdots \\
        n      & 1      & 2      & \cdots & n-1
    \end{vmatrix}
\end{equation*}
\end{problem}
\begin{proof}
    对该行列式进行初等行变换:
    \begin{align*}
          &
        \begin{vmatrix}
            1      & 2      & 3      & \cdots & n      \\
            2      & 3      & 4      & \cdots & 1      \\
            3      & 4      & 5      & \cdots & 2      \\
            \vdots & \vdots & \vdots &        & \vdots \\
            n      & 1      & 2      & \cdots & n-1
        \end{vmatrix}=
        \begin{vmatrix}
            1      & 2      & 3      & \cdots & n      \\
            1      & 1      & 1      & \cdots & 1-n    \\
            1      & 1      & 1      & \cdots & 1      \\
            \vdots & \vdots & \vdots &        & \vdots \\
            1      & 1-n    & 1      & \cdots & 1
        \end{vmatrix}                                                                           \\
        = &
        \begin{vmatrix}
            1      & 1      & 2      & \cdots & n-1    \\
            1      & 0      & 0      & \cdots & -n     \\
            1      & 0      & 0      & \cdots & 0      \\
            \vdots & \vdots & \vdots &        & \vdots \\
            1      & -n     & 0      & \cdots & 0
        \end{vmatrix}=
        \begin{vmatrix}
            1+\frac{1}{n}\sum_{i=1}^{n-1}i & 1      & 2      & \cdots & n-1    \\
            0                              & 0      & 0      & \cdots & -n     \\
            0                              & 0      & 0      & \cdots & 0      \\
            \vdots                         & \vdots & \vdots &        & \vdots \\
            0                              & -n     & 0      & \cdots & 0
        \end{vmatrix}                                                   \\
        = & \spar{1+\frac{n-1}{2}}(-1)^{\frac{(n-1)(n-2)}{2}(-n)^{n-1}}=\spar{1+\frac{n-1}{2}}(-1)^{\frac{n(n-1)}{2}}n^{n-1}
    \end{align*}
\end{proof}

% 3.14
\begin{problem}
计算\(n\)阶行列式:
\begin{equation*}
    \begin{vmatrix}
        x_{1}+a & a       & \cdots & a         & a      \\
        a       & x_{2}+a & \cdots & a         & a      \\
        \vdots  & \vdots  &        & \vdots    & \vdots \\
        a       & a       & \cdots & x_{n-1}+a & a      \\
        a       & a       & \cdots & a         & a
    \end{vmatrix}
\end{equation*}
\end{problem}
\begin{proof}
    对该行列式进行初等行变换:
    \begin{equation*}
        \begin{vmatrix}
            x_{1}+a & a       & \cdots & a         & a      \\
            a       & x_{2}+a & \cdots & a         & a      \\
            \vdots  & \vdots  &        & \vdots    & \vdots \\
            a       & a       & \cdots & x_{n-1}+a & a      \\
            a       & a       & \cdots & a         & a
        \end{vmatrix}=
        \begin{vmatrix}
            x_{1}  & 0      & \cdots & 0       & 0      \\
            0      & x_{2}  & \cdots & 0       & 0      \\
            \vdots & \vdots &        & \vdots  & \vdots \\
            0      & 0      & \cdots & x_{n-1} & 0      \\
            a      & a      & \cdots & a       & a
        \end{vmatrix}=a\prod_{i=1}^{n-1}x_i
    \end{equation*}
\end{proof}

% 3.15
\begin{problem}
计算\(n\)阶行列式:
\begin{equation*}
    \begin{vmatrix}
        x_1+a  & a      & \cdots & a         & a      \\
        a      & x_2+a  & \cdots & a         & a      \\
        \vdots & \vdots &        & \vdots    & \vdots \\
        a      & a      & \cdots & x_{n-1}+a & a      \\
        a      & a      & \cdots & a         & a
    \end{vmatrix}
\end{equation*}
\end{problem}
\begin{proof}
\end{proof}

% 3.16
\begin{problem}
计算行列式:
\begin{equation*}
    \begin{vmatrix}
        1 & 1  & 1  & 1  \\
        1 & 1  & -1 & -1 \\
        1 & -1 & 1  & -1 \\
        1 & -1 & -1 & 1
    \end{vmatrix}
\end{equation*}
\end{problem}
\begin{proof}
    对行列式反复进行初等行(列)变换和按行(列)展开:
    \begin{align*}
          &
        \begin{vmatrix}
            1 & 1  & 1  & 1  \\
            1 & 1  & -1 & -1 \\
            1 & -1 & 1  & -1 \\
            1 & -1 & -1 & 1
        \end{vmatrix}=
        \begin{vmatrix}
            1 & 1  & 1  & 1  \\
            0 & 0  & -2 & -2 \\
            0 & -2 & 0  & -2 \\
            0 & -2 & -2 & 0
        \end{vmatrix}=
        \begin{vmatrix}
            0  & -2 & -2 \\
            -2 & 0  & -2 \\
            -2 & -2 & 0
        \end{vmatrix} \\
        = & 2
        \begin{vmatrix}
            -2 & -2 \\
            -2 & 0
        \end{vmatrix}-2
        \begin{vmatrix}
            -2 & -2 \\
            0  & -2
        \end{vmatrix}=-16
    \end{align*}
\end{proof}

% 3.17
\begin{problem}
计算行列式:
\begin{equation*}
    D_n=
    \begin{vmatrix}
        a_{1}+b_{1} & a_{1}+b_{2} & \cdots & a_{1}+b_{n} \\
        a_{2}+b_{1} & a_{2}+b_{2} & \cdots & a_{2}+b_{n} \\
        \vdots      & \vdots      &        & \vdots      \\
        a_{n}+b_{1} & a_{n}+b_{2} & \cdots & a_{n}+b_{n}
    \end{vmatrix}
\end{equation*}
\end{problem}
\begin{proof}
    当\(n=1\)时,\(D_n=a_1+b_1\);

    当\(n=2\)时,\(D_n=a_1b_2+a_2b_1-a_1b_1-a_2b_2\);

    当\(n\geq3\)时,
    \begin{align*}
        D_n & =
        \begin{vmatrix}
            a_{1}       & a_{1}       & \cdots & a_{1}       \\
            a_{2}+b_{1} & a_{2}+b_{2} & \cdots & a_{2}+b_{n} \\
            \vdots      & \vdots      &        & \vdots      \\
            a_{n}+b_{1} & a_{n}+b_{2} & \cdots & a_{n}+b_{n}
        \end{vmatrix}+
        \begin{vmatrix}
            b_{1}       & b_{2}       & \cdots & b_{n}       \\
            a_{2}+b_{1} & a_{2}+b_{2} & \cdots & a_{2}+b_{n} \\
            \vdots      & \vdots      &        & \vdots      \\
            a_{n}+b_{1} & a_{n}+b_{2} & \cdots & a_{n}+b_{n}
        \end{vmatrix} \\
            & =
        \begin{vmatrix}
            a_{1}       & a_{1}       & \cdots & a_{1}       \\
            a_{2}+b_{1} & b_{2}-b_{1} & \cdots & b_{n}-b_{1} \\
            \vdots      & \vdots      &        & \vdots      \\
            a_{n}+b_{1} & b_{2}-b_{1} & \cdots & b_{n}-b_{1}
        \end{vmatrix}+
        \begin{vmatrix}
            b_{1}  & b_{2}  & \cdots & b_{n}  \\
            a_{2}  & a_{2}  & \cdots & a_{2}  \\
            \vdots & \vdots &        & \vdots \\
            a_{n}  & a_{n}  & \cdots & a_{n}
        \end{vmatrix}                \\
            & =0+0=0
    \end{align*}
\end{proof}

% 3.18
\begin{problem}
计算行列式:
\begin{equation*}
    D_n=
    \begin{vmatrix}
        a_1+1  & 1      & \cdots & 1      \\
        1      & a_2+1  & \cdots & 1      \\
        \vdots & \vdots &        & \vdots \\
        1      & 1      & \cdots & a_n+1
    \end{vmatrix}
\end{equation*}
\end{problem}
\begin{proof}
\end{proof}

% 3.19
\begin{problem}
计算行列式:
\begin{equation*}
    D_n=
    \begin{vmatrix}
        a_1    & x      & x      & x      & \cdots & x      \\
        x      & a_2    & x      & x      & \cdots & x      \\
        x      & x      & a_3    & x      & \cdots & x      \\
        \vdots & \vdots & \vdots & \vdots &        & \vdots \\
        x      & x      & x      & x      & \cdots & x      \\
        x      & x      & x      & x      & \cdots & a_n
    \end{vmatrix}
\end{equation*}
\end{problem}
\begin{proof}
\end{proof}

% 3.20
\begin{problem}
计算\(n\)阶行列式:
\begin{equation*}
    \begin{vmatrix}
        x+y    & xy     & 0      & 0      & 0      & \cdots & 0      & 0      & 0      \\
        1      & x+y    & xy     & 0      & 0      & \cdots & 0      & 0      & 0      \\
        0      & 1      & x+y    & xy     & 0      & \cdots & 0      & 0      & 0      \\
        \vdots & \vdots & \vdots & \vdots & \vdots &        & \vdots & \vdots & \vdots \\
        0      & 0      & 0      & 0      & 0      & \cdots & 1      & x+y    & xy
    \end{vmatrix}
\end{equation*}
\end{problem}
\begin{proof}
\end{proof}

% 3.21
\begin{problem}
设分块\(n\)阶方阵\(\matm=\begin{bmatrix}\mata&\matc\\\mato&\matb\end{bmatrix}\),其中\(\mata\)为\(k\)阶方阵,证明:\(\det{\matm}=\det{\mata}\det{\matb}\)。
\end{problem}
\begin{proof}

\end{proof}

% 3.22
\begin{problem}
计算\(2n\)阶行列式:
\begin{equation*}
    D_{2n}=
    \begin{vmatrix}
        a_1 &     &                                          &     &     &                                          &     & c_1 \\
            & a_2 &                                          &     &     &                                          & c_2 &     \\
            &     & \ddots                                   &     &     & \begin{sideways}\(\ddots\)\end{sideways} &     &     \\
            &     &                                          & a_n & c_n &                                          &     &     \\
            &     &                                          & d_n & b_n &                                          &     &     \\
            &     & \begin{sideways}\(\ddots\)\end{sideways} &     &     & \ddots                                   &     &     \\
            & d_2 &                                          &     &     &                                          & b_2 &     \\
        d_1 &     &                                          &     &     &                                          &     & b_1
    \end{vmatrix}
\end{equation*}
\end{problem}
\begin{proof}
\end{proof}

% 3.23
\begin{problem}
计算行列式(设\(n>2\)):
\begin{equation*}
    \begin{vmatrix}
        \sin 2\alpha_1                   & \sin\spar{\alpha_{1}+\alpha_{2}} & \cdots & \sin\spar{\alpha_{1}+\alpha_{n}} \\
        \sin\spar{\alpha_{2}+\alpha_{1}} & \sin 2\alpha_2                   & \cdots & \sin\spar{\alpha_{2}+\alpha_{n}} \\
        \vdots                           & \vdots                           &        & \vdots                           \\
        \sin\spar{\alpha_{n}+\alpha_{1}} & \sin\spar{\alpha_{n}+\alpha_{2}} & \cdots & \sin 2\alpha_n
    \end{vmatrix}
\end{equation*}
\end{problem}
\begin{proof}
    因为\(\sin\spar{\alpha_i+\alpha_j}=\sin\alpha_i\cos\alpha_j+\cos\alpha_i\sin\alpha_j\),定义\(n\)阶方阵如下:
    \begin{equation*}
        \mata_n=
        \begin{bmatrix}
            \sin\alpha_{1} & \cos\alpha_{1} & 0 & \cdots & 0 \\
            \sin\alpha_{2} & \cos\alpha_{2} & 0 & \cdots & 0 \\
            \vdots         & \vdots         &   & \vdots     \\
            \sin\alpha_{n} & \cos\alpha_{n} & 0 & \cdots & 0
        \end{bmatrix},
        \matb_n=
        \begin{bmatrix}
            \cos\alpha_{1} & \cos\alpha_{2} & \cdots & \cos\alpha_{n} \\
            \sin\alpha_{1} & \sin\alpha_{2} & \cdots & \sin\alpha_{n} \\
            0              & 0              & \cdots & 0              \\
            \vdots         & \vdots         &        & \vdots         \\
            0              & 0              & \cdots & 0
        \end{bmatrix}
    \end{equation*}
    容易发现\(\mat{\mata_n\matb_n}_{ij}=\sin\spar{\alpha_i+\alpha_j}\)。因此有
    \begin{equation*}
        \det{\mata_n\matb_n}=\det{\mata_n}\det{\matb_n}=0
    \end{equation*}
\end{proof}

% 3.24
\begin{problem}
证明\(n\)阶行列式:
\begin{equation*}
    \begin{vmatrix}
        \cos\alpha & 1           & 0           & \cdots & 0           & 0           \\
        1          & 2\cos\alpha & 1           & \cdots & 0           & 0           \\
        0          & 1           & 2\cos\alpha & \cdots & 0           & 0           \\
        \vdots     & \vdots      & \vdots      &        & \vdots      & \vdots      \\
        0          & 0           & 0           & \cdots & 2\cos\alpha & 1           \\
        0          & 0           & 0           & \cdots & 1           & 2\cos\alpha
    \end{vmatrix}=\cos n\alpha
\end{equation*}
\end{problem}
\begin{proof}
\end{proof}

% 3.25
\begin{problem}
计算以下\(n\)阶行列式:
\begin{equation*}
    D_n=
    \begin{vmatrix}
        a^{}+x_1^{}   & a^{}+x_2^{}   & \cdots & a^{}+x_n^{}   \\
        a^{2}+x_1^{2} & a^{2}+x_2^{2} & \cdots & a^{2}+x_n^{2} \\
        \vdots        & \vdots        &        & \vdots        \\
        a^{n}+x_1^{n} & a^{n}+x_2^{n} & \cdots & a^{n}+x_n^{n}
    \end{vmatrix}
\end{equation*}
\end{problem}
\begin{proof}
\end{proof}

% 3.26
\begin{problem}
计算以下\(n\)阶行列式:
\begin{equation*}
    D_n=
    \begin{vmatrix}
        a      & b      & b      & \cdots & b      \\
        c      & a      & b      & \cdots & b      \\
        c      & c      & a      & \cdots & b      \\
        \vdots & \vdots & \vdots &        & \vdots \\
        c      & c      & c      & \cdots & a
    \end{vmatrix}
\end{equation*}
\end{problem}
\begin{proof}
\end{proof}

% 3.27
\begin{problem}
计算\(n\)阶行列式:
\begin{equation*}
    D_n=
    \begin{vmatrix}
        1+x_{1}y_{1} & 1+x_{1}y_{2} & \cdots & 1+x_{1}y_{n} \\
        1+x_{2}y_{1} & 1+x_{2}y_{2} & \cdots & 1+x_{2}y_{n} \\
        \vdots       & \vdots       &        & \vdots       \\
        1+x_{n}y_{1} & 1+x_{n}y_{2} & \cdots & 1+x_{n}y_{n}
    \end{vmatrix}
\end{equation*}
\end{problem}
\begin{proof}
    当\(n=1\)时,\(D_n=1+x_1y_1\);

    当\(n=2\)时,\(D_n=x_1y_1+x_2y_2-x_1y_2-x_2y_1\);

    当\(n\geq3\)时,定义\(n\)阶方阵如下:
    \begin{equation*}
        \matx_n=
        \begin{bmatrix}
            x_{1}  & 1      & 0      & \cdots & 0      \\
            x_{2}  & 1      & 0      & \cdots & 0      \\
            \vdots & \vdots & \vdots &        & \vdots \\
            x_{n}  & 1      & 0      & \cdots & 0
        \end{bmatrix},
        \maty_n=
        \begin{bmatrix}
            y_{1}  & y_{2}  & \cdots & y_{n}  \\
            1      & 1      & \cdots & 1      \\
            0      & 0      & \cdots & 0      \\
            \vdots & \vdots &        & \vdots \\
            0      & 0      & \cdots & 0
        \end{bmatrix}
    \end{equation*}
    容易发现\(\mat{\matx_n\maty_n}_{ij}=x_iy_j+1\)。因此有
    \begin{equation*}
        \det{\matx_n\maty_n}=\det{\matx_n}\det{\maty_n}=0
    \end{equation*}
\end{proof}

% 3.28
\begin{problem}
利用伴随矩阵求以下矩阵的逆矩阵:
\begin{enumerate}
    \item \(\mata=\begin{bmatrix}
              1 & 1  & -1 \\
              2 & 1  & 0  \\
              1 & -1 & 0
          \end{bmatrix}\);
    \item \(\matb=\begin{bmatrix}
              2  & 2  & 3 \\
              1  & -1 & 0 \\
              -1 & 2  & 1
          \end{bmatrix}\);
    \item \(\matc=\begin{bmatrix}
              1 & 0 & -1 & 0  \\
              0 & 1 & 0  & 0  \\
              0 & 0 & -1 & 1  \\
              0 & 0 & 0  & -1
          \end{bmatrix}\)
\end{enumerate}
\end{problem}
\begin{proof}
\end{proof}

% 3.29
\begin{problem}
设\(\mata\)为\(n\)阶方阵,\(\mata^*\)为\(\mata\)的伴随矩阵。已知\(\abs{\mata}=3\),求\(\abs{\inv{\spar{\frac{1}{4}\mata}}-2\mata^*}\)。
\end{problem}
\begin{proof}
\end{proof}

% 3.30
\begin{problem}
设矩阵\(\mata\)的伴随矩阵
\begin{equation*}
    \mata^*=
    \begin{bmatrix}
        1 & 0  & 0 & 0 \\
        0 & 1  & 0 & 0 \\
        1 & 0  & 1 & 0 \\
        0 & -3 & 0 & 8
    \end{bmatrix}
\end{equation*}
且\(\mata\matb\inv{\mata}=\matb\inv{\mata}+3\mati\),其中\(\mati\)为\(4\)阶单位矩阵,求矩阵\(\matb\)。
\end{problem}
\begin{proof}
\end{proof}

% 3.31
\begin{problem}
设\(\mata\)是\(n\)阶可逆矩阵,证明:
\begin{enumerate}
    \item \(\det{\mata^*}=\spar{\det{\mata}}^{n-1}\);
    \item \(\pinv{\mata^*}=\spar{\inv{\mata}}^*\);
    \item \(\spar{\mata^*}^\top=\spar{\mata^\top}^*\)。
\end{enumerate}
\end{problem}
\begin{proof}
\end{proof}

% 3.32
\begin{problem}
设\(\enums{\vecal}{s}\)是线性无关向量组,而
\begin{equation*}
    \vecbeta_i=\sum_{j=1}^sa_{ij}\vecal_j,i=1,2,\cdots,s
\end{equation*}
证明\(\enums{\vecbeta}{s}\)线性无关的充分必要条件是下面\(s\)阶行列式
\begin{equation*}
    \begin{vmatrix}
        a_{11} & a_{12} & \cdots & a_{1s} \\
        a_{21} & a_{22} & \cdots & a_{2s} \\
        \vdots & \vdots &        & \vdots \\
        a_{s1} & a_{s2} & \cdots & a_{ss}
    \end{vmatrix}\neq0
\end{equation*}
\end{problem}
\begin{proof}
\end{proof}

% 3.33
\begin{problem}
设\(a_{ij}\spar{t}\)为自变量\(t\)的实连续函数,\(i,j=1,2,\cdots,n\)。证明:
\begin{equation*}
    \fracdif{t}
    \begin{vmatrix}
        a_{11}\spar{t} & \cdots & a_{1n}\spar{t} \\
        \vdots         &        & \vdots         \\
        a_{n1}\spar{t} & \cdots & a_{nn}\spar{t}
    \end{vmatrix}=\sum_{j=1}^n
    \begin{vmatrix}
        a_{11}\spar{t} & \cdots & \fracdif{t}a_{1j}\spar{t} & \cdots & a_{1n}\spar{t} \\
        \vdots         &        & \vdots                    &        & \vdots         \\
        a_{n1}\spar{t} & \cdots & \fracdif{t}a_{nj}\spar{t} & \cdots & a_{nn}\spar{t}
    \end{vmatrix}
\end{equation*}
\end{problem}
\begin{proof}
\end{proof}

% 3.34
\begin{problem}
利用子矩阵的行列式计算下列矩阵的秩:
\begin{enumerate}
    \item \(\begin{bmatrix}
              17 & 18 & 40 & 10 \\
              3  & 7  & 17 & 1  \\
              1  & 4  & 10 & 0  \\
              7  & 8  & 18 & 4
          \end{bmatrix}\);
    \item \(\begin{bmatrix}
              3 & -5 & 4 & 2 & 7 \\
              2 & 0  & 3 & 1 & 4 \\
              1 & 5  & 2 & 0 & 1
          \end{bmatrix}\)。
\end{enumerate}
\end{problem}
\begin{proof}
\end{proof}