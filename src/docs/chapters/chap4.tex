\section{线性空间与线性变换}

% 4.1
\begin{problem}
在次数不大于\(2\)的多项式线性空间\(P_2[x]\)中,试证:\(\vecf_1=1\),\(\vecf_2=x-1\),\(\vecf_3=\mypar{x-2}\mypar{x-1}\)线性无关。
\end{problem}
\begin{proof}
    由例4.2.2可知,\(\vecveps_1=1\),\(\vecveps_2=x\),\(\vecveps_3=x^2\)是\(P_2[x]\)的一组基。又因为
    \begin{align*}
        \begin{bmatrix}
            \vecf_1 & \vecf_2 & \vecf_3
        \end{bmatrix}
         & =
        \begin{bmatrix}
            \vecveps_1 & \vecveps_2 & \vecveps_3
        \end{bmatrix}
        \begin{bmatrix}
            1 & -1 & 2  \\
            0 & 1  & -3 \\
            0 & 0  & 1
        \end{bmatrix} \\
         & =
        \begin{bmatrix}
            \vecveps_1 & \vecveps_2 & \vecveps_3
        \end{bmatrix}\mata
    \end{align*}
    其中\(\rank{\mata}=3\)。由例4.2.4可知\(\vecf_1,\vecf_2,\vecf_3\)线性无关。
\end{proof}

% 4.2
\begin{problem}
证明:以下三个多项式为\(P_2[x]\)的一组基:
\begin{equation*}
    f_1=1,f_2=x-1,f_3=\mypar{x-1}^2
\end{equation*}
再求\(g\mypar{x}=5x^2+x+3\)在此基下的坐标。
\end{problem}
\begin{proof}
    和习题4.1类似,令\(\vecveps_1=1\),\(\vecveps_2=x\),\(\vecveps_3=x^2\)表示\(P_2[x]\)的一组基。则有
    \begin{align*}
        \begin{bmatrix}
            \vecf_1 & \vecf_2 & \vecf_3
        \end{bmatrix}
         & =
        \begin{bmatrix}
            \vecveps_1 & \vecveps_2 & \vecveps_3
        \end{bmatrix}
        \begin{bmatrix}
            1 & -1 & 1  \\
            0 & 1  & -2 \\
            0 & 0  & 1
        \end{bmatrix} \\
         & =
        \begin{bmatrix}
            \vecveps_1 & \vecveps_2 & \vecveps_3
        \end{bmatrix}\mata
    \end{align*}
    其中\(\rank{\mata}=3\),因此\(\mata\)可逆,\(\vecf_1,\vecf_2,\vecf_3\)线性无关。

    对\(\forall\vecf\in P_2[x]\),设其坐标为\(\vecx\)。则有
    \begin{equation*}
        \vecf=
        \begin{bmatrix}
            \vecveps_1 & \vecveps_2 & \vecveps_3
        \end{bmatrix}\vecx=
        \begin{bmatrix}
            \vecf_1 & \vecf_2 & \vecf_3
        \end{bmatrix}\inv{\mata}\vecx
    \end{equation*}
    因此\(P_2[x]\)中的任意向量可由\(\vecf_1,\vecf_2,\vecf_3\)线性表出。
    将\(g\mypar{x}\)在基\(\vecveps_1,\vecveps_2,\vecveps_3\)下的坐标代入即可得到\(g\mypar{x}\)在基\(\vecf_1,\vecf_2,\vecf_3\)下的坐标:
    \begin{equation*}
        \inv{\mata}\begin{bmatrix}3&1&5\end{bmatrix}^\top=\begin{bmatrix}9&11&5\end{bmatrix}^\top
    \end{equation*}

    综上,\(\vecf_1,\vecf_2,\vecf_3\)为\(P_2[x]\)的一组基,\(g\mypar{x}\)在此基下的坐标为\(\myvec{9,11,5}\)。
\end{proof}

% 4.3
\begin{problem}
在次数不大于\(3\)的多项式空间\(P_3[x]\)中,
\begin{enumerate}
    \item 求由基\(1,x,x^2,x^3\)到基\(1,1+x,\mypar{1+x}^2,\mypar{1+x}^3\)的过渡矩阵;
    \item 求\(f\mypar{x}=a_0+a_1x+a_2x^2+a_3x^3\)在基\(1,1+x,\mypar{1+x}^2,\mypar{1+x}^3\)下的坐标。
\end{enumerate}
\end{problem}
\begin{proof}
    \begin{enumerate}
        \item {
              容易发现
              \begin{equation*}
                  \begin{bmatrix}
                      1 & 1+x & \mypar{1+x}^2 & \mypar{1+x}^3
                  \end{bmatrix}=
                  \begin{bmatrix}
                      1 \\ x \\ x^2 \\ x^3
                  \end{bmatrix}^\top
                  \begin{bmatrix}
                      1 & 1 & 1 & 1 \\
                      0 & 1 & 2 & 3 \\
                      0 & 0 & 1 & 3 \\
                      0 & 0 & 0 & 1
                  \end{bmatrix}
              \end{equation*}
              因此过渡矩阵为\(\matm=
              \begin{bmatrix}
                  1 & 1 & 1 & 1 \\
                  0 & 1 & 2 & 3 \\
                  0 & 0 & 1 & 3 \\
                  0 & 0 & 0 & 1
              \end{bmatrix}\);
              }
        \item {
              令\(\veca=\myvec{a_0,a_1,a_2,a_3}\),则有
              \begin{align*}
                  f\mypar{x}
                   & =
                  \begin{bmatrix}
                      1 & x & x^2 & x^3
                  \end{bmatrix}\veca \\
                   & =
                  \begin{bmatrix}
                      1 & 1+x & \mypar{1+x}^2 & \mypar{1+x}^3
                  \end{bmatrix}\inv{\matm}\veca
              \end{align*}
              因此\(f\mypar{x}\)在基\(1,1+x,\mypar{1+x}^2,\mypar{1+x}^3\)下的坐标为
              \begin{equation*}
                  \inv{\matm}\veca=
                  \begin{bmatrix}
                      a_0-a_1+a_2-a_3 \\
                      a_1-2a_2+3a_3   \\
                      a_2-3a_3        \\
                      a_3
                  \end{bmatrix}
              \end{equation*}
              }
    \end{enumerate}
\end{proof}

% 4.4
\begin{problem}
在\(P_3[x]\)的多项式空间中,旧基为\(1,x,x^2,x^3\);新基为\(1\),\(1+x\),\(1+x+x^2\),\(1+x+x^2+x^3\)。
\begin{enumerate}
    \item 求旧基到新基的过渡矩阵;
    \item 求多项式\(1+2x+3x^2+4x^3\)在新基下的坐标;
    \item 若多项式\(f\mypar{x}\)在新基下的坐标为\(\myvec{1,2,3,4}\),求它在旧基下的坐标。
\end{enumerate}
\end{problem}
\begin{proof}
    \begin{enumerate}
        \item 过渡矩阵为\(\matm=\begin{bmatrix}1&1&1&1\\0&1&1&1\\0&0&1&1\\0&0&0&1\end{bmatrix}\);
        \item 因为该多项式在旧基下的坐标为\(\vecx=\myvec{1,2,3,4}\),因此该多项式在新基下的坐标为\(\inv{\matm}\vecx=\myvec{-1,-1,-1,4}\);
        \item 同理可得该多项式在旧基下的坐标为\(\matm\vecx=\myvec{10,9,7,4}\)。
    \end{enumerate}
\end{proof}

% 4.5
\begin{problem}
已知\(\vecxi\)在基\(\matb_1=\setof{\vecal_1,\vecal_2,\vecal_3}\)下的坐标为\(\vecxi_{\matb_1}=\myvec{1,-2,2}\),求\(\vecxi\)在基\(\matb_2=\setof{\vecbeta_1,\vecbeta_2,\vecbeta_3}\)下的坐标\(\vecxi_{\matb_2}\),其中\(\vecbeta_1=\vecal_1+\vecal_2\),\(\vecbeta_2=\vecal_2+\vecal_3\),\(\vecbeta_3=\vecal_3+\vecal_1\)。
\end{problem}
\begin{proof}
    由题设可知
    \begin{equation*}
        \begin{bmatrix}\vecbeta_1&\vecbeta_2&\vecbeta_3\end{bmatrix}=
        \begin{bmatrix}\vecal_1&\vecal_2&\vecal_3\end{bmatrix}
        \begin{bmatrix}
            1 & 0 & 1 \\
            1 & 1 & 0 \\
            0 & 1 & 1
        \end{bmatrix}
    \end{equation*}
    因此从基\(\matb_1\)到基\(\matb_2\)的过渡矩阵为\(\matm=\begin{bmatrix}1&0&1\\1&1&0\\0&1&1\end{bmatrix}\)。
    因此有
    \begin{equation*}
        \vecxi_{\matb_2}=\inv{\matm}\vecxi_{\matb_1}=\begin{bmatrix}-1.5&-0.5&2.5\end{bmatrix}^\top
    \end{equation*}
\end{proof}

% 4.6
\begin{problem}
设\(\vecveps_1,\vecveps_2,\vecveps_3\)是线性空间\(V\)的一组基,且
\begin{equation*}
    \begin{cases}
        \vecxi_1=\vecveps_1+\vecveps_3 \\
        \vecxi_2=\vecveps_2            \\
        \vecxi_3=\vecveps_1+2\vecveps_2+2\vecveps_3
    \end{cases}
\end{equation*}
\begin{equation*}
    \begin{cases}
        \veceta_1=\vecveps_1            \\
        \veceta_2=\vecveps_1+\vecveps_2 \\
        \veceta_3=\vecveps_1+\vecveps_2+\vecveps_3
    \end{cases}
\end{equation*}
\begin{enumerate}
    \item 试证\(\vecxi_1,\vecxi_2,\vecxi_3\)及\(\veceta_1,\veceta_2,\veceta_3\)都是\(V\)的一组基;
    \item 求由基\(\vecxi_1,\vecxi_2,\vecxi_3\)到基\(\veceta_1,\veceta_2,\veceta_3\)的过渡矩阵。
\end{enumerate}
\end{problem}
\begin{proof}
    \begin{enumerate}
        \item {
              容易发现
              \begin{align*}
                  \begin{bmatrix}\vecxi_1&\vecxi_2&\vecxi_3\end{bmatrix}
                   & =
                  \begin{bmatrix}\vecveps_1&\vecveps_2&\vecveps_3\end{bmatrix}
                  \begin{bmatrix}
                      1 & 0 & 1 \\
                      0 & 1 & 2 \\
                      1 & 0 & 2
                  \end{bmatrix}                                                        \\
                   & =\begin{bmatrix}\vecveps_1&\vecveps_2&\vecveps_3\end{bmatrix}\mata
              \end{align*}
              \begin{align*}
                  \begin{bmatrix}\veceta_1&\veceta_2&\veceta_3\end{bmatrix}
                   & =
                  \begin{bmatrix}\vecveps_1&\vecveps_2&\vecveps_3\end{bmatrix}
                  \begin{bmatrix}
                      1 & 1 & 1 \\
                      0 & 1 & 1 \\
                      0 & 0 & 1
                  \end{bmatrix}                                                        \\
                   & =\begin{bmatrix}\vecveps_1&\vecveps_2&\vecveps_3\end{bmatrix}\matb
              \end{align*}
              又因为\(\rank{\mata}=\rank{\matb}=3\),由习题4.2可知\(\vecxi_1,\vecxi_2,\vecxi_3\)和\(\veceta_1,\veceta_2,\veceta_3\)均线性无关,且任意\(V\)中的向量均可由上述两个向量组线性表出。所以\(\vecxi_1,\vecxi_2,\vecxi_3\)和\(\veceta_1,\veceta_2,\veceta_3\)都是\(V\)的一组基;
              }
        \item {
              由(1)可得
              \begin{align*}
                  \begin{bmatrix}\veceta_1&\veceta_2&\veceta_3\end{bmatrix}
                   & =
                  \begin{bmatrix}\vecveps_1&\vecveps_2&\vecveps_3\end{bmatrix}\matb          \\
                   & =\begin{bmatrix}\vecxi_1&\vecxi_2&\vecxi_3\end{bmatrix}\inv{\mata}\matb
              \end{align*}
              因此由基\(\vecxi_1,\vecxi_2,\vecxi_3\)到基\(\veceta_1,\veceta_2,\veceta_3\)的过渡矩阵为
              \begin{equation*}
                  \inv{\mata}\matb=
                  \begin{bmatrix}
                      2  & 2  & 1 \\
                      2  & 3  & 1 \\
                      -1 & -1 & 0
                  \end{bmatrix}
              \end{equation*}
              }
    \end{enumerate}
\end{proof}

% 4.7
\begin{problem}
在线性空间\(\rea^{2\times2}\)中,已知
\begin{equation*}
    \vecal_1=\begin{bmatrix}1&0\\0&0\end{bmatrix},
    \vecal_2=\begin{bmatrix}0&1\\0&0\end{bmatrix},
    \vecal_3=\begin{bmatrix}0&0\\1&0\end{bmatrix},
    \vecal_4=\begin{bmatrix}0&0\\0&1\end{bmatrix}
\end{equation*}
为其一组基,若\(\rea^{2\times2}\)的另一组基为\(\vecbeta_1,\vecbeta_2,\vecbeta_3,\vecbeta_4\),由\(\vecal_1,\vecal_2,\vecal_3,\vecal_4\)到\(\vecbeta_1,\vecbeta_2,\vecbeta_3,\vecbeta_4\)的过渡矩阵为
\begin{equation*}
    \mata=
    \begin{bmatrix}
        0 & 1 & 1 & 1 \\
        1 & 0 & 1 & 1 \\
        1 & 1 & 0 & 1 \\
        1 & 1 & 1 & 0
    \end{bmatrix}
\end{equation*}
求:
\begin{enumerate}
    \item 基\(\vecbeta_1,\vecbeta_2,\vecbeta_3,\vecbeta_4\);
    \item 矩阵\begin{equation*}\begin{bmatrix}0&1\\2&-3\end{bmatrix}\end{equation*}在基\(\vecbeta_1,\vecbeta_2,\vecbeta_3,\vecbeta_4\)下的坐标。
\end{enumerate}
\end{problem}
\begin{proof}
    \begin{enumerate}
        \item {
              由题设可知
              \begin{align*}
                  \vecbeta_1 & =\vecal_2+\vecal_3+\vecal_4=\begin{bmatrix}0&1\\1&1\end{bmatrix},
                  \vecbeta_2=\vecal_1+\vecal_3+\vecal_4=\begin{bmatrix}1&0\\1&1\end{bmatrix}     \\
                  \vecbeta_3 & =\vecal_1+\vecal_2+\vecal_4=\begin{bmatrix}1&1\\0&1\end{bmatrix},
                  \vecbeta_4=\vecal_1+\vecal_2+\vecal_3=\begin{bmatrix}1&1\\1&0\end{bmatrix}
              \end{align*}
              }
        \item {
              容易得到该矩阵在基\(\vecal_1,\vecal_2,\vecal_3,\vecal_4\)下的坐标为\(\myvec{0,1,2,-3}\)。因此该矩阵在基\(\vecbeta_1,\vecbeta_2,\vecbeta_3,\vecbeta_4\)下的坐标为
              \begin{equation*}
                  \inv{\mata}\begin{bmatrix}0&1&2&-3\end{bmatrix}^\top=\begin{bmatrix}0&-1&-2&3\end{bmatrix}^\top
              \end{equation*}
              }
    \end{enumerate}
\end{proof}

% 4.8
\begin{problem}
已知实数域\(\rea\)上的所有\(2\)阶矩阵,对于矩阵的加法和数乘,构成\(\rea\)上的四维线性空间,记作\(V=\rea^{2\times2}\)。
\begin{enumerate}
    \item {
          分别证明
          \begin{equation*}
              \vecal_1=\begin{bmatrix}1&0\\0&0\end{bmatrix},
              \vecal_2=\begin{bmatrix}1&1\\0&0\end{bmatrix},
              \vecal_3=\begin{bmatrix}1&1\\1&0\end{bmatrix},
              \vecal_4=\begin{bmatrix}1&1\\1&1\end{bmatrix}
          \end{equation*}
          与
          \begin{equation*}
              \vecbeta_1=\begin{bmatrix}-1&1\\1&1\end{bmatrix},
              \vecbeta_2=\begin{bmatrix}1&-1\\1&1\end{bmatrix},
              \vecbeta_3=\begin{bmatrix}1&1\\-1&1\end{bmatrix},
              \vecbeta_4=\begin{bmatrix}1&1\\1&-1\end{bmatrix}
          \end{equation*}
          均为\(\rea^{2\times2}\)的基;}
    \item 对于\(\rea^{2\times2}\)中的任意元素\(\vecal\),求\(\vecal\)在基\(\vecal_1,\vecal_2,\vecal_3,\vecal_4\)下的坐标;
    \item 求由基\(\vecal_1,\vecal_2,\vecal_3,\vecal_4\)到\(\vecbeta_1,\vecbeta_2,\vecbeta_3,\vecbeta_4\)的过渡矩阵。
\end{enumerate}
\end{problem}
\begin{proof}
    \begin{enumerate}
        \item {
              由习题4.7可知,
              \begin{equation*}
                  \vecveps_1=\begin{bmatrix}1&0\\0&0\end{bmatrix},
                  \vecveps_2=\begin{bmatrix}0&1\\0&0\end{bmatrix},
                  \vecveps_3=\begin{bmatrix}0&0\\1&0\end{bmatrix},
                  \vecveps_4=\begin{bmatrix}0&0\\0&1\end{bmatrix}
              \end{equation*}
              为\(\rea^{2\times2}\)的一组基。又因为
              \begin{align*}
                  \begin{bmatrix}\vecal_1&\vecal_2&\vecal_3&\vecal_4\end{bmatrix}
                   & =
                  \begin{bmatrix}\vecveps_1&\vecveps_2&\vecveps_3&\vecveps_4\end{bmatrix}
                  \begin{bmatrix}
                      1 & 1 & 1 & 1 \\
                      0 & 1 & 1 & 1 \\
                      0 & 0 & 1 & 1 \\
                      0 & 0 & 0 & 1
                  \end{bmatrix} \\
                   & =
                  \begin{bmatrix}\vecveps_1&\vecveps_2&\vecveps_3&\vecveps_4\end{bmatrix}\mata
              \end{align*}
              \begin{align*}
                  \begin{bmatrix}\vecbeta_1&\vecbeta_2&\vecbeta_3&\vecbeta_4\end{bmatrix}
                   & =
                  \begin{bmatrix}\vecveps_1&\vecveps_2&\vecveps_3&\vecveps_4\end{bmatrix}
                  \begin{bmatrix}
                      -1 & 1  & 1  & 1  \\
                      1  & -1 & 1  & 1  \\
                      1  & 1  & -1 & 1  \\
                      1  & 1  & 1  & -1
                  \end{bmatrix}                                                                \\
                   & =\begin{bmatrix}\vecveps_1&\vecveps_2&\vecveps_3&\vecveps_4\end{bmatrix}\matb
              \end{align*}
              其中\(\rank{\mata}=\rank{\matb}=4\)。所以\(\vecal_1,\vecal_2,\vecal_3,\vecal_4\)和\(\vecbeta_1,\vecbeta_2,\vecbeta_3,\vecbeta_4\)线性无关,且\(\rea^{2\times2}\)中的任意元素可由这两个向量组线性表出。
              由基的定义可知上述两个向量组均为\(\rea^{2\times2}\)的基;
              }
        \item {
              设\(\vecal=\begin{bmatrix}\alpha_{11}&\alpha_{12}\\\alpha_{21}&\alpha_{22}\end{bmatrix}\),则\(\vecal\)在基\(\vecveps_1,\vecveps_2,\vecveps_3,\vecveps_4\)下的坐标为\(\myvec{\alpha_{11},\alpha_{12},\alpha_{21},\alpha_{22}}\)。由(1)可得\(\vecal\)在基\(\vecal_1,\vecal_2,\vecal_3,\vecal_4\)下的坐标为
              \begin{equation*}
                  \inv{\mata}\begin{bmatrix}\alpha_{11}\\\alpha_{12}\\\alpha_{21}\\\alpha_{22}\end{bmatrix}=
                  \begin{bmatrix}\alpha_{11}-\alpha_{12}\\\alpha_{12}-\alpha_{21}\\\alpha_{21}-\alpha_{22}\\\alpha_{22}\end{bmatrix}
              \end{equation*}
              }
        \item {
              由(1)可得
              \begin{align*}
                  \begin{bmatrix}\vecbeta_1&\vecbeta_2&\vecbeta_3&\vecbeta_4\end{bmatrix}
                   & =\begin{bmatrix}\vecveps_1&\vecveps_2&\vecveps_3&\vecveps_4\end{bmatrix}\matb    \\
                   & =\begin{bmatrix}\vecal_1&\vecal_2&\vecal_3&\vecal_4\end{bmatrix}\inv{\mata}\matb
              \end{align*}
              所以由基\(\vecal_1,\vecal_2,\vecal_3,\vecal_4\)到\(\vecbeta_1,\vecbeta_2,\vecbeta_3,\vecbeta_4\)的过渡矩阵为
              \begin{equation*}
                  \inv{\mata}\matb=
                  \begin{bmatrix}
                      -2 & 2  & 0  & 0  \\
                      0  & -2 & 2  & 0  \\
                      0  & 0  & -2 & 2  \\
                      1  & 1  & 1  & -1
                  \end{bmatrix}
              \end{equation*}
              }
    \end{enumerate}
\end{proof}

% 4.9
\begin{problem}
求下列两个齐次线性方程组的解空间的基和维数:
\begin{enumerate}
    \item \(x_1+x_2+\cdots+x_n=0\);
    \item \begin{equation*}
              \begin{cases}
                  2x_1-4x_2+5x_3+3x_4=0 \\
                  3x_1-6x_2+4x_3+2x_4=0 \\
                  4x_1-8x_2+17x_3+11x_4=0
              \end{cases}
          \end{equation*}
\end{enumerate}
\end{problem}
\begin{proof}
    \begin{enumerate}
        \item {
              方程组对应的系数矩阵的最简阶梯型矩阵为\(\begin{bmatrix}1&1&\cdots&1\end{bmatrix}\),因此解空间的基为
              \begin{equation*}
                  \begin{bmatrix}-1\\1\\0\\0\\\vdots\\0\\0\end{bmatrix},
                  \begin{bmatrix}-1\\0\\1\\0\\\vdots\\0\\0\end{bmatrix},\cdots
                  \begin{bmatrix}-1\\0\\0\\0\\\vdots\\0\\1\end{bmatrix}
              \end{equation*}
              维数为\(n-1\);
              }
        \item {
              方程组对应的系数矩阵的最简阶梯型矩阵为
              \begin{equation*}
                  \begin{bmatrix}
                      1 & -2 & 0 & -\frac{2}{7} \\
                      0 & 0  & 1 & \frac{5}{7}  \\
                      0 & 0  & 0 & 0
                  \end{bmatrix}
              \end{equation*}
              因此解空间的基为\(\myvec{2,1,0,0}\),\(\myvec{2,0,-5,7}\),维数为\(2\)。
              }
    \end{enumerate}
\end{proof}

% 4.10
\begin{problem}
设\(\enums{\vecal}{n}\)是\(n\)维线性空间\(V\)的一组基,又\(V\)中向量\(\vecal_{n+1}\)在这组基下坐标\(\mypar{\enums{x}{n}}\)全都不为零。

证明\(\enums{\vecal}{n},\vecal_{n+1}\)中任意\(n\)个向量必构成\(V\)的一组基,并求\(\vecal_1\)在基\(\vecal_2,\dots,\vecal_n,\vecal_{n+1}\)下的坐标。
\end{problem}
\begin{proof}
    因为\(\enums{\vecal}{n}\)是\(V\)的一组基,下面证明对\(\forall i=1,2,\dots,n\),
    \begin{equation*}
        \mata_i=\setof{\vecal_1,\dots,\vecal_{i-1},\vecal_{i+1},\dots,\vecal_{n+1}}
    \end{equation*}
    都是\(V\)的一组基。

    由题设可知
    \begin{align*}
        \begin{bmatrix}\vecal_1\\\vdots\\\vecal_{i+1}\\\vdots\\\vecal_{n+1}\end{bmatrix}^\top
         & =
        \begin{bmatrix}\vecal_1\\\vecal_2\\\vdots\\\vecal_n\end{bmatrix}^\top
        \begin{bmatrix}
            1      & \cdots & 0      & 0      & \cdots & 0      & x_1     \\
            \vdots & \ddots & \vdots & \vdots & \cdots & \vdots & \vdots  \\
            0      & \cdots & 1      & 0      & \cdots & 0      & x_{i-1} \\
            0      & \cdots & 0      & 0      & \cdots & 0      & x_i     \\
            0      & \cdots & 0      & 1      & \cdots & 0      & x_{i+1} \\
            \vdots &        & \vdots & \vdots & \ddots & \vdots & \vdots  \\
            0      & \cdots & 0      & 0      & \cdots & 1      & x_n
        \end{bmatrix} \\
         & =
        \begin{bmatrix}\vecal_1\\\vecal_2\\\vdots\\\vecal_n\end{bmatrix}^\top\matm_i
    \end{align*}
    容易发现\(\rank{\matm_i}=n\),所以\(\mata_i\)线性无关,且\(V\)中任意向量可由\(\mata_i\)线性表出。

    因此\(\enums{\vecal}{n},\vecal_{n+1}\)中任意\(n\)个向量必构成\(V\)的一组基。

    同时,由等式\(\vecal_{n+1}=x_1\vecal_1+x_2\vecal_2+\cdots+x_n\vecal_n\)可得
    \begin{equation*}
        \vecal_1=-\frac{x_2}{x_1}\vecal_2-\cdots-\frac{x_n}{x_1}\vecal_n+\frac{1}{x_1}\vecal_{n+1}
    \end{equation*}
    所以\(\vecal_1\)在在基\(\vecal_2,\dots,\vecal_n,\vecal_{n+1}\)下的坐标为\(\myvec{-\frac{x_2}{x_1},\cdots,-\frac{x_n}{x_1},\frac{1}{x_1}}\)。
\end{proof}

% 4.11
\begin{problem}
设\(\mathrm{K}^{n\times n}\)为数域\(\mathrm{K}\)上全体\(n\)阶方阵构成的\(n^2\)维线性空间,又\(\mata\in\mathrm{K}^{n\times n}\)。
\begin{enumerate}
    \item 证明:\(\mathrm{K}^{n\times n}\)中与\(\mata\)可交换(即\(\mata\matb=\matb\mata\),称\(\mata\)与\(\matb\)可交换)的全体方阵构成子空间(该子空间记为\(\mathrm{W}\mypar{\mata}\));
    \item 问:当\(\mata\)是主对角线上元素为\(\enums{a}{n}\)(\(a_i\in\mathrm{K}\)且互异)的对角阵时,\(\mathrm{W}\mypar{\mata}\)是几维的?并给出其一基。
\end{enumerate}
\end{problem}
\begin{proof}
\end{proof}

% 4.12
\begin{problem}
设\(V_1,V_2\)是\(\rea^n\)的两个非平凡子空间,证明:在\(\rea^n\)中存在向量\(\vecal\),使\(\vecal\notin V_1\),且\(\vecal\notin V_2\),并在\(\rea^3\)中举例说明此结论。
\end{problem}
\begin{proof}
    \begin{enumerate}
        \item {
              若\(V_1,V_2\)其中一个是另一个的子空间,则不妨设\(V_1\subseteq V_2\)。

              因为\(V_2\)非平凡,存在\(\vecal\in\rea^n\)满足\(\vecal\notin V_2\),从而\(\vecal\notin V_1\);
              }
        \item {
              若\(V_1,V_2\)中任意空间都不是另一个的子空间,则存在\(\vecal_1\in V_1\cap\overline{V_2}\),\(\vecal_2\in V_2\cap\overline{V_1}\)。此时构造\(\vecal=\vecal_1+\vecal_2\):

              若\(\vecal\in V_1\),则有\(\vecal_2=\vecal-\vecal_1\in V_1\),与假设矛盾。因此\(\vecal\notin V_1\)。

              同理可得\(\vecal\notin V_2\),结论得证。

              举例:\(V_1=\myspan{\myvec{1,0,0},\myvec{0,1,0}}\),\(V_2=\myspan{\myvec{0,1,0},\myvec{0,0,1}}\),\(\vecal=\myvec{1,0,1}\)。
              }
    \end{enumerate}
\end{proof}

% 4.13
\begin{problem}
设\(\vecal,\vecbeta,\vecgamma\in\rea^n\),\(c_1,c_2,c_3\in\rea\),且\(c_1c_3\neq0\),证明:若\(c_1\vecal+c_2\vecbeta+c_3\vecgamma=\veczero\),则\(\myspan{\vecal,\vecbeta}=\myspan{\vecbeta,\vecgamma}\)。
\end{problem}
\begin{proof}
    因为\(c_1\vecal+c_2\vecbeta+c_3\vecgamma=\veczero\),且\(c_1c_3\neq0\),所以有\(\vecal=-\frac{c_2}{c_1}\vecbeta-\frac{c_3}{c_1}\vecgamma\)。

    因此对\(\forall\veceta\in\myspan{\vecal,\vecbeta}\),不妨设\(\veceta=\lambda_1\vecal+\lambda_2\vecbeta\),则有
    \begin{align*}
        \veceta & =\lambda_1\mypar{-\frac{c_2}{c_1}\vecbeta-\frac{c_3}{c_1}\vecgamma}+\lambda_2\vecbeta \\
                & =\mypar{\lambda_2-\frac{c_2\lambda_1}{c_1}}\vecbeta-\frac{c_3\lambda_1}{c_1}\vecgamma
    \end{align*}
    所以\(\veceta\in\myspan{\vecbeta,\vecgamma}\),即\(\myspan{\vecal,\vecbeta}\subseteq\myspan{\vecbeta,\vecgamma}\)。

    同理可得\(\myspan{\vecbeta,\vecgamma}\subseteq\myspan{\vecal,\vecbeta}\)。因此\(\myspan{\vecal,\vecbeta}=\myspan{\vecbeta,\vecgamma}\)。
\end{proof}

% 4.14
\begin{problem}
若
\begin{equation*}
    \mata=\begin{bmatrix}1&1&1&0\\2&1&0&1\end{bmatrix}
\end{equation*}
求\(\mata\)的零空间\(\nullsp{\mata}\)的一组基。
\end{problem}
\begin{proof}
    对\(A\)使用初等变换,将其化为最简阶梯型矩阵:
    \begin{equation*}
        \begin{bmatrix}
            1 & 0 & -1 & 1  \\
            0 & 1 & 2  & -1
        \end{bmatrix}
    \end{equation*}
    因此得到\(\nullsp{\mata}\)的一组基为\(\myvec{1,-2,1,0}\),\(\mat{-1,1,0,1}\)。
\end{proof}

% 4.15
\begin{problem}
令\(\mata\)和\(\matb\)为\(n\times n\)矩阵。
\begin{enumerate}
    \item 证明:当且仅当\(\matb\)的列空间是\(\mata\)的零空间的子空间时,有\(\mata\matb=\mato\);
    \item 证明:若\(\mata\matb=\mato\),则\(\mata\)和\(\matb\)的秩的和不会超过\(n\)。
\end{enumerate}
\end{problem}
\begin{proof}
\end{proof}

% 4.16
\begin{problem}
已知向量组\(\vecal_1=\myvec{2,0,1,3,-1}\),\(\vecal_2=\mat{0,-2,1,5,-3}^\top\),\(\vecbeta_1=\mat{1,1,0,-1,1}^\top\),\(\vecbeta_2=\mat{1,-3,2,0,5}\),且\(W_1=\myspan{\vecal_1,\vecal_2}\),\(W_2=\myspan{\vecbeta_1,\vecbeta_2}\)。试求\(W_1\cap W_2\)与\(W_1+W_2\)的维数以及各自的一组基。
\end{problem}
\begin{proof}
    因为\(W_1+W_2=\myspan{\vecal_1,\vecal_2,\vecbeta_1,\vecbeta_2}\),因此考虑对\(\mat{\vecal_1,\vecal_2,\vecbeta_1,\vecbeta_2}\)作初等行变换,化为最简阶梯型矩阵:
    \begin{equation*}
        \begin{bmatrix}
            2  & 0  & 1  & 1  \\
            0  & -2 & 1  & -3 \\
            1  & 1  & 0  & 2  \\
            3  & 5  & -1 & 0  \\
            -1 & -3 & 1  & 5
        \end{bmatrix}\Rightarrow
        \begin{bmatrix}
            1 & 0 & \frac{1}{2}  & 0 \\
            0 & 1 & -\frac{1}{2} & 0 \\
            0 & 0 & 0            & 1 \\
            0 & 0 & 0            & 0 \\
            0 & 0 & 0            & 0
        \end{bmatrix}
    \end{equation*}
    所以\(W_1+W_2\)的维数为\(3\),一组基为\(\vecal_1,\vecal_2,\vecbeta_2\)。

    由维数公式可知\(W_1\cap W_2\)的维数为\(1\),又因为\(\vecbeta_1=\frac{1}{2}\vecal_1-\frac{1}{2}\vecal_2\in W_1\),所以\(\vecbeta_1\in W_1\cap W_2\)为其一组基。
\end{proof}

% 4.17
\begin{problem}
设\(V\)是内积空间,试证明:
\begin{enumerate}
    \item 对\(\forall\vecal\in V\),都有\(\indot{\veczero}{\vecal}=0\);
    \item 若对\(\forall\vecal\in V\),恒有\(\indot{\vecal}{\vecbeta}=0\),则\(\vecbeta=\veczero\);
    \item 已知\(\vecal,\vecbeta\in V\),如对\(\forall\vecgamma\in V\),都有\(\indot{\vecal}{\vecgamma}=\indot{\vecbeta}{\vecgamma}\)。试证明\(\vecal=\vecbeta\)。
\end{enumerate}
\end{problem}
\begin{proof}
    \begin{enumerate}
        \item {
              对\(\forall\vecbeta\in V\),有
              \begin{equation*}
                  \indot{\veczero}{\vecal}=\indot{\vecbeta+\mypar{-\vecbeta}}{\vecal}=\indot{\vecbeta}{\vecal}-\indot{\vecbeta}{\vecal}=0
              \end{equation*}
              }
        \item 令\(\vecal=\vecbeta\),得到\(\indot{\vecbeta}{\vecbeta}=0\),由内积定义知\(\vecbeta=\veczero\);
        \item 由\(\indot{\vecal}{\vecgamma}=\indot{\vecbeta}{\vecgamma}\)可得\(\indot{\vecal-\vecbeta}{\vecgamma}=0\)。由\(\vecgamma\)的任意性及(2)结论可知\(\vecal-\vecbeta=\veczero\),即\(\vecal=\vecbeta\)。
    \end{enumerate}
\end{proof}

% 4.18
\begin{problem}
设\(\vecal=\myvec{a_1,a_2,a_3}\in\rea^3\),\(\vecbeta=\myvec{b_1,b_2,b_3}\in\rea^3\),定义内积\(\indot{\vecal}{\vecbeta}=\vecal^\top\mata\vecbeta\)。当\(\mata\)取以下哪个矩阵时,使\(\rea^3\)成为Euclid空间。
\begin{enumerate}
    \item \(\mata_1=\begin{bmatrix}1&0&0\\0&2&0\\0&0&1\end{bmatrix}\);
    \item \(\mata_2=\begin{bmatrix}1&-1&1\\-1&1&0\\1&0&0\end{bmatrix}\);
    \item \(\mata_3=\begin{bmatrix}1&-1&3\\-1&2&0\\1&0&3\end{bmatrix}\)。
\end{enumerate}
\end{problem}
\begin{proof}
\end{proof}

% 4.19
\begin{problem}
在\(\rea^2\)中,设\(\vecal=\myvec{a_1,a_2}\),\(\vecbeta=\myvec{b_1,b_2}\),令
\begin{equation*}
    \indot{\vecal}{\vecbeta}=a_1b_1-a_2b_1-a_1b_2+3a_2b_2
\end{equation*}
试验证这是\(\rea^2\)的一个内积。
\end{problem}
\begin{proof}
\end{proof}

% 4.20
\begin{problem}
设\(\enums{\vecal}{n}\)是内积空间的\(n\)个向量,行列式
\begin{equation*}
    \gram{\enums{\vecal}{n}}=
    \begin{vmatrix}
        \indot{\vecal_1}{\vecal_1} & \indot{\vecal_1}{\vecal_2} & \cdots & \indot{\vecal_1}{\vecal_n} \\
        \indot{\vecal_2}{\vecal_1} & \indot{\vecal_2}{\vecal_2} & \cdots & \indot{\vecal_2}{\vecal_n} \\
        \vdots                     & \vdots                     &        & \vdots                     \\
        \indot{\vecal_n}{\vecal_1} & \indot{\vecal_n}{\vecal_2} & \cdots & \indot{\vecal_n}{\vecal_n}
    \end{vmatrix}
\end{equation*}
称为\(\enums{\vecal}{n}\)的Gram行列式。证明\(\gram{\enums{\vecal}{n}}=0\)的充分必要条件是\(\enums{\vecal}{n}\)线性相关。
\end{problem}
\begin{proof}
    \begin{itemize}
        \item {
              首先证明充分性。

              若\(\enums{\vecal}{n}\)线性相关,则有\(\sum_{i=1}^nk_i\vecal_i=\veczero\),其中\(k_i\)不全为零。
              因此对\(\forall j=1,2,\dots,n\),有
              \begin{equation*}
                  \indot{\vecal_j}{\sum_{i=1}^nk_i\vecal_i}=\sum_{i=1}^nk_i\indot{\vecal_j}{\vecal_i}=0
              \end{equation*}
              所以\(\gram{\enums{\vecal}{n}}\)对应矩阵的列线性相关,进一步得到
              \begin{equation*}
                  \gram{\enums{\vecal}{n}}=0
              \end{equation*}
              }
        \item {
              随后证明必要性。
              若\(\gram{\enums{\vecal}{n}}=0\),则其对应矩阵的列线性相关。因此对\(\forall j=1,2,\dots,n\),有\(\sum_{i=1}^nk_i\indot{\vecal_j}{\vecal_i}=0\),其中\(k_i\)不全为零,从而进一步得到\(\indot{\vecal_j}{\sum_{i=1}^nk_i\vecal_i}=0\)。由内积性质得到
              \begin{equation*}
                  \indot{\sum_{i=1}^nk_i\vecal_i}{\sum_{i=1}^nk_i\vecal_i}=0
              \end{equation*}
              因此\(\sum_{i=1}^nk_i\vecal_i=\veczero\),\(\enums{\vecal}{n}\)线性相关。
              }
    \end{itemize}
\end{proof}

% 4.21
\begin{problem}
设\(\enums{\vecal}{r}\)线性无关,\(\vecbeta_1,\vecbeta_2\)也线性无关,且后者与前者中的每个向量都正交,证明:\(\vecbeta_1,\vecbeta_2,\enums{\vecal}{r}\)也线性无关。
\end{problem}
\begin{proof}
    反设结论不成立,则存在一组不全为零的数\(l_1,l_2,k_1,k_2,\dots,k_r\)满足
    \begin{equation*}
        l_1\vecbeta_1+l_2\vecbeta_2+\sum_{i=1}^rk_i\vecal_i=\veczero
    \end{equation*}
    此时\(l_1,l_2\)不全为零,否则\(\enums{\vecal}{r}\)线性相关,与题设矛盾。
    又因为\(\vecbeta_1,\vecbeta_2\)和\(\enums{\vecal}{r}\)中的每个向量正交,因此有
    \begin{align*}
          & \indot{l_1\vecbeta_1+l_2\vecbeta_2+\sum_{i=1}^rk_i\vecal_i}{l_1\vecbeta_1+l_2\vecbeta_2}   \\
        = & \indot{l_1\vecbeta_1+l_2\vecbeta_2}{l_1\vecbeta_1+l_2\vecbeta_2}
        +l_1\sum_{i=1}^r\indot{k_i\vecal_i}{\vecbeta_1}+l_2\sum_{i=1}^r\indot{k_i\vecal_i}{\vecbeta_1} \\
        = & \indot{l_1\vecbeta_1+l_2\vecbeta_2}{l_1\vecbeta_1+l_2\vecbeta_2}=0
    \end{align*}
    由内积定义知\(l_1\vecbeta_1+l_2\vecbeta_2=\veczero\),即\(\vecbeta_1,\vecbeta_2\)线性相关。这与题设矛盾,因此结论成立。
\end{proof}

% 4.22
\begin{problem}
试把\(\myvec{1,0,1,0}\),\(\mat{0,1,0,2}\)扩充成为\(\rea^4\)的一组标准正交基。
\end{problem}
\begin{proof}
    设\(\vecal_1=\myvec{1,0,1,0}\),\(\vecal_2=\mat{0,1,0,2}\)。容易发现\(\vecal_1,\vecal_2\)正交。

    设\(\rea^4\)的剩余正交基为\(\vecx\),则有\(\begin{bmatrix}\vecal_1^\top\\\vecal_2^\top\end{bmatrix}\vecx=\veczero\)。

    对该齐次线性方程组求解,得到基础解系:\(\vecal_3=\myvec{-1,0,1,0}\),\(\vecal_4=\mat{0,-2,0,1}\)。

    将上述正交基标准化,得到\(\rea^4\)的一组标准正交基:

    \begin{equation*}
        \begin{bmatrix}
            \vecq_1 & \vecq_2 & \vecq_3 & \vecq_4
        \end{bmatrix}=
        \begin{bmatrix}
            \frac{\sqrt{2}}{2} & 0                   & -\frac{\sqrt{2}}{2} & 0                    \\
            0                  & \frac{\sqrt{5}}{5}  & 0                   & -\frac{2\sqrt{5}}{5} \\
            \frac{\sqrt{2}}{2} & 0                   & \frac{\sqrt{2}}{2}  & 0                    \\
            0                  & \frac{2\sqrt{5}}{5} & 0                   & \frac{\sqrt{5}}{5}
        \end{bmatrix}
    \end{equation*}
\end{proof}

% 4.23
\begin{problem}
已知\(\vecf_1=1\),\(\vecf_2=x-1\),\(\vecf_3=\mypar{x-1}^2\)为三维内积空间\(C\mat{0,1}\)的一组基,内积定义为
\begin{equation*}
    \indot{\vecf}{\vecg}=\int_0^1\vecf\mypar{x}\vecg\mypar{x}\mathrm{d}x
\end{equation*}
利用Schmidt正交化方法,求与\(\vecf_1,\vecf_2,\vecf_3\)等价的一组标准正交基(两个向量组可相互线性表出称为等价)。
\end{problem}
\begin{proof}
    \begin{align*}
        \vecg_1 & =\frac{\vecf_1}{\Abs{\vecf_1}}=\frac{\vecf_1}{\indot{\vecf_1}{\vecf_1}}=1                                                                                      \\
        \vecg_2 & =\frac{\vecf_2-\indot{\vecg_1}{\vecf_2}\vecg_1}{\Abs{\vecf_2-\indot{\vecg_1}{\vecf_2}\vecg_1}}                                                                 \\
                & =\frac{x-\frac{1}{2}}{\sqrt{\frac{1}{12}}}=\sqrt3\mypar{2x-1}                                                                                                  \\
        \vecg_3 & =\frac{\vecf_3-\indot{\vecg_1}{\vecf_3}\vecg_1-\indot{\vecg_2}{\vecf_3}\vecg_2}{\Abs{\vecf_3-\indot{\vecg_1}{\vecf_3}\vecg_1-\indot{\vecg_2}{\vecf_3}\vecg_2}} \\
                & =\frac{x^2-x+\frac{1}{6}}{\sqrt{\frac{1}{180}}}=\sqrt5\mypar{6x^2-6x+1}
    \end{align*}
\end{proof}

% 4.24
\begin{problem}
设\(\enums{\vecal}{n}\)与\(\enums{\vecbeta}{n}\)为\(n\)维欧氏空间的两组基,其中\(\enums{\vecal}{n}\)为标准正交基,又若由\(\enums{\vecal}{n}\)到\(\enums{\vecbeta}{n}\)的过渡矩阵为\(\matm\),证明:\(\enums{\vecbeta}{n}\)为\(V\)的标准正交基的充分必要条件是\(\matm\)为正交矩阵。
\end{problem}
\begin{proof}
    设\(\mata=\mat{\enums{\vecal}{n}}\),\(\matb=\mat{\enums{\vecbeta}{n}}\)。

    因为\(\enums{\vecal}{n}\)为标准正交基,所以\(\mata\)满足\(\mata^\top\mata=\mati\)。

    又由过渡矩阵的定义知\(\matb=\mata\matm\),因此有

    \begin{align*}
        \matb^\top\matb & =\mypar{\mata\matm}^\top\mata\matm      \\
                        & =\matm^\top\mypar{\mata^\top\mata}\matm \\
                        & =\matm^\top\matm
    \end{align*}
    因此\(\matm\)为正交矩阵等价于\(\matb\)为正交矩阵。又因为\(\matb\)为正交矩阵等价于\(\enums{\vecbeta}{n}\)为\(V\)的标准正交基,所以\(\enums{\vecbeta}{n}\)为\(V\)的标准正交基的充分必要条件是\(\matm\)为正交矩阵。
\end{proof}

% 4.25
\begin{problem}
求下列线性方程组的最小二乘解。
\begin{equation*}
    \begin{cases}
        x_1+x_2=1  \\
        2x_1-x_2=2 \\
        -2x_1+4x_2=7
    \end{cases}
\end{equation*}
\end{problem}
\begin{proof}
\end{proof}

% 4.26
\begin{problem}
设
\begin{equation*}
    \mata=
    \begin{bmatrix}
        1 & 2  & -1 \\
        2 & 0  & 1  \\
        2 & -4 & 2  \\
        4 & 0  & 0
    \end{bmatrix},
    \vecb=\begin{bmatrix}-1\\1\\1\\-2\end{bmatrix}
\end{equation*}
问线性方程组\(\mata\vecx=\vecb\)是否有解?如果无解,求其最小二乘解。
\end{problem}
\begin{proof}
    分别对\(\mata\)和\(\mat{\mata,\vecb}\)作初等行变换,可得\(\rank{\mata}=3\),\(\rank{\mat{\mata,\vecb}}=4\)。所以线性方程组\(\mata\vecx=\vecb\)无解。

    下面考虑求方程的最小二乘解:令
    \begin{equation*}
        \mata'=\mata^\top\mata=
        \begin{bmatrix}
            25 & -6  & 5   \\
            -6 & 20  & -10 \\
            5  & -10 & 6
        \end{bmatrix}
    \end{equation*}
    \begin{equation*}
        \vecb'=\mata^\top\vecb=\begin{bmatrix}-5&6&4\end{bmatrix}^\top
    \end{equation*}
    解线性方程组\(\mata'\hat{\vecx}=\vecb'\),得到最小二乘解为
    \begin{equation*}
        \hat{\vecx}=
        \begin{bmatrix}
            -\frac{11}{24} & \frac{25}{48} & \frac{23}{12}
        \end{bmatrix}^\top
    \end{equation*}
\end{proof}

% 4.27
\begin{problem}
求向量\(\vecb\)在矩阵\(\mata\)的列空间中的正交投影。
\begin{equation*}
    \mata=
    \begin{bmatrix}
        1 & -2 \\
        1 & 0  \\
        1 & 1  \\
        1 & 3
    \end{bmatrix},
    \vecb=\begin{bmatrix}-4\\-3\\3\\0\end{bmatrix}
\end{equation*}
\end{problem}
\begin{proof}
    分别对\(\mata\)和\(\mat{\mata,\vecb}\)作初等行变换,可得\(\rank{\mata}=2\),\(\rank{\mat{\mata,\vecb}}=3\)。所以\(\vecb\notin\colsp{\mata}\)。

    由例4.4.11可知\(\vecb\)在\(\colsp{\mata}\)上的正交投影为
    \begin{equation*}
        \hat{\vecb}=\mata\pinv{\mata^\top\mata}\mata^\top\vecb=
        \begin{bmatrix}
            -\frac{7}{2} & -\frac{3}{2} & -\frac{1}{2} & \frac{3}{2}
        \end{bmatrix}^\top
    \end{equation*}
\end{proof}

% 4.28
\begin{problem}
设齐次线性方程组
\begin{equation*}
    \begin{cases}
        2x_1+x_2-x_3+x_4-3x_5=0 \\
        x_1+x_2-x_3+x_5=0
    \end{cases}
\end{equation*}
\begin{enumerate}
    \item 求解空间的标准正交基;
    \item 求解空间的正交补空间的标准正交基。
\end{enumerate}
\end{problem}
\begin{proof}
    记该方程组为\(\mata\vecx=\veczero\),其中\(\mata\)为系数矩阵。
    \begin{enumerate}
        \item {
              将\(\mata\)通过初等行变换化为最简阶梯型矩阵:
              \begin{equation*}
                  \begin{bmatrix}
                      1 & 0 & 0  & 1  & -4 \\
                      0 & 1 & -1 & -1 & 5
                  \end{bmatrix}
              \end{equation*}
              从而求得\(\nullsp{\mata}\)的一组基为
              \begin{equation*}
                  \vecal_1=\begin{bmatrix}0\\1\\1\\0\\0\end{bmatrix},
                  \vecal_2=\begin{bmatrix}-1\\1\\0\\1\\0\end{bmatrix},
                  \vecal_3=\begin{bmatrix}4\\-5\\0\\0\\1\end{bmatrix}
              \end{equation*}
              将\(\vecal_1,\vecal_2,\vecal_3\) Schmidt正交化:
              \begin{align*}
                  \vecq_1 & =\frac{\vecal_1}{\Abs{\vecal_1}}=\frac{1}{\sqrt2}\begin{bmatrix}0&1&1&0&0\end{bmatrix}^\top                                                                                                                      \\
                  \vecq_2 & =\frac{\vecal_2-\vecq_1\vecq_1^\top\vecal_2}{\Abs{\vecal_2-\vecq_1\vecq_1^\top\vecal_2}}=\frac{1}{\sqrt{10}}\begin{bmatrix}-2&1&-1&2&0\end{bmatrix}^\top                                                         \\
                  \vecq_3 & =\frac{\vecal_3-\vecq_1\vecq_1^\top\vecal_3-\vecq_2\vecq_2^\top\vecal_3}{\Abs{\vecal_3-\vecq_1\vecq_1^\top\vecal_3-\vecq_2\vecq_2^\top\vecal_3}}=\frac{1}{3\sqrt{35}}\begin{bmatrix}4&-5&0&0&1\end{bmatrix}^\top
              \end{align*}
              \(\vecq_1,\vecq_2,\vecq_3\)构成\(\nullsp{\mata}\)的一组标准正交基。
              }
        \item {
              由推论4.4.1可知\(\nullsp{\mata}\)的正交补空间为\(\colsp{\mata^\top}\),即\(\mata\)的行空间。

              由(1)可知\(\dim\mypar{\colsp{\mata^\top}}=2\),因此\(\colsp{\mata^\top}\)的一组基为
              \begin{equation*}
                  \vecbeta_1=\begin{bmatrix}2&1&-1&1&-3\end{bmatrix}^\top,
                  \vecbeta_2=\begin{bmatrix}1&1&-1&0&1\end{bmatrix}^\top
              \end{equation*}
              同(1)理,将\(\vecbeta_1,\vecbeta_2\) Schmidt正交化后得到\(\colsp{\mata^\top}\)的标准正交基为
              \begin{equation*}
                  \vecp_1=\frac{1}{4}\begin{bmatrix}2&1&-1&1&-3\end{bmatrix}^\top,
                  \vecp_2=\frac{1}{12\sqrt{7}}\begin{bmatrix}14&15&-15&-1&19\end{bmatrix}^\top
              \end{equation*}
              }
    \end{enumerate}
\end{proof}

% 4.29
\begin{problem}
设\(\vecal_1,\vecal_2\)和\(\vecal\)是Euclid空间\(\rea^3\)中的向量,
\begin{equation*}
    \vecal_1=\begin{bmatrix}1\\1\\2\end{bmatrix},
    \vecal_2=\begin{bmatrix}1\\0\\1\end{bmatrix},
    \vecal=\begin{bmatrix}3\\2\\1\end{bmatrix}
\end{equation*}
求\(\myspan{\vecal_1,\vecal_2}\)的正交补空间以及向量\(\vecal\)分别在\(\myspan{\vecal_1,\vecal_2}\)及其正交补空间中的正交投影。
\end{problem}
\begin{proof}
    记\(\mata=\mat{\vecal_1,\vecal_2}\)。
    \begin{itemize}
        \item {
              容易发现\(\vecal_1,\vecal_2\)线性无关,因此它们构成\(\myspan{\vecal_1,\vecal_2}\)的一组基。

              由例4.4.11可知\(\vecal\)在\(\myspan{\vecal_1,\vecal_2}\)中的正交投影为
              \begin{equation*}
                  \mata\pinv{\mata^\top\mata}\mata^\top\vecal=\begin{bmatrix}\frac{5}{3}&\frac{2}{3}&\frac{7}{3}\end{bmatrix}^\top
              \end{equation*}
              }
        \item {
              由推论4.4.1可知\(\myspan{\vecal_1,\vecal_2}\)的正交补空间为\(\nullsp{\mata^\top}\)。

              将\(\mata^\top\)通过初等行变换化为最简阶梯型矩阵\(\begin{bmatrix}1 & 0 & 1 \\0 & 1 & 1\end{bmatrix}\),从而求得\(\nullsp{\mata^\top}\)的基\(\vecbeta_1=\myvec{-1,-1,1}\)。

              所以\(\vecal\)在\(\nullsp{\mata^\top}\)中的正交投影为
              \begin{equation*}
                  \frac{\vecbeta\vecbeta^\top}{\vecbeta^\top\vecbeta}\vecal=\begin{bmatrix}\frac{4}{3}&\frac{4}{3}&-\frac{4}{3}\end{bmatrix}^\top
              \end{equation*}
              }
    \end{itemize}
\end{proof}

% 4.30
\begin{problem}
设\(T\)是线性空间\(\mathrm{K}^{n\times n}\)的如下变换:
\begin{equation*}
    T(\mata)=\matc\mata-\mata\matc \mypar{\matc\in\mathrm{K}^{n\times n},\forall\mata\in\mathrm{K}^{n\times n}}
\end{equation*}
证明:\(T\)是\(\mathrm{K}^{n\times n}\)的线性变换且
\begin{equation*}
    T\mypar{\mata\matb}=T\mypar{\mata}\matb+\mata T\mypar{\matb}
\end{equation*}
\end{problem}
\begin{proof}
\end{proof}

% 4.31
\begin{problem}
已知在\(\rea^3\)中,线性变换\(T\)在基
\begin{equation*}
    \vecveps_1=\begin{bmatrix}8\\-6\\7\end{bmatrix},
    \vecveps_2=\begin{bmatrix}-16\\7\\-13\end{bmatrix},
    \vecveps_3=\begin{bmatrix}9\\-3\\7\end{bmatrix}
\end{equation*}
下的表示矩阵
\begin{equation*}
    \mata=
    \begin{bmatrix}
        1  & -18 & 15 \\
        -1 & -22 & 20 \\
        1  & -25 & 22
    \end{bmatrix}
\end{equation*}
求\(T\)在基
\begin{equation*}
    \veceta_1=\begin{bmatrix}1\\-2\\1\end{bmatrix},
    \veceta_2=\begin{bmatrix}3\\-1\\2\end{bmatrix},
    \veceta_3=\begin{bmatrix}2\\1\\2\end{bmatrix}
\end{equation*}
下的表示矩阵。
\end{problem}
\begin{proof}
    设从基\(\vecveps_1,\vecveps_2,\vecveps_3\)到基\(\veceta_1,\veceta_2,\veceta_3\)的过渡矩阵为\(\matm\),则有
    \begin{equation*}
        \begin{bmatrix}\veceta_1&\veceta_2&\veceta_3\end{bmatrix}=
        \begin{bmatrix}\vecveps_1&\vecveps_2&\vecveps_3\end{bmatrix}\matm
    \end{equation*}
    \begin{equation*}
        \matm=\inv{\begin{bmatrix}\vecveps_1&\vecveps_2&\vecveps_3\end{bmatrix}}\begin{bmatrix}\veceta_1&\veceta_2&\veceta_3\end{bmatrix}=
        \begin{bmatrix}
            1 & 1 & -3 \\
            1 & 2 & -5 \\
            1 & 3 & -6
        \end{bmatrix}
    \end{equation*}
    由定理4.5.4知\(T\)在基\(\veceta_1,\veceta_2,\veceta_3\)下的表示矩阵为
    \begin{equation*}
        \inv{\matm}\mata\matm=
        \begin{bmatrix}
            1 & 2  & 2  \\
            3 & -1 & -2 \\
            2 & -3 & 1
        \end{bmatrix}
    \end{equation*}
\end{proof}

% 4.32
\begin{problem}
在\(P_n[x]\)中,取基为\(\epsilon_0=1\),\(\epsilon_1=\mypar{x-a}\),\(\epsilon_2=\frac{\mypar{x-a}^2}{2!}\),\dots,\(\epsilon_n=\frac{\mypar{x-a}^n}{n!}\)(其中\(a\)为实数),求线性变换\(D\mypar{f\mypar{x}}=f'\mypar{x}\)在此基下的表示矩阵。
\end{problem}
\begin{proof}
\end{proof}

% 4.33
\begin{problem}
给定\(\rea^3\)的两组基:
\begin{equation*}
    \vecveps_1=\begin{bmatrix}1\\0\\1\end{bmatrix},
    \vecveps_2=\begin{bmatrix}2\\1\\0\end{bmatrix},
    \vecveps_3=\begin{bmatrix}1\\1\\1\end{bmatrix}
\end{equation*}
和
\begin{equation*}
    \veceta_1=\begin{bmatrix}1\\2\\-1\end{bmatrix},
    \veceta_2=\begin{bmatrix}2\\2\\-1\end{bmatrix},
    \veceta_3=\begin{bmatrix}2\\-1\\-1\end{bmatrix}
\end{equation*}
定义线性变换:\(T\mypar{\vecveps_i}=\veceta_i\),\(i\in\setof{1,2,3}\)。分别求\(T\)在基\(\vecveps_1,\vecveps_2,\vecveps_3\)与\(\veceta_1,\veceta_2,\veceta_3\)下的表示矩阵。
\end{problem}
\begin{proof}
    设从基\(\vecveps_1,\vecveps_2,\vecveps_3\)到基\(\veceta_1,\veceta_2,\veceta_3\)的过渡矩阵为\(\matm\)。则有
    \begin{equation*}
        \begin{bmatrix}\veceta_1&\veceta_2&\veceta_3\end{bmatrix}=
        \begin{bmatrix}\vecveps_1&\vecveps_2&\vecveps_3\end{bmatrix}\matm
    \end{equation*}
    \begin{equation*}
        \matm=\inv{\begin{bmatrix}\vecveps_1&\vecveps_2&\vecveps_3\end{bmatrix}}\begin{bmatrix}\veceta_1&\veceta_2&\veceta_3\end{bmatrix}=
        \begin{bmatrix}
            -2 & -\frac{3}{2} & \frac{3}{2}  \\
            1  & \frac{3}{2}  & \frac{3}{2}  \\
            1  & \frac{1}{2}  & -\frac{5}{2}
        \end{bmatrix}
    \end{equation*}
    设\(T\)在基\(\vecveps_1,\vecveps_2,\vecveps_3\)下的表示矩阵为\(\mata\)。由题设可知
    \begin{equation*}
        \begin{bmatrix}\veceta_1&\veceta_2&\veceta_3\end{bmatrix}=T\mypar{\vecveps_1,\vecveps_2,\vecveps_3}=\begin{bmatrix}\vecveps_1&\vecveps_2&\vecveps_3\end{bmatrix}\mata
    \end{equation*}
    因此有\(\mata=\matm\)。由定理4.5.4知由定理4.5.4知\(T\)在基\(\veceta_1,\veceta_2,\veceta_3\)下的表示矩阵为\(\inv{\matm}\mata\matm=\matm\)。

    综上,\(T\)在基\(\vecveps_1,\vecveps_2,\vecveps_3\)与\(\veceta_1,\veceta_2,\veceta_3\)下的表示矩阵均为
    \begin{equation*}
        \matm=
        \begin{bmatrix}
            -2 & -\frac{3}{2} & \frac{3}{2}  \\
            1  & \frac{3}{2}  & \frac{3}{2}  \\
            1  & \frac{1}{2}  & -\frac{5}{2}
        \end{bmatrix}
    \end{equation*}
\end{proof}

% 4.34
\begin{problem}
在实数域上所有\(2\)阶矩阵所构成的线性空间\(\rea^{2\times2}\)中,定义变换\(T\mypar{\vecal}=\vecgamma\vecal-\vecal\vecgamma\),其中\(\vecal\)为\(\rea^{2\times2}\)中任意一个元素(矩阵),\(\vecgamma\)为\(\rea^{2\times2}\)中一个固定的元素(矩阵)。
\begin{enumerate}
    \item 证明:\(T\)是\(\rea^{2\times2}\)上的一个线性变换;
    \item 对于\(\rea^{2\times2}\)的一组基
          \begin{equation*}
              \vecal_1=\begin{bmatrix}1&0\\0&0\end{bmatrix},
              \vecal_2=\begin{bmatrix}0&1\\0&0\end{bmatrix},
              \vecal_3=\begin{bmatrix}0&0\\1&0\end{bmatrix},
              \vecal_4=\begin{bmatrix}0&0\\0&1\end{bmatrix}
          \end{equation*}
          求\(T\)在这组基下的表示矩阵。
\end{enumerate}
\end{problem}
\begin{proof}
\end{proof}