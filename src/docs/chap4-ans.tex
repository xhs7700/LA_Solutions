\documentclass{ctexart}
\usepackage{anyfontsize}
\usepackage{hyperref}
\usepackage{graphicx}
\usepackage{amsmath,amsthm,amsfonts,amssymb}
\usepackage{rotating}

\author{夏海淞}
\title{线性代数习题答案}
\date{\today}

\ctexset { section = { name={第,章} } }
\ctexset { section = { number={\chinese {section}} } }

\newtheorem{problem}{习题}[section]
\newtheorem{extraprob}{附加习题}[section]
\newtheorem{suplprob}{补充习题}[section]

\renewcommand{\labelenumi}{(\theenumi)}
\renewcommand{\proofname}{解}

% \newtheorem{theorem}{定理}[subsection]
% \newtheorem{lemma}{引理}[subsection]
% \newtheorem*{definition}{定义}
% \newtheorem{property}{性质}[subsection]
% \newtheorem{infer}{推论}[subsection]

\usepackage{color}

\newcommand{\bsym}[1]{\boldsymbol{#1}}
\newcommand{\mypar}[1]{\left( #1 \right)}
\newcommand{\gram}[1]{\bsym{G}\mypar{#1}}
\newcommand{\abs}[1]{\left|#1 \right|}
\newcommand{\Abs}[1]{\left\Vert#1\right\Vert}
\newcommand{\setof}[1]{\left\{#1 \right\}}
\newcommand{\indot}[2]{\left\langle #1,#2 \right\rangle}
\newcommand{\mat}[1]{\left[ #1 \right]}
\newcommand{\myvec}[1]{\left[ #1 \right]^\top}
\newcommand{\sqmat}[3]{\begin{#1}
{#2}_{11} & {#2}_{12} & \cdots & {#2}_{1{#3}} \\
{#2}_{21} & {#2}_{22} & \cdots & {#2}_{2{#3}} \\
\vdots & \vdots &   \ddots     & \vdots \\
{#2}_{{#3}1} & {#2}_{{#3}2} & \cdots & {#2}_{{#3}{#3}}
\end{#1}}
\newcommand{\normmat}[4]{\begin{#1}
{#2}_{11} & {#2}_{12} & \cdots & {#2}_{1{#4}} \\
{#2}_{21} & {#2}_{22} & \cdots & {#2}_{2{#4}} \\
\vdots & \vdots &   \ddots     & \vdots \\
{#2}_{{#3}1} & {#2}_{{#3}2} & \cdots & {#2}_{{#3}{#4}}
\end{#1}}
\newcommand{\nullsp}[1]{\bsym{\mathrm{N}}\mypar{#1}}
\newcommand{\colsp}[1]{\bsym{\mathrm{C}}\mypar{#1}}
\newcommand{\func}[2]{\mathrm{#1}\mypar{#2}}
\newcommand{\entry}[3]{\func{entry}{#1,#2,#3}}
\newcommand{\row}[2]{\func{row}{#1,#2}}
\newcommand{\col}[2]{\func{col}{#1,#2}}
\newcommand{\trace}[1]{\func{Tr}{#1}}
\newcommand{\diag}[1]{\func{diag}{#1}}
\newcommand{\rank}[1]{\func{rank}{#1}}
\newcommand{\adj}[1]{\func{adj}{#1}}
\newcommand{\myspan}[1]{\func{span}{#1}}
\newcommand{\enums}[2]{{#1}_1,{#1}_2,\dots,{#1}_{#2}}
\newcommand{\inv}[1]{{#1}^{-1}}
\newcommand{\pinv}[1]{\inv{\mypar{#1}}}
\newcommand{\ortcom}[1]{{#1}^{\bot}}

\renewcommand{\det}[1]{\func{det}{#1}}

\newcommand{\dif}{\mathrm{d}}
\newcommand{\fracdif}[1]{\frac{\dif}{\dif #1}}

\newcommand{\todo}[1]{{ \textcolor{red}{ TODO: #1}}}

\newcommand{\mata}{\bsym{A}}
\newcommand{\matb}{\bsym{B}}
\newcommand{\matc}{\bsym{C}}
\newcommand{\matd}{\bsym{D}}
\newcommand{\mate}{\bsym{E}}
\newcommand{\matf}{\bsym{F}}
\newcommand{\matfstar}{\bsym{F^*}}
\newcommand{\matg}{\bsym{G}}
\newcommand{\matH}{\bsym{H}}
\newcommand{\mati}{\bsym{I}}
\newcommand{\matj}{\bsym{J}}
\newcommand{\matk}{\bsym{K}}
\newcommand{\matl}{\bsym{L}}
\newcommand{\matlam}{\bsym{\Lambda}}
\newcommand{\matm}{\bsym{M}}
\newcommand{\matn}{\bsym{N}}
\newcommand{\mato}{\bsym{O}}
\newcommand{\matp}{\bsym{P}}
\newcommand{\matq}{\bsym{Q}}
\newcommand{\matr}{\bsym{R}}
\newcommand{\mats}{\bsym{S}}
\newcommand{\matsig}{\bsym{\Sigma}}
\newcommand{\matt}{\bsym{T}}
\newcommand{\matu}{\bsym{U}}
\newcommand{\matv}{\bsym{V}}
\newcommand{\matw}{\bsym{W}}
\newcommand{\matx}{\bsym{X}}
\newcommand{\maty}{\bsym{Y}}
\newcommand{\matz}{\bsym{Z}}

\newcommand{\field}{\bsym{\mathrm{F}}}
\newcommand{\rea}{\bsym{\mathbb{R}}}

\newcommand{\veca}{\bsym{a}}
\newcommand{\vecal}{\bsym{\alpha}}
\newcommand{\vecb}{\bsym{b}}
\newcommand{\vecbeta}{\bsym{\beta}}
\newcommand{\vecc}{\bsym{c}}
\newcommand{\vecd}{\bsym{d}}
\newcommand{\vecdelta}{\bsym{\delta}}
\newcommand{\vece}{\bsym{e}}
\newcommand{\veceps}{\bsym{\epsilon}}
\newcommand{\vecveps}{\bsym{\varepsilon}}
\newcommand{\vecf}{\bsym{f}}
\newcommand{\vecg}{\bsym{g}}
\newcommand{\vecgamma}{\bsym{\gamma}}
\newcommand{\vech}{\bsym{h}}
\newcommand{\veceta}{\bsym{\eta}}
\newcommand{\veci}{\bsym{i}}
\newcommand{\vecj}{\bsym{j}}
\newcommand{\veck}{\bsym{k}}
\newcommand{\vecl}{\bsym{l}}
\newcommand{\vecm}{\bsym{m}}
\newcommand{\vecn}{\bsym{n}}
\newcommand{\veco}{\bsym{o}}
\newcommand{\vecone}{\bsym{1}}
\newcommand{\vecp}{\bsym{p}}
\newcommand{\vecq}{\bsym{q}}
\newcommand{\vecr}{\bsym{r}}
\newcommand{\vecs}{\bsym{s}}
\newcommand{\vect}{\bsym{t}}
\newcommand{\vecu}{\bsym{u}}
\newcommand{\vecv}{\bsym{v}}
\newcommand{\vecw}{\bsym{w}}
\newcommand{\vecx}{\bsym{x}}
\newcommand{\vecxi}{\bsym{\xi}}
\newcommand{\vecy}{\bsym{y}}
\newcommand{\vecz}{\bsym{z}}
\newcommand{\veczero}{\bsym{0}}


\begin{document}

\setcounter{section}{3}
\section{线性空间与线性变换}

% 4.2
\setcounter{problem}{1}
\begin{problem}
证明:以下三个多项式为\(P_2[x]\)的一组基:
\begin{equation*}
    f_1=1,f_2=x-1,f_3=\spar{x-1}^2
\end{equation*}
再求\(g\spar{x}=5x^2+x+3\)在此基下的坐标。
\end{problem}
\begin{proof}
    \(\vecf_1,\vecf_2,\vecf_3\)为\(P_2[x]\)的一组基,\(g\spar{x}\)在此基下的坐标为\(\myvec{9,11,5}\)。
\end{proof}

% 4.3
\begin{problem}
在次数不大于\(3\)的多项式空间\(P_3[x]\)中,
\begin{enumerate}
    \item 求由基\(1,x,x^2,x^3\)到基\(1,1+x,\spar{1+x}^2,\spar{1+x}^3\)的过渡矩阵;
    \item 求\(f\spar{x}=a_0+a_1x+a_2x^2+a_3x^3\)在基\(1,1+x,\spar{1+x}^2,\spar{1+x}^3\)下的坐标。
\end{enumerate}
\end{problem}
\begin{proof}
    \begin{enumerate}
        \item {
              过渡矩阵为\(\matm=
              \begin{bmatrix}
                  1 & 1 & 1 & 1 \\
                  0 & 1 & 2 & 3 \\
                  0 & 0 & 1 & 3 \\
                  0 & 0 & 0 & 1
              \end{bmatrix}\);
              }
        \item {
              令\(\veca=\myvec{a_0,a_1,a_2,a_3}\),则\(f\spar{x}\)在基\(1,1+x,\spar{1+x}^2,\spar{1+x}^3\)下的坐标为
              \begin{equation*}
                  \inv{\matm}\veca=
                  \begin{bmatrix}
                      a_0-a_1+a_2-a_3 \\
                      a_1-2a_2+3a_3   \\
                      a_2-3a_3        \\
                      a_3
                  \end{bmatrix}
              \end{equation*}
              }
    \end{enumerate}
\end{proof}

% 4.4
\begin{problem}
在\(P_3[x]\)的多项式空间中,旧基为\(1,x,x^2,x^3\);新基为\(1\),\(1+x\),\(1+x+x^2\),\(1+x+x^2+x^3\)。
\begin{enumerate}
    \item 求旧基到新基的过渡矩阵;
    \item 求多项式\(1+2x+3x^2+4x^3\)在新基下的坐标;
    \item 若多项式\(f\spar{x}\)在新基下的坐标为\(\myvec{1,2,3,4}\),求它在旧基下的坐标。
\end{enumerate}
\end{problem}
\begin{proof}
    \begin{enumerate}
        \item 过渡矩阵为\(\matm=\begin{bmatrix}1&1&1&1\\0&1&1&1\\0&0&1&1\\0&0&0&1\end{bmatrix}\);
        \item 该多项式在新基下的坐标为\(\inv{\matm}\vecx=\myvec{-1,-1,-1,4}\);
        \item 该多项式在旧基下的坐标为\(\matm\vecx=\myvec{10,9,7,4}\)。
    \end{enumerate}
\end{proof}

% 4.5
\begin{problem}
已知\(\vecxi\)在基\(\matb_1=\setof{\vecal_1,\vecal_2,\vecal_3}\)下的坐标为\(\vecxi_{\matb_1}=\myvec{1,-2,2}\),求\(\vecxi\)在基\(\matb_2=\setof{\vecbeta_1,\vecbeta_2,\vecbeta_3}\)下的坐标\(\vecxi_{\matb_2}\),其中\(\vecbeta_1=\vecal_1+\vecal_2\),\(\vecbeta_2=\vecal_2+\vecal_3\),\(\vecbeta_3=\vecal_3+\vecal_1\)。
\end{problem}
\begin{proof}
    \begin{equation*}
        \vecxi_{\matb_2}=\inv{\matm}\vecxi_{\matb_1}=\begin{bmatrix}-1.5&-0.5&2.5\end{bmatrix}^\top
    \end{equation*}
\end{proof}

% 4.6
\begin{problem}
设\(\vecveps_1,\vecveps_2,\vecveps_3\)是线性空间\(V\)的一组基,且
\begin{equation*}
    \begin{cases}
        \vecxi_1=\vecveps_1+\vecveps_3 \\
        \vecxi_2=\vecveps_2            \\
        \vecxi_3=\vecveps_1+2\vecveps_2+2\vecveps_3
    \end{cases}
\end{equation*}
\begin{equation*}
    \begin{cases}
        \veceta_1=\vecveps_1            \\
        \veceta_2=\vecveps_1+\vecveps_2 \\
        \veceta_3=\vecveps_1+\vecveps_2+\vecveps_3
    \end{cases}
\end{equation*}
\begin{enumerate}
    \item 试证\(\vecxi_1,\vecxi_2,\vecxi_3\)及\(\veceta_1,\veceta_2,\veceta_3\)都是\(V\)的一组基;
    \item 求由基\(\vecxi_1,\vecxi_2,\vecxi_3\)到基\(\veceta_1,\veceta_2,\veceta_3\)的过渡矩阵。
\end{enumerate}
\end{problem}
\begin{proof}
    \begin{enumerate}
        \item[(2)] {
              由基\(\vecxi_1,\vecxi_2,\vecxi_3\)到基\(\veceta_1,\veceta_2,\veceta_3\)的过渡矩阵为
              \begin{equation*}
                  \begin{bmatrix}
                      2  & 2  & 1 \\
                      2  & 3  & 1 \\
                      -1 & -1 & 0
                  \end{bmatrix}
              \end{equation*}
              }
    \end{enumerate}
\end{proof}

% 4.7
\begin{problem}
在线性空间\(\rea^{2\times2}\)中,已知
\begin{equation*}
    \vecal_1=\begin{bmatrix}1&0\\0&0\end{bmatrix},
    \vecal_2=\begin{bmatrix}0&1\\0&0\end{bmatrix},
    \vecal_3=\begin{bmatrix}0&0\\1&0\end{bmatrix},
    \vecal_4=\begin{bmatrix}0&0\\0&1\end{bmatrix}
\end{equation*}
为其一组基,若\(\rea^{2\times2}\)的另一组基为\(\vecbeta_1,\vecbeta_2,\vecbeta_3,\vecbeta_4\),由\(\vecal_1,\vecal_2,\vecal_3,\vecal_4\)到\(\vecbeta_1,\vecbeta_2,\vecbeta_3,\vecbeta_4\)的过渡矩阵为
\begin{equation*}
    \mata=
    \begin{bmatrix}
        0 & 1 & 1 & 1 \\
        1 & 0 & 1 & 1 \\
        1 & 1 & 0 & 1 \\
        1 & 1 & 1 & 0
    \end{bmatrix}
\end{equation*}
求:
\begin{enumerate}
    \item 基\(\vecbeta_1,\vecbeta_2,\vecbeta_3,\vecbeta_4\);
    \item 矩阵\begin{equation*}\begin{bmatrix}0&1\\2&-3\end{bmatrix}\end{equation*}在基\(\vecbeta_1,\vecbeta_2,\vecbeta_3,\vecbeta_4\)下的坐标。
\end{enumerate}
\end{problem}
\begin{proof}
    \begin{enumerate}
        \item {
              由题设可知
              \begin{align*}
                  \vecbeta_1 & =\begin{bmatrix}0&1\\1&1\end{bmatrix},
                  \vecbeta_2=\begin{bmatrix}1&0\\1&1\end{bmatrix}     \\
                  \vecbeta_3 & =\begin{bmatrix}1&1\\0&1\end{bmatrix},
                  \vecbeta_4=\begin{bmatrix}1&1\\1&0\end{bmatrix}
              \end{align*}
              }
        \item {
              该矩阵在基\(\vecbeta_1,\vecbeta_2,\vecbeta_3,\vecbeta_4\)下的坐标为
              \begin{equation*}
                  \begin{bmatrix}0&-1&-2&3\end{bmatrix}^\top
              \end{equation*}
              }
    \end{enumerate}
\end{proof}

% 4.8
\begin{problem}
已知实数域\(\rea\)上的所有\(2\)阶矩阵,对于矩阵的加法和数乘,构成\(\rea\)上的四维线性空间,记作\(V=\rea^{2\times2}\)。
\begin{enumerate}
    \item {
          分别证明
          \begin{equation*}
              \vecal_1=\begin{bmatrix}1&0\\0&0\end{bmatrix},
              \vecal_2=\begin{bmatrix}1&1\\0&0\end{bmatrix},
              \vecal_3=\begin{bmatrix}1&1\\1&0\end{bmatrix},
              \vecal_4=\begin{bmatrix}1&1\\1&1\end{bmatrix}
          \end{equation*}
          与
          \begin{equation*}
              \vecbeta_1=\begin{bmatrix}-1&1\\1&1\end{bmatrix},
              \vecbeta_2=\begin{bmatrix}1&-1\\1&1\end{bmatrix},
              \vecbeta_3=\begin{bmatrix}1&1\\-1&1\end{bmatrix},
              \vecbeta_4=\begin{bmatrix}1&1\\1&-1\end{bmatrix}
          \end{equation*}
          均为\(\rea^{2\times2}\)的基;}
    \item 对于\(\rea^{2\times2}\)中的任意元素\(\vecal\),求\(\vecal\)在基\(\vecal_1,\vecal_2,\vecal_3,\vecal_4\)下的坐标;
    \item 求由基\(\vecal_1,\vecal_2,\vecal_3,\vecal_4\)到\(\vecbeta_1,\vecbeta_2,\vecbeta_3,\vecbeta_4\)的过渡矩阵。
\end{enumerate}
\end{problem}
\begin{proof}
    \begin{enumerate}
        \item[(2)] {
              设\(\vecal=\begin{bmatrix}\alpha_{11}&\alpha_{12}\\\alpha_{21}&\alpha_{22}\end{bmatrix}\),则\(\vecal\)在基\(\vecveps_1,\vecveps_2,\vecveps_3,\vecveps_4\)下的坐标为\(\myvec{\alpha_{11},\alpha_{12},\alpha_{21},\alpha_{22}}\)。则\(\vecal\)在基\(\vecal_1,\vecal_2,\vecal_3,\vecal_4\)下的坐标为
              \begin{equation*}
                  \inv{\mata}\begin{bmatrix}\alpha_{11}\\\alpha_{12}\\\alpha_{21}\\\alpha_{22}\end{bmatrix}=
                  \begin{bmatrix}\alpha_{11}-\alpha_{12}\\\alpha_{12}-\alpha_{21}\\\alpha_{21}-\alpha_{22}\\\alpha_{22}\end{bmatrix}
              \end{equation*}
              }
        \item[(3)] {
              由基\(\vecal_1,\vecal_2,\vecal_3,\vecal_4\)到\(\vecbeta_1,\vecbeta_2,\vecbeta_3,\vecbeta_4\)的过渡矩阵为
              \begin{equation*}
                  \begin{bmatrix}
                      -2 & 2  & 0  & 0  \\
                      0  & -2 & 2  & 0  \\
                      0  & 0  & -2 & 2  \\
                      1  & 1  & 1  & -1
                  \end{bmatrix}
              \end{equation*}
              }
    \end{enumerate}
\end{proof}

% 4.9
\begin{problem}
求下列两个齐次线性方程组的解空间的基和维数:
\begin{enumerate}
    \item \(x_1+x_2+\cdots+x_n=0\);
    \item \begin{equation*}
              \begin{cases}
                  2x_1-4x_2+5x_3+3x_4=0 \\
                  3x_1-6x_2+4x_3+2x_4=0 \\
                  4x_1-8x_2+17x_3+11x_4=0
              \end{cases}
          \end{equation*}
\end{enumerate}
\end{problem}
\begin{proof}
    \begin{enumerate}
        \item {
              解空间的基为
              \begin{equation*}
                  \begin{bmatrix}-1\\1\\0\\0\\\vdots\\0\\0\end{bmatrix},
                  \begin{bmatrix}-1\\0\\1\\0\\\vdots\\0\\0\end{bmatrix},\cdots
                  \begin{bmatrix}-1\\0\\0\\0\\\vdots\\0\\1\end{bmatrix}
              \end{equation*}
              维数为\(n-1\);
              }
        \item {
              解空间的基为\(\myvec{2,1,0,0}\),\(\myvec{2,0,-5,7}\),维数为\(2\)。
              }
    \end{enumerate}
\end{proof}

% 4.10
\begin{problem}
设\(\enums{\vecal}{n}\)是\(n\)维线性空间\(V\)的一组基,又\(V\)中向量\(\vecal_{n+1}\)在这组基下坐标\(\spar{\enums{x}{n}}\)全都不为零。

证明\(\enums{\vecal}{n},\vecal_{n+1}\)中任意\(n\)个向量必构成\(V\)的一组基,并求\(\vecal_1\)在基\(\vecal_2,\dots,\vecal_n,\vecal_{n+1}\)下的坐标。
\end{problem}
\begin{proof}
    \(\vecal_1\)在在基\(\vecal_2,\dots,\vecal_n,\vecal_{n+1}\)下的坐标为\(\myvec{-\frac{x_2}{x_1},\cdots,-\frac{x_n}{x_1},\frac{1}{x_1}}\)。
\end{proof}

\setcounter{problem}{11}
% 4.12
\begin{problem}
设\(V_1,V_2\)是\(\rea^n\)的两个非平凡子空间,证明:在\(\rea^n\)中存在向量\(\vecal\),使\(\vecal\notin V_1\),且\(\vecal\notin V_2\),并在\(\rea^3\)中举例说明此结论。
\end{problem}
\begin{proof}
    举例:\(V_1=\myspan{\myvec{1,0,0},\myvec{0,1,0}}\),\(V_2=\myspan{\myvec{0,1,0},\myvec{0,0,1}}\),\(\vecal=\myvec{1,0,1}\)。
\end{proof}

\setcounter{problem}{13}
% 4.14
\begin{problem}
若
\begin{equation*}
    \mata=\begin{bmatrix}1&1&1&0\\2&1&0&1\end{bmatrix}
\end{equation*}
求\(\mata\)的零空间\(\nullsp{\mata}\)的一组基。
\end{problem}
\begin{proof}
    \(\nullsp{\mata}\)的一组基为\(\myvec{1,-2,1,0}\),\(\mat{-1,1,0,1}\)。
\end{proof}

\setcounter{problem}{15}
% 4.16
\begin{problem}
已知向量组\(\vecal_1=\myvec{2,0,1,3,-1}\),\(\vecal_2=\mat{0,-2,1,5,-3}^\top\),\(\vecbeta_1=\mat{1,1,0,-1,1}^\top\),\(\vecbeta_2=\mat{1,-3,2,0,5}\),且\(W_1=\myspan{\vecal_1,\vecal_2}\),\(W_2=\myspan{\vecbeta_1,\vecbeta_2}\)。试求\(W_1\cap W_2\)与\(W_1+W_2\)的维数以及各自的一组基。
\end{problem}
\begin{proof}
    \(W_1+W_2\)的维数为\(3\),一组基为\(\vecal_1,\vecal_2,\vecbeta_2\)。

    \(W_1\cap W_2\)的维数为\(1\),一组基为\(\vecbeta_1\)。
\end{proof}

\setcounter{problem}{21}
% 4.22
\begin{problem}
试把\(\myvec{1,0,1,0}\),\(\mat{0,1,0,2}\)扩充成为\(\rea^4\)的一组标准正交基。
\end{problem}
\begin{proof}
    \(\rea^4\)的一组标准正交基为:

    \begin{equation*}
        \begin{bmatrix}
            \vecq_1 & \vecq_2 & \vecq_3 & \vecq_4
        \end{bmatrix}=
        \begin{bmatrix}
            \frac{\sqrt{2}}{2} & 0                   & -\frac{\sqrt{2}}{2} & 0                    \\
            0                  & \frac{\sqrt{5}}{5}  & 0                   & -\frac{2\sqrt{5}}{5} \\
            \frac{\sqrt{2}}{2} & 0                   & \frac{\sqrt{2}}{2}  & 0                    \\
            0                  & \frac{2\sqrt{5}}{5} & 0                   & \frac{\sqrt{5}}{5}
        \end{bmatrix}
    \end{equation*}
\end{proof}

% 4.23
\begin{problem}
已知\(\vecf_1=1\),\(\vecf_2=x-1\),\(\vecf_3=\spar{x-1}^2\)为三维内积空间\(C\mat{0,1}\)的一组基,内积定义为
\begin{equation*}
    \indot{\vecf}{\vecg}=\int_0^1\vecf\spar{x}\vecg\spar{x}\mathrm{d}x
\end{equation*}
利用Schmidt正交化方法,求与\(\vecf_1,\vecf_2,\vecf_3\)等价的一组标准正交基(两个向量组可相互线性表出称为等价)。
\end{problem}
\begin{proof}
    \begin{align*}
        \vecg_1 & =\frac{\vecf_1}{\Abs{\vecf_1}}=\frac{\vecf_1}{\indot{\vecf_1}{\vecf_1}}=1                                                                                      \\
        \vecg_2 & =\frac{\vecf_2-\indot{\vecg_1}{\vecf_2}\vecg_1}{\Abs{\vecf_2-\indot{\vecg_1}{\vecf_2}\vecg_1}}                                                                 \\
                & =\frac{x-\frac{1}{2}}{\sqrt{\frac{1}{12}}}=\sqrt3\spar{2x-1}                                                                                                   \\
        \vecg_3 & =\frac{\vecf_3-\indot{\vecg_1}{\vecf_3}\vecg_1-\indot{\vecg_2}{\vecf_3}\vecg_2}{\Abs{\vecf_3-\indot{\vecg_1}{\vecf_3}\vecg_1-\indot{\vecg_2}{\vecf_3}\vecg_2}} \\
                & =\frac{x^2-x+\frac{1}{6}}{\sqrt{\frac{1}{180}}}=\sqrt5\spar{6x^2-6x+1}
    \end{align*}
\end{proof}

\setcounter{problem}{25}
% 4.26
\begin{problem}
设
\begin{equation*}
    \mata=
    \begin{bmatrix}
        1 & 2  & -1 \\
        2 & 0  & 1  \\
        2 & -4 & 2  \\
        4 & 0  & 0
    \end{bmatrix},
    \vecb=\begin{bmatrix}-1\\1\\1\\-2\end{bmatrix}
\end{equation*}
问线性方程组\(\mata\vecx=\vecb\)是否有解?如果无解,求其最小二乘解。
\end{problem}
\begin{proof}
    线性方程组\(\mata\vecx=\vecb\)无解。

    该方程组的最小二乘解为
    \begin{equation*}
        \hat{\vecx}=
        \begin{bmatrix}
            -\frac{11}{24} & \frac{25}{48} & \frac{23}{12}
        \end{bmatrix}^\top
    \end{equation*}
\end{proof}

% 4.27
\begin{problem}
求向量\(\vecb\)在矩阵\(\mata\)的列空间中的正交投影。
\begin{equation*}
    \mata=
    \begin{bmatrix}
        1 & -2 \\
        1 & 0  \\
        1 & 1  \\
        1 & 3
    \end{bmatrix},
    \vecb=\begin{bmatrix}-4\\-3\\3\\0\end{bmatrix}
\end{equation*}
\end{problem}
\begin{proof}
    \(\vecb\notin\colsp{\mata}\)。

    \(\vecb\)在\(\colsp{\mata}\)上的正交投影为
    \begin{equation*}
        \hat{\vecb}=\mata\pinv{\mata^\top\mata}\mata^\top\vecb=
        \begin{bmatrix}
            -\frac{7}{2} & -\frac{3}{2} & -\frac{1}{2} & \frac{3}{2}
        \end{bmatrix}^\top
    \end{equation*}
\end{proof}

% 4.28
\begin{problem}
设齐次线性方程组
\begin{equation*}
    \begin{cases}
        2x_1+x_2-x_3+x_4-3x_5=0 \\
        x_1+x_2-x_3+x_5=0
    \end{cases}
\end{equation*}
\begin{enumerate}
    \item 求解空间的标准正交基;
    \item 求解空间的正交补空间的标准正交基。
\end{enumerate}
\end{problem}
\begin{proof}
    记该方程组为\(\mata\vecx=\veczero\),其中\(\mata\)为系数矩阵。
    \begin{enumerate}
        \item {
              \(\nullsp{\mata}\)的一组基为
              \begin{equation*}
                  \vecal_1=\begin{bmatrix}0\\1\\1\\0\\0\end{bmatrix},
                  \vecal_2=\begin{bmatrix}-1\\1\\0\\1\\0\end{bmatrix},
                  \vecal_3=\begin{bmatrix}4\\-5\\0\\0\\1\end{bmatrix}
              \end{equation*}
              将\(\vecal_1,\vecal_2,\vecal_3\) Schmidt正交化:
              \begin{align*}
                  \vecq_1 & =\frac{\vecal_1}{\Abs{\vecal_1}}=\frac{1}{\sqrt2}\begin{bmatrix}0&1&1&0&0\end{bmatrix}^\top                                                                                                                      \\
                  \vecq_2 & =\frac{\vecal_2-\vecq_1\vecq_1^\top\vecal_2}{\Abs{\vecal_2-\vecq_1\vecq_1^\top\vecal_2}}=\frac{1}{\sqrt{10}}\begin{bmatrix}-2&1&-1&2&0\end{bmatrix}^\top                                                         \\
                  \vecq_3 & =\frac{\vecal_3-\vecq_1\vecq_1^\top\vecal_3-\vecq_2\vecq_2^\top\vecal_3}{\Abs{\vecal_3-\vecq_1\vecq_1^\top\vecal_3-\vecq_2\vecq_2^\top\vecal_3}}=\frac{1}{3\sqrt{35}}\begin{bmatrix}4&-5&0&0&1\end{bmatrix}^\top
              \end{align*}
              \(\vecq_1,\vecq_2,\vecq_3\)构成\(\nullsp{\mata}\)的一组标准正交基。
              }
        \item {
              \(\nullsp{\mata}\)的正交补空间为\(\colsp{\mata^\top}\),\(\colsp{\mata^\top}\)的一组基为
              \begin{equation*}
                  \vecbeta_1=\begin{bmatrix}2&1&-1&1&-3\end{bmatrix}^\top,
                  \vecbeta_2=\begin{bmatrix}1&1&-1&0&1\end{bmatrix}^\top
              \end{equation*}
              将\(\vecbeta_1,\vecbeta_2\) Schmidt正交化后得到\(\colsp{\mata^\top}\)的标准正交基为
              \begin{equation*}
                  \vecp_1=\frac{1}{4}\begin{bmatrix}2&1&-1&1&-3\end{bmatrix}^\top,
                  \vecp_2=\frac{1}{12\sqrt{7}}\begin{bmatrix}14&15&-15&-1&19\end{bmatrix}^\top
              \end{equation*}
              }
    \end{enumerate}
\end{proof}

% 4.29
\begin{problem}
设\(\vecal_1,\vecal_2\)和\(\vecal\)是Euclid空间\(\rea^3\)中的向量,
\begin{equation*}
    \vecal_1=\begin{bmatrix}1\\1\\2\end{bmatrix},
    \vecal_2=\begin{bmatrix}1\\0\\1\end{bmatrix},
    \vecal=\begin{bmatrix}3\\2\\1\end{bmatrix}
\end{equation*}
求\(\myspan{\vecal_1,\vecal_2}\)的正交补空间以及向量\(\vecal\)分别在\(\myspan{\vecal_1,\vecal_2}\)及其正交补空间中的正交投影。
\end{problem}
\begin{proof}
    记\(\mata=\mat{\vecal_1,\vecal_2}\)。
    \begin{itemize}
        \item {
              \(\vecal\)在\(\myspan{\vecal_1,\vecal_2}\)中的正交投影为
              \begin{equation*}
                  \mata\pinv{\mata^\top\mata}\mata^\top\vecal=\begin{bmatrix}\frac{5}{3}&\frac{2}{3}&\frac{7}{3}\end{bmatrix}^\top
              \end{equation*}
              }
        \item {
              由推论4.4.1可知\(\myspan{\vecal_1,\vecal_2}\)的正交补空间为\(\nullsp{\mata^\top}\)。

              \(\nullsp{\mata^\top}\)的基为\(\vecbeta=\myvec{-1,-1,1}\)。

              所以\(\vecal\)在\(\nullsp{\mata^\top}\)中的正交投影为
              \begin{equation*}
                  \frac{\vecbeta\vecbeta^\top}{\vecbeta^\top\vecbeta}\vecal=\begin{bmatrix}\frac{4}{3}&\frac{4}{3}&-\frac{4}{3}\end{bmatrix}^\top
              \end{equation*}
              }
    \end{itemize}
\end{proof}

\setcounter{problem}{30}
% 4.31
\begin{problem}
已知在\(\rea^3\)中,线性变换\(T\)在基
\begin{equation*}
    \vecveps_1=\begin{bmatrix}8\\-6\\7\end{bmatrix},
    \vecveps_2=\begin{bmatrix}-16\\7\\-13\end{bmatrix},
    \vecveps_3=\begin{bmatrix}9\\-3\\7\end{bmatrix}
\end{equation*}
下的表示矩阵
\begin{equation*}
    \mata=
    \begin{bmatrix}
        1  & -18 & 15 \\
        -1 & -22 & 20 \\
        1  & -25 & 22
    \end{bmatrix}
\end{equation*}
求\(T\)在基
\begin{equation*}
    \veceta_1=\begin{bmatrix}1\\-2\\1\end{bmatrix},
    \veceta_2=\begin{bmatrix}3\\-1\\2\end{bmatrix},
    \veceta_3=\begin{bmatrix}2\\1\\2\end{bmatrix}
\end{equation*}
下的表示矩阵。
\end{problem}
\begin{proof}
    \(T\)在基\(\veceta_1,\veceta_2,\veceta_3\)下的表示矩阵为
    \begin{equation*}
        \begin{bmatrix}
            1 & 2  & 2  \\
            3 & -1 & -2 \\
            2 & -3 & 1
        \end{bmatrix}
    \end{equation*}
\end{proof}

\setcounter{problem}{32}
% 4.33
\begin{problem}
给定\(\rea^3\)的两组基:
\begin{equation*}
    \vecveps_1=\begin{bmatrix}1\\0\\1\end{bmatrix},
    \vecveps_2=\begin{bmatrix}2\\1\\0\end{bmatrix},
    \vecveps_3=\begin{bmatrix}1\\1\\1\end{bmatrix}
\end{equation*}
和
\begin{equation*}
    \veceta_1=\begin{bmatrix}1\\2\\-1\end{bmatrix},
    \veceta_2=\begin{bmatrix}2\\2\\-1\end{bmatrix},
    \veceta_3=\begin{bmatrix}2\\-1\\-1\end{bmatrix}
\end{equation*}
定义线性变换:\(T\spar{\vecveps_i}=\veceta_i\),\(i\in\setof{1,2,3}\)。分别求\(T\)在基\(\vecveps_1,\vecveps_2,\vecveps_3\)与\(\veceta_1,\veceta_2,\veceta_3\)下的表示矩阵。
\end{problem}
\begin{proof}
    \(T\)在基\(\vecveps_1,\vecveps_2,\vecveps_3\)与\(\veceta_1,\veceta_2,\veceta_3\)下的表示矩阵均为
    \begin{equation*}
        \begin{bmatrix}
            -2 & -\frac{3}{2} & \frac{3}{2}  \\
            1  & \frac{3}{2}  & \frac{3}{2}  \\
            1  & \frac{1}{2}  & -\frac{5}{2}
        \end{bmatrix}
    \end{equation*}
\end{proof}

\end{document}