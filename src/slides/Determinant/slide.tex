\documentclass[9pt,xcolor=svgnames]{beamer} %Beamer
\usepackage{palatino} %font type
\usefonttheme{metropolis} %Type of slides
\usefonttheme[onlymath]{serif} %font type Mathematical expressions
\usetheme[progressbar=frametitle,numbering=counter]{metropolis} %This adds a bar at the beginning of each section.
\useoutertheme{infolines} %Circles in the top of each frame, showing the slide of each section you are at

\usepackage{xeCJK}
\setCJKmainfont{Noto Serif CJK SC}
\setCJKsansfont[BoldFont=Noto Sans CJK SC]{Noto Sans CJK SC Light}

\usepackage{appendixnumberbeamer} %enumerate each slide without counting the appendix
\setbeamercolor{progress bar}{fg=Maroon!70!Coral} %These are the colours of the progress bar. Notice that the names used are the svgnames
\setbeamercolor{title separator}{fg=DarkSalmon} %This is the line colour in the title slide
\setbeamercolor{structure}{fg=black} %Colour of the text of structure, numbers, items, blah. Not the big text.
\setbeamercolor{normal text}{fg=black!87} %Colour of normal text
\setbeamercolor{alerted text}{fg=DarkRed!60!Gainsboro} %Color of the alert box
\setbeamercolor{example text}{fg=Maroon!70!Coral} %Colour of the Example block text


\setbeamercolor{palette primary}{bg=NavyBlue!50!DarkOliveGreen, fg=white} %These are the colours of the background. Being this the main combination and so one. 
\setbeamercolor{palette secondary}{bg=NavyBlue!50!DarkOliveGreen, fg=white}
\setbeamercolor{palette tertiary}{bg=NavyBlue!40!Black, fg= white}
\setbeamercolor{section in toc}{fg=NavyBlue!40!Black} %Color of the text in the table of contents (toc)

%These next packages are the useful for Physics in general, you can add the extras here. 
\usepackage{amsmath,amssymb}
\usepackage{slashed}
\usepackage{cite}
\usepackage{relsize}
\usepackage{caption}
\usepackage{multicol}
\usepackage{multirow}
\usepackage{booktabs}
\usepackage[scale=2]{ccicons}
\usepackage{pgfplots}
\usepgfplotslibrary{dateplot}
\usepackage{geometry}
\usepackage{xspace}
\usepackage{graphicx,subfig,float}
\usepackage{color}
\newcommand{\themename}{\textbf{\textsc{bluetemp}\xspace}}%metropolis}}\xspace}

% \newtheorem{theorem}{定理}[subsection]
% \newtheorem{lemma}{引理}[subsection]
% \newtheorem*{definition}{定义}
% \newtheorem{property}{性质}[subsection]
% \newtheorem{infer}{推论}[subsection]
\newtheorem{problem}{习题}[section]

\newcommand{\bsym}[1]{\boldsymbol{#1}}
\newcommand{\mypar}[1]{\left( #1 \right)}
\newcommand{\abs}[1]{\left|#1 \right|}
\newcommand{\Abs}[1]{\left\Vert#1\right\Vert}
\newcommand{\setof}[1]{\left\{#1 \right\}}
\newcommand{\indot}[2]{\left\langle #1,#2 \right\rangle}
\newcommand{\mat}[1]{\left[ #1 \right]}
\newcommand{\sqmat}[3]{\begin{#1}
{#2}_{11} & {#2}_{12} & \cdots & {#2}_{1{#3}} \\
{#2}_{21} & {#2}_{22} & \cdots & {#2}_{2{#3}} \\
\vdots & \vdots &        & \vdots \\
{#2}_{{#3}1} & {#2}_{{#3}2} & \cdots & {#2}_{{#3}{#3}}
\end{#1}}
\newcommand{\nullsp}[1]{\bsym{\mathrm{N}}\mypar{#1}}
\newcommand{\colsp}[1]{\bsym{\mathrm{C}}\mypar{#1}}
\newcommand{\func}[2]{\mathrm{#1}\mypar{#2}}
\newcommand{\entry}[3]{\func{entry}{#1,#2,#3}}
\newcommand{\row}[2]{\func{row}{#1,#2}}
\newcommand{\col}[2]{\func{col}{#1,#2}}
\newcommand{\trace}[1]{\func{Tr}{#1}}
\newcommand{\diag}[1]{\func{diag}{#1}}
\newcommand{\rank}[1]{\func{rank}{#1}}
\newcommand{\adj}[1]{\func{adj}{#1}}
\newcommand{\myspan}[1]{\func{span}{#1}}
\newcommand{\enums}[2]{{#1}_1,{#1}_2,\dots,{#1}_{#2}}
\newcommand{\inv}[1]{{#1}^{-1}}
\newcommand{\ortcom}[1]{{#1}^{\bot}}

\renewcommand{\det}[1]{\func{det}{#1}}

\newcommand{\mata}{\bsym{A}}
\newcommand{\matb}{\bsym{B}}
\newcommand{\matc}{\bsym{C}}
\newcommand{\matd}{\bsym{D}}
\newcommand{\mate}{\bsym{E}}
\newcommand{\matf}{\bsym{F}}
\newcommand{\matg}{\bsym{G}}
\newcommand{\matH}{\bsym{H}}
\newcommand{\mati}{\bsym{I}}
\newcommand{\matj}{\bsym{J}}
\newcommand{\matk}{\bsym{K}}
\newcommand{\matl}{\bsym{L}}
\newcommand{\matlam}{\bsym{\Lambda}}
\newcommand{\matm}{\bsym{M}}
\newcommand{\matn}{\bsym{N}}
\newcommand{\mato}{\bsym{O}}
\newcommand{\matp}{\bsym{P}}
\newcommand{\matq}{\bsym{Q}}
\newcommand{\matr}{\bsym{R}}
\newcommand{\mats}{\bsym{S}}
\newcommand{\matsig}{\bsym{\Sigma}}
\newcommand{\matt}{\bsym{T}}
\newcommand{\matu}{\bsym{U}}
\newcommand{\matv}{\bsym{V}}
\newcommand{\matw}{\bsym{W}}
\newcommand{\matx}{\bsym{X}}
\newcommand{\maty}{\bsym{Y}}
\newcommand{\matz}{\bsym{Z}}

\newcommand{\field}{\bsym{\mathrm{F}}}
\newcommand{\rea}{\bsym{\mathrm{R}}}

\newcommand{\veca}{\bsym{a}}
\newcommand{\vecal}{\bsym{\alpha}}
\newcommand{\vecb}{\bsym{b}}
\newcommand{\vecbeta}{\bsym{\beta}}
\newcommand{\vecc}{\bsym{c}}
\newcommand{\vecd}{\bsym{d}}
\newcommand{\vecdelta}{\bsym{\delta}}
\newcommand{\vece}{\bsym{e}}
\newcommand{\veceps}{\bsym{\epsilon}}
\newcommand{\vecf}{\bsym{f}}
\newcommand{\vecg}{\bsym{g}}
\newcommand{\vecgamma}{\bsym{\gamma}}
\newcommand{\vech}{\bsym{h}}
\newcommand{\veceta}{\bsym{\eta}}
\newcommand{\veci}{\bsym{i}}
\newcommand{\vecj}{\bsym{j}}
\newcommand{\veck}{\bsym{k}}
\newcommand{\vecl}{\bsym{l}}
\newcommand{\vecm}{\bsym{m}}
\newcommand{\vecn}{\bsym{n}}
\newcommand{\veco}{\bsym{o}}
\newcommand{\vecp}{\bsym{p}}
\newcommand{\vecq}{\bsym{q}}
\newcommand{\vecr}{\bsym{r}}
\newcommand{\vecs}{\bsym{s}}
\newcommand{\vect}{\bsym{t}}
\newcommand{\vecu}{\bsym{u}}
\newcommand{\vecv}{\bsym{v}}
\newcommand{\vecw}{\bsym{w}}
\newcommand{\vecx}{\bsym{x}}
\newcommand{\vecy}{\bsym{y}}
\newcommand{\vecz}{\bsym{z}}
\newcommand{\veczero}{\bsym{0}}


\title{线性代数习题课}
\author[夏海淞]{夏海淞\\ \texttt{hsxia22@m.fudan.edu.cn}}
\subtitle{行列式}
\institute[复旦大学]{计算机科学技术学院\\ 复旦大学}
\date{2024年9月19日}

\begin{document}
{
\setbeamercolor{background canvas}{bg=NavyBlue!50!DarkOliveGreen, fg=white}
\setbeamercolor{normal text}{fg=white}
\maketitle
}%This is the colour of the first slide. bg= background and fg=foreground

% \metroset{titleformat frame=smallcaps} %This changes the titles for small caps

\begin{frame}
    \setbeamertemplate{section in toc}[sections numbered] %This is numbering the sections
    \tableofcontents[hideallsubsections] %You can comment this line if you want to show the subsections in the table of contents
\end{frame}

\section{作业要求}
\begin{frame}{作业要求}
    \begin{block}{作业/解题要求}
        \begin{enumerate}
            \item 作业本写明学号和姓名;
            \item \textbf{不建议}提交作业纸;
            \item 作业准确率/完成程度不是考察要求;
            \item \textbf{建议}将过程写得详细清楚一些。
        \end{enumerate}
    \end{block}
\end{frame}

\section{行列式的定义}

\subsection*{逆序数和行列式}
\begin{frame}
    \begin{block}{逆序数的定义}
        \begin{itemize}
            \item \textbf{\(n\)级排列}:由数\(1,2,\dots,n\)组成的有序数组\(i_1i_2\cdots i_n\)。
            \item \textbf{逆序数}:\(\tau\spar{i_1i_2\cdots i_n}\triangleq\abs{\setof{\spar{i_a,i_b}\mid 1\leq a<b\leq n,i_a>i_b}}\)。
        \end{itemize}
    \end{block}
    \begin{block}{行列式的定义}
        \begin{equation*}
            \sqmat{vmatrix}{a}{n}=\sum_{i_1i_2\cdots i_n}(-1)^{\tau\spar{i_1i_2\cdots i_n}}a_{i_11}a_{i_22}\cdots a_{i_nn}
        \end{equation*}
    \end{block}
\end{frame}

\subsection*{习题1}
\begin{frame}
    \begin{block}{题面}
        根据行列式定义,计算
        \begin{equation*}
            f(x)=
            \begin{vmatrix}
                2x & x & 1 & 2  \\
                1  & x & 1 & -1 \\
                3  & 2 & x & 1  \\
                1  & 1 & 1 & x
            \end{vmatrix}
        \end{equation*}
        中\(x^4\)与\(x^3\)的系数。
    \end{block}
    \pause
    \begin{block}{解答}
        \(x^4\)的系数为\((-1)^{\tau(1234)}\times2\times1\times1\times1=2\);

        \(x^3\)的系数为\((-1)^{\tau(2134)}\times1\times1\times1\times1=-1\)。
    \end{block}
\end{frame}

\subsection*{习题2}
\begin{frame}{题面}
    使用行列式的定义证明:
    \begin{equation*}
        D=
        \begin{vmatrix}
            a_1 & a_2 & a_3 & a_4 & a_5 \\
            b_1 & b_2 & b_3 & b_4 & b_5 \\
            c_1 & c_2 & 0   & 0   & 0   \\
            d_1 & d_2 & 0   & 0   & 0   \\
            e_1 & e_2 & 0   & 0   & 0   \\
        \end{vmatrix}=0
    \end{equation*}
\end{frame}
\begin{frame}{解答}
    记\(c_i=d_i=e_i=0\),\(i\in\setof{3,4,5}\)。则由行列式定义可知
    \begin{equation*}
        D=\sum_{i_1i_2i_3i_4i_5}(-1)^{\tau(i_1i_2i_3i_4i_5)}a_{i_1}b_{i_2}c_{i_3}d_{i_4}e_{i_5}
    \end{equation*}
    因为\(i_1i_2i_3i_4i_5\)是长度为\(5\)的排列,\(\setof{i_3i_4i_5}\cap\setof{3,4,5}\neq\varnothing\)。

    所以\(a_{i_1}b_{i_2}c_{i_3}d_{i_4}e_{i_5}=0\)恒成立,即\(D=0\)。
\end{frame}

\subsection*{习题3}
\begin{frame}
    \begin{block}{题面}
        证明:一个\(n\)阶行列式中等于零的元素个数如果比\(n^2-n\)多,则此行列式必等于零。
    \end{block}
    \pause
    \begin{block}{一种解法}
        由题设可知该\(n\)阶行列式中非零元素个数小于\(n^2-\spar{n^2-n}=n\)。

        记该行列式\(i\)行\(j\)列元素为\(a_{ij}\),则对任意\(n\)阶排列\(i_1i_2\cdots i_n\),均满足\(\prod_{j=1}^{n}a_{i_jj}=0\)。

        由行列式定义可知该行列式为\(0\)。
    \end{block}
    \pause
    \begin{block}{另一种解法}
        由题设可知该\(n\)阶行列式中非零元素个数小于\(n^2-\spar{n^2-n}=n\)。

        由抽屉原理可知该行列式中至少有一行元素均为\(0\)。根据行列式的性质可知该行列式为\(0\)。
    \end{block}
\end{frame}


\section{行列式的性质}
\begin{frame}{行列式的性质}
    \begin{itemize}
        \item 设\(\mata\)是\(n\)阶方阵,则\(\det{\mata^\top}=\det{\mata}\)。
        \item 方阵的任意两行(列)互换,其行列式的值只改变正负号。
        \item \underline{方阵的行列式具有分行(列)相加性}。
        \item 若方阵中有两行(列)对应元素相等,则其行列式为\(0\)。
        \item 方阵\(\mata\)任意行乘常数\(k\)所得新方阵的行列式为\(k\det{\mata}\)。
        \item \underline{将方阵的第\(j\)行(列)乘以常数\(k\)后加到第\(i\)行(列)所得矩阵的行列式不变}。
        \item 设\(\mata\)和\(\matb\)都是\(n\)阶方阵,则\(\det{\mata\matb}=\det{\mata}\det{\matb}\)。
        \item 设\(n\)阶方阵\(\mata\)可逆,则\(\det{\inv{\mata}}=\inv{\det{\mata}}\)。
    \end{itemize}
\end{frame}

\subsection*{习题1}
\begin{frame}
    \frametitle{题面}
    计算\(n\)阶行列式:
    \begin{equation*}
        \begin{vmatrix}
            1      & 2      & 3      & \cdots & n      \\
            2      & 3      & 4      & \cdots & 1      \\
            3      & 4      & 5      & \cdots & 2      \\
            \vdots & \vdots & \vdots &        & \vdots \\
            n      & 1      & 2      & \cdots & n-1
        \end{vmatrix}
    \end{equation*}
\end{frame}
\begin{frame}
    \frametitle{解答}
    对该行列式进行初等行变换:
    \begin{align*}
          &
        \begin{vmatrix}
            1      & 2      & 3      & \cdots & n      \\
            2      & 3      & 4      & \cdots & 1      \\
            3      & 4      & 5      & \cdots & 2      \\
            \vdots & \vdots & \vdots &        & \vdots \\
            n      & 1      & 2      & \cdots & n-1
        \end{vmatrix}=
        \begin{vmatrix}
            1      & 2      & 3      & \cdots & n      \\
            1      & 1      & 1      & \cdots & 1-n    \\
            1      & 1      & 1      & \cdots & 1      \\
            \vdots & \vdots & \vdots &        & \vdots \\
            1      & 1-n    & 1      & \cdots & 1
        \end{vmatrix}                                                                  \\
        = &
        \begin{vmatrix}
            1      & 1      & 2      & \cdots & n-1    \\
            1      & 0      & 0      & \cdots & -n     \\
            1      & 0      & 0      & \cdots & 0      \\
            \vdots & \vdots & \vdots &        & \vdots \\
            1      & -n     & 0      & \cdots & 0
        \end{vmatrix}=
        \begin{vmatrix}
            1+\frac{1}{n}\sum_{i=1}^{n-1}i & 1      & 2      & \cdots & n-1    \\
            0                              & 0      & 0      & \cdots & -n     \\
            0                              & 0      & 0      & \cdots & 0      \\
            \vdots                         & \vdots & \vdots &        & \vdots \\
            0                              & -n     & 0      & \cdots & 0
        \end{vmatrix}                                          \\
        = & \spar{1+\frac{n-1}{2}}(-1)^{\frac{(n-1)(n-2)}{2}(-n)^{n-1}}=\frac{n+1}{2}(-1)^{\frac{n(n-1)}{2}}n^{n-1}
    \end{align*}
\end{frame}

\subsection*{习题2}
\begin{frame}
    \frametitle{题面}
    计算行列式:
    \begin{equation*}
        D_n=
        \begin{vmatrix}
            a_{1}+b_{1} & a_{1}+b_{2} & \cdots & a_{1}+b_{n} \\
            a_{2}+b_{1} & a_{2}+b_{2} & \cdots & a_{2}+b_{n} \\
            \vdots      & \vdots      &        & \vdots      \\
            a_{n}+b_{1} & a_{n}+b_{2} & \cdots & a_{n}+b_{n}
        \end{vmatrix}
    \end{equation*}
\end{frame}
\begin{frame}
    \frametitle{解答}
    当\(n=1\)时,\(D_n=a_1+b_1\);

    当\(n=2\)时,\(D_n=a_1b_2+a_2b_1-a_1b_1-a_2b_2\);
    \pause

    当\(n\geq3\)时,
    \begin{align*}
        D_n & =
        \begin{vmatrix}
            a_{1}       & a_{1}       & \cdots & a_{1}       \\
            a_{2}+b_{1} & a_{2}+b_{2} & \cdots & a_{2}+b_{n} \\
            \vdots      & \vdots      &        & \vdots      \\
            a_{n}+b_{1} & a_{n}+b_{2} & \cdots & a_{n}+b_{n}
        \end{vmatrix}+
        \begin{vmatrix}
            b_{1}       & b_{2}       & \cdots & b_{n}       \\
            a_{2}+b_{1} & a_{2}+b_{2} & \cdots & a_{2}+b_{n} \\
            \vdots      & \vdots      &        & \vdots      \\
            a_{n}+b_{1} & a_{n}+b_{2} & \cdots & a_{n}+b_{n}
        \end{vmatrix} \\
            & =
        \begin{vmatrix}
            a_{1}       & a_{1}       & \cdots & a_{1}       \\
            a_{2}+b_{1} & b_{2}-b_{1} & \cdots & b_{n}-b_{1} \\
            \vdots      & \vdots      &        & \vdots      \\
            a_{n}+b_{1} & b_{2}-b_{1} & \cdots & b_{n}-b_{1}
        \end{vmatrix}+
        \begin{vmatrix}
            b_{1}  & b_{2}  & \cdots & b_{n}  \\
            a_{2}  & a_{2}  & \cdots & a_{2}  \\
            \vdots & \vdots &        & \vdots \\
            a_{n}  & a_{n}  & \cdots & a_{n}
        \end{vmatrix}                \\
            & =0+0=0
    \end{align*}
\end{frame}

\subsection*{习题3}
\begin{frame}
    \frametitle{题面}
    计算行列式(设\(n>2\)):
    \begin{equation*}
        \begin{vmatrix}
            \sin 2\alpha_1                   & \sin\spar{\alpha_{1}+\alpha_{2}} & \cdots & \sin\spar{\alpha_{1}+\alpha_{n}} \\
            \sin\spar{\alpha_{2}+\alpha_{1}} & \sin 2\alpha_2                   & \cdots & \sin\spar{\alpha_{2}+\alpha_{n}} \\
            \vdots                           & \vdots                           &        & \vdots                           \\
            \sin\spar{\alpha_{n}+\alpha_{1}} & \sin\spar{\alpha_{n}+\alpha_{2}} & \cdots & \sin 2\alpha_n
        \end{vmatrix}
    \end{equation*}
\end{frame}
\begin{frame}
    \frametitle{解答}
    因为\(\sin\spar{\alpha_i+\alpha_j}=\sin\alpha_i\cos\alpha_j+\cos\alpha_i\sin\alpha_j\),定义\(n\)阶方阵如下:
    \begin{equation*}
        \mata_n=
        \begin{bmatrix}
            \sin\alpha_{1} & \cos\alpha_{1} & 0 & \cdots & 0 \\
            \sin\alpha_{2} & \cos\alpha_{2} & 0 & \cdots & 0 \\
            \vdots         & \vdots         &   & \vdots     \\
            \sin\alpha_{n} & \cos\alpha_{n} & 0 & \cdots & 0
        \end{bmatrix},
        \matb_n=
        \begin{bmatrix}
            \cos\alpha_{1} & \cos\alpha_{2} & \cdots & \cos\alpha_{n} \\
            \sin\alpha_{1} & \sin\alpha_{2} & \cdots & \sin\alpha_{n} \\
            0              & 0              & \cdots & 0              \\
            \vdots         & \vdots         &        & \vdots         \\
            0              & 0              & \cdots & 0
        \end{bmatrix}
    \end{equation*}
    容易发现\(\mat{\mata_n\matb_n}_{ij}=\sin\spar{\alpha_i+\alpha_j}\)。因此有
    \begin{equation*}
        \det{\mata_n\matb_n}=\det{\mata_n}\det{\matb_n}=0
    \end{equation*}
\end{frame}

\subsection*{习题4}
\begin{frame}
    \frametitle{题面}
    定义\(n\)阶方阵\(\mata_n\)为:\(\mat{\mata_n}_{ij}=\abs{i-j}\)。求\(\det{\mata_n}\)。
\end{frame}
\begin{frame}[allowframebreaks]
    \frametitle{解答}
    由题设可知
    \begin{equation*}
        \mata_n=
        \begin{bmatrix}
            0      & 1      & 2      & \cdots & n-1    \\
            1      & 0      & 1      & \cdots & n-2    \\
            2      & 1      & 0      & \cdots & n-3    \\
            \vdots & \vdots & \vdots &        & \vdots \\
            n-1    & n-2    & n-3    & \cdots & 0
        \end{bmatrix}
    \end{equation*}
    对其进行初等行变换,可得
    \begin{equation*}
        \det{\mata_n}=
        \begin{vmatrix}
            -1     & 1      & 1      & \cdots & 1      \\
            -1     & -1     & 1      & \cdots & 1      \\
            -1     & -1     & -1     & \cdots & 1      \\
            \vdots & \vdots & \vdots &        & \vdots \\
            n-1    & n-2    & n-3    & \cdots & 0
        \end{vmatrix}
    \end{equation*}
    对其进行初等列变换,可得
    \begin{equation*}
        \det{\mata_n}=
        \begin{vmatrix}
            -1     & 0      & 0      & \cdots & 0      \\
            -1     & -2     & 0      & \cdots & 0      \\
            -1     & -1     & -2     & \cdots & 0      \\
            \vdots & \vdots & \vdots &        & \vdots \\
            n-1    & 2n-3   & 2n-4   & \cdots & n-1
        \end{vmatrix}=(-1)^{n-1}(n-1)2^{n-2}
    \end{equation*}
\end{frame}

\section{行列式按行(列)展开}
\subsection*{代数余子式与行列式展开}
\begin{frame}
    \begin{block}{代数余子式}
        将\(n\)阶方阵\(\mata=\mat{a_{ij}}_{n\times n}\)的第\(i\)行第\(j\)列划去后,所得的\(n-1\)阶子矩阵的行列式记作\(M_{ij}\),则称\((-1)^{i+j}M_{ij}\)为元素\(a_{ij}\)的\textbf{代数余子式},记作\(A_{ij}=(-1)^{i+j}M_{ij}\)。
    \end{block}
    \begin{block}{行列式按行展开}
        设
        \begin{equation*}
            \mata=\begin{bmatrix}
                a_{11} & 0      & \cdots & 0      \\
                a_{21} & a_{22} & \cdots & a_{2n} \\
                \vdots & \vdots &        & \vdots \\
                a_{n1} & a_{n2} & \cdots & a_{nn}
            \end{bmatrix}
        \end{equation*}
        则\(\det{\mata}=a_{11}M_{11}\),其中\(M_{11}\)是将方阵\(\mata\)的第一行第一列划去后所得的\(n-1\)阶子矩阵的行列式:
        \begin{equation*}
            M_{11}=\begin{vmatrix}
                a_{22} & \cdots & a_{2n} \\
                \vdots &        & \vdots \\
                a_{n2} & \cdots & a_{nn}
            \end{vmatrix}
        \end{equation*}
    \end{block}
\end{frame}

\subsection*{习题1}

\begin{frame}
    \frametitle{题面}
    计算\(n\spar{n>1}\)阶行列式:
    \begin{equation*}
        D_n=
        \begin{vmatrix}
            x      & y      & 0      & \cdots & 0      & 0      \\
            0      & x      & y      & \cdots & 0      & 0      \\
            \vdots & \vdots & \vdots &        & \vdots & \vdots \\
            0      & 0      & 0      & \cdots & x      & y      \\
            y      & 0      & 0      & \cdots & 0      & x
        \end{vmatrix}
    \end{equation*}
\end{frame}
\begin{frame}
    \frametitle{解答}
    将行列式按第一列展开,可得
    \begin{align*}
        D_n & =x
        \begin{vmatrix}
            x      & y      & \cdots & 0      & 0      \\
            \vdots & \vdots &        & \vdots & \vdots \\
            0      & 0      & \cdots & x      & y      \\
            0      & 0      & \cdots & 0      & x
        \end{vmatrix}+(-1)^{n+1}y
        \begin{vmatrix}
            y      & 0      & \cdots & 0      & 0      \\
            x      & y      & \cdots & 0      & 0      \\
            \vdots & \vdots &        & \vdots & \vdots \\
            0      & 0      & \cdots & x      & y
        \end{vmatrix} \\
            & =x^n+(-1)^{n+1}y^n.
    \end{align*}
\end{frame}

\subsection*{习题2}
\begin{frame}
    \frametitle{题面}
    计算行列式:
    \begin{equation*}
        \begin{vmatrix}
            x      & -1      & 0       & \cdots & 0      & 0      \\
            0      & x       & -1      & \cdots & 0      & 0      \\
            \vdots & \vdots  & \vdots  &        & \vdots & \vdots \\
            0      & 0       & 0       & \cdots & x      & -1     \\
            a_n    & a_{n-1} & a_{n-2} & \cdots & a_2    & a_1+x
        \end{vmatrix}
    \end{equation*}
\end{frame}
\begin{frame}
    \frametitle{解答}

    记上述行列式为\(D_n\)。将\(D_n\)按第一列展开:
    \begin{equation*}
        D_n  =xD_{n-1}+(-1)^{n+1}a_n
        \begin{vmatrix}
            -1     & 0      & \cdots & 0      & 0      \\
            x      & -1     & \cdots & 0      & 0      \\
            \vdots & \vdots &        & \vdots & \vdots \\
            0      & 0      & \cdots & x      & -1
        \end{vmatrix}
        =xD_{n-1}+a_n
    \end{equation*}
    \(n=1\)时,\(D_n=a_1+x\);\(n=2\)时,\(D_n=a_2+a_1x+x^2\)。

    猜想\(D_n=x^n+\sum_{i=1}^{n}a_ix^{n-i}\)对\(n\geq1\)成立。

    当\(n=1\)时,结论成立。设当\(n=k\)\((k\geq1)\)时结论成立,则当\(n=k+1\)时,
    \begin{align*}
        D_{k+1} & =xD_k+a_{k+1}=x\spar{x^k+\sum_{i=1}^{k}a_ix^{k-i}}+a_{k+1} \\
                & =x^{k+1}+\sum_{i=1}^{k+1}a_ix^{k+1-i}
    \end{align*}

    由归纳公理知\(D_n=x^n+\sum_{i=1}^{n}a_ix^{n-i}\)对\(n\geq1\)成立。

\end{frame}

\section{Laplace定理}

\subsection*{习题1}
\begin{frame}
    \begin{theorem}[Laplace定理]
        在行列式对应的方阵中任取\(k\)行,则这\(k\)行元素组成的一切\(k\)阶子式与它们对应的代数余子式的乘积之和等于行列式的值。
    \end{theorem}
    \pause
    \begin{block}{题面}
        设分块\(n\)阶方阵\(\matm=\begin{bmatrix}\mata&\matc\\\mato&\matb\end{bmatrix}\),其中\(\mata\)为\(k\)阶方阵,证明:\(\det{\matm}=\det{\mata}\det{\matb}\)。
    \end{block}
    \pause
    \begin{block}{解答}
        利用Laplace定理完成证明。

        在\(\matm=\begin{bmatrix}\mata&\matc\\\mato&\matb\end{bmatrix}\)中取前\(k\)行,分析\(\matm\)中前\(k\)行元素组成的\(k\)阶子矩阵\(\matm_k\):
        \pause

        当\(\matm_k=\mata\)时,其子式对应的代数余子式为\(\det{\matb}\);

        当\(\matm_k\neq\mata\)时,其子式对应的代数余子式中必然存在全零列,即该代数余子式的值为零。

        由Laplace定理可得\(\det{\matm}=\det{\mata}\det{\matb}\)。
    \end{block}
\end{frame}

\subsection*{习题2}
\begin{frame}
    \begin{block}{题面}
        设\(\mata,\matb\)是数域\(P\)上的\(n\)级矩阵,则\(\abs{\mata\matb}=\abs{\mata}\abs{\matb}\)。

        提示:使用性质\(\col{\mata\matb}{j}=\mata\col{\matb}{j}=\sum_{k=1}^lb_{kj}\col{\mata}{k}\)。
    \end{block}
    \pause
    \begin{block}{解答}
        设矩阵\(\mata=\mat{a_{ij}}_{n\times n}\),矩阵\(\matb=\mat{b_{ij}}_{n\times n}\)。构造\(2n\)阶方阵\(\matf\):
        \begin{equation*}
            \matf=
            \begin{bmatrix}
                \mata & \mato \\\matd&\matb
            \end{bmatrix},\matd=\diag{\mat{-1,-1,\dots,-1}}
        \end{equation*}

        对\(\matf\)作初等列变换,设变换后的矩阵\(\matfstar\)形如:
        \begin{equation*}
            \matfstar=\begin{bmatrix}\mata&\matc\\\matd&\mato\end{bmatrix}
        \end{equation*}

        则容易发现\(\col{\matc}{j}=\sum_{k=1}^nb_{kj}\col{\mata}{k}=\col{\mata\matb}{j}\),即\(\matc=\mata\matb\)。

        又由Laplace定理可知,\(\abs{\matf}=\abs{\mata}\abs{\matb}\),\(\abs{\matfstar}=\spar{-1}^n\abs{\matc}\abs{\matd}=\abs{\mata\matb}\),配合行列式性质即可得\(\abs{\mata\matb}=\abs{\mata}\abs{\matb}\)。
    \end{block}
\end{frame}

\section{其他方法}

\subsection*{升阶法}
\begin{frame}
    \begin{block}{题面}
        计算\(D=\begin{vmatrix}1+x&1&1&1\\1&1-x&1&1\\1&1&1+y&1\\1&1&1&1-y\end{vmatrix}\)。
    \end{block}
    \pause
    \begin{block}{解答}
        将\(D\)对应方阵添加一行和一列,转化为五阶行列式
        \begin{equation*}
            D=
            \begin{vmatrix}
                1 & 1   & 1   & 1   & 1   \\
                0 & 1+x & 1   & 1   & 1   \\
                0 & 1   & 1-x & 1   & 1   \\
                0 & 1   & 1   & 1+y & 1   \\
                0 & 1   & 1   & 1   & 1-y
            \end{vmatrix}=
            \begin{vmatrix}
                1  & 1 & 1  & 1 & 1  \\
                -1 & x & 0  & 0 & 1  \\
                -1 & 0 & -x & 0 & 1  \\
                -1 & 0 & 0  & y & 1  \\
                -1 & 0 & 0  & 0 & -y
            \end{vmatrix}
        \end{equation*}
        当\(xy\neq0\)时,经初等列变换可得\(D=x^2y^2\);当\(xy=0\)时,易知\(D=0\)。
    \end{block}
\end{frame}

\subsection*{Vandermonde行列式}
\begin{frame}
    \begin{block}{Vandermonde行列式}
        \begin{equation*}
            \begin{vmatrix}
                1      & x_1    & x_1^2   & \cdots & x_1^{n-1}   \\
                1      & x_2    & x_2^2   & \cdots & x_2^{n-1}   \\
                1      & x_3    & x_3^2   & \cdots & x_3^{n-1}   \\
                \vdots & \vdots & \vdots  & \ddots & \vdots      \\
                1      & x_{n}  & x_{n}^2 & \cdots & x_{n}^{n-1} \\
            \end{vmatrix}=\prod_{1\leq i<j\leq n}\spar{x_j-x_i}
        \end{equation*}
    \end{block}
    \pause
    \begin{block}{题面}
        设\(a,b,c\)互不相同,\(D=\begin{vmatrix}a&b&c\\a^2&b^2&c^2\\b+c&c+a&a+b\end{vmatrix}\),则\(D=0\)的充要条件是\(a+b+c=0\)。
    \end{block}
    \pause
    \begin{block}{解答}
        将\(D\)的第一行加到第三行后调整行序,得\(D=\spar{a+b+c}\begin{vmatrix}1&1&1\\a&b&c\\a^2&b^2&c^2\end{vmatrix}\)。

        因为\(a,b,c\)互异,由Vandermonde行列式可知\(D=0\)的充要条件是\(a+b+c=0\)。
    \end{block}
\end{frame}

\end{document}
