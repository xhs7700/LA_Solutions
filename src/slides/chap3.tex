\documentclass[10pt,xcolor=svgnames]{beamer} %Beamer
\usepackage{palatino} %font type
\usefonttheme{metropolis} %Type of slides
\usefonttheme[onlymath]{serif} %font type Mathematical expressions
\usetheme[progressbar=frametitle,numbering=counter]{metropolis} %This adds a bar at the beginning of each section.
\useoutertheme{infolines} %Circles in the top of each frame, showing the slide of each section you are at

\usepackage{xeCJK}
\setCJKmainfont{Noto Serif CJK SC}
\setCJKsansfont[BoldFont=Noto Sans CJK SC]{Noto Sans CJK SC Light}

\usepackage{appendixnumberbeamer} %enumerate each slide without counting the appendix
\setbeamercolor{progress bar}{fg=Maroon!70!Coral} %These are the colours of the progress bar. Notice that the names used are the svgnames
\setbeamercolor{title separator}{fg=DarkSalmon} %This is the line colour in the title slide
\setbeamercolor{structure}{fg=black} %Colour of the text of structure, numbers, items, blah. Not the big text.
\setbeamercolor{normal text}{fg=black!87} %Colour of normal text
\setbeamercolor{alerted text}{fg=DarkRed!60!Gainsboro} %Color of the alert box
\setbeamercolor{example text}{fg=Maroon!70!Coral} %Colour of the Example block text


\setbeamercolor{palette primary}{bg=NavyBlue!50!DarkOliveGreen, fg=white} %These are the colours of the background. Being this the main combination and so one. 
\setbeamercolor{palette secondary}{bg=NavyBlue!50!DarkOliveGreen, fg=white}
\setbeamercolor{palette tertiary}{bg=NavyBlue!40!Black, fg= white}
\setbeamercolor{section in toc}{fg=NavyBlue!40!Black} %Color of the text in the table of contents (toc)

%These next packages are the useful for Physics in general, you can add the extras here. 
\usepackage{amsmath,amssymb}
\usepackage{slashed}
\usepackage{cite}
\usepackage{relsize}
\usepackage{caption}
\usepackage{multicol}
\usepackage{multirow}
\usepackage{booktabs}
\usepackage[scale=2]{ccicons}
\usepackage{pgfplots}
\usepgfplotslibrary{dateplot}
\usepackage{geometry}
\usepackage{xspace}
\usepackage{graphicx,subfig,float}
\usepackage{color}
\newcommand{\themename}{\textbf{\textsc{bluetemp}\xspace}}%metropolis}}\xspace}

% \newtheorem{theorem}{定理}[subsection]
% \newtheorem{lemma}{引理}[subsection]
% \newtheorem*{definition}{定义}
% \newtheorem{property}{性质}[subsection]
% \newtheorem{infer}{推论}[subsection]

\usepackage{color}

\newcommand{\bsym}[1]{\boldsymbol{#1}}
\newcommand{\mypar}[1]{\left( #1 \right)}
\newcommand{\gram}[1]{\bsym{G}\mypar{#1}}
\newcommand{\abs}[1]{\left|#1 \right|}
\newcommand{\Abs}[1]{\left\Vert#1\right\Vert}
\newcommand{\setof}[1]{\left\{#1 \right\}}
\newcommand{\indot}[2]{\left\langle #1,#2 \right\rangle}
\newcommand{\mat}[1]{\left[ #1 \right]}
\newcommand{\myvec}[1]{\left[ #1 \right]^\top}
\newcommand{\sqmat}[3]{\begin{#1}
{#2}_{11} & {#2}_{12} & \cdots & {#2}_{1{#3}} \\
{#2}_{21} & {#2}_{22} & \cdots & {#2}_{2{#3}} \\
\vdots & \vdots &   \ddots     & \vdots \\
{#2}_{{#3}1} & {#2}_{{#3}2} & \cdots & {#2}_{{#3}{#3}}
\end{#1}}
\newcommand{\normmat}[4]{\begin{#1}
{#2}_{11} & {#2}_{12} & \cdots & {#2}_{1{#4}} \\
{#2}_{21} & {#2}_{22} & \cdots & {#2}_{2{#4}} \\
\vdots & \vdots &   \ddots     & \vdots \\
{#2}_{{#3}1} & {#2}_{{#3}2} & \cdots & {#2}_{{#3}{#4}}
\end{#1}}
\newcommand{\nullsp}[1]{\bsym{\mathrm{N}}\mypar{#1}}
\newcommand{\colsp}[1]{\bsym{\mathrm{C}}\mypar{#1}}
\newcommand{\func}[2]{\mathrm{#1}\mypar{#2}}
\newcommand{\entry}[3]{\func{entry}{#1,#2,#3}}
\newcommand{\row}[2]{\func{row}{#1,#2}}
\newcommand{\col}[2]{\func{col}{#1,#2}}
\newcommand{\trace}[1]{\func{Tr}{#1}}
\newcommand{\diag}[1]{\func{diag}{#1}}
\newcommand{\rank}[1]{\func{rank}{#1}}
\newcommand{\adj}[1]{\func{adj}{#1}}
\newcommand{\myspan}[1]{\func{span}{#1}}
\newcommand{\enums}[2]{{#1}_1,{#1}_2,\dots,{#1}_{#2}}
\newcommand{\inv}[1]{{#1}^{-1}}
\newcommand{\pinv}[1]{\inv{\mypar{#1}}}
\newcommand{\ortcom}[1]{{#1}^{\bot}}

\renewcommand{\det}[1]{\func{det}{#1}}

\newcommand{\dif}{\mathrm{d}}
\newcommand{\fracdif}[1]{\frac{\dif}{\dif #1}}

\newcommand{\todo}[1]{{ \textcolor{red}{ TODO: #1}}}

\newcommand{\mata}{\bsym{A}}
\newcommand{\matb}{\bsym{B}}
\newcommand{\matc}{\bsym{C}}
\newcommand{\matd}{\bsym{D}}
\newcommand{\mate}{\bsym{E}}
\newcommand{\matf}{\bsym{F}}
\newcommand{\matfstar}{\bsym{F^*}}
\newcommand{\matg}{\bsym{G}}
\newcommand{\matH}{\bsym{H}}
\newcommand{\mati}{\bsym{I}}
\newcommand{\matj}{\bsym{J}}
\newcommand{\matk}{\bsym{K}}
\newcommand{\matl}{\bsym{L}}
\newcommand{\matlam}{\bsym{\Lambda}}
\newcommand{\matm}{\bsym{M}}
\newcommand{\matn}{\bsym{N}}
\newcommand{\mato}{\bsym{O}}
\newcommand{\matp}{\bsym{P}}
\newcommand{\matq}{\bsym{Q}}
\newcommand{\matr}{\bsym{R}}
\newcommand{\mats}{\bsym{S}}
\newcommand{\matsig}{\bsym{\Sigma}}
\newcommand{\matt}{\bsym{T}}
\newcommand{\matu}{\bsym{U}}
\newcommand{\matv}{\bsym{V}}
\newcommand{\matw}{\bsym{W}}
\newcommand{\matx}{\bsym{X}}
\newcommand{\maty}{\bsym{Y}}
\newcommand{\matz}{\bsym{Z}}

\newcommand{\field}{\bsym{\mathrm{F}}}
\newcommand{\rea}{\bsym{\mathbb{R}}}

\newcommand{\veca}{\bsym{a}}
\newcommand{\vecal}{\bsym{\alpha}}
\newcommand{\vecb}{\bsym{b}}
\newcommand{\vecbeta}{\bsym{\beta}}
\newcommand{\vecc}{\bsym{c}}
\newcommand{\vecd}{\bsym{d}}
\newcommand{\vecdelta}{\bsym{\delta}}
\newcommand{\vece}{\bsym{e}}
\newcommand{\veceps}{\bsym{\epsilon}}
\newcommand{\vecveps}{\bsym{\varepsilon}}
\newcommand{\vecf}{\bsym{f}}
\newcommand{\vecg}{\bsym{g}}
\newcommand{\vecgamma}{\bsym{\gamma}}
\newcommand{\vech}{\bsym{h}}
\newcommand{\veceta}{\bsym{\eta}}
\newcommand{\veci}{\bsym{i}}
\newcommand{\vecj}{\bsym{j}}
\newcommand{\veck}{\bsym{k}}
\newcommand{\vecl}{\bsym{l}}
\newcommand{\vecm}{\bsym{m}}
\newcommand{\vecn}{\bsym{n}}
\newcommand{\veco}{\bsym{o}}
\newcommand{\vecone}{\bsym{1}}
\newcommand{\vecp}{\bsym{p}}
\newcommand{\vecq}{\bsym{q}}
\newcommand{\vecr}{\bsym{r}}
\newcommand{\vecs}{\bsym{s}}
\newcommand{\vect}{\bsym{t}}
\newcommand{\vecu}{\bsym{u}}
\newcommand{\vecv}{\bsym{v}}
\newcommand{\vecw}{\bsym{w}}
\newcommand{\vecx}{\bsym{x}}
\newcommand{\vecxi}{\bsym{\xi}}
\newcommand{\vecy}{\bsym{y}}
\newcommand{\vecz}{\bsym{z}}
\newcommand{\veczero}{\bsym{0}}


\title{线性代数习题课}
\author{夏海淞}
\subtitle{第一章}
\institute[复旦大学]{计算机科学技术学院\\ 复旦大学}
\date{2023年9月21日}

\begin{document}
{
\setbeamercolor{background canvas}{bg=NavyBlue!50!DarkOliveGreen, fg=white}
\setbeamercolor{normal text}{fg=white}
\maketitle
}%This is the colour of the first slide. bg= background and fg=foreground

% \metroset{titleformat frame=smallcaps} %This changes the titles for small caps

\begin{frame}
    \setbeamertemplate{section in toc}[sections numbered] %This is numbering the sections
    \tableofcontents[hideallsubsections] %You can comment this line if you want to show the subsections in the table of contents
\end{frame}

\section{习题讲解}

\subsection*{3.1}
\begin{frame}
    \frametitle{题面}
    根据行列式定义,计算
    \begin{equation*}
        f(x)=
        \begin{vmatrix}
            2x & x & 1 & 2  \\
            1  & x & 1 & -1 \\
            3  & 2 & x & 1  \\
            1  & 1 & 1 & x
        \end{vmatrix}
    \end{equation*}
    中\(x^4\)与\(x^3\)的系数。
\end{frame}
\begin{frame}
    \frametitle{解答}
    \(x^4\)的系数为\((-1)^{\tau(1234)}\times2\times1\times1\times1=2\);

    \(x^3\)的系数为\((-1)^{\tau(2134)}\times1\times1\times1\times1=-1\)。
\end{frame}

\subsection*{3.2}
\begin{frame}
    \frametitle{题面}
    使用行列式的定义证明:
    \begin{equation*}
        D=
        \begin{vmatrix}
            a_1 & a_2 & a_3 & a_4 & a_5 \\
            b_1 & b_2 & b_3 & b_4 & b_5 \\
            c_1 & c_2 & 0   & 0   & 0   \\
            d_1 & d_2 & 0   & 0   & 0   \\
            e_1 & e_2 & 0   & 0   & 0   \\
        \end{vmatrix}=0
    \end{equation*}
\end{frame}
\begin{frame}
    \frametitle{解答}
    记\(c_i=d_i=e_i=0\),\(i\in\setof{3,4,5}\)。则由行列式定义可知
    \begin{equation*}
        D=\sum_{i_1i_2i_3i_4i_5}(-1)^{\tau(i_1i_2i_3i_4i_5)}a_{i_1}b_{i_2}c_{i_3}d_{i_4}e_{i_5}
    \end{equation*}
    因为\(i_1i_2i_3i_4i_5\)是长度为\(5\)的排列,\(\setof{i_3i_4i_5}\cap\setof{3,4,5}\neq\varnothing\)。

    所以\(a_{i_1}b_{i_2}c_{i_3}d_{i_4}e_{i_5}=0\)恒成立,即\(D=0\)。
\end{frame}

\subsection*{3.3}
\begin{frame}
    \frametitle{题面}
    证明:一个\(n\)阶行列式中等于零的元素个数如果比\(n^2-n\)多,则此行列式必等于零。
\end{frame}
\begin{frame}
    \frametitle{解答}
    由题设可知该\(n\)阶行列式中非零元素个数小于\(n^2-\spar{n^2-n}=n\)。

    记该行列式\(i\)行\(j\)列元素为\(a_{ij}\),则对任意\(n\)阶排列\(i_1i_2\cdots i_n\),均满足\(\prod_{j=1}^{n}a_{i_jj}=0\)。

    由行列式定义可知该行列式为\(0\)。
\end{frame}

\subsection*{3.4}
\begin{frame}
    \frametitle{题面}
    通过计算以下行列式证明:奇偶排列各半。
    \begin{equation*}
        D=
        \begin{vmatrix}
            1      & 1      & \cdots & 1      \\
            1      & 1      & \cdots & 1      \\
            \vdots & \vdots &        & \vdots \\
            1      & 1      & \cdots & 1
        \end{vmatrix}
    \end{equation*}
\end{frame}
\begin{frame}
    \frametitle{解答}
    注意到\(D\)中存在两行(列)元素相同,由推论3.2.1可知\(D=0\)。

    将\(D\)按定义展开:
    \begin{equation*}
        D=\sum_{i_1i_2\cdots i_n}(-1)^{\tau(i_1i_2\cdots i_n)}=0
    \end{equation*}
    当\(i_1i_2\cdots i_n\)为奇排列时,\((-1)^{\tau(i_1i_2\cdots i_n)}=-1\);

    当\(i_1i_2\cdots i_n\)为偶排列时,\((-1)^{\tau(i_1i_2\cdots i_n)}=1\)。因此可知奇偶排列各半。
\end{frame}

\subsection*{3.5}
\begin{frame}
    \frametitle{题面}
    计算下列行列式的值:
    \begin{enumerate}
        \item \(\begin{vmatrix}2&0&0\\4&1&0\\7&3&-2\end{vmatrix}\);
        \item \(\begin{vmatrix}3&0&0\\2&1&1\\1&2&2\end{vmatrix}\);
        \item \(\begin{vmatrix}4&0&2&1\\5&0&4&2\\2&0&3&4\\1&0&2&3\end{vmatrix}\);
        \item \(\begin{vmatrix}1&1&1&3\\0&3&1&1\\0&0&2&2\\-1&-1&-1&2\end{vmatrix}\);
    \end{enumerate}
\end{frame}
\begin{frame}
    \frametitle{解答}
    \begin{equation*}
        \begin{array}[2]{ll}
            \begin{vmatrix}2&0&0\\4&1&0\\7&3&-2\end{vmatrix}=-4,              & \begin{vmatrix}3&0&0\\2&1&1\\1&2&2\end{vmatrix}=0,                    \\
            \begin{vmatrix}4&0&2&1\\5&0&4&2\\2&0&3&4\\1&0&2&3\end{vmatrix}=0, & \begin{vmatrix}1&1&1&3\\0&3&1&1\\0&0&2&2\\-1&-1&-1&2\end{vmatrix}=30.
        \end{array}
    \end{equation*}
\end{frame}

\subsection*{3.7}
\begin{frame}
    \frametitle{题面}
    计算\(n\spar{n>1}\)阶行列式:
    \begin{equation*}
        D_n=
        \begin{vmatrix}
            x      & y      & 0      & \cdots & 0      & 0      \\
            0      & x      & y      & \cdots & 0      & 0      \\
            \vdots & \vdots & \vdots &        & \vdots & \vdots \\
            0      & 0      & 0      & \cdots & x      & y      \\
            y      & 0      & 0      & \cdots & 0      & x
        \end{vmatrix}
    \end{equation*}
\end{frame}
\begin{frame}
    \frametitle{解答}
    将行列式按第一列展开,可得
    \begin{align*}
        D_n & =x
        \begin{vmatrix}
            x      & y      & \cdots & 0      & 0      \\
            \vdots & \vdots &        & \vdots & \vdots \\
            0      & 0      & \cdots & x      & y      \\
            0      & 0      & \cdots & 0      & x
        \end{vmatrix}+(-1)^{n+1}y
        \begin{vmatrix}
            y      & 0      & \cdots & 0      & 0      \\
            x      & y      & \cdots & 0      & 0      \\
            \vdots & \vdots &        & \vdots & \vdots \\
            0      & 0      & \cdots & x      & y
        \end{vmatrix} \\
            & =x^n+(-1)^{n+1}y^n.
    \end{align*}
\end{frame}

\subsection*{3.9}
\begin{frame}
    \frametitle{题面}
    计算\(n\)阶行列式:
    \begin{equation*}
        D_n=
        \begin{vmatrix}
            x+y    & xy     & 0      & \cdots & 0      & 0      & 0      \\
            1      & x+y    & xy     & \cdots & 0      & 0      & 0      \\
            0      & 1      & x+y    & \cdots & 0      & 0      & 0      \\
            \vdots & \vdots & \vdots &        & \vdots & \vdots & \vdots \\
            0      & 0      & 0      & \cdots & 0      & 1      & x+y
        \end{vmatrix}
    \end{equation*}
\end{frame}
\begin{frame}
    \frametitle{解答}
    将\(D_n\)按第一行展开,后项按第一列展开后,得
    \begin{equation*}
        D_n=\spar{x+y}D_{n-1}-xyD_{n-2}.
    \end{equation*}
    该递推式的特征方程为\(\lambda^2-\spar{x+y}\lambda+xy=0\),根为\(\lambda_1=x\),\(\lambda_2=y\)。
    \pause

    当\(x=y\)时,递推式的通解为\(D_n=c_1x^n+c_2nx^n\)。代入初值解得\(D_n=\spar{1+n}x^n\);

    当\(x\neq y\)时,递推式的通解为\(D_n=c_1x^n+c_2y^n\)。代入初值解得
    \begin{equation*}
        D_n=\frac{x}{x-y}x^n-\frac{y}{x-y}y^n=\sum_{i=0}^{n}x^{n-i}y_i.
    \end{equation*}
    综上,\(D_n=\sum_{i=0}^{n}x^{n-i}y^i\)。
\end{frame}

\subsection*{3.12}
\begin{frame}
    \frametitle{题面}
    计算行列式:
    \begin{equation*}
        \begin{vmatrix}
            x      & -1      & 0       & \cdots & 0      & 0      \\
            0      & x       & -1      & \cdots & 0      & 0      \\
            \vdots & \vdots  & \vdots  &        & \vdots & \vdots \\
            0      & 0       & 0       & \cdots & x      & -1     \\
            a_n    & a_{n-1} & a_{n-2} & \cdots & a_2    & a_1+x
        \end{vmatrix}
    \end{equation*}
\end{frame}
\begin{frame}[allowframebreaks]
    \frametitle{解答}
    当\(x\neq0\)时,可利用初等列变换将行列式转化为:
    \begin{align*}
          &
        \begin{vmatrix}
            x      & 0                     & 0       & \cdots & 0      & 0      \\
            0      & x                     & -1      & \cdots & 0      & 0      \\
            \vdots & \vdots                & \vdots  &        & \vdots & \vdots \\
            0      & 0                     & 0       & \cdots & x      & -1     \\
            a_n    & a_{n-1}+\frac{a_n}{x} & a_{n-2} & \cdots & a_2    & a_1+x
        \end{vmatrix} \\
        = &
        \begin{vmatrix}
            x      & 0                     & 0                                         & \cdots & 0      & 0      \\
            0      & x                     & 0                                         & \cdots & 0      & 0      \\
            \vdots & \vdots                & \vdots                                    &        & \vdots & \vdots \\
            0      & 0                     & 0                                         & \cdots & x      & -1     \\
            a_n    & a_{n-1}+\frac{a_n}{x} & a_{n-2}+\frac{a_{n-1}}{x}+\frac{a_n}{x^2} & \cdots & a_2    & a_1+x
        \end{vmatrix}
    \end{align*}
    \begin{align*}
        = &
        \begin{vmatrix}
            x      & 0      & 0      & \cdots & 0      & 0                                                        \\
            0      & x      & 0      & \cdots & 0      & 0                                                        \\
            \vdots & \vdots & \vdots &        & \vdots & \vdots                                                   \\
            0      & 0      & 0      & \cdots & x      & 0                                                        \\
            a_n    & \cdots & \cdots & \cdots & \cdots & \frac{a_n}{x^{n-1}}+\frac{a_{n-1}}{x^{n-2}}+\cdots+a_1+x
        \end{vmatrix} \\
        = & x^n+\sum_{i=1}^{n}a_ix^{n-i}
    \end{align*}

    当\(x=0\)时,可将行列式按第一列展开:
    \begin{align*}
        \begin{vmatrix}
            0      & -1      & 0       & \cdots & 0      & 0      \\
            0      & 0       & -1      & \cdots & 0      & 0      \\
            \vdots & \vdots  & \vdots  &        & \vdots & \vdots \\
            0      & 0       & 0       & \cdots & 0      & -1     \\
            a_n    & a_{n-1} & a_{n-2} & \cdots & a_2    & a_1
        \end{vmatrix}=(-1)^{n+1}a_n(-1)^{n-1}=a_n
    \end{align*}
    综上,行列式的值为\(x^n+\sum_{i=1}^{n}a_ix^{n-i}\)。
\end{frame}

\subsection*{3.13}
\begin{frame}
    \frametitle{题面}
    计算\(n\)阶行列式:
    \begin{equation*}
        \begin{vmatrix}
            1      & 2      & 3      & \cdots & n      \\
            2      & 3      & 4      & \cdots & 1      \\
            3      & 4      & 5      & \cdots & 2      \\
            \vdots & \vdots & \vdots &        & \vdots \\
            n      & 1      & 2      & \cdots & n-1
        \end{vmatrix}
    \end{equation*}
\end{frame}
\begin{frame}
    \frametitle{解答}
    对该行列式进行初等行变换:
    \begin{align*}
          &
        \begin{vmatrix}
            1      & 2      & 3      & \cdots & n      \\
            2      & 3      & 4      & \cdots & 1      \\
            3      & 4      & 5      & \cdots & 2      \\
            \vdots & \vdots & \vdots &        & \vdots \\
            n      & 1      & 2      & \cdots & n-1
        \end{vmatrix}=
        \begin{vmatrix}
            1      & 2      & 3      & \cdots & n      \\
            1      & 1      & 1      & \cdots & 1-n    \\
            1      & 1      & 1      & \cdots & 1      \\
            \vdots & \vdots & \vdots &        & \vdots \\
            1      & 1-n    & 1      & \cdots & 1
        \end{vmatrix}                                                                           \\
        = &
        \begin{vmatrix}
            1      & 1      & 2      & \cdots & n-1    \\
            1      & 0      & 0      & \cdots & -n     \\
            1      & 0      & 0      & \cdots & 0      \\
            \vdots & \vdots & \vdots &        & \vdots \\
            1      & -n     & 0      & \cdots & 0
        \end{vmatrix}=
        \begin{vmatrix}
            1+\frac{1}{n}\sum_{i=1}^{n-1}i & 1      & 2      & \cdots & n-1    \\
            0                              & 0      & 0      & \cdots & -n     \\
            0                              & 0      & 0      & \cdots & 0      \\
            \vdots                         & \vdots & \vdots &        & \vdots \\
            0                              & -n     & 0      & \cdots & 0
        \end{vmatrix}                                                   \\
        = & \spar{1+\frac{n-1}{2}}(-1)^{\frac{(n-1)(n-2)}{2}(-n)^{n-1}}=\spar{1+\frac{n-1}{2}}(-1)^{\frac{n(n-1)}{2}}n^{n-1}
    \end{align*}
\end{frame}

\subsection*{3.14}
\begin{frame}
    \frametitle{题面}
    计算\(n\)阶行列式:
    \begin{equation*}
        \begin{vmatrix}
            x_{1}+a & a       & \cdots & a         & a      \\
            a       & x_{2}+a & \cdots & a         & a      \\
            \vdots  & \vdots  &        & \vdots    & \vdots \\
            a       & a       & \cdots & x_{n-1}+a & a      \\
            a       & a       & \cdots & a         & a
        \end{vmatrix}
    \end{equation*}
\end{frame}
\begin{frame}
    \frametitle{解答}
    对该行列式进行初等行变换:
    \begin{equation*}
        \begin{vmatrix}
            x_{1}+a & a       & \cdots & a         & a      \\
            a       & x_{2}+a & \cdots & a         & a      \\
            \vdots  & \vdots  &        & \vdots    & \vdots \\
            a       & a       & \cdots & x_{n-1}+a & a      \\
            a       & a       & \cdots & a         & a
        \end{vmatrix}=
        \begin{vmatrix}
            x_{1}  & 0      & \cdots & 0       & 0      \\
            0      & x_{2}  & \cdots & 0       & 0      \\
            \vdots & \vdots &        & \vdots  & \vdots \\
            0      & 0      & \cdots & x_{n-1} & 0      \\
            a      & a      & \cdots & a       & a
        \end{vmatrix}=a\prod_{i=1}^{n-1}x_i
    \end{equation*}
\end{frame}

\subsection*{3.16}
\begin{frame}
    \frametitle{题面}
    计算行列式:
    \begin{equation*}
        \begin{vmatrix}
            1 & 1  & 1  & 1  \\
            1 & 1  & -1 & -1 \\
            1 & -1 & 1  & -1 \\
            1 & -1 & -1 & 1
        \end{vmatrix}
    \end{equation*}
\end{frame}
\begin{frame}
    \frametitle{解答}
    对行列式反复进行初等行(列)变换和按行(列)展开:
    \begin{align*}
          &
        \begin{vmatrix}
            1 & 1  & 1  & 1  \\
            1 & 1  & -1 & -1 \\
            1 & -1 & 1  & -1 \\
            1 & -1 & -1 & 1
        \end{vmatrix}=
        \begin{vmatrix}
            1 & 1  & 1  & 1  \\
            0 & 0  & -2 & -2 \\
            0 & -2 & 0  & -2 \\
            0 & -2 & -2 & 0
        \end{vmatrix}=
        \begin{vmatrix}
            0  & -2 & -2 \\
            -2 & 0  & -2 \\
            -2 & -2 & 0
        \end{vmatrix} \\
        = & 2
        \begin{vmatrix}
            -2 & -2 \\
            -2 & 0
        \end{vmatrix}-2
        \begin{vmatrix}
            -2 & -2 \\
            0  & -2
        \end{vmatrix}=-16
    \end{align*}
\end{frame}

\subsection*{3.17}
\begin{frame}
    \frametitle{题面}
    计算行列式:
    \begin{equation*}
        D_n=
        \begin{vmatrix}
            a_{1}+b_{1} & a_{1}+b_{2} & \cdots & a_{1}+b_{n} \\
            a_{2}+b_{1} & a_{2}+b_{2} & \cdots & a_{2}+b_{n} \\
            \vdots      & \vdots      &        & \vdots      \\
            a_{n}+b_{1} & a_{n}+b_{2} & \cdots & a_{n}+b_{n}
        \end{vmatrix}
    \end{equation*}
\end{frame}
\begin{frame}
    \frametitle{解答}
    当\(n=1\)时,\(D_n=a_1+b_1\);

    当\(n=2\)时,\(D_n=a_1b_2+a_2b_1-a_1b_1-a_2b_2\);
    \pause

    当\(n\geq3\)时,
    \begin{align*}
        D_n & =
        \begin{vmatrix}
            a_{1}       & a_{1}       & \cdots & a_{1}       \\
            a_{2}+b_{1} & a_{2}+b_{2} & \cdots & a_{2}+b_{n} \\
            \vdots      & \vdots      &        & \vdots      \\
            a_{n}+b_{1} & a_{n}+b_{2} & \cdots & a_{n}+b_{n}
        \end{vmatrix}+
        \begin{vmatrix}
            b_{1}       & b_{2}       & \cdots & b_{n}       \\
            a_{2}+b_{1} & a_{2}+b_{2} & \cdots & a_{2}+b_{n} \\
            \vdots      & \vdots      &        & \vdots      \\
            a_{n}+b_{1} & a_{n}+b_{2} & \cdots & a_{n}+b_{n}
        \end{vmatrix} \\
            & =
        \begin{vmatrix}
            a_{1}       & a_{1}       & \cdots & a_{1}       \\
            a_{2}+b_{1} & b_{2}-b_{1} & \cdots & b_{n}-b_{1} \\
            \vdots      & \vdots      &        & \vdots      \\
            a_{n}+b_{1} & b_{2}-b_{1} & \cdots & b_{n}-b_{1}
        \end{vmatrix}+
        \begin{vmatrix}
            b_{1}  & b_{2}  & \cdots & b_{n}  \\
            a_{2}  & a_{2}  & \cdots & a_{2}  \\
            \vdots & \vdots &        & \vdots \\
            a_{n}  & a_{n}  & \cdots & a_{n}
        \end{vmatrix}                \\
            & =0+0=0
    \end{align*}
\end{frame}

\subsection*{3.21}
\begin{frame}
    \frametitle{题面}
    设分块\(n\)阶方阵\(\matm=\begin{bmatrix}\mata&\matc\\\mato&\matb\end{bmatrix}\),其中\(\mata\)为\(k\)阶方阵,证明:\(\det{\matm}=\det{\mata}\det{\matb}\)。
\end{frame}
\begin{frame}
    \frametitle{解答}
\end{frame}

\subsection*{3.23}
\begin{frame}
    \frametitle{题面}
    计算行列式(设\(n>2\)):
    \begin{equation*}
        \begin{vmatrix}
            \sin 2\alpha_1                   & \sin\spar{\alpha_{1}+\alpha_{2}} & \cdots & \sin\spar{\alpha_{1}+\alpha_{n}} \\
            \sin\spar{\alpha_{2}+\alpha_{1}} & \sin 2\alpha_2                   & \cdots & \sin\spar{\alpha_{2}+\alpha_{n}} \\
            \vdots                           & \vdots                           &        & \vdots                           \\
            \sin\spar{\alpha_{n}+\alpha_{1}} & \sin\spar{\alpha_{n}+\alpha_{2}} & \cdots & \sin 2\alpha_n
        \end{vmatrix}
    \end{equation*}
\end{frame}
\begin{frame}
    \frametitle{解答}
    因为\(\sin\spar{\alpha_i+\alpha_j}=\sin\alpha_i\cos\alpha_j+\cos\alpha_i\sin\alpha_j\),定义\(n\)阶方阵如下:
    \begin{equation*}
        \mata_n=
        \begin{bmatrix}
            \sin\alpha_{1} & \cos\alpha_{1} & 0 & \cdots & 0 \\
            \sin\alpha_{2} & \cos\alpha_{2} & 0 & \cdots & 0 \\
            \vdots         & \vdots         &   & \vdots     \\
            \sin\alpha_{n} & \cos\alpha_{n} & 0 & \cdots & 0
        \end{bmatrix},
        \matb_n=
        \begin{bmatrix}
            \cos\alpha_{1} & \cos\alpha_{2} & \cdots & \cos\alpha_{n} \\
            \sin\alpha_{1} & \sin\alpha_{2} & \cdots & \sin\alpha_{n} \\
            0              & 0              & \cdots & 0              \\
            \vdots         & \vdots         &        & \vdots         \\
            0              & 0              & \cdots & 0
        \end{bmatrix}
    \end{equation*}
    容易发现\(\mat{\mata_n\matb_n}_{ij}=\sin\spar{\alpha_i+\alpha_j}\)。因此有
    \begin{equation*}
        \det{\mata_n\matb_n}=\det{\mata_n}\det{\matb_n}=0
    \end{equation*}
\end{frame}

\subsection*{3.27}
\begin{frame}
    \frametitle{题面}
    计算\(n\)阶行列式:
    \begin{equation*}
        D_n=
        \begin{vmatrix}
            1+x_{1}y_{1} & 1+x_{1}y_{2} & \cdots & 1+x_{1}y_{n} \\
            1+x_{2}y_{1} & 1+x_{2}y_{2} & \cdots & 1+x_{2}y_{n} \\
            \vdots       & \vdots       &        & \vdots       \\
            1+x_{n}y_{1} & 1+x_{n}y_{2} & \cdots & 1+x_{n}y_{n}
        \end{vmatrix}
    \end{equation*}
\end{frame}
\begin{frame}
    \frametitle{解答}
    当\(n=1\)时,\(D_n=1+x_1y_1\);

    当\(n=2\)时,\(D_n=x_1y_1+x_2y_2-x_1y_2-x_2y_1\);
    \pause

    当\(n\geq3\)时,定义\(n\)阶方阵如下:
    \begin{equation*}
        \matx_n=
        \begin{bmatrix}
            x_{1}  & 1      & 0      & \cdots & 0      \\
            x_{2}  & 1      & 0      & \cdots & 0      \\
            \vdots & \vdots & \vdots &        & \vdots \\
            x_{n}  & 1      & 0      & \cdots & 0
        \end{bmatrix},
        \maty_n=
        \begin{bmatrix}
            y_{1}  & y_{2}  & \cdots & y_{n}  \\
            1      & 1      & \cdots & 1      \\
            0      & 0      & \cdots & 0      \\
            \vdots & \vdots &        & \vdots \\
            0      & 0      & \cdots & 0
        \end{bmatrix}
    \end{equation*}
    容易发现\(\mat{\matx_n\maty_n}_{ij}=x_iy_j+1\)。因此有
    \begin{equation*}
        \det{\matx_n\maty_n}=\det{\matx_n}\det{\maty_n}=0
    \end{equation*}
\end{frame}

\section{补充习题}

\end{document}
