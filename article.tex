\documentclass{ctexart}
\usepackage{anyfontsize}
\usepackage{hyperref}
\usepackage{graphicx}
\usepackage{amsmath,amsthm}

\author{夏海淞}
\title{线性代数习题答案}
\date{\today}

\ctexset { section = { name={第,章} } }
\ctexset { section = { number={\chinese {section}} } }

\newtheorem{problem}{习题}[section]
\newtheorem{extraprob}{附加习题}[section]

\renewcommand{\labelenumi}{(\theenumi)}

% \newtheorem{theorem}{定理}[subsection]
% \newtheorem{lemma}{引理}[subsection]
% \newtheorem*{definition}{定义}
% \newtheorem{property}{性质}[subsection]
% \newtheorem{infer}{推论}[subsection]

\usepackage{color}

\newcommand{\bsym}[1]{\boldsymbol{#1}}
\newcommand{\mypar}[1]{\left( #1 \right)}
\newcommand{\gram}[1]{\bsym{G}\mypar{#1}}
\newcommand{\abs}[1]{\left|#1 \right|}
\newcommand{\Abs}[1]{\left\Vert#1\right\Vert}
\newcommand{\setof}[1]{\left\{#1 \right\}}
\newcommand{\indot}[2]{\left\langle #1,#2 \right\rangle}
\newcommand{\mat}[1]{\left[ #1 \right]}
\newcommand{\myvec}[1]{\left[ #1 \right]^\top}
\newcommand{\sqmat}[3]{\begin{#1}
{#2}_{11} & {#2}_{12} & \cdots & {#2}_{1{#3}} \\
{#2}_{21} & {#2}_{22} & \cdots & {#2}_{2{#3}} \\
\vdots & \vdots &   \ddots     & \vdots \\
{#2}_{{#3}1} & {#2}_{{#3}2} & \cdots & {#2}_{{#3}{#3}}
\end{#1}}
\newcommand{\normmat}[4]{\begin{#1}
{#2}_{11} & {#2}_{12} & \cdots & {#2}_{1{#4}} \\
{#2}_{21} & {#2}_{22} & \cdots & {#2}_{2{#4}} \\
\vdots & \vdots &   \ddots     & \vdots \\
{#2}_{{#3}1} & {#2}_{{#3}2} & \cdots & {#2}_{{#3}{#4}}
\end{#1}}
\newcommand{\nullsp}[1]{\bsym{\mathrm{N}}\mypar{#1}}
\newcommand{\colsp}[1]{\bsym{\mathrm{C}}\mypar{#1}}
\newcommand{\func}[2]{\mathrm{#1}\mypar{#2}}
\newcommand{\entry}[3]{\func{entry}{#1,#2,#3}}
\newcommand{\row}[2]{\func{row}{#1,#2}}
\newcommand{\col}[2]{\func{col}{#1,#2}}
\newcommand{\trace}[1]{\func{Tr}{#1}}
\newcommand{\diag}[1]{\func{diag}{#1}}
\newcommand{\rank}[1]{\func{rank}{#1}}
\newcommand{\adj}[1]{\func{adj}{#1}}
\newcommand{\myspan}[1]{\func{span}{#1}}
\newcommand{\enums}[2]{{#1}_1,{#1}_2,\dots,{#1}_{#2}}
\newcommand{\inv}[1]{{#1}^{-1}}
\newcommand{\pinv}[1]{\inv{\mypar{#1}}}
\newcommand{\ortcom}[1]{{#1}^{\bot}}

\renewcommand{\det}[1]{\func{det}{#1}}

\newcommand{\dif}{\mathrm{d}}
\newcommand{\fracdif}[1]{\frac{\dif}{\dif #1}}

\newcommand{\todo}[1]{{ \textcolor{red}{ TODO: #1}}}

\newcommand{\mata}{\bsym{A}}
\newcommand{\matb}{\bsym{B}}
\newcommand{\matc}{\bsym{C}}
\newcommand{\matd}{\bsym{D}}
\newcommand{\mate}{\bsym{E}}
\newcommand{\matf}{\bsym{F}}
\newcommand{\matfstar}{\bsym{F^*}}
\newcommand{\matg}{\bsym{G}}
\newcommand{\matH}{\bsym{H}}
\newcommand{\mati}{\bsym{I}}
\newcommand{\matj}{\bsym{J}}
\newcommand{\matk}{\bsym{K}}
\newcommand{\matl}{\bsym{L}}
\newcommand{\matlam}{\bsym{\Lambda}}
\newcommand{\matm}{\bsym{M}}
\newcommand{\matn}{\bsym{N}}
\newcommand{\mato}{\bsym{O}}
\newcommand{\matp}{\bsym{P}}
\newcommand{\matq}{\bsym{Q}}
\newcommand{\matr}{\bsym{R}}
\newcommand{\mats}{\bsym{S}}
\newcommand{\matsig}{\bsym{\Sigma}}
\newcommand{\matt}{\bsym{T}}
\newcommand{\matu}{\bsym{U}}
\newcommand{\matv}{\bsym{V}}
\newcommand{\matw}{\bsym{W}}
\newcommand{\matx}{\bsym{X}}
\newcommand{\maty}{\bsym{Y}}
\newcommand{\matz}{\bsym{Z}}

\newcommand{\field}{\bsym{\mathrm{F}}}
\newcommand{\rea}{\bsym{\mathbb{R}}}

\newcommand{\veca}{\bsym{a}}
\newcommand{\vecal}{\bsym{\alpha}}
\newcommand{\vecb}{\bsym{b}}
\newcommand{\vecbeta}{\bsym{\beta}}
\newcommand{\vecc}{\bsym{c}}
\newcommand{\vecd}{\bsym{d}}
\newcommand{\vecdelta}{\bsym{\delta}}
\newcommand{\vece}{\bsym{e}}
\newcommand{\veceps}{\bsym{\epsilon}}
\newcommand{\vecveps}{\bsym{\varepsilon}}
\newcommand{\vecf}{\bsym{f}}
\newcommand{\vecg}{\bsym{g}}
\newcommand{\vecgamma}{\bsym{\gamma}}
\newcommand{\vech}{\bsym{h}}
\newcommand{\veceta}{\bsym{\eta}}
\newcommand{\veci}{\bsym{i}}
\newcommand{\vecj}{\bsym{j}}
\newcommand{\veck}{\bsym{k}}
\newcommand{\vecl}{\bsym{l}}
\newcommand{\vecm}{\bsym{m}}
\newcommand{\vecn}{\bsym{n}}
\newcommand{\veco}{\bsym{o}}
\newcommand{\vecone}{\bsym{1}}
\newcommand{\vecp}{\bsym{p}}
\newcommand{\vecq}{\bsym{q}}
\newcommand{\vecr}{\bsym{r}}
\newcommand{\vecs}{\bsym{s}}
\newcommand{\vect}{\bsym{t}}
\newcommand{\vecu}{\bsym{u}}
\newcommand{\vecv}{\bsym{v}}
\newcommand{\vecw}{\bsym{w}}
\newcommand{\vecx}{\bsym{x}}
\newcommand{\vecxi}{\bsym{\xi}}
\newcommand{\vecy}{\bsym{y}}
\newcommand{\vecz}{\bsym{z}}
\newcommand{\veczero}{\bsym{0}}


\begin{document}
\maketitle
\tableofcontents

\clearpage
\section{矩阵}

% 1.1
\begin{problem}\

\begin{enumerate}
    \item[(1)] \(\begin{bmatrix}
            4 & 3 & 1 \\1&-2&3\\5&7&0
        \end{bmatrix}\begin{bmatrix}
            7 \\2\\1
        \end{bmatrix}=\begin{bmatrix}
            35 \\6\\49
        \end{bmatrix}\)
    \item[(5)]
        \begin{align*}
              & \begin{bmatrix}
                x_1 & x_2 & x_3
            \end{bmatrix}
            \begin{bmatrix}
                a_{11} & a_{12} & a_{13} \\
                a_{12} & a_{22} & a_{23} \\
                a_{13} & a_{23} & a_{33}
            \end{bmatrix}
            \begin{bmatrix}
                x_1 \\x_2\\x_3
            \end{bmatrix}                                                 \\
            = & \begin{bmatrix}\sum_{i=1}^3a_{1i}x_i+\sum_{i=1}^3a_{2i}x_i+\sum_{i=1}^3a_{3i}x_i\end{bmatrix}
            \begin{bmatrix}x_1 \\x_2 \\x_3\end{bmatrix}                                                 \\
            = & \sum_{1 \leq i<j \leq 3} 2 a_{i j} x_i x_j+\sum_{k=1}^3 a_{k k} x_k^2
        \end{align*}

\end{enumerate}
\end{problem}

% 1.2
\begin{problem}\

\begin{align*}
     & \mata\matb           =
    \left[\begin{array}{ccc}
            1 & 1  & 1  \\
            1 & 1  & -1 \\
            1 & -1 & 1
        \end{array}\right]
    \left[\begin{array}{ccc}
            1  & 2  & 3 \\
            -1 & -2 & 4 \\
            0  & 5  & 1
        \end{array}\right]=
    \left[\begin{array}{ccc}
            0 & 5  & 8 \\
            0 & -5 & 6 \\
            2 & 9  & 0
        \end{array}\right] \\
     & 3\mata\matb-2 \mata  =
    3\left[\begin{array}{ccc}
            0 & 5  & 8 \\
            0 & -5 & 6 \\
            2 & 9  & 0
        \end{array}\right]
    -2\left[\begin{array}{ccc}
            1 & 1  & 1  \\
            1 & 1  & -1 \\
            1 & -1 & 1
        \end{array}\right]
    =\left[\begin{array}{ccc}
            -2 & 13  & 22 \\
            -2 & -17 & 20 \\
            4  & 29  & -2
        \end{array}\right]
\end{align*}
\end{problem}

% 1.3
\begin{problem}\

\begin{equation*}
    \begin{array}{lll}
        \mata^2=\begin{bmatrix}3&1\\1&-3\end{bmatrix}\begin{bmatrix}3&1\\1&-3\end{bmatrix}=\begin{bmatrix}10&0\\0&10\end{bmatrix}      \\
        \mata^{50}=\mypar{\mata^2}^{25}={\begin{bmatrix}10&0\\0&10\end{bmatrix}}^{25}=\begin{bmatrix}10^{25}&0\\0&10^{25}\end{bmatrix} \\
        \mata^{51}=\mata^{50}\mata=\begin{bmatrix}10^{25}&0\\0&10^{25}\end{bmatrix}\begin{bmatrix}3&1\\1&-3\end{bmatrix}=\begin{bmatrix}3\cdot10^{25}&10^{25}\\10^{25}&-3\cdot10^{25}\end{bmatrix}
    \end{array}
\end{equation*}
\end{problem}

% 1.4
\begin{problem}\

\begin{enumerate}
    \item \begin{equation*}
              \begin{array}{lll}
                  \mata\text{为对称矩阵}\Rightarrow \mata^\top=\mata \\
                  \mypar{\matb^\top\mata\matb}^\top=\matb^\top\mypar{\matb^\top\mata}^\top=
                  \matb^\top\mata^\top\mypar{\matb^\top}^\top=\matb^\top\mata\matb
              \end{array}
          \end{equation*}
    \item \begin{itemize}
              \item 充分条件:\begin{equation*}
                        \mata\matb=\matb\mata\Rightarrow\mypar{\mata\matb}^\top=\matb^\top\mata^\top=\matb\mata=\mata\matb
                    \end{equation*}
              \item 必要条件:\begin{equation*}
                        \mata\matb=\mypar{\mata\matb}^\top\Rightarrow\mata\matb=\mypar{\mata\matb}^\top=\matb^\top\mata^\top=\matb\mata
                    \end{equation*}
          \end{itemize}
\end{enumerate}
\end{problem}

% 1.5
\begin{problem}\

必要性显然成立。下面证明充分性。

设矩阵\(\mata=\mat{a_{ij}}_{m\times n}\)。由\(\mata^\top\mata=\mato\)和定义1.2.5,有

\begin{equation*}
    \mat{\mata^\top\mata}_{ii}=\sum_{k=1}^m\mat{\mata^\top}_{ik}\mat{\mata}_{ki}=\sum_{k=1}^ma_{ki}^2=0
\end{equation*}

对\(i=1,2,\dots,n\)均成立。因此有\(\mata=\mato\)。

\end{problem}

% 1.6
\begin{problem}\

\begin{equation*}
    \mata^5=\begin{bmatrix}a^5&0&0\\0&b&5\\0&0&c^5\end{bmatrix},\matb^3=\begin{bmatrix}0&0&0\\0&0&0\\0&0&0\end{bmatrix},\matc^n=\begin{bmatrix}\cos n\theta&\sin n\theta\\-\sin n\theta&\cos n\theta\end{bmatrix}
\end{equation*}

数学归纳法格式:

猜想\(\matc^n=\begin{bmatrix}\cos n\theta&\sin n\theta\\-\sin n\theta&\cos n\theta\end{bmatrix}\)对\(n\in N^+\)成立。

当\(n=1\)时,\(\matc^n=\begin{bmatrix}\cos \theta&\sin \theta\\-\sin \theta&\cos \theta\end{bmatrix}\),结论成立。

设当\(n=k\)时结论成立,则当\(n=k+1\)时,
\begin{align*}
    \matc^{k+1} & =\matc^k\matc=\begin{bmatrix}\cos k\theta&\sin k\theta\\-\sin k\theta&\cos k\theta\end{bmatrix}\begin{bmatrix}\cos \theta&\sin \theta\\-\sin \theta&\cos \theta\end{bmatrix} \\
                & =\begin{bmatrix}\cos (k+1)\theta&\sin (k+1)\theta\\-\sin (k+1)\theta&\cos (k+1)\theta\end{bmatrix}
\end{align*}

由归纳公理知\(\matc^n=\begin{bmatrix}\cos n\theta&\sin n\theta\\-\sin n\theta&\cos n\theta\end{bmatrix}\)对\(n\in N^+\)成立。

\end{problem}

% 1.7
\begin{problem}\

计算矩阵乘法,由等式可得

\begin{equation*}
    \begin{cases}
        3a+a-3=b       \\
        9+0\cdot a-3=6 \\
        2a+3=b
    \end{cases}
\end{equation*}

解线性方程组得\(a=3,b=9\)。

\end{problem}

% 1.8
\begin{problem}\

因为\(\mata^n=\mato\),我们得到

\begin{align*}
      & \mypar{\mati_n-\mata}\mypar{\mati_n+\sum_{i=1}^{n-1}\mata^i} \\
    = & \mati_n+\sum_{i=1}^{n-1}\mata^i-\sum_{i=1}^n\mata^i          \\
    = & \mati_n-\mata^n=\mati_n
\end{align*}

\end{problem}

% 1.9
\begin{problem}\

\begin{enumerate}

    \item[(3)]
        {
        容易发现
        \begin{equation*}
            \matc^2=
            \begin{bmatrix}
                4 & 0 & 0 & 0 \\
                0 & 4 & 0 & 0 \\
                0 & 0 & 4 & 0 \\
                0 & 0 & 0 & 4
            \end{bmatrix}=4\mati_4
        \end{equation*}
        因此根据奇偶性讨论,有
        \begin{equation*}
            \matc^n=
            \begin{cases}
                2^{n-1}\matc & n=2k-1 \\
                2^n\mati_4   & n=2k
            \end{cases}
            (k\in N^+)
        \end{equation*}
        }

    \item[(4)]
        {
        记\(\matd'=\begin{bmatrix}0&1&0\\0&0&1\\0&0&0\end{bmatrix}\),容易发现
        \begin{equation*}
            \matd'^2=\begin{bmatrix}0&0&1\\0&0&0\\0&0&0\end{bmatrix},\matd'^n=\mato(n\ge3)
        \end{equation*}
        因此有
        \begin{align*}
            \matd^n & =\mypar{\mati+\matd'}^n                 \\
                    & =\mati+\sum_{i=1}^n\binom{n}{i}\matd'^i \\
                    & =\mati+n\matd'+\binom{n}{2}\matd'^2     \\
                    & =\begin{bmatrix}
                1 & n & n(n-1)/2 \\
                0 & 1 & n        \\
                0 & 0 & 1
            \end{bmatrix}
        \end{align*}
        }
\end{enumerate}
\end{problem}

% 1.10
\begin{problem}\

\begin{enumerate}
    \item
          {
          设矩阵\(\mata=\mat{a_{ij}}_{n\times n}\)为对角阵,矩阵\(\matb=\mat{b_{ij}}_{n\times n}\)。则有
          \begin{equation*}
              \begin{array}{lll}
                  \mat{\mata\matb}_{ij}=\sum_{k=1}^na_{ik}b_{kj}=a_{ii}b_{ij} \\
                  \mat{\matb\mata}_{ij}=\sum_{k=1}^nb_{ik}a_{kj}=a_{jj}b_{ij}
              \end{array}
          \end{equation*}
          由题设知\(\mat{\mata\matb}_{ij}=\mat{\matb\mata}_{ij}\)对任意\(i,j\in\setof{1,2,\dots,n}\)成立。

          当\(i=j\)时,等式成立;当\(i\neq j\)时,由\(\mata\)的任意性知\(b_{ij}=0\),即\(\matb\)为对角阵。
          }
    \item
          {
          由(1)知,满足要求的矩阵为对角阵。

          设矩阵\(\mata=\mat{a_{ij}}_{n\times n}\)为对角阵,矩阵\(\matb=\mat{b_{ij}}_{n\times n}\)。

          同(1)理,可得\(a_{ii}b_{ij}=a_{jj}b_{ij}\)对任意\(i,j\in\setof{1,2,\dots,n}\)成立。由\(\matb\)的任意性可知\(a_{ii}=a_{jj}\)对任意\(i,j\in\setof{1,2,\dots,n}\)成立,即\(\mata\)为纯量阵。
          }
\end{enumerate}

\end{problem}

% 1.11
\begin{problem}\

设矩阵\(\mata=\mat{a_{ij}}_{n\times n}\),矩阵\(\matb=\mat{b_{ij}}_{n\times n}\)。则有\(\mat{\mata\matb}_{ij}=\sum_{k=1}^na_{ik}b_{kj}=\sum_{k=1}^{i-1}a_{ik}b_{kj}+a_{ii}b_{ij}+\sum_{t=i+1}^na_{it}b_{tj}\)。

当\(i>j\)时,因为\(\mata,\matb\)均为上三角矩阵,因此有\(a_{ik}=b_{tj}=0(k<i,t>=i)\),代入上式可得\(\mat{\mata\matb}_{ij}=0\);

当\(i=j\)时,因为\(\mata,\matb\)均为对角元为\(1\)的上三角矩阵,因此有\(a_{ik}=b_{tj}=0(k<i,t>i)\),代入上式可得\(\mat{\mata\matb}_{ij}=a_{ii}b_{ij}=1\)。

综上,\(\mata\matb\)为对角元为\(1\)的上三角矩阵。

\end{problem}

% 1.12
\begin{problem}\


\begin{equation*}
    \begin{array}{lll}
        \mata=\vecy\vecx^\top=
        \begin{bmatrix}
            x_1y_1 & x_2y_1 & \cdots & x_ny_1 \\
            x_1y_2 & x_2y_2 & \cdots & x_ny_2 \\
            \vdots & \vdots & \ddots & \vdots \\
            x_1y_n & x_2y_n & \cdots & x_ny_n \\
        \end{bmatrix}         \\
        \vecx^\top\vecy=\sum_{i=1}^nx_iy_i \\
    \end{array}
\end{equation*}

因此有
\begin{align*}
    \mata^k & =\underbrace{\mypar{\vecy\vecx^\top}\mypar{\vecy\vecx^\top}\cdots\mypar{\vecy\vecx^\top}}_k                    \\
            & =\vecy\underbrace{\mypar{\vecx^\top\vecy}\mypar{\vecx^\top\vecy}\cdots\mypar{\vecx^\top\vecy}}_{k-1}\vecx^\top \\
            & =\mypar{\vecx^\top\vecy}^{k-1}\vecy\vecx^\top=\mypar{\sum_{i=1}^nx_iy_i}^{k-1}
    \begin{bmatrix}
        x_1y_1 & x_2y_1 & \cdots & x_ny_1 \\
        x_1y_2 & x_2y_2 & \cdots & x_ny_2 \\
        \vdots & \vdots & \ddots & \vdots \\
        x_1y_n & x_2y_n & \cdots & x_ny_n \\
    \end{bmatrix}
\end{align*}

\end{problem}

% 1.13
\begin{problem}\

\begin{align*}
    \vecx^\top\vecx & =
    \begin{bmatrix}
        \frac{1}{2} & 0 & \cdots & 0 & \frac{1}{2}
    \end{bmatrix}
    \begin{bmatrix}
        \frac{1}{2} & 0 & \cdots & 0 & \frac{1}{2}
    \end{bmatrix}^\top=\frac{1}{2}                                        \\
    \mata\matb      & =\mypar{\mati_n-\vecx\vecx^\top}\mypar{\mati_n+2\vecx\vecx^\top} \\
                    & =\mati_n+\vecx\vecx^\top-2\vecx\mypar{\vecx^\top\vecx}\vecx^\top \\
                    & =\mati_n+\vecx\vecx^\top-\vecx\vecx^\top=\mati_n
\end{align*}

\end{problem}

% 1.14
\begin{problem}\

设\(\mata=\begin{bmatrix}\veca_1&\veca_2&\cdots&\veca_n\end{bmatrix}\),\(\vece_i\)表示第\(i\)个分量为\(1\),其余分量为\(0\)的\(n\)阶列向量。由题设可知

\begin{equation*}
    \mata\vece_i=\veca_i=\veczero
\end{equation*}

对\(i=1,2,\dots,n\)均成立。因此有\(\veca_1=\veca_2=\cdots=\veca_n=\veczero\),即\(\mata=\mato\)。

\end{problem}

% 1.15
\begin{problem}\

根据\(\mata^2=\mata,\matb^2=\matb\),可将\(\mypar{\mata+\matb}^2\)展开:

\begin{align*}
    \mypar{\mata+\matb}^2 & =\mata^2+\matb^2+\mata\matb+\matb\mata \\
                          & =\mata+\matb+\mata\matb+\matb\mata
\end{align*}

又因为\(\mypar{\mata+\matb}^2=\mata+\matb\),可得

\begin{equation}\label{eq-1.15}
    \mata\matb+\matb\mata=\mato
\end{equation}

将\eqref{eq-1.15}式左乘\(\mata\),得到\(\mata\matb+\mata\matb\mata=\mato\);将\eqref{eq-1.15}式式左右各乘\(\mata\),得到\(2\mata\matb\mata=\mato\)。将上述两式联立解得\(\mata\matb=\mato\)。

\end{problem}

% 1.16
\begin{problem}\

因为\(\mata^n-2\mata^{n-1}=\mata^{n-1}(\mata-2\mati)\),容易发现

\begin{equation*}
    \mata(\mata-2\mati)=
    \begin{bmatrix}
        1 & 0 & 1 \\0&2&0\\1&0&1
    \end{bmatrix}
    \begin{bmatrix}
        -1 & 0 & 1 \\0&0&0\\1&0&-1
    \end{bmatrix}=\mato
\end{equation*}

因此当\(n\ge2\)时,\(\mata^n-2\mata^{n-1}=\mata^{n-2}\mata(\mata-2\mati)=\mato\)。

\end{problem}

% 1.17
\begin{problem}\

根据\(\mata^2=-\mata,\matb^2=-\matb\),可将\(\mypar{\mata+\matb}^2\)展开:

\begin{align*}
    \mypar{\mata+\matb}^2 & =\mata^2+\matb^2+\mata\matb+\matb\mata \\
                          & =-\mata-\matb+\mata\matb+\matb\mata
\end{align*}

又因为\(\mypar{\mata+\matb}^2=-\mata-\matb\),可得

\begin{equation}\label{eq-1.17}
    \mata\matb+\matb\mata=\mato
\end{equation}

将\eqref{eq-1.17}式左乘\(\mata\),得到\(\mata\matb+\mata\matb\mata=\mato\);将\eqref{eq-1.17}式左右各乘\(\mata\),得到\(2\mata\matb\mata=\mato\)。将上述两式联立解得\(\mata\matb=\mato\)。

\end{problem}

\setcounter{problem}{18}
% 1.19
\begin{problem}\

\begin{itemize}
    \item 首先证明充分性:
          因为\(\vecal^\top\vecal=1\),因此有
          \begin{align*}
              \mata^2 & =\mypar{\mati-\vecal\vecal^\top}^2                                   \\
                      & =\mati-2\vecal\vecal^\top+\vecal\mypar{\vecal^\top\vecal}\vecal^\top \\
                      & =\mati-\vecal\vecal^\top=\mata
          \end{align*}

          充分性得证。

    \item 随后证明必要性:
          因为\(\mata^2=\mata\),因此有
          \begin{align*}
              \mata^2-\mata & =\mat{\mati-2\vecal\vecal^\top+\vecal\mypar{\vecal^\top\vecal}\vecal^\top}-\mypar{\mati-\vecal\vecal^\top} \\
                            & =\mypar{\vecal^\top\vecal-1}\vecal\vecal^\top=\mato
          \end{align*}

          因为\(\vecal\neq\veczero\),因此\(\vecal^\top\vecal-1=0\),即\(\vecal^\top\vecal=1\)。
\end{itemize}

\end{problem}

\setcounter{problem}{21}
% 1.22
\begin{problem}\

\begin{enumerate}
    \item 记\(\mata'=\begin{bmatrix}0&a&0\\0&0&a\\0&0&0\end{bmatrix}\),容易发现
          \begin{equation*}
              \mata'^2=\begin{bmatrix}0&0&a^2\\0&0&0\\0&0&0\end{bmatrix},\mata'^n=\mato(n\ge3)
          \end{equation*}
          因此有
          \begin{align*}
              \mata^k & =\mypar{\mati+\mata'}^k                 \\
                      & =\mati+\sum_{i=1}^k\binom{k}{i}\mata'^i \\
                      & =\mati+k\mata'+\binom{k}{2}\mata'^2     \\
                      & =\begin{bmatrix}
                  1 & ka & \frac{k(k-1)}{2}a^2 \\
                  0 & 1  & ka                  \\
                  0 & 0  & 1
              \end{bmatrix}
          \end{align*}
    \item 记\(\vecal=\begin{bmatrix}1\\2\\3\end{bmatrix}\),\(\vecbeta=\begin{bmatrix}1\\2\\4\end{bmatrix}\)。容易发现\(\mata=\vecal\vecbeta^\top\),因此有
          \begin{align*}
              \mata^k & =\mypar{\vecal\vecbeta^\top}^k=\vecal\mypar{\vecbeta^\top\vecal}^{k-1}\vecbeta^\top \\
                      & =\mypar{17}^{k-1}\vecal\vecbeta^\top=\mypar{17}^{k-1}
              \begin{bmatrix}
                  1 & 2 & 4 \\2&4&8\\3&6&12
              \end{bmatrix}
          \end{align*}
\end{enumerate}

\end{problem}

% 1.23
\begin{problem}\

% 记\(\mata_1=\mata\matb\),\(\mata_2=\matb\mata\)。

令\(\matc=\vece_i\vece_j\),其中\(\vece_i\)表示第\(i\)个分量为\(1\),其余分量为\(0\)的\(n\)阶列向量。因此有
\begin{align*}
    \mat{\mata\matb\matc}_{tt} & =\sum_{k=1}^n\mat{\mata\matb}_{tk}\mat{\matc}_{kt}=
    \begin{cases}
        \mat{\mata\matb}_{ji} & t=j     \\
        0                     & t\neq j
    \end{cases}                                                       \\
    \mat{\matc\matb\mata}_{tt} & =\sum_{k=1}^n\mat{\matc}_{tk}\mat{\matb\mata}_{kt}=
    \begin{cases}
        \mat{\matb\mata}_{ji} & t=i     \\
        0                     & t\neq i
    \end{cases}
\end{align*}

因为\(\trace{\mata\matb\matc}=\trace{\matc\matb\mata}\),因此有\(\mat{\mata\matb}_{ji}=\mat{\matb\mata}_{ji}\)对\(i,j\in\setof{1,2,\dots,n}\)成立,即\(\mata\matb=\matb\mata\)。

\end{problem}

% 1.24
\begin{problem}\

定义子矩阵如下:
\begin{equation*}
    \mata_3=
    \begin{bmatrix}
        -1 & 2 \\1&1
    \end{bmatrix},
    \matb_1=
    \begin{bmatrix}
        1 & 0 \\-1&2
    \end{bmatrix},
    \matb_3=
    \begin{bmatrix}
        1 & 0 \\-1&-1
    \end{bmatrix},
    \matb_4=
    \begin{bmatrix}
        4 & 1 \\2&0
    \end{bmatrix}
\end{equation*}

则有
\begin{align*}
    \mata\matb & =
    \begin{bmatrix}
        \mati & \mato \\\mata_3&\mati
    \end{bmatrix}
    \begin{bmatrix}
        \matb_1 & \mati \\\matb_3&\matb_4
    \end{bmatrix} \\&=
    \begin{bmatrix}
        \matb_1 & \mati \\\mata_3\matb_1+\matb_3&\mata_3+\matb_4
    \end{bmatrix} \\&=
    \begin{bmatrix}
        1  & 0 & 1 & 0 \\
        -1 & 2 & 0 & 1 \\
        -2 & 4 & 3 & 3 \\
        -1 & 1 & 3 & 1
    \end{bmatrix}
\end{align*}

\end{problem}

\clearpage
\section{线性方程组}

\setcounter{problem}{3}
% 2.4(2)
\begin{problem}\



\end{problem}

% 2.5
\begin{problem}\



\end{problem}

% 2.6
\begin{problem}\



\end{problem}

% 2.7
\begin{problem}\



\end{problem}

% 2.8
\begin{problem}\



\end{problem}

\setcounter{problem}{13}
% 2.14
\begin{problem}\



\end{problem}

% 2.15
\begin{problem}\



\end{problem}

\setcounter{problem}{20}
% 2.21
\begin{problem}\



\end{problem}

% 2.22
\begin{problem}\



\end{problem}

\setcounter{problem}{25}
% 2.26(3)
\begin{problem}\



\end{problem}

\setcounter{problem}{27}
% 2.28(1)
\begin{problem}\



\end{problem}

% 2.29(1)
\begin{problem}\



\end{problem}

% 2.30
\begin{problem}\



\end{problem}

% 2.31
\begin{problem}\



\end{problem}

% 2.32
\begin{problem}\



\end{problem}

% 2.33
\begin{problem}\



\end{problem}

\setcounter{problem}{34}
% 2.35
\begin{problem}\



\end{problem}

\setcounter{problem}{36}
% 2.37
\begin{problem}\



\end{problem}

\setcounter{problem}{39}
% 2.40(1)
\begin{problem}\



\end{problem}

% 2.41
\begin{problem}\



\end{problem}

\setcounter{problem}{43}
% 2.44
\begin{problem}\



\end{problem}

\setcounter{problem}{45}
% 2.46
\begin{problem}\



\end{problem}

% 2.47(2)
\begin{problem}\



\end{problem}

\end{document}
