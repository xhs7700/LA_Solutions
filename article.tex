\documentclass{ctexart}
\usepackage{anyfontsize}
\usepackage{hyperref}
\usepackage{graphicx}
\usepackage{amsmath,amsthm}

\author{夏海淞}
\title{线性代数习题答案}
\date{\today}

\ctexset { section = { name={第,章} } }
\ctexset { section = { number={\chinese {section}} } }

% \newtheorem{theorem}{定理}[subsection]
% \newtheorem{lemma}{引理}[subsection]
% \newtheorem*{definition}{定义}
% \newtheorem{property}{性质}[subsection]
% \newtheorem{infer}{推论}[subsection]

\usepackage{color}

\newcommand{\bsym}[1]{\boldsymbol{#1}}
\newcommand{\mypar}[1]{\left( #1 \right)}
\newcommand{\gram}[1]{\bsym{G}\mypar{#1}}
\newcommand{\abs}[1]{\left|#1 \right|}
\newcommand{\Abs}[1]{\left\Vert#1\right\Vert}
\newcommand{\setof}[1]{\left\{#1 \right\}}
\newcommand{\indot}[2]{\left\langle #1,#2 \right\rangle}
\newcommand{\mat}[1]{\left[ #1 \right]}
\newcommand{\myvec}[1]{\left[ #1 \right]^\top}
\newcommand{\sqmat}[3]{\begin{#1}
{#2}_{11} & {#2}_{12} & \cdots & {#2}_{1{#3}} \\
{#2}_{21} & {#2}_{22} & \cdots & {#2}_{2{#3}} \\
\vdots & \vdots &   \ddots     & \vdots \\
{#2}_{{#3}1} & {#2}_{{#3}2} & \cdots & {#2}_{{#3}{#3}}
\end{#1}}
\newcommand{\normmat}[4]{\begin{#1}
{#2}_{11} & {#2}_{12} & \cdots & {#2}_{1{#4}} \\
{#2}_{21} & {#2}_{22} & \cdots & {#2}_{2{#4}} \\
\vdots & \vdots &   \ddots     & \vdots \\
{#2}_{{#3}1} & {#2}_{{#3}2} & \cdots & {#2}_{{#3}{#4}}
\end{#1}}
\newcommand{\nullsp}[1]{\bsym{\mathrm{N}}\mypar{#1}}
\newcommand{\colsp}[1]{\bsym{\mathrm{C}}\mypar{#1}}
\newcommand{\func}[2]{\mathrm{#1}\mypar{#2}}
\newcommand{\entry}[3]{\func{entry}{#1,#2,#3}}
\newcommand{\row}[2]{\func{row}{#1,#2}}
\newcommand{\col}[2]{\func{col}{#1,#2}}
\newcommand{\trace}[1]{\func{Tr}{#1}}
\newcommand{\diag}[1]{\func{diag}{#1}}
\newcommand{\rank}[1]{\func{rank}{#1}}
\newcommand{\adj}[1]{\func{adj}{#1}}
\newcommand{\myspan}[1]{\func{span}{#1}}
\newcommand{\enums}[2]{{#1}_1,{#1}_2,\dots,{#1}_{#2}}
\newcommand{\inv}[1]{{#1}^{-1}}
\newcommand{\pinv}[1]{\inv{\mypar{#1}}}
\newcommand{\ortcom}[1]{{#1}^{\bot}}

\renewcommand{\det}[1]{\func{det}{#1}}

\newcommand{\dif}{\mathrm{d}}
\newcommand{\fracdif}[1]{\frac{\dif}{\dif #1}}

\newcommand{\todo}[1]{{ \textcolor{red}{ TODO: #1}}}

\newcommand{\mata}{\bsym{A}}
\newcommand{\matb}{\bsym{B}}
\newcommand{\matc}{\bsym{C}}
\newcommand{\matd}{\bsym{D}}
\newcommand{\mate}{\bsym{E}}
\newcommand{\matf}{\bsym{F}}
\newcommand{\matfstar}{\bsym{F^*}}
\newcommand{\matg}{\bsym{G}}
\newcommand{\matH}{\bsym{H}}
\newcommand{\mati}{\bsym{I}}
\newcommand{\matj}{\bsym{J}}
\newcommand{\matk}{\bsym{K}}
\newcommand{\matl}{\bsym{L}}
\newcommand{\matlam}{\bsym{\Lambda}}
\newcommand{\matm}{\bsym{M}}
\newcommand{\matn}{\bsym{N}}
\newcommand{\mato}{\bsym{O}}
\newcommand{\matp}{\bsym{P}}
\newcommand{\matq}{\bsym{Q}}
\newcommand{\matr}{\bsym{R}}
\newcommand{\mats}{\bsym{S}}
\newcommand{\matsig}{\bsym{\Sigma}}
\newcommand{\matt}{\bsym{T}}
\newcommand{\matu}{\bsym{U}}
\newcommand{\matv}{\bsym{V}}
\newcommand{\matw}{\bsym{W}}
\newcommand{\matx}{\bsym{X}}
\newcommand{\maty}{\bsym{Y}}
\newcommand{\matz}{\bsym{Z}}

\newcommand{\field}{\bsym{\mathrm{F}}}
\newcommand{\rea}{\bsym{\mathbb{R}}}

\newcommand{\veca}{\bsym{a}}
\newcommand{\vecal}{\bsym{\alpha}}
\newcommand{\vecb}{\bsym{b}}
\newcommand{\vecbeta}{\bsym{\beta}}
\newcommand{\vecc}{\bsym{c}}
\newcommand{\vecd}{\bsym{d}}
\newcommand{\vecdelta}{\bsym{\delta}}
\newcommand{\vece}{\bsym{e}}
\newcommand{\veceps}{\bsym{\epsilon}}
\newcommand{\vecveps}{\bsym{\varepsilon}}
\newcommand{\vecf}{\bsym{f}}
\newcommand{\vecg}{\bsym{g}}
\newcommand{\vecgamma}{\bsym{\gamma}}
\newcommand{\vech}{\bsym{h}}
\newcommand{\veceta}{\bsym{\eta}}
\newcommand{\veci}{\bsym{i}}
\newcommand{\vecj}{\bsym{j}}
\newcommand{\veck}{\bsym{k}}
\newcommand{\vecl}{\bsym{l}}
\newcommand{\vecm}{\bsym{m}}
\newcommand{\vecn}{\bsym{n}}
\newcommand{\veco}{\bsym{o}}
\newcommand{\vecone}{\bsym{1}}
\newcommand{\vecp}{\bsym{p}}
\newcommand{\vecq}{\bsym{q}}
\newcommand{\vecr}{\bsym{r}}
\newcommand{\vecs}{\bsym{s}}
\newcommand{\vect}{\bsym{t}}
\newcommand{\vecu}{\bsym{u}}
\newcommand{\vecv}{\bsym{v}}
\newcommand{\vecw}{\bsym{w}}
\newcommand{\vecx}{\bsym{x}}
\newcommand{\vecxi}{\bsym{\xi}}
\newcommand{\vecy}{\bsym{y}}
\newcommand{\vecz}{\bsym{z}}
\newcommand{\veczero}{\bsym{0}}


\begin{document}
\maketitle
\tableofcontents
\section{矩阵}

% 1.1
\begin{problem}\

\begin{enumerate}
    \item[(1)] \(\begin{bmatrix}
            4 & 3 & 1 \\1&-2&3\\5&7&0
        \end{bmatrix}\begin{bmatrix}
            7 \\2\\1
        \end{bmatrix}=\begin{bmatrix}
            35 \\6\\49
        \end{bmatrix}\)
    \item[(5)]
        \begin{align*}
              & \begin{bmatrix}
                    x_1 & x_2 & x_3
                \end{bmatrix}
            \begin{bmatrix}
                a_{11} & a_{12} & a_{13} \\
                a_{12} & a_{22} & a_{23} \\
                a_{13} & a_{23} & a_{33}
            \end{bmatrix}
            \begin{bmatrix}
                x_1 \\x_2\\x_3
            \end{bmatrix}                                                                                    \\
            = & \begin{bmatrix}\sum_{i=1}^3a_{1i}x_i+\sum_{i=1}^3a_{2i}x_i+\sum_{i=1}^3a_{3i}x_i\end{bmatrix}
            \begin{bmatrix}x_1 \\x_2 \\x_3\end{bmatrix}                                                       \\
            = & \sum_{1 \leq i<j \leq 3} 2 a_{i j} x_i x_j+\sum_{k=1}^3 a_{k k} x_k^2
        \end{align*}

\end{enumerate}
\end{problem}

% 1.2
\begin{problem}\

\begin{align*}
     & \mata\matb           =
    \left[\begin{array}{ccc}
                  1 & 1  & 1  \\
                  1 & 1  & -1 \\
                  1 & -1 & 1
              \end{array}\right]
    \left[\begin{array}{ccc}
                  1  & 2  & 3 \\
                  -1 & -2 & 4 \\
                  0  & 5  & 1
              \end{array}\right]=
    \left[\begin{array}{ccc}
                  0 & 5  & 8 \\
                  0 & -5 & 6 \\
                  2 & 9  & 0
              \end{array}\right]  \\
     & 3\mata\matb-2 \mata  =
    3\left[\begin{array}{ccc}
                   0 & 5  & 8 \\
                   0 & -5 & 6 \\
                   2 & 9  & 0
               \end{array}\right]
    -2\left[\begin{array}{ccc}
                    1 & 1  & 1  \\
                    1 & 1  & -1 \\
                    1 & -1 & 1
                \end{array}\right]
    =\left[\begin{array}{ccc}
                   -2 & 13  & 22 \\
                   -2 & -17 & 20 \\
                   4  & 29  & -2
               \end{array}\right]
\end{align*}
\end{problem}

% 1.3
\begin{problem}\

\begin{equation*}
    \begin{array}{lll}
        \mata^2=\begin{bmatrix}3&1\\1&-3\end{bmatrix}\begin{bmatrix}3&1\\1&-3\end{bmatrix}=\begin{bmatrix}10&0\\0&10\end{bmatrix}      \\
        \mata^{50}=\mypar{\mata^2}^{25}={\begin{bmatrix}10&0\\0&10\end{bmatrix}}^{25}=\begin{bmatrix}10^{25}&0\\0&10^{25}\end{bmatrix} \\
        \mata^{51}=\mata^{50}\mata=\begin{bmatrix}10^{25}&0\\0&10^{25}\end{bmatrix}\begin{bmatrix}3&1\\1&-3\end{bmatrix}=\begin{bmatrix}3\cdot10^{25}&10^{25}\\10^{25}&-3\cdot10^{25}\end{bmatrix}
    \end{array}
\end{equation*}
\end{problem}

% 1.4
\begin{problem}\

\begin{enumerate}
    \item \begin{equation*}
              \begin{array}{lll}
                  \mata\text{为对称矩阵}\Rightarrow \mata^\top=\mata \\
                  \mypar{\matb^\top\mata\matb}^\top=\matb^\top\mypar{\matb^\top\mata}^\top=
                  \matb^\top\mata^\top\mypar{\matb^\top}^\top=\matb^\top\mata\matb
              \end{array}
          \end{equation*}
    \item \begin{itemize}
              \item 充分条件:\begin{equation*}
                        \mata\matb=\matb\mata\Rightarrow\mypar{\mata\matb}^\top=\matb^\top\mata^\top=\matb\mata=\mata\matb
                    \end{equation*}
              \item 必要条件:\begin{equation*}
                        \mata\matb=\mypar{\mata\matb}^\top\Rightarrow\mata\matb=\mypar{\mata\matb}^\top=\matb^\top\mata^\top=\matb\mata
                    \end{equation*}
          \end{itemize}
\end{enumerate}
\end{problem}

% 1.5
\begin{problem}\

必要性显然成立。下面证明充分性。

设\(\mata\in\rea^{m\times n}\),\(\entry{\mata}{i}{j}=a_{ij}\)。

由\(\mata^\top\mata=\mato\)和定义1.2.5,有

\begin{equation*}
    \mat{\mata^\top\mata}_{ii}=\sum_{k=1}^m\mat{\mata^\top}_{ik}\mat{\mata}_{ki}=\sum_{k=1}^ma_{ki}^2=0
\end{equation*}

对\(i=1,2,\dots,n\)均成立。因此有\(\mata=\mato\)。

\end{problem}

% 1.6
\begin{problem}\

\begin{equation*}
    \mata^5=\begin{bmatrix}a^5&0&0\\0&b&5\\0&0&c^5\end{bmatrix},\matb^3=\begin{bmatrix}0&0&0\\0&0&0\\0&0&0\end{bmatrix},\matc^n=\begin{bmatrix}\cos n\theta&\sin n\theta\\-\sin n\theta&\cos n\theta\end{bmatrix}
\end{equation*}

数学归纳法格式:

猜想\(\matc^n=\begin{bmatrix}\cos n\theta&\sin n\theta\\-\sin n\theta&\cos n\theta\end{bmatrix}\)对\(n\in N^+\)成立。

当\(n=1\)时,\(\matc^n=\begin{bmatrix}\cos \theta&\sin \theta\\-\sin \theta&\cos \theta\end{bmatrix}\),结论成立。

设当\(n=k\)时结论成立,则当\(n=k+1\)时,
\begin{align*}
    \matc^{k+1} & =\matc^k\matc=\begin{bmatrix}\cos k\theta&\sin k\theta\\-\sin k\theta&\cos k\theta\end{bmatrix}\begin{bmatrix}\cos \theta&\sin \theta\\-\sin \theta&\cos \theta\end{bmatrix} \\
                & =\begin{bmatrix}\cos (k+1)\theta&\sin (k+1)\theta\\-\sin (k+1)\theta&\cos (k+1)\theta\end{bmatrix}
\end{align*}

由归纳公理知\(\matc^n=\begin{bmatrix}\cos n\theta&\sin n\theta\\-\sin n\theta&\cos n\theta\end{bmatrix}\)对\(n\in N^+\)成立。

\end{problem}

% 1.7
\begin{problem}

\end{problem}

% 1.8
\begin{problem}

\end{problem}

% 1.9
\begin{problem}\

\begin{enumerate}
    \item[(3)]
    \item[(4)]
\end{enumerate}
\end{problem}

% 1.10
\begin{problem}

\end{problem}

% 1.11
\begin{problem}

\end{problem}

% 1.12
\begin{problem}

\end{problem}

% 1.13
\begin{problem}

\end{problem}

% 1.14
\begin{problem}

\end{problem}

% 1.15
\begin{problem}

\end{problem}

% 1.16
\begin{problem}

\end{problem}

% 1.17
\begin{problem}

\end{problem}

\setcounter{problem}{18}
% 1.19
\begin{problem}

\end{problem}

\setcounter{problem}{21}
% 1.22
\begin{problem}

\end{problem}

% 1.23
\begin{problem}

\end{problem}

% 1.24
\begin{problem}

\end{problem}

\end{document}
