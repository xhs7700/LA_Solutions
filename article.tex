\documentclass{ctexart}
\usepackage{anyfontsize}
\usepackage{hyperref}
\usepackage{graphicx}
\usepackage{amsmath,amsthm}

\author{夏海淞}
\title{线性代数习题答案}
\date{\today}

\ctexset { section = { name={第,章} } }
\ctexset { section = { number={\chinese {section}} } }

\newtheorem{problem}{习题}[section]
\newtheorem{extraprob}{附加习题}[section]
\newtheorem{suplprob}{补充习题}[section]

\renewcommand{\labelenumi}{(\theenumi)}
\renewcommand{\proofname}{解}

% \newtheorem{theorem}{定理}[subsection]
% \newtheorem{lemma}{引理}[subsection]
% \newtheorem*{definition}{定义}
% \newtheorem{property}{性质}[subsection]
% \newtheorem{infer}{推论}[subsection]

\usepackage{color}

\newcommand{\bsym}[1]{\boldsymbol{#1}}
\newcommand{\mypar}[1]{\left( #1 \right)}
\newcommand{\gram}[1]{\bsym{G}\mypar{#1}}
\newcommand{\abs}[1]{\left|#1 \right|}
\newcommand{\Abs}[1]{\left\Vert#1\right\Vert}
\newcommand{\setof}[1]{\left\{#1 \right\}}
\newcommand{\indot}[2]{\left\langle #1,#2 \right\rangle}
\newcommand{\mat}[1]{\left[ #1 \right]}
\newcommand{\myvec}[1]{\left[ #1 \right]^\top}
\newcommand{\sqmat}[3]{\begin{#1}
{#2}_{11} & {#2}_{12} & \cdots & {#2}_{1{#3}} \\
{#2}_{21} & {#2}_{22} & \cdots & {#2}_{2{#3}} \\
\vdots & \vdots &   \ddots     & \vdots \\
{#2}_{{#3}1} & {#2}_{{#3}2} & \cdots & {#2}_{{#3}{#3}}
\end{#1}}
\newcommand{\normmat}[4]{\begin{#1}
{#2}_{11} & {#2}_{12} & \cdots & {#2}_{1{#4}} \\
{#2}_{21} & {#2}_{22} & \cdots & {#2}_{2{#4}} \\
\vdots & \vdots &   \ddots     & \vdots \\
{#2}_{{#3}1} & {#2}_{{#3}2} & \cdots & {#2}_{{#3}{#4}}
\end{#1}}
\newcommand{\nullsp}[1]{\bsym{\mathrm{N}}\mypar{#1}}
\newcommand{\colsp}[1]{\bsym{\mathrm{C}}\mypar{#1}}
\newcommand{\func}[2]{\mathrm{#1}\mypar{#2}}
\newcommand{\entry}[3]{\func{entry}{#1,#2,#3}}
\newcommand{\row}[2]{\func{row}{#1,#2}}
\newcommand{\col}[2]{\func{col}{#1,#2}}
\newcommand{\trace}[1]{\func{Tr}{#1}}
\newcommand{\diag}[1]{\func{diag}{#1}}
\newcommand{\rank}[1]{\func{rank}{#1}}
\newcommand{\adj}[1]{\func{adj}{#1}}
\newcommand{\myspan}[1]{\func{span}{#1}}
\newcommand{\enums}[2]{{#1}_1,{#1}_2,\dots,{#1}_{#2}}
\newcommand{\inv}[1]{{#1}^{-1}}
\newcommand{\pinv}[1]{\inv{\mypar{#1}}}
\newcommand{\ortcom}[1]{{#1}^{\bot}}

\renewcommand{\det}[1]{\func{det}{#1}}

\newcommand{\dif}{\mathrm{d}}
\newcommand{\fracdif}[1]{\frac{\dif}{\dif #1}}

\newcommand{\todo}[1]{{ \textcolor{red}{ TODO: #1}}}

\newcommand{\mata}{\bsym{A}}
\newcommand{\matb}{\bsym{B}}
\newcommand{\matc}{\bsym{C}}
\newcommand{\matd}{\bsym{D}}
\newcommand{\mate}{\bsym{E}}
\newcommand{\matf}{\bsym{F}}
\newcommand{\matfstar}{\bsym{F^*}}
\newcommand{\matg}{\bsym{G}}
\newcommand{\matH}{\bsym{H}}
\newcommand{\mati}{\bsym{I}}
\newcommand{\matj}{\bsym{J}}
\newcommand{\matk}{\bsym{K}}
\newcommand{\matl}{\bsym{L}}
\newcommand{\matlam}{\bsym{\Lambda}}
\newcommand{\matm}{\bsym{M}}
\newcommand{\matn}{\bsym{N}}
\newcommand{\mato}{\bsym{O}}
\newcommand{\matp}{\bsym{P}}
\newcommand{\matq}{\bsym{Q}}
\newcommand{\matr}{\bsym{R}}
\newcommand{\mats}{\bsym{S}}
\newcommand{\matsig}{\bsym{\Sigma}}
\newcommand{\matt}{\bsym{T}}
\newcommand{\matu}{\bsym{U}}
\newcommand{\matv}{\bsym{V}}
\newcommand{\matw}{\bsym{W}}
\newcommand{\matx}{\bsym{X}}
\newcommand{\maty}{\bsym{Y}}
\newcommand{\matz}{\bsym{Z}}

\newcommand{\field}{\bsym{\mathrm{F}}}
\newcommand{\rea}{\bsym{\mathbb{R}}}

\newcommand{\veca}{\bsym{a}}
\newcommand{\vecal}{\bsym{\alpha}}
\newcommand{\vecb}{\bsym{b}}
\newcommand{\vecbeta}{\bsym{\beta}}
\newcommand{\vecc}{\bsym{c}}
\newcommand{\vecd}{\bsym{d}}
\newcommand{\vecdelta}{\bsym{\delta}}
\newcommand{\vece}{\bsym{e}}
\newcommand{\veceps}{\bsym{\epsilon}}
\newcommand{\vecveps}{\bsym{\varepsilon}}
\newcommand{\vecf}{\bsym{f}}
\newcommand{\vecg}{\bsym{g}}
\newcommand{\vecgamma}{\bsym{\gamma}}
\newcommand{\vech}{\bsym{h}}
\newcommand{\veceta}{\bsym{\eta}}
\newcommand{\veci}{\bsym{i}}
\newcommand{\vecj}{\bsym{j}}
\newcommand{\veck}{\bsym{k}}
\newcommand{\vecl}{\bsym{l}}
\newcommand{\vecm}{\bsym{m}}
\newcommand{\vecn}{\bsym{n}}
\newcommand{\veco}{\bsym{o}}
\newcommand{\vecone}{\bsym{1}}
\newcommand{\vecp}{\bsym{p}}
\newcommand{\vecq}{\bsym{q}}
\newcommand{\vecr}{\bsym{r}}
\newcommand{\vecs}{\bsym{s}}
\newcommand{\vect}{\bsym{t}}
\newcommand{\vecu}{\bsym{u}}
\newcommand{\vecv}{\bsym{v}}
\newcommand{\vecw}{\bsym{w}}
\newcommand{\vecx}{\bsym{x}}
\newcommand{\vecxi}{\bsym{\xi}}
\newcommand{\vecy}{\bsym{y}}
\newcommand{\vecz}{\bsym{z}}
\newcommand{\veczero}{\bsym{0}}


\begin{document}
\maketitle
\tableofcontents

\clearpage
\section{矩阵}

% 1.1
\begin{problem}
计算下列矩阵的乘积:

\begin{enumerate}
    \item[(1)] \(\begin{bmatrix}
            4 & 3 & 1 \\1&-2&3\\5&7&0
        \end{bmatrix}\begin{bmatrix}
            7 \\2\\1
        \end{bmatrix}\)
    \item[(5)] \(\begin{bmatrix}
            x_1 & x_2 & x_3
        \end{bmatrix}
        \begin{bmatrix}
            a_{11} & a_{12} & a_{13} \\
            a_{12} & a_{22} & a_{23} \\
            a_{13} & a_{23} & a_{33}
        \end{bmatrix}
        \begin{bmatrix}
            x_1 \\x_2\\x_3
        \end{bmatrix}\)
\end{enumerate}
\end{problem}
\begin{proof}
    \begin{enumerate}
        \item[(1)] \(\begin{bmatrix}
                4 & 3 & 1 \\1&-2&3\\5&7&0
            \end{bmatrix}\begin{bmatrix}
                7 \\2\\1
            \end{bmatrix}=\begin{bmatrix}
                35 \\6\\49
            \end{bmatrix}\)
        \item[(5)]
            \begin{align*}
                  & \begin{bmatrix}
                    x_1 & x_2 & x_3
                \end{bmatrix}
                \begin{bmatrix}
                    a_{11} & a_{12} & a_{13} \\
                    a_{12} & a_{22} & a_{23} \\
                    a_{13} & a_{23} & a_{33}
                \end{bmatrix}
                \begin{bmatrix}
                    x_1 \\x_2\\x_3
                \end{bmatrix}                                                 \\
                = & \begin{bmatrix}\sum_{i=1}^3a_{1i}x_i+\sum_{i=1}^3a_{2i}x_i+\sum_{i=1}^3a_{3i}x_i\end{bmatrix}
                \begin{bmatrix}x_1 \\x_2 \\x_3\end{bmatrix}                                                 \\
                = & \sum_{1 \leq i<j \leq 3} 2 a_{i j} x_i x_j+\sum_{k=1}^3 a_{k k} x_k^2
            \end{align*}

    \end{enumerate}
\end{proof}

% 1.2
\begin{problem}
设\(\mata=\begin{bmatrix}
    1 & 1  & 1  \\
    1 & 1  & -1 \\
    1 & -1 & 1
\end{bmatrix}\),\(\matb=\begin{bmatrix}
    1  & 2  & 3 \\
    -1 & -2 & 4 \\
    0  & 5  & 1
\end{bmatrix}\),求\(3\mata\matb-2\mata\)。
\end{problem}
\begin{proof}
    \begin{align*}
         & \mata\matb           =
        \left[\begin{array}{ccc}
                1 & 1  & 1  \\
                1 & 1  & -1 \\
                1 & -1 & 1
            \end{array}\right]
        \left[\begin{array}{ccc}
                1  & 2  & 3 \\
                -1 & -2 & 4 \\
                0  & 5  & 1
            \end{array}\right]=
        \left[\begin{array}{ccc}
                0 & 5  & 8 \\
                0 & -5 & 6 \\
                2 & 9  & 0
            \end{array}\right] \\
         & 3\mata\matb-2 \mata  =
        3\left[\begin{array}{ccc}
                0 & 5  & 8 \\
                0 & -5 & 6 \\
                2 & 9  & 0
            \end{array}\right]
        -2\left[\begin{array}{ccc}
                1 & 1  & 1  \\
                1 & 1  & -1 \\
                1 & -1 & 1
            \end{array}\right]
        =\left[\begin{array}{ccc}
                -2 & 13  & 22 \\
                -2 & -17 & 20 \\
                4  & 29  & -2
            \end{array}\right]
    \end{align*}
\end{proof}

% 1.3
\begin{problem}
设\(\mata=\begin{bmatrix}3&1\\1&-3\end{bmatrix}\),求\(\mata^{50}\)和\(\mata^{51}\)。
\end{problem}
\begin{proof}
    \begin{equation*}
        \begin{array}{lll}
            \mata^2=\begin{bmatrix}3&1\\1&-3\end{bmatrix}\begin{bmatrix}3&1\\1&-3\end{bmatrix}=\begin{bmatrix}10&0\\0&10\end{bmatrix}      \\
            \mata^{50}=\mypar{\mata^2}^{25}={\begin{bmatrix}10&0\\0&10\end{bmatrix}}^{25}=\begin{bmatrix}10^{25}&0\\0&10^{25}\end{bmatrix} \\
            \mata^{51}=\mata^{50}\mata=\begin{bmatrix}10^{25}&0\\0&10^{25}\end{bmatrix}\begin{bmatrix}3&1\\1&-3\end{bmatrix}=\begin{bmatrix}3\cdot10^{25}&10^{25}\\10^{25}&-3\cdot10^{25}\end{bmatrix}
        \end{array}
    \end{equation*}
\end{proof}

% 1.4
\begin{problem}
\begin{enumerate}
    \item 设\(\mata,\matb\)为\(n\)阶矩阵,且\(\mata\)为对称矩阵,证明\(\matb^\top\mata\matb\)也是对称矩阵;
    \item 设\(\mata,\matb\)为\(n\)阶对称矩阵,证明\(\mata\matb\)是对称矩阵的充要条件是\(\mata\matb=\matb\mata\)。
\end{enumerate}
\end{problem}
\begin{proof}
    \begin{enumerate}
        \item \begin{equation*}
                  \begin{array}{lll}
                      \mata\text{为对称矩阵}\Rightarrow \mata^\top=\mata \\
                      \mypar{\matb^\top\mata\matb}^\top=\matb^\top\mypar{\matb^\top\mata}^\top=
                      \matb^\top\mata^\top\mypar{\matb^\top}^\top=\matb^\top\mata\matb
                  \end{array}
              \end{equation*}
        \item \begin{itemize}
                  \item 充分条件:\begin{equation*}
                            \mata\matb=\matb\mata\Rightarrow\mypar{\mata\matb}^\top=\matb^\top\mata^\top=\matb\mata=\mata\matb
                        \end{equation*}
                  \item 必要条件:\begin{equation*}
                            \mata\matb=\mypar{\mata\matb}^\top\Rightarrow\mata\matb=\mypar{\mata\matb}^\top=\matb^\top\mata^\top=\matb\mata
                        \end{equation*}
              \end{itemize}
    \end{enumerate}
\end{proof}

% 1.5
\begin{problem}
证明矩阵\(\mata=\mato\)的充分必要条件是方阵\(\mata^\top\mata=\mato\)。
\end{problem}
\begin{proof}
    必要性显然成立。下面证明充分性。

    设矩阵\(\mata=\mat{a_{ij}}_{m\times n}\)。由\(\mata^\top\mata=\mato\)和定义1.2.5,有

    \begin{equation*}
        \mat{\mata^\top\mata}_{ii}=\sum_{k=1}^m\mat{\mata^\top}_{ik}\mat{\mata}_{ki}=\sum_{k=1}^ma_{ki}^2=0
    \end{equation*}

    对\(i=1,2,\dots,n\)均成立。因此有\(\mata=\mato\)。
\end{proof}

% 1.6
\begin{problem}
设

\begin{equation*}
    \mata=\begin{bmatrix}a&0&0\\0&b&0\\0&0&c\end{bmatrix},
    \matb=\begin{bmatrix}0&1&0\\0&0&1\\0&0&0\end{bmatrix},
    \matc=\begin{bmatrix}\cos \theta&\sin \theta\\-\sin \theta&\cos \theta\end{bmatrix}
\end{equation*}

求\(\mata^5,\matb^3,\matc^n\)。
\end{problem}
\begin{proof}
    \begin{equation*}
\mata^5=\begin{bmatrix}a^5&0&0\\0&b^5&0\\0&0&c^5\end{bmatrix},\matb^3=\begin{bmatrix}0&0&0\\0&0&0\\0&0&0\end{bmatrix},\matc^n=\begin{bmatrix}\cos n\theta&\sin n\theta\\-\sin n\theta&\cos n\theta\end{bmatrix}
    \end{equation*}

    数学归纳法格式:

    猜想\(\matc^n=\begin{bmatrix}\cos n\theta&\sin n\theta\\-\sin n\theta&\cos n\theta\end{bmatrix}\)对\(n\in N^+\)成立。

    当\(n=1\)时,\(\matc^n=\begin{bmatrix}\cos \theta&\sin \theta\\-\sin \theta&\cos \theta\end{bmatrix}\),结论成立。

    设当\(n=k\)时结论成立,则当\(n=k+1\)时,
    \begin{align*}
        \matc^{k+1} & =\matc^k\matc=\begin{bmatrix}\cos k\theta&\sin k\theta\\-\sin k\theta&\cos k\theta\end{bmatrix}\begin{bmatrix}\cos \theta&\sin \theta\\-\sin \theta&\cos \theta\end{bmatrix} \\
                    & =\begin{bmatrix}\cos (k+1)\theta&\sin (k+1)\theta\\-\sin (k+1)\theta&\cos (k+1)\theta\end{bmatrix}
    \end{align*}

    由归纳公理知\(\matc^n=\begin{bmatrix}\cos n\theta&\sin n\theta\\-\sin n\theta&\cos n\theta\end{bmatrix}\)对\(n\in N^+\)成立。
\end{proof}

% 1.7
\begin{problem}
已知\(\begin{bmatrix}a&1&1\\3&0&1\\0&2&-1\end{bmatrix}\begin{bmatrix}3\\a\\-3\end{bmatrix}=\begin{bmatrix}b\\6\\b\end{bmatrix}\),求\(a\)和\(b\)。
\end{problem}
\begin{proof}
    计算矩阵乘法,由等式可得

    \begin{equation*}
        \begin{cases}
            3a+a-3=b       \\
            9+0\cdot a-3=6 \\
            2a+3=b
        \end{cases}
    \end{equation*}

    解线性方程组得\(a=3,b=9\)。
\end{proof}

% 1.8
\begin{problem}
若\(\mata\)是\(n\)阶方阵且\(\mata^n=\mato\),试证:

\begin{equation*}
    \mypar{\mati_n-\mata}\mypar{\mati_n+\sum_{i=1}^{n-1}\mata^i}
\end{equation*}
\end{problem}
\begin{proof}
    因为\(\mata^n=\mato\),我们得到

    \begin{align*}
          & \mypar{\mati_n-\mata}\mypar{\mati_n+\sum_{i=1}^{n-1}\mata^i} \\
        = & \mati_n+\sum_{i=1}^{n-1}\mata^i-\sum_{i=1}^n\mata^i          \\
        = & \mati_n-\mata^n=\mati_n
    \end{align*}
\end{proof}

% 1.9
\begin{problem}
计算下列矩阵的\(n\)次方幂:

\begin{enumerate}
    \item[(3)] 设\(\matc=\begin{bmatrix}1&-1&-1&-1\\-1&1&-1&-1\\-1&-1&1&-1\\-1&-1&-1&1\end{bmatrix}\),求\(\matc^n\);
    \item[(4)] 设\(\matd=\begin{bmatrix}1&1&0\\0&1&1\\0&0&1\end{bmatrix}\),求\(\matd^n\)。
\end{enumerate}
\end{problem}
\begin{proof}
    \begin{enumerate}

        \item[(3)]
            {
            容易发现
            \begin{equation*}
                \matc^2=
                \begin{bmatrix}
                    4 & 0 & 0 & 0 \\
                    0 & 4 & 0 & 0 \\
                    0 & 0 & 4 & 0 \\
                    0 & 0 & 0 & 4
                \end{bmatrix}=4\mati_4
            \end{equation*}
            因此根据奇偶性讨论,有
            \begin{equation*}
                \matc^n=
                \begin{cases}
                    2^{n-1}\matc & n=2k-1 \\
                    2^n\mati_4   & n=2k
                \end{cases}
                (k\in N^+)
            \end{equation*}
            }

        \item[(4)]
            {
            记\(\matd'=\begin{bmatrix}0&1&0\\0&0&1\\0&0&0\end{bmatrix}\),容易发现
            \begin{equation*}
                \matd'^2=\begin{bmatrix}0&0&1\\0&0&0\\0&0&0\end{bmatrix},\matd'^n=\mato(n\ge3)
            \end{equation*}
            因此有
            \begin{align*}
                \matd^n & =\mypar{\mati+\matd'}^n                 \\
                        & =\mati+\sum_{i=1}^n\binom{n}{i}\matd'^i \\
                        & =\mati+n\matd'+\binom{n}{2}\matd'^2     \\
                        & =\begin{bmatrix}
                    1 & n & n(n-1)/2 \\
                    0 & 1 & n        \\
                    0 & 0 & 1
                \end{bmatrix}
            \end{align*}
            }
    \end{enumerate}
\end{proof}

% 1.10
\begin{problem}
试证:
\begin{enumerate}
    \item 与所有\(n\)阶对角阵乘法可交换的矩阵也必是\(n\)阶对角阵;
    \item 与所有\(n\)阶矩阵乘法可交换的矩阵是纯量阵。
\end{enumerate}
\end{problem}
\begin{proof}
    \begin{enumerate}
        \item
              {
              设矩阵\(\mata=\mat{a_{ij}}_{n\times n}\)为对角阵,矩阵\(\matb=\mat{b_{ij}}_{n\times n}\)。则有
              \begin{equation*}
                  \begin{array}{lll}
                      \mat{\mata\matb}_{ij}=\sum_{k=1}^na_{ik}b_{kj}=a_{ii}b_{ij} \\
                      \mat{\matb\mata}_{ij}=\sum_{k=1}^nb_{ik}a_{kj}=a_{jj}b_{ij}
                  \end{array}
              \end{equation*}
              由题设知\(\mat{\mata\matb}_{ij}=\mat{\matb\mata}_{ij}\)对任意\(i,j\in\setof{1,2,\dots,n}\)成立。

              当\(i=j\)时,等式成立;当\(i\neq j\)时,由\(\mata\)的任意性知\(b_{ij}=0\),即\(\matb\)为对角阵。
              }
        \item
              {
              由(1)知,满足要求的矩阵为对角阵。

              设矩阵\(\mata=\mat{a_{ij}}_{n\times n}\)为对角阵,矩阵\(\matb=\mat{b_{ij}}_{n\times n}\)。

              同(1)理,可得\(a_{ii}b_{ij}=a_{jj}b_{ij}\)对任意\(i,j\in\setof{1,2,\dots,n}\)成立。由\(\matb\)的任意性可知\(a_{ii}=a_{jj}\)对任意\(i,j\in\setof{1,2,\dots,n}\)成立,即\(\mata\)为纯量阵。
              }
    \end{enumerate}
\end{proof}

% 1.11
\begin{problem}
证明:两个对角元为\(1\)的上三角阵乘积仍是对角元为\(1\)的上三角阵。
\end{problem}
\begin{proof}
    设矩阵\(\mata=\mat{a_{ij}}_{n\times n}\),矩阵\(\matb=\mat{b_{ij}}_{n\times n}\)。则有\(\mat{\mata\matb}_{ij}=\sum_{k=1}^na_{ik}b_{kj}=\sum_{k=1}^{i-1}a_{ik}b_{kj}+a_{ii}b_{ij}+\sum_{t=i+1}^na_{it}b_{tj}\)。

    当\(i>j\)时,因为\(\mata,\matb\)均为上三角矩阵,因此有\(a_{ik}=b_{tj}=0(k<i,t>=i)\),代入上式可得\(\mat{\mata\matb}_{ij}=0\);

    当\(i=j\)时,因为\(\mata,\matb\)均为对角元为\(1\)的上三角矩阵,因此有\(a_{ik}=b_{tj}=0(k<i,t>i)\),代入上式可得\(\mat{\mata\matb}_{ij}=a_{ii}b_{ij}=1\)。

    综上,\(\mata\matb\)为对角元为\(1\)的上三角矩阵。
\end{proof}

% 1.12
\begin{problem}
设\(n\)元向量\(\vecx=\begin{bmatrix}x_1\\x_2\\\vdots\\x_n\end{bmatrix}\),\(\vecy=\begin{bmatrix}y_1\\y_2\\\vdots\\y_n\end{bmatrix}\)。若\(\mata=\vecy\vecx^\top\),求\(\mata^k\mypar{k \in N^+}\)。
\end{problem}
\begin{proof}
    \begin{equation*}
        \begin{array}{lll}
            \mata=\vecy\vecx^\top=
            \begin{bmatrix}
                x_1y_1 & x_2y_1 & \cdots & x_ny_1 \\
                x_1y_2 & x_2y_2 & \cdots & x_ny_2 \\
                \vdots & \vdots & \ddots & \vdots \\
                x_1y_n & x_2y_n & \cdots & x_ny_n \\
            \end{bmatrix}         \\
            \vecx^\top\vecy=\sum_{i=1}^nx_iy_i \\
        \end{array}
    \end{equation*}

    因此有
    \begin{align*}
        \mata^k & =\underbrace{\mypar{\vecy\vecx^\top}\mypar{\vecy\vecx^\top}\cdots\mypar{\vecy\vecx^\top}}_k                    \\
                & =\vecy\underbrace{\mypar{\vecx^\top\vecy}\mypar{\vecx^\top\vecy}\cdots\mypar{\vecx^\top\vecy}}_{k-1}\vecx^\top \\
                & =\mypar{\vecx^\top\vecy}^{k-1}\vecy\vecx^\top=\mypar{\sum_{i=1}^nx_iy_i}^{k-1}
        \begin{bmatrix}
            x_1y_1 & x_2y_1 & \cdots & x_ny_1 \\
            x_1y_2 & x_2y_2 & \cdots & x_ny_2 \\
            \vdots & \vdots & \ddots & \vdots \\
            x_1y_n & x_2y_n & \cdots & x_ny_n \\
        \end{bmatrix}
    \end{align*}
\end{proof}

% 1.13
\begin{problem}
设\(n\mypar{n\ge2}\)元向量\(\vecx=\begin{bmatrix}\frac12\\0\\\vdots\\0\\\frac12\end{bmatrix}\),\(\mata=\mati_n-\vecx\vecx^\top\),\(\matb=\mati_n+2\vecx\vecx^\top\),求\(\mata\matb\)。
\end{problem}
\begin{proof}
    \begin{align*}
        \vecx^\top\vecx & =
        \begin{bmatrix}
            \frac{1}{2} & 0 & \cdots & 0 & \frac{1}{2}
        \end{bmatrix}
        \begin{bmatrix}
            \frac{1}{2} & 0 & \cdots & 0 & \frac{1}{2}
        \end{bmatrix}^\top=\frac{1}{2}                                        \\
        \mata\matb      & =\mypar{\mati_n-\vecx\vecx^\top}\mypar{\mati_n+2\vecx\vecx^\top} \\
                        & =\mati_n+\vecx\vecx^\top-2\vecx\mypar{\vecx^\top\vecx}\vecx^\top \\
                        & =\mati_n+\vecx\vecx^\top-\vecx\vecx^\top=\mati_n
    \end{align*}
\end{proof}

% 1.14
\begin{problem}
设\(\mata\)是\(m\times n\)矩阵。证明:若对于任何\(n\)元列向量\(\vecx\)成立\(\mata\vecx=\veczero\),则\(\mata=\mato_{m\times n}\)。
\end{problem}
\begin{proof}
    设\(\mata=\begin{bmatrix}\veca_1&\veca_2&\cdots&\veca_n\end{bmatrix}\),\(\vece_i\)表示第\(i\)个分量为\(1\),其余分量为\(0\)的\(n\)阶列向量。由题设可知

    \begin{equation*}
        \mata\vece_i=\veca_i=\veczero
    \end{equation*}

    对\(i=1,2,\dots,n\)均成立。因此有\(\veca_1=\veca_2=\cdots=\veca_n=\veczero\),即\(\mata=\mato\)。
\end{proof}

% 1.15
\begin{problem}
设\(n\)阶方阵\(\mata,\matb\)满足\(\mata^2=\mata\),\(\matb^2=\matb\),且\(\mypar{\mata+\matb}^2=\mata+\matb\),证明:\(\mata\matb=\mato\)。
\end{problem}
\begin{proof}
    根据\(\mata^2=\mata,\matb^2=\matb\),可将\(\mypar{\mata+\matb}^2\)展开:

    \begin{align*}
        \mypar{\mata+\matb}^2 & =\mata^2+\matb^2+\mata\matb+\matb\mata \\
                              & =\mata+\matb+\mata\matb+\matb\mata
    \end{align*}

    又因为\(\mypar{\mata+\matb}^2=\mata+\matb\),可得

    \begin{equation}\label{eq-1.15}
        \mata\matb+\matb\mata=\mato
    \end{equation}

    将\eqref{eq-1.15}式左乘\(\mata\),得到\(\mata\matb+\mata\matb\mata=\mato\);将\eqref{eq-1.15}式式左右各乘\(\mata\),得到\(2\mata\matb\mata=\mato\)。将上述两式联立解得\(\mata\matb=\mato\)。
\end{proof}

% 1.16
\begin{problem}
设\(\mata=\begin{bmatrix}1&0&1\\0&2&0\\1&0&1\end{bmatrix}\),求\(\mata^n-2\mata^{n-1}\)。
\end{problem}
\begin{proof}
    因为\(\mata^n-2\mata^{n-1}=\mata^{n-1}(\mata-2\mati)\),容易发现

    \begin{equation*}
        \mata(\mata-2\mati)=
        \begin{bmatrix}
            1 & 0 & 1 \\0&2&0\\1&0&1
        \end{bmatrix}
        \begin{bmatrix}
            -1 & 0 & 1 \\0&0&0\\1&0&-1
        \end{bmatrix}=\mato
    \end{equation*}

    因此当\(n\ge2\)时,\(\mata^n-2\mata^{n-1}=\mata^{n-2}\mata(\mata-2\mati)=\mato\)。
\end{proof}

% 1.17
\begin{problem}
设\(n\)阶方阵\(\mata,\matb\)满足\(\mata^2=-\mata\),\(\matb^2=-\matb\),且\(\mypar{\mata+\matb}^2=-\mata-\matb\),证明:\(\mata\matb=\mato\)。
\end{problem}
\begin{proof}
    根据\(\mata^2=-\mata,\matb^2=-\matb\),可将\(\mypar{\mata+\matb}^2\)展开:

    \begin{align*}
        \mypar{\mata+\matb}^2 & =\mata^2+\matb^2+\mata\matb+\matb\mata \\
                              & =-\mata-\matb+\mata\matb+\matb\mata
    \end{align*}

    又因为\(\mypar{\mata+\matb}^2=-\mata-\matb\),可得

    \begin{equation}\label{eq-1.17}
        \mata\matb+\matb\mata=\mato
    \end{equation}

    将\eqref{eq-1.17}式左乘\(\mata\),得到\(\mata\matb+\mata\matb\mata=\mato\);将\eqref{eq-1.17}式左右各乘\(\mata\),得到\(2\mata\matb\mata=\mato\)。将上述两式联立解得\(\mata\matb=\mato\)。
\end{proof}

\setcounter{problem}{18}
% 1.19
\begin{problem}
设\(\mata=\mati-\vecal\vecal^\top\),其中\(\vecal\)为非零\(n\times1\)矩阵,试证:\(\mata^2=\mata\)的充要条件是\(\vecal^\top\vecal=1\)。
\end{problem}
\begin{proof}
    \begin{itemize}
        \item 首先证明充分性:
              因为\(\vecal^\top\vecal=1\),因此有
              \begin{align*}
                  \mata^2 & =\mypar{\mati-\vecal\vecal^\top}^2                                   \\
                          & =\mati-2\vecal\vecal^\top+\vecal\mypar{\vecal^\top\vecal}\vecal^\top \\
                          & =\mati-\vecal\vecal^\top=\mata
              \end{align*}

              充分性得证。

        \item 随后证明必要性:
              因为\(\mata^2=\mata\),因此有
              \begin{align*}
                  \mata^2-\mata & =\mat{\mati-2\vecal\vecal^\top+\vecal\mypar{\vecal^\top\vecal}\vecal^\top}-\mypar{\mati-\vecal\vecal^\top} \\
                                & =\mypar{\vecal^\top\vecal-1}\vecal\vecal^\top=\mato
              \end{align*}

              因为\(\vecal\neq\veczero\),因此\(\vecal^\top\vecal-1=0\),即\(\vecal^\top\vecal=1\)。
    \end{itemize}
\end{proof}

\setcounter{problem}{21}
% 1.22
\begin{problem}
计算下列矩阵的\(k\)次幂,其中\(k\)为正整数:

\begin{enumerate}
    \item \(\mata=\begin{bmatrix}1&a&0\\0&1&a\\0&0&1\end{bmatrix}\);
    \item \(\mata=\begin{bmatrix}1&2&4\\2&4&8\\3&6&12\end{bmatrix}\)。
\end{enumerate}
\end{problem}
\begin{proof}
    \begin{enumerate}
        \item 记\(\mata'=\begin{bmatrix}0&a&0\\0&0&a\\0&0&0\end{bmatrix}\),容易发现
              \begin{equation*}
                  \mata'^2=\begin{bmatrix}0&0&a^2\\0&0&0\\0&0&0\end{bmatrix},\mata'^n=\mato(n\ge3)
              \end{equation*}
              因此有
              \begin{align*}
                  \mata^k & =\mypar{\mati+\mata'}^k                 \\
                          & =\mati+\sum_{i=1}^k\binom{k}{i}\mata'^i \\
                          & =\mati+k\mata'+\binom{k}{2}\mata'^2     \\
                          & =\begin{bmatrix}
                      1 & ka & \frac{k(k-1)}{2}a^2 \\
                      0 & 1  & ka                  \\
                      0 & 0  & 1
                  \end{bmatrix}
              \end{align*}
        \item 记\(\vecal=\begin{bmatrix}1\\2\\3\end{bmatrix}\),\(\vecbeta=\begin{bmatrix}1\\2\\4\end{bmatrix}\)。容易发现\(\mata=\vecal\vecbeta^\top\),因此有
              \begin{align*}
                  \mata^k & =\mypar{\vecal\vecbeta^\top}^k=\vecal\mypar{\vecbeta^\top\vecal}^{k-1}\vecbeta^\top \\
                          & =\mypar{17}^{k-1}\vecal\vecbeta^\top=\mypar{17}^{k-1}
                  \begin{bmatrix}
                      1 & 2 & 4 \\2&4&8\\3&6&12
                  \end{bmatrix}
              \end{align*}
    \end{enumerate}
\end{proof}

% 1.23
\begin{problem}
设\(\mata,\matb\)是两个\(n\)阶矩阵,若\(\trace{\mata\matb\matc}=\trace{\matc\matb\mata}\)对任意\(n\)阶矩阵\(\matc\)成立,求证\(\mata\matb=\matb\mata\)。
\end{problem}
\begin{proof}
令\(\matc=\vece_i\vece_j^\top\),其中\(\vece_i\)表示第\(i\)个分量为\(1\),其余分量为\(0\)的\(n\)阶列向量。因此有
    \begin{align*}
        \mat{\mata\matb\matc}_{tt} & =\sum_{k=1}^n\mat{\mata\matb}_{tk}\mat{\matc}_{kt}=
        \begin{cases}
            \mat{\mata\matb}_{ji} & t=j     \\
            0                     & t\neq j
        \end{cases}                                                       \\
        \mat{\matc\matb\mata}_{tt} & =\sum_{k=1}^n\mat{\matc}_{tk}\mat{\matb\mata}_{kt}=
        \begin{cases}
            \mat{\matb\mata}_{ji} & t=i     \\
            0                     & t\neq i
        \end{cases}
    \end{align*}

    因为\(\trace{\mata\matb\matc}=\trace{\matc\matb\mata}\),因此有\(\mat{\mata\matb}_{ji}=\mat{\matb\mata}_{ji}\)对\(i,j\in\setof{1,2,\dots,n}\)成立,即\(\mata\matb=\matb\mata\)。
\end{proof}

% 1.24
\begin{problem}
设

\begin{equation*}
    \mata=\begin{bmatrix}1&0&0&0\\0&1&0&0\\-1&2&1&0\\1&1&0&1\end{bmatrix},
    \matb=\begin{bmatrix}1&0&1&0\\-1&2&0&1\\1&0&4&1\\-1&-1&2&0\end{bmatrix}
\end{equation*}

利用分块矩阵求\(\mata\matb\)。
\end{problem}
\begin{proof}
    定义子矩阵如下:
    \begin{equation*}
        \mata_3=
        \begin{bmatrix}
            -1 & 2 \\1&1
        \end{bmatrix},
        \matb_1=
        \begin{bmatrix}
            1 & 0 \\-1&2
        \end{bmatrix},
        \matb_3=
        \begin{bmatrix}
            1 & 0 \\-1&-1
        \end{bmatrix},
        \matb_4=
        \begin{bmatrix}
            4 & 1 \\2&0
        \end{bmatrix}
    \end{equation*}

    则有
    \begin{align*}
        \mata\matb & =
        \begin{bmatrix}
            \mati & \mato \\\mata_3&\mati
        \end{bmatrix}
        \begin{bmatrix}
            \matb_1 & \mati \\\matb_3&\matb_4
        \end{bmatrix} \\&=
        \begin{bmatrix}
            \matb_1 & \mati \\\mata_3\matb_1+\matb_3&\mata_3+\matb_4
        \end{bmatrix} \\&=
        \begin{bmatrix}
            1  & 0 & 1 & 0 \\
            -1 & 2 & 0 & 1 \\
            -2 & 4 & 3 & 3 \\
            -1 & 1 & 3 & 1
        \end{bmatrix}
    \end{align*}
\end{proof}

% 附加1.1
\begin{extraprob}\label{extra-1.1}
    证明\(\row{\mata\matb}{i}=\row{\mata}{i}\matb=\sum_{k=1}^la_{ik}\row{\matb}{k}\)。
\end{extraprob}
\begin{proof}
    设\(\mata=\normmat{bmatrix}{a}{m}{l}\),\(\matb=\normmat{bmatrix}{b}{l}{n}\)。则有
    \begin{equation*}
        \mata\matb=
        \begin{bmatrix}
            \sum_{k=1}^la_{1k}b_{k1} & \sum_{k=1}^la_{1k}b_{k2} & \cdots & \sum_{k=1}^la_{1k}b_{kn} \\
            \sum_{k=1}^la_{2k}b_{k1} & \sum_{k=1}^la_{2k}b_{k2} & \cdots & \sum_{k=1}^la_{2k}b_{kn} \\
            \vdots                   & \vdots                   & \ddots & \vdots                   \\
            \sum_{k=1}^la_{mk}b_{k1} & \sum_{k=1}^la_{mk}b_{k2} & \cdots & \sum_{k=1}^la_{mk}b_{kn}
        \end{bmatrix}
    \end{equation*}
    \begin{equation*}
        \row{\mata\matb}{i}=
        \begin{bmatrix}
            \sum_{k=1}^la_{ik}b_{k1} & \sum_{k=1}^la_{ik}b_{k2} & \cdots\sum_{k=1}^la_{ik}b_{kn}
        \end{bmatrix}
    \end{equation*}
    \begin{align*}
        \row{\mata}{i}\matb & =
        \begin{bmatrix}
            a_{i1} & a_{i2} & \cdots & a_{il}
        \end{bmatrix}\normmat{bmatrix}{b}{l}{n} \\
                            & =
        \begin{bmatrix}
            \sum_{k=1}^la_{ik}b_{k1} & \sum_{k=1}^la_{ik}b_{k2} & \cdots\sum_{k=1}^la_{ik}b_{kn}
        \end{bmatrix}
    \end{align*}
    \begin{align*}
        \sum_{k=1}^la_{ik}\row{\matb}{k} & =\sum_{k=1}^la_{ik}
        \begin{bmatrix}
            b_{k1} & b_{k2} & \cdots & b_{kn}
        \end{bmatrix}                            \\
                                         & =
        \begin{bmatrix}
            \sum_{k=1}^la_{ik}b_{k1} & \sum_{k=1}^la_{ik}b_{k2} & \cdots\sum_{k=1}^la_{ik}b_{kn}
        \end{bmatrix}
    \end{align*}

    因此有\(\row{\mata\matb}{i}=\row{\mata}{i}\matb=\sum_{k=1}^la_{ik}\row{\matb}{k}\)。
\end{proof}

% 附加1.2
\begin{extraprob}
    证明\(\mata\matb=\sum_{k=1}^l\col{\mata}{k}\row{\matb}{k}\)。
\end{extraprob}
\begin{proof}
    和附加习题\ref{extra-1.1}解法类似,定义矩阵\(\mata=\mat{a_ij}_{m\times l}\),\(\matb=\mat{b_{ij}}_{l\times n}\)。则有
    \begin{equation*}
        \mata\matb=
        \begin{bmatrix}
            \sum_{k=1}^la_{1k}b_{k1} & \sum_{k=1}^la_{1k}b_{k2} & \cdots & \sum_{k=1}^la_{1k}b_{kn} \\
            \sum_{k=1}^la_{2k}b_{k1} & \sum_{k=1}^la_{2k}b_{k2} & \cdots & \sum_{k=1}^la_{2k}b_{kn} \\
            \vdots                   & \vdots                   & \ddots & \vdots                   \\
            \sum_{k=1}^la_{mk}b_{k1} & \sum_{k=1}^la_{mk}b_{k2} & \cdots & \sum_{k=1}^la_{mk}b_{kn}
        \end{bmatrix}
    \end{equation*}
    \begin{align*}
          & \sum_{k=1}^l\col{\mata}{k}\row{\matb}{k} \\ =&\sum_{k=1}^l
        \mypar{
            \begin{bmatrix}
                a_{1k} \\a_{2k}\\\vdots\\a_{mk}
            \end{bmatrix}
            \begin{bmatrix}
                b_{k1} & b_{k2} & \cdots & b_{kn}
            \end{bmatrix}
        }                                            \\
        = & \sum_{k=1}^l
        \begin{bmatrix}
            a_{1k}b_{k1} & a_{1k}b_{k2} & \cdots & a_{1k}b_{kn} \\
            a_{2k}b_{k1} & a_{2k}b_{k2} & \cdots & a_{2k}b_{kn} \\
            \vdots       & \vdots       & \ddots & \vdots       \\
            a_{mk}b_{k1} & a_{mk}b_{k2} & \cdots & a_{mk}b_{kn}
        \end{bmatrix}                  \\
        = &
        \begin{bmatrix}
            \sum_{k=1}^la_{1k}b_{k1} & \sum_{k=1}^la_{1k}b_{k2} & \cdots & \sum_{k=1}^la_{1k}b_{kn} \\
            \sum_{k=1}^la_{2k}b_{k1} & \sum_{k=1}^la_{2k}b_{k2} & \cdots & \sum_{k=1}^la_{2k}b_{kn} \\
            \vdots                   & \vdots                   & \ddots & \vdots                   \\
            \sum_{k=1}^la_{mk}b_{k1} & \sum_{k=1}^la_{mk}b_{k2} & \cdots & \sum_{k=1}^la_{mk}b_{kn}
        \end{bmatrix}
    \end{align*}
    因此\(\mata\matb=\sum_{k=1}^l\col{\mata}{k}\row{\matb}{k}\)。
\end{proof}

% 附加1.3
\begin{extraprob}
    设\(\mata,\matb\)是数域\(P\)上的\(n\)级矩阵,则\(\abs{\mata\matb}=\abs{\mata}\abs{\matb}\)。

    提示:使用性质\(\col{\mata\matb}{j}=\mata\col{\matb}{j}=\sum_{k=1}^lb_{kj}\col{\mata}{k}\)。
\end{extraprob}
\begin{proof}
设矩阵\(\mata=\mat{a_{ij}}_{n\times n}\),矩阵\(\matb=\mat{b_{ij}}_{n\times n}\)。构造\(2n\)阶方阵\(\matf\):
\begin{equation*}
    \matf=
    \begin{bmatrix}
        \mata & \mato \\\matd&\matb
    \end{bmatrix},\matd=\diag{\mat{-1,-1,\dots,-1}}
\end{equation*}

对\(\matf\)作初等列变换,设变换后的矩阵\(\matfstar\)形如:
\begin{equation*}
    \matfstar=\begin{bmatrix}\mata&\matc\\\matd&\mato\end{bmatrix}
\end{equation*}

则容易发现\(\col{\matc}{j}=\sum_{k=1}^nb_{kj}\col{\mata}{k}=\col{\mata\matb}{j}\),即\(\matc=\mata\matb\)。

又由Laplace定理可知,\(\abs{\matf}=\abs{\mata}\abs{\matb}\),\(\abs{\matfstar}=-\abs{\matc}\abs{\matd}=\abs{\mata\matb}\),配合行列式性质即可得\(\abs{\mata\matb}=\abs{\mata}\abs{\matb}\)。

\end{proof}

% 附加1.4
\begin{extraprob}
    证明:
    \begin{enumerate}
        \item \(\matc\mypar{\mata+\matb}=\matc\mata+\matc\matb\)
        \item \(\mypar{\mata+\matb}\matc=\mata\matc+\matb\matc\)
        \item \(k\mypar{\mata\matb}=\mypar{k\mata}\matb=\mata\mypar{k\matb}\)
    \end{enumerate}
    其中\(k\)为数,矩阵\(\mata,\matb,\matc\)的阶数使上述各式有意义。
\end{extraprob}
\begin{proof}
    \begin{enumerate}
        \item
              {
              设\(\matc=\mat{c_{ij}}_{m\times l}\),\(\mata=\mat{a_{ij}}_{l\times n}\),\(\matb=\mat{b_{ij}}_{l\times n}\)。则有
              \begin{align*}
                  \entry{\matc\mypar{\mata+\matb}}{i}{j} & =\row{\matc}{i}\col{\mata+\matb}{j}                        \\
                                                         & =\sum_{k=1}^lc_{ik}\mypar{a_{kj}+b_{kj}}                   \\
                                                         & =\sum_{k=1}^lc_{ik}a_{kj}+\sum_{k=1}^lc_{ik}b_{kj}         \\
                                                         & =\row{\matc}{i}\col{\mata}{j}+\row{\matc}{i}\col{\matb}{j} \\
                                                         & =\entry{\matc\mata}{i}{j}+\entry{\matc\matb}{i}{j}
              \end{align*}
              因此\(\matc\mypar{\mata+\matb}=\matc\mata+\matc\matb\)成立。
              }

        \item
              {
              设\(\mata=\mat{a_{ij}}_{m\times l}\),\(\matb=\mat{b_{ij}}_{m\times l}\),\(\matc=\mat{c_{ij}}_{l\times n}\)。则有
              \begin{align*}
                  \entry{\mypar{\mata+\matb}\matc}{i}{j} & =\row{\mata+\matb}{i}\col{\matc}{j}                        \\
                                                         & =\sum_{k=1}^l\mypar{a_{ik}+b_{ik}}c_{kj}                   \\
                                                         & =\sum_{k=1}^la_{ik}c_{kj}+\sum_{k=1}^lb_{ik}c_{kj}         \\
                                                         & =\row{\mata}{i}\col{\matc}{j}+\row{\matb}{i}\col{\matc}{j} \\
                                                         & =\entry{\mata\matc}{i}{j}+\entry{\matb\matc}{i}{j}
              \end{align*}
              因此\(\mypar{\mata+\matb}\matc=\mata\matc+\matb\matc\)成立。
              }

        \item
              {
              设\(\mata=\mat{a_{ij}}_{m\times l}\),\(\matb=\mat{b_{ij}}_{l\times n}\)。则有
              \begin{align*}
                  \entry{\mypar{k\mata}\matb}{i}{j} & =\row{k\mata}{i}\col{\matb}{j}     \\
                                                    & =\sum_{t=1}^l\mypar{ka_{it}}b_{tj} \\
                                                    & =k\sum_{t=1}^la_{it}b_{tj}         \\
                                                    & =k\row{\mata}{i}\col{\matb}{j}     \\
                                                    & =k\entry{\mata\matb}{i}{j}
              \end{align*}
              因此\(\mypar{k\mata}\matb=k\mypar{\mata\matb}\)成立。同理可得\(\mata\mypar{k\matb}=k\mypar{\mata\matb}\)成立。
              }
    \end{enumerate}
\end{proof}

% 补充1.1
\begin{suplprob}
    设\(\mata,\matb\)为\(n\)阶方阵,且\(\mata\matb=\mata+\matb\)。证明:
    \begin{enumerate}
        \item \(\mata-\mati\)可逆;
        \item \(\mata\matb=\matb\mata\)。
    \end{enumerate}
\end{suplprob}
\begin{proof}
    \begin{enumerate}
        \item
              {
              由\(\mata\matb=\mata+\matb\)可得\(\mata\matb-\mata-\matb+\mati=\mati\),即\(\mypar{\mata-\mati}\mypar{\matb-\mati}=\mati\)。由可逆矩阵定义知\(\mata-\mati\)可逆。
              }
        \item
              {
              由可逆矩阵定义可知\(\mypar{\mata-\mati}\mypar{\matb-\mati}=\mypar{\matb-\mati}\mypar{\mata-\mati}\)。

              等式两边展开化简后即得证。
              }
    \end{enumerate}
\end{proof}

% 补充1.2
\begin{suplprob}
    设\(\mata,\matb\)为\(n\)阶方阵,\(\mata\)对称且可逆,且\(\mypar{\mata-\matb}^2=\mati\)。化简:
    \begin{equation*}
        \mypar{\mati+\inv{\mata}\matb^\top}^\top\inv{\mypar{\mati-\matb\inv{\mata}}}
    \end{equation*}
\end{suplprob}
\begin{proof}
    \begin{align*}
          & \mypar{\mati+\inv{\mata}\matb^\top}^\top\inv{\mypar{\mati-\matb\inv{\mata}}}          \\
        = & \mypar{\mati+\matb\inv{\mata}}\inv{\mypar{\mata\inv{\mata}-\matb\inv{\mata}}}         \\
        = & \mypar{\mata\inv{\mata}+\matb\inv{\mata}}\inv{\mypar{\mypar{\mata-\matb}\inv{\mata}}} \\
        = & \mypar{\mypar{\mata+\matb}\inv{\mata}}\mata\inv{\mypar{\mata-\matb}}                  \\
        = & \mypar{\mata+\matb}\mypar{\mata-\matb}
    \end{align*}
\end{proof}

\clearpage
\section{线性方程组}

\setcounter{problem}{3}
% 2.4(2)
\begin{problem}
\end{problem}
\begin{proof}

\end{proof}

% 2.5
\begin{problem}
\end{problem}
\begin{proof}

\end{proof}

% 2.6
\begin{problem}
\end{problem}
\begin{proof}

\end{proof}

% 2.7
\begin{problem}
\end{problem}
\begin{proof}

\end{proof}

% 2.8
\begin{problem}
\end{problem}
\begin{proof}

\end{proof}

\setcounter{problem}{13}
% 2.14
\begin{problem}
\end{problem}
\begin{proof}

\end{proof}

% 2.15
\begin{problem}
\end{problem}
\begin{proof}

\end{proof}

\setcounter{problem}{20}
% 2.21
\begin{problem}
\end{problem}
\begin{proof}

\end{proof}

% 2.22
\begin{problem}
\end{problem}
\begin{proof}

\end{proof}

\setcounter{problem}{25}
% 2.26(3)
\begin{problem}
\end{problem}
\begin{proof}

\end{proof}

\setcounter{problem}{27}
% 2.28(1)
\begin{problem}
\end{problem}
\begin{proof}

\end{proof}

% 2.29(1)
\begin{problem}
\end{problem}
\begin{proof}

\end{proof}

% 2.30
\begin{problem}
\end{problem}
\begin{proof}

\end{proof}

% 2.31
\begin{problem}
\end{problem}
\begin{proof}

\end{proof}

% 2.32
\begin{problem}
\end{problem}
\begin{proof}

\end{proof}

% 2.33
\begin{problem}
\end{problem}
\begin{proof}

\end{proof}

\setcounter{problem}{34}
% 2.35
\begin{problem}
\end{problem}
\begin{proof}

\end{proof}

\setcounter{problem}{36}
% 2.37
\begin{problem}
\end{problem}
\begin{proof}

\end{proof}

\setcounter{problem}{39}
% 2.40(1)
\begin{problem}
\end{problem}
\begin{proof}

\end{proof}

% 2.41
\begin{problem}
\end{problem}
\begin{proof}

\end{proof}

\setcounter{problem}{43}
% 2.44
\begin{problem}
\end{problem}
\begin{proof}

\end{proof}

\setcounter{problem}{45}
% 2.46
\begin{problem}
\end{problem}
\begin{proof}

\end{proof}

% 2.47(2)
\begin{problem}
\end{problem}
\begin{proof}

\end{proof}

\clearpage

\section{行列式}

% 3.1
\begin{problem}
\end{problem}
\begin{proof}

\end{proof}

% 3.2
\begin{problem}
\end{problem}
\begin{proof}

\end{proof}

% 3.3
\begin{problem}
\end{problem}
\begin{proof}

\end{proof}

% 3.4
\begin{problem}
\end{problem}
\begin{proof}

\end{proof}

% 3.5
\begin{problem}
\end{problem}
\begin{proof}

\end{proof}

\setcounter{problem}{6}
% 3.7
\begin{problem}
\end{problem}
\begin{proof}

\end{proof}

\setcounter{problem}{8}
% 3.9
\begin{problem}
\end{problem}
\begin{proof}

\end{proof}

\setcounter{problem}{11}
% 3.12
\begin{problem}
\end{problem}
\begin{proof}

\end{proof}

% 3.13
\begin{problem}
\end{problem}
\begin{proof}

\end{proof}

% 3.14
\begin{problem}
\end{problem}
\begin{proof}

\end{proof}

\setcounter{problem}{15}
% 3.16
\begin{problem}
\end{problem}
\begin{proof}

\end{proof}

% 3.17
\begin{problem}
\end{problem}
\begin{proof}

\end{proof}

\setcounter{problem}{20}
% 3.21
\begin{problem}
\end{problem}
\begin{proof}

\end{proof}

\setcounter{problem}{22}
% 3.23
\begin{problem}
\end{problem}
\begin{proof}

\end{proof}

\setcounter{problem}{26}
% 3.27
\begin{problem}
\end{problem}
\begin{proof}

\end{proof}

\clearpage

\section{线性空间与线性变换}

% 4.1
\begin{problem}
在次数不大于\(2\)的多项式线性空间\(P_2[x]\)中,试证:\(f_1=1\),\(f_2=x-1\),\(f_3=\mypar{x-2}\mypar{x-1}\)线性无关。
\end{problem}
\begin{proof}

\end{proof}

% 4.2
\begin{problem}
证明:以下三个多项式为\(P_2[x]\)的一组基:
\begin{equation*}
    f_1=1,f_2=x-1,f_3=\mypar{x-1}^2
\end{equation*}
再求\(g(x)=5x^2+x+3\)在此基下的坐标。
\end{problem}
\begin{proof}

\end{proof}

% 4.3
\begin{problem}
在次数不大于\(3\)的多项式空间\(P_3[x]\)中,
\begin{enumerate}
    \item 求由基\(1,x,x^2,x^3\)到基\(1,1+x,\mypar{1+x}^2,\mypar{1+x}^3\)的过渡矩阵;
    \item 求\(f(x)=a_0+a_1x+a_2x^2+a_3x^3\)在基\(1,1+x,\mypar{1+x}^2,\mypar{1+x}^3\)下的坐标。
\end{enumerate}
\end{problem}
\begin{proof}

\end{proof}

% 4.4
\begin{problem}
在\(P_3[x]\)的多项式空间中,旧基为\(1,x,x^2,x^3\);新基为\(1\),\(1+x\),\(1+x+x^2\),\(1+x+x^2+x^3\)。
\begin{enumerate}
    \item 求旧基到新基的过渡矩阵;
    \item 求多项式\(1+2x+3x^2+4x^3\)在新基下的坐标;
    \item 若多项式\(f(x)\)在新基下的坐标为\(\mat{1,2,3,4}^\top\),求它在旧基下的坐标。
\end{enumerate}
\end{problem}
\begin{proof}

\end{proof}

% 4.5
\begin{problem}
已知\(\vecxi\)在基\(\matb_1=\setof{\vecal_1,\vecal_2,\vecal_3}\)下的坐标为\(\vecxi_{\matb_1}=\mat{1,-2,2}^\top\),求\(\vecxi\)在基\(\matb_2=\setof{\vecbeta_1,\vecbeta_2,\vecbeta_3}\)下的坐标\(\vecxi_{\matb_2}\),其中\(\vecbeta_1=\vecal_1+\vecal_2\),\(\vecbeta_2=\vecal_2+\vecal_3\),\(\vecbeta_3=\vecal_3+\vecal_1\)。
\end{problem}
\begin{proof}

\end{proof}

% 4.6
\begin{problem}
设\(\vecveps_1,\vecveps_2,\vecveps_3\)是线性空间\(V\)的一组基,且
\begin{equation*}
    \begin{cases}
        \vecxi_1=\vecveps_1+\vecveps_3 \\
        \vecxi_2=\vecveps_2            \\
        \vecxi_3=\vecveps_1+2\vecveps_2+2\vecveps_3
    \end{cases}
\end{equation*}
\begin{equation*}
    \begin{cases}
        \veceta_1=\vecveps_1            \\
        \veceta_2=\vecveps_1+\vecveps_2 \\
        \veceta_3=\vecveps_1+\vecveps_2+\vecveps_3
    \end{cases}
\end{equation*}
\begin{enumerate}
    \item 试证\(\vecxi_1,\vecxi_2,\vecxi_3\)及\(\veceta_1,\veceta_2,\veceta_3\)都是\(V\)的一组基;
    \item 求由基\(\vecxi_1,\vecxi_2,\vecxi_3\)到基\(\veceta_1,\veceta_2,\veceta_3\)的过渡矩阵。
\end{enumerate}
\end{problem}
\begin{proof}

\end{proof}

% 4.7
\begin{problem}
在线性空间\(\rea^{2\times2}\)中,已知
\begin{equation*}
    \vecal_1=\begin{bmatrix}1&0\\0&0\end{bmatrix},
    \vecal_2=\begin{bmatrix}0&1\\0&0\end{bmatrix},
    \vecal_3=\begin{bmatrix}0&0\\1&0\end{bmatrix},
    \vecal_4=\begin{bmatrix}0&0\\0&1\end{bmatrix}
\end{equation*}
为其一组基,若\(\rea^{2\times2}\)的另一组基为\(\vecbeta_1,\vecbeta_2,\vecbeta_3,\vecbeta_4\),由\(\vecal_1,\vecal_2,\vecal_3,\vecal_4\)到\(\vecbeta_1,\vecbeta_2,\vecbeta_3,\vecbeta_4\)的过渡矩阵为
\begin{equation*}
    \mata=
    \begin{bmatrix}
        0 & 1 & 1 & 1 \\
        1 & 0 & 1 & 1 \\
        1 & 1 & 0 & 1 \\
        1 & 1 & 1 & 0
    \end{bmatrix}
\end{equation*}
求:
\begin{enumerate}
    \item 基\(\vecbeta_1,\vecbeta_2,\vecbeta_3,\vecbeta_4\);
    \item 矩阵\begin{equation*}\begin{bmatrix}0&1\\2&-3\end{bmatrix}\end{equation*}在基\(\vecbeta_1,\vecbeta_2,\vecbeta_3,\vecbeta_4\)下的坐标。
\end{enumerate}
\end{problem}
\begin{proof}

\end{proof}

% 4.8
\begin{problem}
已知实数域\(\rea\)上的所有\(2\)阶矩阵,对于矩阵的加法和数乘,构成\(\rea\)上的四维线性空间,记作\(V=\rea^{2\times2}\)。
\begin{enumerate}
    \item {
          分别证明
          \begin{equation*}
              \vecal_1=\begin{bmatrix}1&0\\0&0\end{bmatrix},
              \vecal_2=\begin{bmatrix}1&1\\0&0\end{bmatrix},
              \vecal_3=\begin{bmatrix}1&1\\1&0\end{bmatrix},
              \vecal_4=\begin{bmatrix}1&1\\1&1\end{bmatrix}
          \end{equation*}
          与
          \begin{equation*}
              \vecbeta_1=\begin{bmatrix}-1&1\\1&1\end{bmatrix},
              \vecbeta_2=\begin{bmatrix}1&-1\\1&1\end{bmatrix},
              \vecbeta_3=\begin{bmatrix}1&1\\-1&1\end{bmatrix},
              \vecbeta_4=\begin{bmatrix}1&1\\1&-1\end{bmatrix}
          \end{equation*}
          均为\(\rea^{2\times2}\)的基;}
    \item 对于\(\rea^{2\times2}\)中的任意元素\(\vecal\),求\(\vecal\)在基\(\vecal_1,\vecal_2,\vecal_3,\vecal_4\)下的坐标;
    \item 求由基\(\vecal_1,\vecal_2,\vecal_3,\vecal_4\)到\(\vecbeta_1,\vecbeta_2,\vecbeta_3,\vecbeta_4\)的过渡矩阵。
\end{enumerate}
\end{problem}
\begin{proof}

\end{proof}

% 4.9
\begin{problem}
求下列两个齐次线性方程组的解空间的基和维数:
\begin{enumerate}
    \item \(x_1+x_2+\cdots+x_n=0\);
    \item \begin{equation*}
              \begin{cases}
                  2x_1-4x_2+5x_3+3x_4=0 \\
                  3x_1-6x_2+4x_3+2x_4=0 \\
                  4x_1-8x_2+17x_3+11x_4=0
              \end{cases}
          \end{equation*}
\end{enumerate}
\end{problem}
\begin{proof}

\end{proof}

% 4.10
\begin{problem}
设\(\enums{\vecal}{n}\)是\(n\)维线性空间\(V\)的一组基,又\(V\)中向量\(\vecal_{n+1}\)在这组基下坐标\(\mypar{\enums{x}{n}}\)全都不为零。证明\(\enums{\vecal}{n},\vecal_{n+1}\)中任意\(n\)个向量必构成\(V\)的一组基,并求\(\vecal_1\)在基\(\vecal_2,\dots,\vecal_n,\vecal_{n+1}\)下的坐标。
\end{problem}
\begin{proof}

\end{proof}

\setcounter{problem}{11}
% 4.12
\begin{problem}
设\(V_1,V_2\)是\(\rea^n\)的两个非平凡子空间,证明:在\(\rea^n\)中存在向量\(\vecal\),使\(\vecal\notin V_1\),且\(\vecal\notin V_2\),并在\(\rea^3\)中举例说明此结论。
\end{problem}
\begin{proof}

\end{proof}

% 4.13
\begin{problem}
设\(\vecal,\vecbeta,\vecgamma\in\rea^n\),\(c_1,c_2,c_3\in\rea\),且\(c_1c_3\neq0\),证明:若\(c_1\vecal+c_2\vecbeta+c_3\vecgamma=\veczero\),则\(\myspan{\vecal,\vecbeta}=\myspan{\vecbeta,\vecgamma}\)。
\end{problem}
\begin{proof}

\end{proof}

% 4.14
\begin{problem}
若
\begin{equation*}
    \mata=\begin{bmatrix}1&1&1&0\\2&1&0&1\end{bmatrix}
\end{equation*}
求\(\mata\)的零空间\(\nullsp{\mata}\)的一组基。
\end{problem}
\begin{proof}

\end{proof}

\setcounter{problem}{15}
% 4.16
\begin{problem}
已知向量组\(\vecal_1=\mat{2,0,1,3,-1}^\top\),\(\vecal_2=\mat{0,-2,1,5,-3}^\top\),\(\vecbeta_1=\mat{1,1,0,-1,1}^\top\),\(\vecbeta_2=\mat{1,-3,2,0,5}^\top\),且\(W_1=\myspan{\vecal_1,\vecal_2}\),\(W_2=\myspan{\vecbeta_1,\vecbeta_2}\)。试求\(W_1\cap W_2\)与\(W_1+W_2\)的维数以及各自的一组基。
\end{problem}
\begin{proof}

\end{proof}

% 4.17
\begin{problem}
\end{problem}
\begin{proof}

\end{proof}

\setcounter{problem}{19}
% 4.20
\begin{problem}
\end{problem}
\begin{proof}

\end{proof}

% 4.21
\begin{problem}
\end{problem}
\begin{proof}

\end{proof}

% 4.22
\begin{problem}
\end{problem}
\begin{proof}

\end{proof}

% 4.23
\begin{problem}
\end{problem}
\begin{proof}

\end{proof}

% 4.24
\begin{problem}
\end{problem}
\begin{proof}

\end{proof}

\setcounter{problem}{25}
% 4.26
\begin{problem}
\end{problem}
\begin{proof}

\end{proof}

% 4.27
\begin{problem}
\end{problem}
\begin{proof}

\end{proof}

% 4.28
\begin{problem}
\end{problem}
\begin{proof}

\end{proof}

% 4.29
\begin{problem}
\end{problem}
\begin{proof}

\end{proof}

\setcounter{problem}{30}
% 4.31
\begin{problem}
\end{problem}
\begin{proof}

\end{proof}

\setcounter{problem}{32}
% 4.33
\begin{problem}
\end{problem}
\begin{proof}

\end{proof}

\end{document}
